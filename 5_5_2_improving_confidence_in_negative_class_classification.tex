陰性クラスの分類においても、確信度の向上が観察された。ベースラインでは、確信度1.000(陰性割合1.000)のノードは、Elasticsearch、Hazelcast、Netty、OrientDBの4プロジェクトで合計6個存在した。ステップ3では、この数は1個に減少したが、これは分類ルールがより複雑になったためである。一方で、確信度0.9以上のノードを見ると、ベースラインでは2個(Nettyの0.965、Neo4jの0.917)であったのに対し、ステップ3では8個に増加した。特に、Elasticsearchでは0.973と1.000、Hazelcastでは0.965、Neo4jでは0.952、Nettyでは0.975と0.963と0.912、OrientDBでは0.937という高確信度のノードが出現した。

この変化は、時系列変化メトリクスにより、バグが混入していない変更の特徴をより正確に捉えられるようになったことを示唆している。ベースラインでは静的な特徴量のみを用いていたため、完全な確信度(1.000)でしか陰性に分類できない場合があったが、ステップ3では変更メトリクスという追加情報により、0.9以上の高い確信度で多様な陰性パターンを識別できるようになった。