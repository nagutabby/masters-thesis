構築したモデルがどのような判断基準で予測を行っているかを理解するため、以下の分析を実施する。

特徴量の寄与度の算出: ランダムフォレストが提供するFeature Importanceを計算し、どの特徴量が予測に最も寄与しているかを明らかにする。Feature Importanceは、各特徴量が決定木の分岐においてどの程度情報利得をもたらしたかを示す指標である。この分析により、メソッド単位の変更メトリクスとコミット単位の変更メトリクスのうち、どの特徴量がバグ予測に重要であるかを定量的に評価できる。

Partial Dependence Plot(PDP)の生成: 各特徴量と陽性予測確率の関係を可視化するため、PDPを生成する。PDPは、特定の特徴量の値を変化させたときに予測確率がどのように変化するかを示すグラフであり、特徴量と予測結果の関係を直感的に理解できる。

決定木の可視化: ランダムフォレストを構成する決定木の一つを可視化し、どのような分類条件でバグの有無を判断しているかを確認する。決定木の可視化により、モデルの判断基準と確信度を具体的に把握できる。

これらの実験手順により、提案手法の有効性を多角的に評価し、時系列変化を考慮することがソフトウェア欠陥予測にどのような効果をもたらすかを明らかにする。