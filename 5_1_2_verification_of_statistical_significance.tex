ベースラインとステップ3の予測結果に統計的に有意な差があるかを検証するため、マクネマー検定を実施した。マクネマー検定は、同じテストデータに対する2つの分類器の予測結果を比較するための統計的検定手法である。有意水準は0.01とし、p値が0.01未満の場合に有意な差があると判断する。

検定の結果、全てのプロジェクトにおいてp値は0.000000となり、有意水準0.01を大きく下回った。これは、ベースラインとステップ3の予測結果の差が偶然ではなく、統計的に有意であることを示している。すなわち、メソッド・コミット単位の変更メトリクスを追加することで、予測性能が統計的に有意に向上したことが実証された。