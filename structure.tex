
システムは学習対象を保持する部分、学習を管理する部分、情報提供する部分から構成される。
図\ref{fig:システム構成}にシステム構成を示す。

\begin{itemize}

\item{情報保持部(Library)}

情報保持部は課程、教材、学習の過程で作成される全ての情報を保持する。
学習対象で最大のものは課程(Course)である。
課程は複数の教材(Book)から構成される。
教材には初級、中級などの段階(rank)が付随する。
教材の最後には完了試験(CompletionExam)が、
課程内の全ての教材を習得した後には修了試験(QualificationExam)が含まれる。

学習が進行すると最初に教材に対応した
学習記録(ScoreRec)が作成される。
複数の学習者が同じ教材を学習すると、結果が学習進捗記録(ScoreBook)に追加される。
教育担当者が教材を改善すると改善前後の正答率の変換が教材改善進捗記録(ScoreHistory)に追加される。


\item{学習管理部(Runner)}

学習管理部は学習者に教材選択、学習、完了試験受験の各機能を提供する。
システムが起動し、教材が登録された後でまだ学習者がいない場合は学習記録は空である。
学習が行われると学習者ごとに学習記録が追加される。
教材を複数の学習者が学習した時に、教材ごとに学習進捗状況が追加される。
教材が更新される時に課程ごとに学習進捗状況が教材改善進捗状況に追加される。

\item{情報分析部(Viewer)}

情報分析部は各利用者が必要とする情報を分析した結果を提供する。
学習者には学習記録から教材の学習状況(合否)を、
教育担当者には教材ごと、学習者集合ごとの正答率及び合格率を、
課程設計者には課程に対する教材の改善履歴及び改善による正答率及び合格率の上昇を与える。

\end{itemize}

\clearpage

\begin{figure}[H]
  \vspace{0.5cm}
  \hspace*{0.5cm}
%%  \includegraphics[width=13cm,bb=0 350 800 600]{../system.pdf}
  % \includegraphics[width=13cm]{../system.pdf}
  \caption{システム構成}
  \label{fig:システム構成}
\end{figure}


課程(course1)とそれに含まれる教材群はすでに登録されているとする。
学習者1(st1)が初めてこの課程を学習する場合を考える。
課程の最初の教材(book1)を選択する。
完了試験後に結果が学習記録(sr1)に記録される。
別の学習者2(st2)が同様に最初の教材book1を学習し、
試験結果がsr2に記録される。

教育担当者(pf1)が自分が作成した教材(book1)の学習進捗記録を要求する。
今までbook1を学習した学習記録(sr1,sr2)を含む記録s-book1を返す。
ある程度の人数が学習した後で
合格率が必要な場合は学習記録に含まれる情報から計算した結果を返す。
この時に正答数が著しく低い問題が発見されたら、
教材の改善を流すメッセージを返す。
教育担当者は説明を改善した教材(book1n)を
book1の更新として課程に登録する。
以降の学習者はbook1の代わりにbook1nを学習し、
学習進捗はs-book1nに記録される。

課程設計者の要求に応じて
その課程に含まれるすべての教材に対する改善進捗記録s-history1を返す。
教材の改善前後における合格率の上昇度を必要とする場合は
s-history1から計算した結果を共に返す。

課程の更新の一例として最初に学習する教材(book1/book1n/...)の前に
より基礎的な内容の教材(book0)が追加された場合を考える。
以降の学習者はbook1の代わりにbook0から学習を開始し、
その完了試験に合格した後でbook1を学習する。
これによりbook1の学習効果が改善されることが期待される。

\clearpage