3.4節で述べたナップサック問題アプローチに基づき、提案手法によるレビュー工数削減効果を評価する。

\paragraph{レビュー労力の計算}
各コミット $i$ のレビュー労力 $W_i$ を、3.4節で定義した以下の式に従って計算する。

まず、コードチャーン $C_i$ を計算する。
\[
C_i = LA_i + LD_i
\]
ここで、$LA_i$ は追加されたコード行数、$LD_i$ は削除されたコード行数である。

次に、変更の分散度を表すEntropy $H_i$ を計算する。
\[
H_i = -\sum_{k=1}^{n_i} p_k \log_2 p_k
\]
ここで、$n_i$ はコミット $i$ で変更されたファイル数、$p_k$ はファイル $k$ が変更全体に占める割合である。
\[
p_k = \frac{\text{file}_k \text{の変更行数}}{\text{全変更行数}}
\]

Entropyを正規化する。
\[
H_i^{\text{norm}} = \frac{H_i}{\log_2 n_i}
\]

ベース労力 $E_{\text{raw}, i}$ を計算する。
\[
E_{\text{raw}, i} = C_i \times N_i^{H_i^{\text{norm}}}
\]
ここで、$N_i$ はコミット $i$ で変更されたファイル数である。

最後に、対数変換を適用して補正済み労力 $W_i$ を得る。
\[
W_i = E_{\text{adj}, i} = \ln(E_{\text{raw}, i} + 1)
\]

\paragraph{レビュー総労力の設定}
ナップサックの容量 $C_{total}$(レビューに使える総労力)を以下の手順で設定する。

\begin{enumerate}
    \item 全コミットをレビュー労力 $W_i$ の昇順にソートする
    \item 上位80\%のコミット(レビュー労力が小さい方から80\%)を選択する
    \item 選択されたコミットのレビュー労力の和を $C_{total}$ とする
\end{enumerate}

\[
C_{total} = \sum_{i \in S_{80\%}} W_i
\]

この設定により、極端に大きな労力を要するコミット(上位20\%)の影響を除外し、より現実的なレビュー労力の制約をモデル化できる。

\paragraph{貪欲法によるレビュー対象の選択}
各モデル(ベースライン、ステップ2、ステップ3)について、以下の手順でレビュー対象を選択する。

\begin{enumerate}
    \item 各コミット $i$ について、モデルが予測したバグ混入確率 $\hat{y}_i$ と補正済み労力 $E_{\text{adj}, i}$ から密度 $D_i$ を計算する
    \[
    D_i = \frac{V_i}{W_i} = \frac{\hat{y}_i}{E_{\text{adj}, i}}
    \]
    \item 密度の降順にコミットをソートする
    \item 累積労力 $W_{\text{累積}} = 0$ とする
    \item ソートされた順にコミットを選択し、以下を実行する:
    \begin{itemize}
        \item $W_{\text{累積}} + W_i \leq C_{total}$ であれば、コミット $i$ をレビュー対象に追加し、$W_{\text{累積}} \leftarrow W_{\text{累積}} + W_i$ とする
        \item そうでなければ、コミット $i$ をスキップする
    \end{itemize}
    \item 累積労力が容量を超えるまで、または全てのコミットを検討するまで繰り返す
    \item 各コミットをレビューするごとに、累積レビュー労力と累積発見バグ数を記録する
\end{enumerate}

\paragraph{Cost-Benefit Curveの作成}
提案手法のレビュー労力削減効果を評価するため、Cost-Benefit Curveを作成する。

Cost-Benefit Curveは、横軸に投入したレビュー労力、縦軸に発見したバグ数をプロットしたグラフである。各モデル(ベースライン、ステップ2、ステップ3)について、上記の貪欲法により選択されたコミットの順序で累積レビュー労力と累積発見バグ数を記録し、曲線を描く。

この比較により、労力を考慮したコードレビューモデルという統一的な評価枠組みの中で、特徴量エンジニアリング(メソッド単位とコミット単位の変更メトリクスの追加)の効果を測定できる。提案手法がベースライン手法よりも左上に位置する曲線を描く場合、同じレビュー労力でより多くのバグを発見できることを意味し、提案した特徴量の有効性が実証される。

\paragraph{評価指標}
Cost-Benefit Curveに加えて、以下の指標を用いて評価を行う。

\begin{itemize}
    \item \textbf{AUC(Area Under Curve)}: Cost-Benefit Curveの下の面積。大きいほど効率的にバグを発見できることを示す
    \item \textbf{特定労力時点でのバグ発見数}: 例えば、総労力の20\%、50\%時点で発見できたバグ数を比較する
    \item \textbf{特定バグ数到達に要する労力}: 例えば、全バグの50\%、80\%を発見するのに要した労力を比較する
\end{itemize}

これらの実験手順により、提案手法の有効性を多角的に評価し、時系列変化を考慮することがソフトウェア欠陥予測およびレビュー労力削減にどのような効果をもたらすかを明らかにする。