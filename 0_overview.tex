In software development, continuous code changes bring about defect risk. Existing defect prediction methods use structural metrics that cannot capture the characteristics of changes. Recent studies have investigated change metrics from the version control history and have shown that change-related features are more strongly correlated with defects than structural metrics alone. However, these approaches typically 1) do not consider irregularly occurring change events and collect change information in specific versions, and 2) do not combine the characteristics of different components and focus only on the metrics of specific components.

In this work, we propose a defect prediction method that differs from existing approaches in three aspects. First, unlike traditional methods that partition the state at regular intervals, we treat commits as irregularly occurring events and extract features from the differences between consecutive commits. Thus, we capture the irregular development processes in software projects. Second, unlike existing studies that focus on single components, our method combines per-method metrics that reflect local changes and per-commit metrics that reflect global changes, and captures both local and global trends simultaneously. Third, our method selects commits for review by considering both the review effort and the probability of defect introduction based on the scale of changes, the complexity of changes, and the spread of changes. This provides a more effective review prioritization method under review effort constraints.

Experiments on five open-source projects validate the effectiveness of our method. We confirm that the prediction accuracy is improved in all projects, and the proposed method outperforms the baseline method using only structural metrics. The defect discovery rate is improved by the review prioritization method considering more features. This study investigates a method for analyzing software changes as irregular time-series data and presents a method for combining change metrics of multiple components, which provides a defect prediction method that can be easily applied to actual quality assurance activities.

ソフトウェア開発では、継続的なコード変更は欠陥リスクを伴う。既存の欠陥予測手法は構造的メトリクスを用いており、変更の特徴を捉えることができない。最近の研究では、バージョン管理履歴からの変更メトリクスが調査されており、変更に関連する特徴は、構造的メトリクスのみよりも欠陥と強く相関することが示されている。しかし、これらのアプローチは大抵、1) 不規則に発生する変更イベントを考慮せず、特定のバージョンにおける変更情報を収集しており、2) 異なる構成要素の特徴を組み合わせず、特定の構成要素のメトリクスのみに焦点を当てている。

本研究では、3つの点で既存のアプローチと異なる欠陥予測手法を提案する。第一に、規則的な間隔で状態を分割する従来の方法とは異なり、コミットを不規則に発生するイベントとして扱い、連続するコミット間の差から特徴を抽出する。これにより、ソフトウェア開発プロジェクトにおける不規則な開発プロセスを捉える。第二に、既存の研究が単一の構成要素に焦点を当てているのに対して、この手法では局所的な変更を反映するメソッド単位のメトリクスと、全体的な変更を反映するコミット単位のメトリクスを組み合わせ、局所的・全体的な傾向の両方を同時に捉える。第三に、この方法は、変更の規模、変更の複雑さ、および変更の広がりに基づくレビュー労力と欠陥混入確率を考慮してレビュー対象のコミットを選択する。これにより、レビュー労力の制約の下でより効果的なレビュー優先度付け手法を提供する。

5つのオープンソースプロジェクトでの実験で、この方法の有効性を検証した。すべてのプロジェクトで予測精度が向上し、構造的メトリクスのみを使用した既存手法よりも提案手法がより優れていることを確認した。より多くの特徴量を考慮したレビューの優先度付け手法により、欠陥発見率が改善した。本研究は、ソフトウェアの変更を不規則な時系列データとして解析する方法を検討し、複数の構成要素の変更メトリクスを組み合わせる方法を提示することで、実際の品質保証活動に適用しやすい欠陥予測手法を提供する。