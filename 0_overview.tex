In software development, continuous code changes bring about defect risk. Existing defect prediction methods use structural metrics. However, these metrics cannot capture the characteristics of changes. Recent studies have investigated change metrics from the version control history. They have shown that change-related features are more strongly correlated with defects than structural metrics. However, these approaches have two problems. First, they do not consider irregularly occurring change events. Instead, they collect change information at specific versions. Second, they do not combine the characteristics of different components. Instead, they focus on the metrics of specific components.

In this work, we propose a defect prediction method that differs from existing approaches in three aspects. First, traditional methods partition the state at regular intervals. In contrast, we treat commits as irregularly occurring events and extract features from the differences between consecutive commits. Thus, we capture the irregular development processes in software projects. Second, existing studies focus on single components. In contrast, our method combines per-method metrics that reflect local changes and per-commit metrics that reflect global changes. This allows us to capture both local and global trends simultaneously. Third, our method selects commits for review by considering both the review effort and the probability of defect introduction. This consideration is based on the scale of the changes, the complexity of the changes, and the spread of the changes. Consequently, this provides a more effective review prioritization method under review effort constraints. 

Experiments on five open-source projects validate the effectiveness of our method. We confirm that the prediction accuracy has been improved in all projects. Moreover, the proposed method outperforms existing methods using only structural metrics. The defect discovery rate is improved by the review prioritization method considering more features.