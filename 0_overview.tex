本研究では、コードの時系列変化を考慮したソフトウェア欠陥予測手法を提案する。提案手法では、コミットを不規則なタイミングで発生するイベントとして位置付け、直前のコミットとの差分に基づく変化量を特徴量として活用する。具体的には、メソッド単位での変化量(コード行数、トークン数、循環的複雑度の変化)とコミット単位での変化率(変更ファイル数、追加・削除行数の割合など)という、異なる粒度のメトリクスを統合することで、ミクロ的視点とマクロ的視点の両面から変更の特性を捉える。また、レビュー対象の選択をナップサック問題として定式化し、各コミットのレビュー労力を考慮した上で、バグ発見期待値を最大化する手法を提案する。

5つのオープンソースプロジェクトを用いた実験により、提案手法の有効性を検証した。ランダムフォレストを用いた予測モデルにおいて、時系列変化を考慮することでF1スコアが向上することを確認した。また、レビュー労力に対する欠陥発見数の分析の結果、同じレビュー労力の下で従来手法よりも多くのバグを検出できることが示された。特に、Neo4jプロジェクトではレビュー労力40\%時点で17.0ポイント、Nettyプロジェクトでは20\%時点で19.8ポイントの改善が確認された。これにより、継続的な機能拡張が行われるソフトウェア開発において、効率的な品質保証活動を支援する実用的な手法を提供する。