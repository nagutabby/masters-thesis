ソフトウェア開発において、継続的な変更は欠陥混入のリスクを伴う。静的メトリクスに依存する既存の欠陥予測手法では、変更頻度やパターンといった動的なリスク要因を捉えられず、予測精度に限界がある。本研究では、コードの時系列変化を考慮した欠陥予測手法を提案する。

提案手法では、コミットを不規則なタイミングで発生するイベントとして扱い、直前のコミットとの差分に基づく変化を特徴量として活用する。メソッド単位やコミット単位で計測される、異なる構成要素の変更メトリクスを統合することで、局所的な変化とプロジェクト全体の動向を捉える。また、レビュー対象のコミットの選択をナップサック問題として定式化し、限られた労力でより多くの欠陥を発見できるようにする。

5つのオープンソースプロジェクトを用いた実験により、提案手法の有効性を検証した。全プロジェクトでF1スコアの向上が確認され、平均改善幅は0.21であった。レビュー労力に対する欠陥発見数の分析では、Neo4jプロジェクトでレビュー労力40\%時点で17.0\%、Nettyプロジェクトで20\%時点で19.8\%の改善が見られた。本手法は、限られたレビューリソースの効率的な配分を支援し、ソフトウェア開発における品質保証活動の実用性向上に貢献する。