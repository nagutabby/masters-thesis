In software development, continuous code changes introduce defect risks. Existing defect prediction methods rely on structural metrics, which fail to capture features of the changes. Recent research has explored change metrics from version control history, showing that change-related features correlate more strongly with defects than structural metrics alone. However, these approaches typically aggregate change data over time windows without explicitly modeling the irregular temporal structure of commits, and focus on either method-level or commit-level metrics in isolation without integrating insights from different granularities.

This research proposes a defect prediction method that differs from existing approaches in three key aspects. First, unlike conventional methods that impose artificial time windows or assume regular intervals, the proposed method treats commits as irregularly-timed events, deriving features from differences between consecutive commits. This formulation captures the irregular development frequency in software projects. Second, while existing research focuses on single granularity, the proposed method integrates method-level metrics tracking localized modifications with commit-level metrics reflecting overall scope, capturing both local changes and global trends simultaneously. Third, the method formulates commit selection in reviews as a knapsack problem that explicitly accounts for review effort based on change volume, modification complexity, and change scope to provide better resource allocation under constraints.

Experiments on five open-source projects validate the method's effectiveness, demonstrating consistent improvements in prediction accuracy across all projects and outperforming baseline methods using only structural metrics. The knapsack-based review strategy shows improvements in defect discovery rates at various effort levels. This research contributes by presenting a method for analyzing commits as irregular time-series data, developing a method for integrating multi-granularity change metrics, and providing evidence of applicability in real-world quality assurance.

ソフトウェア開発では、継続的なコード変更は欠陥リスクをもたらす。既存の欠陥予測方法は構造的メトリクスに依存しており、変更の特徴を捉えることができない。最近の研究では、バージョン管理履歴からの変更メトリクスが調査されており、変更に関連する特徴は、構造的メトリクスのみよりも欠陥と強く相関することが示されている。しかし、これらのアプローチは通常、コミットの不規則な時間構造を明示的にモデル化することなく、時間ウィンドウにわたって変更データを集約し、異なる粒度からの洞察を組み合わせることなく、メソッド単位やコミット単位のメトリクスのいずれかに分離して焦点を当てる。

本研究では、3つの重要な点で既存のアプローチと異なる欠陥予測法を提案する。第一に、人工的な時間窓を課したり、規則的な間隔を仮定したりする従来の方法とは異なり、提案した方法はコミットを不規則なタイミングのイベントとして扱い、連続するコミット間の差から特徴を導出する。この定式化は、ソフトウェアプロジェクトにおける不規則な開発頻度を捕捉する。第二に、既存の研究が単一の粒度に焦点を当てているのに対して、提案された方法は、局所的な修正を追跡するメソッド単位のメトリクスと、全体的な範囲を反映するコミット単位のメトリクスを組み合わせ、局所的な変更とグローバルな傾向の両方を同時に捕捉する。第三に、この方法は、レビューにおけるコミット選択を、変更量、変更の複雑さ、および変更範囲に基づいてレビュー作業を明示的に説明するナップサック問題として定式化し、制約の下でより良いリソース割り当てを提供する。

5つのオープンソースプロジェクトでの実験で、この方法の有効性を検証し、すべてのプロジェクトで一貫して予測精度が向上し、構造的メトリクスのみを使用したベースライン方法よりも優れていることを実証した。ナップサックベースのレビュー戦略は、さまざまな作業単位で欠陥発見率の改善を示している。本研究は、コミットを不規則な時系列データとして解析する方法を提示し、複数の粒度の変更メトリクスを組み合わせる方法を開発し、実世界の品質保証に適用可能であることの証拠を提供することに貢献する。