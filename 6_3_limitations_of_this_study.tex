本研究にはいくつかの限界が存在する。

第一に、評価に用いたデータセットが特定のオープンソースプロジェクトに限定されている点である。企業の商用プロジェクトでは、開発プロセスやチーム構成が異なるため、同様の結果が得られるとは限らない。特に、ドメイン固有の知識が重要となるプロジェクトでは、汎用的なメトリクスだけでは不十分な可能性がある。

第二に、レビューコストのモデル化が簡略化されている点である。実際のレビュープロセスでは、コードの行数以外にも、変更の複雑さ、レビュアーの専門性、コミュニケーションコストなど、多様な要因がレビュー工数に影響する。より精緻なコストモデルの構築は今後の課題である。

第三に、予測モデルの解釈可能性が十分に検討されていない点である。機械学習モデルは高い予測精度を達成できるが、なぜそのような予測をしたのかを説明することが難しい場合がある。開発現場で受け入れられるためには、予測の根拠を示すことが重要である。

第四に、時間的な要因を考慮していない点である。本研究では静的なデータセットを用いた評価を行ったが、実際のプロジェクトでは時間経過とともにコードベースや開発チームが変化する。モデルの経年劣化や再学習の戦略については、さらなる検討が必要である。