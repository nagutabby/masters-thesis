%%概要

本研究は、現代社会で様々な利用背景に対応できる、
利用形態が多様化するeラーニングシステムのアーキテクチャの構築を目的にする。
具体的には、eラーニングシステム例を対象にする、
必要とされる機能を実現するための仕組み
(アーキテクチャ)を調査し、
適切なアーキテクチャを選択した上で、
アーキテクチャの改良も行う。
最終的にステークホルダーを確定し、
アーキテクチャを提案する。
提案したアーキテクチャを評価するために、
既存システムと同様の機能及び特徴を持つシステムを作成し、
不足している機能の実現を図る。
その不足している機能を実現するための
システムの改良に必要な作業量をプロトタイプ上で計測する。
プロトタイプの作成にはJavaを使用する。
