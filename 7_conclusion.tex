本研究では、ソフトウェア開発におけるコードレビューの効率化を目的として、変更行数、変更の複雑さ、変更の広がりの3つの要素を考慮した欠陥予測モデルを構築し、レビュー労力に対する欠陥発見数を分析する手法を提案した。具体的には、コミットレベルとメソッドレベルの両方のメトリクスを活用し、限られたレビューリソースの下で欠陥検出を実現する手法を開発した。

提案手法の有効性を検証するため、複数のオープンソースプロジェクトを対象とした実験を行った。実験では、従来のメトリクスのみに基づく手法と比較評価を実施した。その結果、提案手法は同じレビュー工数で従来手法よりも多くの欠陥を検出できることが示された。特に、レビューリソースが限定的な状況において、提案手法の優位性が顕著であった。さらに、レビュー労力に対する欠陥発見数の分析により、レビュー戦略を改善できることを示した。この分析手法は、プロジェクトの特性や利用可能なリソースに応じて、適切なレビュー対象の選定を支援する。

本研究の主な貢献は以下の3点である。第一に、コミットレベルとメソッドレベルのメトリクスを統合した欠陥予測モデルを構築した。従来研究では、いずれか一方のレベルに焦点を当てたものが多かったが、本研究では両者を組み合わせることで予測精度の向上を実現した。これにより、より実用的な欠陥予測が可能となった。第二に、レビュー労力を考慮した欠陥予測モデルによるレビュー戦略の改善手法を提案した。この手法により、レビューにかけられる工数に応じた適切なレビュー対象の選定が可能となる。また、ナップサック問題としての定式化により、効率的な最適化アルゴリズムの適用が可能となった。第三に、複数のプロジェクトでの実証実験を通じて、提案手法の有効性を定量的に示した。実験結果から、提案手法が従来手法と比較して優れた性能を持つことが確認された。また、プロジェクトごとの性能差についても分析を行い、提案手法の適用条件に関する知見を得た。

本研究の成果を基盤として、今後さらなる発展が期待される。まず、提案手法の汎用性を高めるため、より多様なプロジェクトでの評価が必要である。本研究ではオープンソースプロジェクトを対象としたが、商用プロジェクトや異なるドメインのプロジェクトでの検証により、適用範囲を拡大できる。次に、動的な環境への適応が重要な課題となる。プロジェクトの進行に伴い、コードベースや開発チームは変化する。これらの変化に対応するため、継続的な学習を行う仕組みの構築が求められる。また、レビュープロセス全体の最適化への拡張も有望な研究方向である。本研究ではレビュー対象の選定に焦点を当てたが、レビュアーの割り当てやレビューの深さの調整なども含めた総合的な最適化により、さらなる効率化が期待できる。最後に、他の品質保証活動との統合も重要である。コードレビューは品質保証活動の一部であり、自動テストやソフトウェア構造の分析などの他の活動と組み合わせることで、より効果的な品質保証が可能となる。複数の活動を統合的に最適化する枠組みの構築は、ソフトウェア工学における重要な研究課題である。

本研究で提案した手法は、ソフトウェア開発における品質保証活動の効率化に貢献する。今後、実際の開発現場での適用を通じて、さらなる改善と発展が期待される。