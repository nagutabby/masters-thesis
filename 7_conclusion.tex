本研究では、ソフトウェア開発におけるコードレビューの効率化を目的として、コミット単位とメソッド単位の変更メトリクスを活用した欠陥予測手法を提案した。提案手法では、コミットを不規則なタイミングで発生するイベントとして扱い、直前の状態からの差分を特徴量として学習する。さらに、新たなレビュー労力の計算式とレビュー対象コミットの選択基準を設けることで、少ないレビュー労力で高い欠陥発見率を実現する。

5つのオープンソースプロジェクトを対象とした実験により、提案手法の有効性が示された。全てのプロジェクトにおいて、構造的メトリクスのみのベースラインと比較してF1スコアが向上し、平均改善幅は0.21、最終的なF1スコアは0.70以上を達成した。ROC-AUCはすべてのプロジェクトで0.91を超え、高い識別性能を示した。レビュー労力の分析では、20\%のレビュー労力で、全ての欠陥のうち平均66.9\%を検出できることが示され、提案手法がリソース配分の効率化に有効であることが確認された。

今後の課題として、より多様なプログラミング言語やドメインへの適用が挙げられる。本研究はJavaプロジェクトを対象としたが、型システムやメモリ管理が異なる言語では、メトリクスの調整が必要となる可能性がある。また、本研究で対象としていない金融システムや医療システムでは、より厳格な品質要求や規制要件が存在する。これらのドメインでは、欠陥の特徴が異なる可能性があり、ドメイン固有の調整が必要となる。さらに、プロジェクトの進行に伴うコードベースの変化に対応するため、欠陥予測モデルを継続的に改善する仕組みが求められる。