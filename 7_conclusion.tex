本研究では、変更履歴を活用し、欠陥予測とレビューの優先度付けの効率化を目指した。現代のソフトウェア開発では頻繁なコード変更が不可欠である一方、変更に伴う欠陥混入リスクが品質保証上の課題となっている。従来の構造的メトリクスのみでは変更の特徴を十分に捉えられず、レビューに費やされる労力を効率的に配分する手法も改善が必要であった。本研究では、メソッド単位とコミット単位という異なる構成要素からなる変更メトリクスの活用、不規則な時系列変化を考慮した欠陥予測モデルの構築、レビュー労力の測定と欠陥発見率の向上という3つの研究目的を設定した。

これらの目的を達成するため、コミットを点過程データとして扱い、直前の状態からの差分を特徴量に変換する手法を提案した。5つのオープンソースプロジェクトを用いた実験により、提案手法が従来の構造的メトリクスのみを用いる手法と比較して予測精度を向上させることを示した。また、レビュー労力を考慮した優先度付け手法により、同一の労力でより多くの欠陥を発見可能であることを確認した。最終的に、欠陥予測モデルのクラス分類の詳細とレビューの優先度付け手法を関連付けることで、開発者の意思決定を支援するモデルを構築した。

本研究では、設定した3つの研究目的を達成することができた。メソッド単位とコミット単位の変更メトリクスを組み合わせることで、全プロジェクトで予測精度の向上を確認した。不規則な時系列変化を考慮した差分ベースのアプローチにより、全プロジェクトでF1スコアが向上した。レビュー労力を考慮した優先度付け手法により、レビュー労力を効果的に配分し、欠陥発見率を向上させられることを示した。第6章では、変更規模と欠陥混入確率の関係性、プロジェクト間の特徴量重要度の相違、予測性能の差異とその要因について考察した。

ただし、変更の目的を変更の特徴と関連付け、欠陥の原因を特定することは十分に達成できていない。変更の目的と欠陥の種類の関係を明らかにすることで、開発者に対して目的別の注意点を提示するガイドラインを作成し、欠陥の予防に繋げられる可能性がある。

今後の研究課題として、変更目的の推定による欠陥混入の原因の特定が挙げられる。6.6節で整理したように、変更目的の推定を効率的かつ迅速に行うためには、熟練した開発者が暗黙的に参照している情報を明らかにし、体系的に活用する必要がある。変更目的の推定に必要な要素としては、コミットに関する情報、コード変更の構造に関する情報、変更の文脈に関する情報、開発者に関する情報がある。これらの情報を収集し、プロジェクトごとに最も信頼性の高い情報源を優先的に活用することで、効率的かつ迅速な変更目的の推定が可能になると考えられる。