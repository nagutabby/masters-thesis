本研究では、ソフトウェア開発におけるレビューの効率化を目的として、コミット単位とメソッド単位の変更メトリクスを活用した欠陥予測手法を提案した。不規則なコード変更の特徴を捉えるためにコミットを点過程データとして扱い、直前の状態からの差分を学習することで、コードの一時的な構造分析だけでは発見しにくい、欠陥の予兆を検出する手法を確立した。また、メソッド単位の局所的な変化とコミット単位の全体的な変化という異なる構成要素の特徴を統合することで、変更の規模と複雑度の両面からリスクを識別できるようにした。さらに、変更の分散度を考慮した新たなレビュー労力指標を定義し、組み合わせ最適化アルゴリズムを適用することで、少ないレビュー労力でで高い欠陥発見率を達成する意思決定支援モデルを提示した。5つのオープンソースプロジェクトを用いた検証の結果、提案手法は従来の構造的メトリクスのみを用いる手法と比較して、予測精度が一貫して向上することが明らかになった。レビュー労力の分析においても、同一の労力でより多くの欠陥を特定できることが分かり、実際の品質保証活動における工数配分の改善に寄与する可能性が示された。今後の課題としては、多言語への展開や欠陥の重大度を考慮したモデルへの拡張、そして開発の進展に伴う動的なモデル更新の仕組みの構築が挙げられる。本研究で得られた知見が、ソフトウェアの保守性向上のための改善策の検討に寄与することを期待する。