本節では、提案手法の予測性能を評価した結果を報告する。評価は、ベースライン(BugHunter Datasetに前処理を適用したもの)、ステップ2(メソッド単位の変化量メトリクスを追加)、ステップ3(コミット単位の変更率メトリクスをさらに追加)の3段階で実施した。各段階でランダムフォレストを用いてモデルを構築し、10分割交差検証による性能評価とテストデータによる最終評価を行った。

\paragraph{テストデータによる評価}
表\ref{tab:evaluation}に、5つのプロジェクトにおける各段階のF1スコア、適合率、再現率、正解率、ROC-AUCを示す。ベースラインと比較して、ステップ2では全てのプロジェクトでF1スコアが向上し、ステップ3ではさらなる性能向上が確認された。

\begin{table}[ht]
\centering
\caption{テストデータによる評価}
\label{tab:evaluation}
\begin{tabular}{|l|l|r|r|r|r|r|}
\hline
プロジェクト & モデル & F1スコア & 適合率 & 再現率 & 正解率 & ROC-AUC \\
\hline
Elasticsearch & ベースライン & 0.5754 & 0.4576 & 0.7750 & 0.6340 & 0.7746 \\
 & ステップ2 & 0.7075 & 0.5983 & 0.8656 & 0.7710 & 0.8802 \\
 & ステップ3 & 0.7669 & 0.6770 & 0.8844 & 0.8280 & 0.9254 \\
\hline
Hazelcast & ベースライン & 0.6783 & 0.5824 & 0.8119 & 0.7420 & 0.8747 \\
 & ステップ2 & 0.7332 & 0.6549 & 0.8328 & 0.7970 & 0.8998 \\
 & ステップ3 & 0.7901 & 0.7014 & 0.9045 & 0.8390 & 0.9316 \\
\hline
Neo4j & ベースライン & 0.4782 & 0.3664 & 0.6883 & 0.6290 & 0.7130 \\
 & ステップ2 & 0.6158 & 0.4882 & 0.8340 & 0.7430 & 0.8296 \\
 & ステップ3 & 0.7420 & 0.6358 & 0.8907 & 0.8470 & 0.9382 \\
\hline
Netty & ベースライン & 0.4548 & 0.3279 & 0.7419 & 0.6140 & 0.7054 \\
 & ステップ2 & 0.6607 & 0.5394 & 0.8525 & 0.8100 & 0.8815 \\
 & ステップ3 & 0.7466 & 0.6433 & 0.8894 & 0.8690 & 0.9337 \\
\hline
OrientDB & ベースライン & 0.4833 & 0.3602 & 0.7342 & 0.6280 & 0.7376 \\
 & ステップ2 & 0.5241 & 0.3945 & 0.7806 & 0.6640 & 0.7647 \\
 & ステップ3 & 0.7012 & 0.5801 & 0.8861 & 0.8210 & 0.9141 \\
\hline
\end{tabular}
\end{table}

Elasticsearchでは、F1スコアがベースラインの0.5754からステップ2で0.7075、ステップ3で0.7669へと段階的に向上した。ベースラインと比較したステップ3の改善幅は0.1915ポイントである。適合率は0.4576から0.6770へ0.2194ポイント向上し、再現率は0.7750から0.8844へ0.1094ポイント改善した。正解率も0.6340から0.8280へ0.1940ポイント向上した。ROC-AUCは0.7746から0.9254へ0.1508ポイント向上し、モデルの識別能力が大きく改善されたことが示された。

Hazelcastでは、F1スコアがベースラインの0.6783からステップ2で0.7332、ステップ3で0.7901へと向上した。ベースラインと比較したステップ3の改善幅は0.1118ポイントである。適合率は0.5824から0.7014へ0.1190ポイント向上し、再現率は0.8119から0.9045へ0.0926ポイント改善した。正解率は0.7420から0.8390へ0.0970ポイント向上し、ROC-AUCは0.8747から0.9316へ0.0569ポイント改善された。

Neo4jでは、F1スコアがベースラインの0.4782からステップ2で0.6158、ステップ3で0.7420へと大幅に向上した。ベースラインと比較したステップ3の改善幅は0.2638ポイントであり、5つのプロジェクトの中で最も大きな改善幅である。適合率は0.3664から0.6358へ0.2694ポイント向上し、再現率は0.6883から0.8907へ0.2024ポイント改善した。正解率は0.6290から0.8470へ0.2180ポイント向上し、ROC-AUCは0.7130から0.9382へ0.2252ポイント大幅に改善された。

Nettyでは、F1スコアがベースラインの0.4548からステップ2で0.6607、ステップ3で0.7466へと向上した。ベースラインと比較したステップ3の改善幅は0.2918ポイントであり、5つのプロジェクトの中で最も大きな改善幅である。適合率は0.3279から0.6433へ0.3154ポイント向上し、再現率は0.7419から0.8894へ0.1475ポイント改善した。正解率は0.6140から0.8690へ0.2550ポイント向上し、ROC-AUCは0.7054から0.9337へ0.2283ポイント大幅に改善された。

OrientDBでは、F1スコアがベースラインの0.4833からステップ2で0.5241、ステップ3で0.7012へと向上した。ベースラインと比較したステップ3の改善幅は0.2179ポイントである。適合率は0.3602から0.5801へ0.2199ポイント向上し、再現率は0.7342から0.8861へ0.1519ポイント改善した。正解率は0.6280から0.8210へ0.1930ポイント向上し、ROC-AUCは0.7376から0.9141へ0.1765ポイント改善された。

\paragraph{統計的有意性の検証}
ベースラインとステップ3の予測結果に統計的に有意な差があるかを検証するため、マクネマー検定を実施した。マクネマー検定は、同じテストデータに対する2つの分類器の予測結果を比較するための統計的検定手法である。有意水準は0.01とし、p値が0.01未満の場合に有意な差があると判断する。

検定の結果、全てのプロジェクトにおいてp値は0.000000となり、有意水準0.01を大きく下回った。これは、ベースラインとステップ3の予測結果の差が偶然ではなく、統計的に有意であることを示している。すなわち、メソッド・コミット単位の変更メトリクスを追加することで、予測性能が統計的に有意に向上したことが実証された。

\paragraph{交差検証による性能変動の評価}
10分割交差検証の結果を表\ref{tab:crossvalidation}に示す。各プロジェクトにおいて、ベースライン、ステップ2、ステップ3の順にF1スコアとROC-AUCの平均値と標準偏差を報告する。

\begin{table}[ht]
\centering
\caption{10分割交差検証の結果(平均値 ± 標準偏差)}
\label{tab:crossvalidation}
\begin{tabular}{|l|l|r|r|}
\hline
プロジェクト & モデル & F1スコア & ROC-AUC \\
\hline
Elasticsearch & ベースライン & 0.6016 ± 0.0152 & 0.7988 ± 0.0179 \\
 & ステップ2 & 0.7165 ± 0.0242 & 0.8918 ± 0.0136 \\
 & ステップ3 & 0.7910 ± 0.0242 & 0.9313 ± 0.0161 \\
\hline
Hazelcast & ベースライン & 0.6934 ± 0.0234 & 0.8919 ± 0.0160 \\
 & ステップ2 & 0.7489 ± 0.0250 & 0.9215 ± 0.0145 \\
 & ステップ3 & 0.7833 ± 0.0185 & 0.9314 ± 0.0111 \\
\hline
Neo4j & ベースライン & 0.4472 ± 0.0244 & 0.6775 ± 0.0186 \\
 & ステップ2 & 0.5765 ± 0.0232 & 0.8101 ± 0.0136 \\
 & ステップ3 & 0.7393 ± 0.0329 & 0.9360 ± 0.0111 \\
\hline
Netty & ベースライン & 0.4257 ± 0.0280 & 0.6864 ± 0.0312 \\
 & ステップ2 & 0.6212 ± 0.0331 & 0.8707 ± 0.0290 \\
 & ステップ3 & 0.7076 ± 0.0275 & 0.9252 ± 0.0160 \\
\hline
OrientDB & ベースライン & 0.4734 ± 0.0392 & 0.7246 ± 0.0364 \\
 & ステップ2 & 0.5086 ± 0.0402 & 0.7535 ± 0.0326 \\
 & ステップ3 & 0.6636 ± 0.0251 & 0.8985 ± 0.0192 \\
\hline
\end{tabular}
\end{table}

Elasticsearchでは、ベースラインのF1スコアが0.6016±0.0152、ステップ2で0.7165±0.0242、ステップ3で0.7910±0.0242となった。ROC-AUCはベースラインの0.7988±0.0179からステップ2で0.8918±0.0136、ステップ3で0.9313±0.0161へと向上した。標準偏差は全ての段階で0.02前後と比較的小さく、性能が安定していることが確認された。

Hazelcastでは、F1スコアがベースラインの0.6934±0.0234からステップ2で0.7489±0.0250、ステップ3で0.7833±0.0185へと向上した。ROC-AUCはベースラインの0.8919±0.0160からステップ2で0.9215±0.0145、ステップ3で0.9314±0.0111へと改善された。特にステップ3では標準偏差が0.0111と最も小さく、性能の安定性が向上したことが示された。

Neo4jでは、F1スコアがベースラインの0.4472±0.0244からステップ2で0.5765±0.0232、ステップ3で0.7393±0.0329へと大幅に向上した。ROC-AUCはベースラインの0.6775±0.0186からステップ2で0.8101±0.0136、ステップ3で0.9360±0.0111へと改善された。ステップ3では標準偏差が0.0329とやや大きいが、これはプロジェクトの特性やデータの偏りに起因する可能性がある。

Nettyでは、F1スコアがベースラインの0.4257±0.0280からステップ2で0.6212±0.0331、ステップ3で0.7076±0.0275へと向上した。ROC-AUCはベースラインの0.6864±0.0312からステップ2で0.8707±0.0290、ステップ3で0.9252±0.0160へと改善された。ステップ3では標準偏差が0.0160と小さくなり、性能の安定性が向上した。

OrientDBでは、F1スコアがベースラインの0.4734±0.0392からステップ2で0.5086±0.0402、ステップ3で0.6636±0.0251へと向上した。ROC-AUCはベースラインの0.7246±0.0364からステップ2で0.7535±0.0326、ステップ3で0.8985±0.0192へと改善された。ステップ3では標準偏差が0.0251と小さくなり、性能の安定性が向上した。

全てのプロジェクトにおいて、交差検証の標準偏差は0.04以下と比較的小さく、提案手法が異なるデータ分割に対しても安定した性能を発揮することが確認された。特に、ステップ3ではROC-AUCの標準偏差が全プロジェクトで0.02以下となり、モデルの識別能力が安定していることが示された。

図\ref{fig:cv_performance_elasticsearch}から図\ref{fig:cv_performance_orientdb}に、各プロジェクトにおける10分割交差検証の評価指標の推移を示す。

\begin{figure}[ht]
\centering
\includegraphics[width=0.6\textwidth]{figures/elasticsearch/cv_performance_chart.png}
\caption{10分割交差検証における各ステップの性能比較(Elasticsearch)}
\label{fig:cv_performance_elasticsearch}
\end{figure}

\begin{figure}[ht]
\centering
\includegraphics[width=0.6\textwidth]{figures/hazelcast/cv_performance_chart.png}
\caption{10分割交差検証における各ステップの性能比較(Hazelcast)}
\label{fig:cv_performance_hazelcast}
\end{figure}

\begin{figure}[ht]
\centering
\includegraphics[width=0.6\textwidth]{figures/neo4j/cv_performance_chart.png}
\caption{10分割交差検証における各ステップの性能比較(Neo4j)}
\label{fig:cv_performance_neo4j}
\end{figure}

\begin{figure}[ht]
\centering
\includegraphics[width=0.6\textwidth]{figures/netty/cv_performance_chart.png}
\caption{10分割交差検証における各ステップの性能比較(Netty)}
\label{fig:cv_performance_netty}
\end{figure}

\begin{figure}[ht]
\centering
\includegraphics[width=0.6\textwidth]{figures/orientdb/cv_performance_chart.png}
\caption{10分割交差検証における各ステップの性能比較(OrientDB)}
\label{fig:cv_performance_orientdb}
\end{figure}