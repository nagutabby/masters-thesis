静的メトリクスは、ソースコードの構造的特徴を定量化する。代表的なメトリクスとして、複雑度メトリクス、規模メトリクス、結合度・凝集度メトリクスがある。

McCabe\cite{mccabe1976}が提案した循環的複雑度は、プログラムの制御フローの複雑さを測定する。これは制御フローグラフにおける独立したパスの数として定義され、$V(G) = E - N + 2P$で計算される。循環的複雑度が高いほど、欠陥混入リスクが高まる。

規模メトリクスは、コードの量的な大きさを測定する。最も基本的なものはLOC(Lines of Code)であり、多くの研究でコードサイズと欠陥数の正の相関が報告されている\cite{fenton1999}。

オブジェクト指向プログラムに対しては、ChidamberとKemerer\cite{chidamber1994}が提案したCKメトリクスが広く用いられる。CBO(Coupling Between Objects)はクラス間の結合度を示し、RFC(Response For Class)はクラスの複雑性を反映する。LCOM(Lack of Cohesion of Methods)はクラス内メソッド間の凝集性を測定する。

静的メトリクスは、ある時点でのコードのスナップショットのみを分析するため、変更の頻度やパターンといった動的な情報を捉えられない。変更の頻度やパターンは静的メトリクスよりも欠陥混入との関連性が高いことが知られており\cite{graves2000}、静的メトリクスだけでは予測精度に限界がある。