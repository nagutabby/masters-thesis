構造的メトリクスは、ソースコードの構造的な特徴を定量化したものである。これらは主に、複雑度、規模、およびコンポーネント間の関係性の3つの観点から定義される。第一に、論理構造の複雑さを測る指標として、McCabe \cite{mccabe1976} が提案した循環的複雑度が広く用いられる。これは制御フローグラフにおける独立したパスの数として定義され、以下の式で計算される。

$$V(G) = E - N + 2P$$

ここで、$E$はグラフにおけるエッジの数、$N$はグラフにおけるノードの数、$P$は連結成分の数(通常、単一のメソッドを対象とする場合は $P=1$)である。循環的複雑度が高いほど、コードの分岐やループが複雑であることを示し、欠陥混入リスクが高まる傾向にある。

第二に、規模を測る指標は、コードの量的な大きさを測定する。最も基本的なものはLOC(Lines of Code)であり、多くの研究でコードサイズと欠陥数の正の相関が報告されている \cite{fenton1999}。

第三に、コード内の要素間の関係性を測る指標として、結合度や凝集度に関するメトリクスがある。ChidamberとKemerer \cite{chidamber1994} が提案したメトリクス群がその代表例である。CBO(Coupling Between Objects)は要素間の結合度を、RFC(Response For Class)は要素が呼び出すメソッドの範囲を、LCOM(Lack of Cohesion of Methods)は内部要素間の凝集度をそれぞれ測定する。

これらの構造的メトリクスは、ある時点でのコードの状態を分析するのに適している。しかし、コードのスナップショットのみに依存するため、変更の頻度やパターンといった動的な情報を捉えられない。変更の特徴は構造の特徴よりも欠陥混入との関連性が高いことが指摘されており \cite{graves2000}、構造的メトリクスのみによる予測には限界がある。