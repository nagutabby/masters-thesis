コードの変更履歴を活用した欠陥予測の研究は、バージョン管理システムの普及とともに発展してきた。Gravesらは、大規模な電話交換システムを対象とした研究で、変更の頻度が欠陥の強い予測因子であることを示した\cite{graves2000}。

NagappanとBallは、変更の規模(追加・削除行数)や変更ファイル数に基づく欠陥予測手法を考え、システムの欠陥密度の推定における有効性を示した\cite{nagappan2005}。

Hassanは、エントロピーの概念を用いて変更の複雑さを測定した\cite{hassan2009}。高エントロピーの変更ほど欠陥を含みやすいことが実証された。

Kameiらは、Just-In-Time(JIT)欠陥予測モデルを提案した\cite{kamei2013}。JITモデルは、コミット時点で欠陥混入の有無を予測し、即座にレビューやテストの対象とする。彼らは6つのオープンソースプロジェクトと5つの商用プロジェクトを用いた実証研究により、平均正解率68\%、平均再現率64\%で欠陥誘発コミットを予測できることを示し、レビュー労力の20\%で全欠陥誘発コミットの35\%を特定できることを実証した。

しかし、既存の変更メトリクス研究では、一連の変更データから全体の変化量を算出しており、コミットのタイミングや間隔といった時系列的な構造を明示的には考慮していない。ソフトウェアの変更要求やその承認は不規則なタイミングで発生する活動であり、この特徴を考慮した手法は十分に探求されていない。