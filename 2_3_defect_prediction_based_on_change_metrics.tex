コードの変更履歴を活用した欠陥予測の研究は、バージョン管理システムの普及とともに発展してきた。Gravesらは、大規模な電話交換システムを対象とした研究で、変更の頻度が欠陥の強い予測因子であることを示した\cite{graves2000}。特に、最近変更された箇所ほど欠陥を含む可能性が高いという知見は、変更履歴の重要性を示す初期の証拠となった。

NagappanとBallは、変更の複雑さを定量化するため、変更の規模(追加・削除行数)や変更したファイル数などに基づくメトリクスを提案した\cite{nagappan2005}。彼らの研究では、変更の規模やその広がりを示すメトリクスがシステムの欠陥密度の推定において有効であることが示された。

Hassan\cite{hassan2009}は、エントロピーの概念を用いて変更の複雑さを測定した。情報理論におけるエントロピーは不確実性の度合いを表し、多数のファイルに散在する変更ほど高いエントロピーを持つ。Hassanの研究では、高エントロピーの変更ほど欠陥を含みやすいことが実証された。これは、変更の影響範囲が広いほど開発者の理解が困難になり、予期しない相互作用が生じやすいためと考えられる。

Kameiらは、Just-In-Time(JIT)欠陥予測モデルを提案した\cite{kamei2013}。JITモデルは、コミット時点で欠陥混入の有無を予測し、即座にレビューやテストの対象とすることで、早期の品質保証を可能にする。彼らは6つのオープンソースプロジェクトと5つの商用プロジェクトを用いた大規模な実証研究により、変更メトリクスの有効性を確認した。その結果、平均正解率68\%、平均再現率64\%で欠陥誘発コミットを予測できることを示し、レビュー労力の20\%で全欠陥誘発コミットの35\%を特定できることを実証した。

しかし、既存の変更メトリクス研究には課題が残されている。多くの研究では、変更を累積的な統計量(累積変更回数、累積追加行数など)として扱っており、コミットのタイミングや間隔といった時系列的な構造を明示的には考慮していない。開発活動は等間隔で行われるわけではなく、不規則なタイミングで発生するイベントである。このような不規則な時系列データの特性を適切に扱う手法は、欠陥予測研究において未だ十分に探求されていない。