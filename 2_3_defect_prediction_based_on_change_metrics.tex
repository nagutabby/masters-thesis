コードの変更履歴を活用した欠陥予測の研究は、バージョン管理システムの普及とともに発展してきた。Gravesらは、大規模な電話交換システムを対象とした研究で、変更の頻度が欠陥の強い予測因子であることを示した\cite{graves2000}。

NagappanとBallは、変更の規模(追加・削除行数)や変更ファイル数に基づくメトリクスを提案し、システムの欠陥密度の推定における有効性を示した\cite{nagappan2005}。

Hassanは、エントロピーの概念を用いて変更の複雑さを測定した\cite{hassan2009}。高エントロピーの変更ほど欠陥を含みやすいことが実証された。

Kameiらは、Just-In-Time(JIT)欠陥予測モデルを提案した\cite{kamei2013}。JITモデルは、コミット時点で欠陥混入の有無を予測し、即座にレビューやテストの対象とする。彼らは6つのオープンソースプロジェクトと5つの商用プロジェクトを用いた実証研究により、平均正解率68\%、平均再現率64\%で欠陥誘発コミットを予測できることを示し、レビュー労力の20\%で全欠陥誘発コミットの35\%を特定できることを実証した。

しかし、既存の変更メトリクス研究では、変更を累積的な統計量(累積変更回数、累積追加行数など)として扱っており、コミットのタイミングや間隔といった時系列的な構造を明示的には考慮していない。開発活動は不規則なタイミングで発生するイベントであり、この特性を適切に扱う手法は十分に探求されていない。