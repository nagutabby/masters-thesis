コードの変更履歴を活用した欠陥予測の研究は、バージョン管理システムの普及とともに発展してきた。これらの研究では、VCS上のコミットログから「以前の状態と今の状態」を比較した差分を抽出し、それを特徴量に変換する。これにより、ソースコードの構造だけでなく、開発者が「いつ、どの程度の頻度で、どのような規模の修正を試みたか」という開発プロセスに潜む欠陥混入リスクをモデル化することが可能になる。Gravesらは、大規模な電話交換システムを対象とした研究で、変更の頻度が欠陥の強い予測因子であることを示した\cite{graves2000}。

NagappanとBallは、変更の規模(コードの追加・削除行数)や変更ファイル数に基づく欠陥予測手法を考え、システムの欠陥密度の推定における有効性を示した\cite{nagappan2005}。

Hassanは、エントロピーの概念を用いて変更の複雑さを測定した\cite{hassan2009}。エントロピーが高い変更ほど欠陥を含みやすいことが示された。

Kameiらは、Just-In-Time(JIT)欠陥予測モデルを提案した\cite{kamei2013}。JITモデルは、コミット時に欠陥混入の有無を予測し、即座にレビューやテストの対象とする。彼らは6つのオープンソースプロジェクトと5つの商用プロジェクトを用いた実証研究により、平均正解率68\%、平均再現率64\%で欠陥混入コミットを予測可能であることを示し、20\%のレビュー労力で欠陥混入コミットの35\%を特定可能であることを示した。

しかし、既存の変更メトリクス研究では、一連の変更データから全体の変化量を算出しており、コミットのタイミングや間隔といった時系列構造を明示的に考慮していない。ソフトウェアの変更要求やその承認は不規則なタイミングで発生する活動であり、この特徴を考慮した手法は十分に探求されていない。