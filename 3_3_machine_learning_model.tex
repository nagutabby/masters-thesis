提案する時系列変化を考慮したメトリクスを用いて欠陥予測モデルを構築するため、機械学習アルゴリズムとしてランダムフォレストを採用する。

ランダムフォレストは、複数の決定木を構築し、それらの予測結果を統合するアンサンブル学習手法である。本研究でランダムフォレストを選定した理由は以下の通りである。

第一に、非線形な関係を捉える能力が高い。ソフトウェア欠陥予測において、特徴量とバグの有無の関係は必ずしも線形ではない。例えば、コード行数の変化量が小さい場合と極端に大きい場合の両方でバグ混入リスクが高まる可能性がある。また、複数の特徴量が複雑に相互作用してバグ混入リスクに影響を与えることもある。ランダムフォレストは、このような非線形な関係や特徴量間の相互作用を効果的にモデル化できる。

第二に、アンサンブル学習による高い予測精度である。ランダムフォレストは、データのランダムサンプリングと特徴量のランダム選択により、多様な決定木を構築する。各決定木は異なる側面からデータを学習するため、それらの予測結果を統合することで、単一の決定木よりも頑健で精度の高い予測が可能になる。特に、本研究で扱うような高次元の特徴量を持つデータセットにおいて、過学習を抑制しながら高い汎化性能を達成できる。

第三に、特徴量の寄与度を評価できる。ランダムフォレストは、各特徴量が予測にどの程度寄与しているかを定量的に評価できる。これにより、メソッド単位の変化量メトリクスとコミット単位の変更メトリクスのうち、どの特徴量がバグ予測により貢献するのかを明らかにできる。この情報は、モデルの解釈性を高めるだけでなく、実務での意思決定を支援する知見を提供する。

モデルの性能を評価するため、本研究では主にF1スコアを用いる。F1スコアは、適合率(Precision)と再現率(Recall)の調和平均であり、以下の式で計算される。

\[
\text{F1スコア} = 2 \times \frac{\text{適合率} \times \text{再現率}}{\text{適合率} + \text{再現率}}
\]

ソフトウェア欠陥予測においてF1スコアを採用する理由は、バグありクラスとバグなしクラスの不均衡が存在するためである。一般的に、バグを含むメソッドの数は、バグを含まないメソッドの数よりも少ない。このような不均衡データでは、正確率(Accuracy)のみでは適切な評価ができない。例えば、全てのインスタンスをバグなしと予測しても、高い正確率が得られる可能性がある。F1スコアは、適合率(予測したバグありのうち、実際にバグありの割合)と再現率(実際のバグありのうち、バグありと予測できた割合)の両方を考慮するため、不均衡データにおいても適切な評価が可能である。Zhaoら[6]の調査によれば、欠陥予測研究においてF1スコアによる評価が効果的であることが示されている。

提案手法がベースライン手法と比較して統計的に有意な改善をもたらしているかを検証するため、マクネマー検定を実施する。マクネマー検定は、同じデータセットに対する2つの分類器の予測結果を比較するための統計的検定手法である。この検定により、提案手法による予測性能の改善が偶然ではなく、統計的に有意であることを確認できる。

本研究では、10分割交差検証を用いてモデルの性能を評価する。データセットを10個のサブセットに分割し、そのうち9個を訓練データ、1個をテストデータとして使用する。この過程を10回繰り返し、各サブセットが1度だけテストデータとして使用されるようにする。これにより、データセット全体での予測性能を頑健に評価できる。