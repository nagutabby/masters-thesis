提案する時系列変化を考慮したメトリクスを用いて欠陥予測モデルを構築するため、機械学習アルゴリズムとしてランダムフォレストを採用する。

ランダムフォレストを選定した理由は以下の通りである。第一に、非線形な関係を捉える能力が高い。ソフトウェア欠陥予測において、特徴量と欠陥の有無の関係は必ずしも線形ではない。第二に、アンサンブル学習による高い予測精度である。第三に、特徴量重要度を評価できるため、モデルの解釈性を高めることができる。

モデルの性能を評価するため、本研究では主にF1スコアを用いる。F1スコアは、適合率(Precision)と再現率(Recall)の調和平均であり、不均衡データにおいても適切な評価が可能である。

提案手法がベースライン手法と比較して統計的に有意な改善をもたらしているかを検証するため、マクネマー検定を実施する。また、10分割交差検証を用いてモデルの性能を評価する。具体的な実装と評価手順については第4章で述べる。

ソフトウェア欠陥予測においてF1スコアを採用する理由は、欠陥を含むデータと含まないデータの不均衡が存在するためである。一般的に、欠陥を含むメソッドの数は、欠陥を含まないメソッドの数よりも少ない。このような不均衡データでは、正確率(Accuracy)のみでは適切な評価ができない。例えば、全てのインスタンスを欠陥なしと予測しても、高い正確率が得られる可能性がある。F1スコアは、適合率(欠陥を含むと予測したデータのうち、実際に欠陥を含むデータの割合)と再現率(実際に欠陥を含むデータのうち、欠陥を含むと予測できた割合)の両方を考慮するため、不均衡データにおいても適切な評価が可能である。

提案手法がベースライン手法と比較して統計的に有意な改善をもたらしているかを検証するため、マクネマー検定を実施する。マクネマー検定は、同じデータセットに対する2つの分類器の予測結果を比較するための統計的検定手法である。この検定により、提案手法による予測性能の改善が偶然ではなく、統計的に有意であることを確認できる。

本研究では、10分割交差検証を用いてモデルの性能を評価する。データセットを10個のサブセットに分割し、そのうち9個を訓練データ、1個をテストデータとして使用する。この過程を10回繰り返し、各サブセットが1度だけテストデータとして使用されるようにする。これにより、データセット全体での予測性能を頑健に評価できる。