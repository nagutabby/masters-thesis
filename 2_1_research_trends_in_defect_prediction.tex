ソフトウェア開発において、保守フェーズはライフサイクル全体のコストの大部分を占める。JIS X 0161では、保守作業を是正保守、適応保守、完全化保守、予防保守の4つに分類しており\cite{jisx0161}、特に不具合を修正する是正保守の効率化がコスト削減の鍵となる。Microsoft Researchの調査では、76\%の開発者がリファクタリングによる欠陥混入を懸念していると報告されており\cite{kim2014}、変更に伴うリスク管理が保守活動における重要な課題である。Ostrandらは、欠陥密度が最も高い構成要素の20\%に平均83\%の欠陥が集中することを示し\cite{ostrand2005}、限られたリソースを効果的に配分する戦略の重要性を示唆した。このような保守コストの削減とリスク管理の観点から、欠陥を早期に予測し、重点的にレビューすべき箇所を特定する技術が求められている。

ソフトウェア欠陥予測の初期研究では、主に構造的メトリクスが活用されてきた。1970年代のMcCabeによる循環的複雑度の提案\cite{mccabe1976}以降、コードの複雑さと欠陥の関係が注目され、様々なメトリクスが開発されてきた。
1990年代には、オブジェクト指向設計の普及に伴い、CKメトリクス\cite{chidamber1994}をはじめとする結合度・凝集度に関する研究が進展した。2000年代以降、バージョン管理システムの普及により、コードの変更履歴を活用した研究が盛んになった。Hassan\cite{hassan2009}は、変更ファイル数に基づく複雑さが欠陥予測精度の向上に貢献することを示した。近年では、機械学習技術の発展により、複数のメトリクスを組み合わせた高精度な予測モデルの構築が可能となっている。
多くの研究では欠陥混入コミットと欠陥修正コミットを特定し、それらのコミット時点でのメトリクスを記録している。しかし、欠陥混入から修正までの期間に発生した変更を考慮した研究は限定的である。また、コミットを不規則な時系列データとして明示的に扱う研究や、レビュー労力を考慮した欠陥予測に関する研究も十分ではない。