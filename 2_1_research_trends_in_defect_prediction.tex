ソフトウェア開発において、保守フェーズはライフサイクル全体のコストの大部分を占める。JIS X 0161では、保守作業を是正保守、適応保守、完全化保守、予防保守の4つに分類しており\cite{jisx0161}、特に不具合を修正する是正保守の効率化がコスト削減の鍵となる。Microsoft Researchの調査では、76\%の開発者がリファクタリングによる欠陥混入を懸念していると報告されており\cite{kim2014}、変更に伴うリスク管理が保守活動における重要な課題である。Ostrandらは、欠陥密度が最も高い構成要素の20\%に平均83\%の欠陥が集中することを示し\cite{ostrand2005}、限られたリソースを効果的に配分する戦略の重要性を示唆した。

ソフトウェア欠陥予測の初期研究では、主に構造的メトリクスが活用されてきた。1970年代のMcCabeによる循環的複雑度の提案\cite{mccabe1976}以降、コードの論理構造と欠陥の関係が注目され、様々な指標が開発された。

2000年代以降、バージョン管理システム(VCS)の普及により、コードの変更履歴を活用した研究が盛んになった。元来、ソフトウェアの変更管理はソースコードの手動バックアップや差分ファイルの受け渡しに依存していた。しかし、プロジェクトの規模が拡大し、複数の開発者が同時に作業を行う環境では、変更の競合や、不具合発生時の過去の状態への切り戻しが困難になるという課題が顕在化した。これらの問題を解決し、「いつ」、「誰が」、「どこを」「どのように」変更したのかを記録するためにVCSが登場した。 このVCSの普及により、研究者は膨大な過去の変更データ(コミット履歴)に容易にアクセス可能となった。Hassan\cite{hassan2009}らは、変更ファイル数に基づく複雑さが欠陥予測精度の向上に貢献することを示した。近年では、機械学習技術の発展により、構造的メトリクスとVCSから得られる変更履歴を組み合わせた高精度な予測モデルの構築も試みられている。