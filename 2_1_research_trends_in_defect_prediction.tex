ソフトウェア欠陥予測は、ソフトウェア工学における長年の研究課題である。初期の研究では、コードの構造的特性に着目した静的メトリクスが主流であった。1970年代のMcCabeによる循環的複雑度の提案\cite{mccabe1976}以降、コードの複雑さと欠陥の関係が注目され、様々なメトリクスが開発されてきた。

1990年代には、オブジェクト指向設計の普及に伴い、CKメトリクス\cite{chidamber1994}をはじめとする結合度・凝集度に関する研究が進展した。これらの静的メトリクスは、ある時点でのコードの構造的特徴を捉えることで、欠陥の存在を予測しようとする。

2000年代以降、バージョン管理システムの普及により、コードの変更履歴を活用した研究が盛んになった。Hassan\cite{hassan2009}は、変更ファイル数に基づく複雑さが欠陥予測精度の向上に貢献することを示し、静的な特徴だけでなく動的な変化の過程に着目する重要性を明らかにした。近年では、機械学習技術の発展により、複数のメトリクスを組み合わせた高精度な予測モデルの構築が可能となっている。

多くの研究では欠陥混入コミットと欠陥修正コミットを特定し、それらのコミット時点での静的メトリクスや変更メトリクスを記録している。しかし、欠陥が混入されてから修正されるまでの期間に発生した変更内容を考慮した研究は限定的である。欠陥が存在する期間中の変更は、欠陥の顕在化や影響範囲の拡大に関与する可能性があるにもかかわらず、この時系列的な変化の過程は十分に分析されていない。また、コミットを不規則な時系列データとして明示的に扱う研究や、レビュー労力を考慮した最適化に関する研究も十分ではない。