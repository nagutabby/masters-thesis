ソフトウェア欠陥予測の初期研究では、コードの構造的特性に着目した静的メトリクスが主流であった。1970年代のMcCabeによる循環的複雑度の提案\cite{mccabe1976}以降、コードの複雑さと欠陥の関係が注目され、様々なメトリクスが開発されてきた。

1990年代には、オブジェクト指向設計の普及に伴い、CKメトリクス\cite{chidamber1994}をはじめとする結合度・凝集度に関する研究が進展した。

2000年代以降、バージョン管理システムの普及により、コードの変更履歴を活用した研究が盛んになった。Hassan\cite{hassan2009}は、変更ファイル数に基づく複雑さが欠陥予測精度の向上に貢献することを示した。近年では、機械学習技術の発展により、複数のメトリクスを組み合わせた高精度な予測モデルの構築が可能となっている。

多くの研究では欠陥混入コミットと欠陥修正コミットを特定し、それらのコミット時点でのメトリクスを記録している。しかし、欠陥混入から修正までの期間に発生した変更を考慮した研究は限定的である。また、コミットを不規則な時系列データとして明示的に扱う研究や、レビュー労力を考慮した欠陥予測に関する研究も十分ではない。