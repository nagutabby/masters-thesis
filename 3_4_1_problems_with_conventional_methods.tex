従来のレビュー優先度付け手法では、欠陥予測モデルが出力したバグ混入確率に基づいてコミットを順位付けし、上位からレビューを行うアプローチが一般的であった。Kameiらの研究では、レビュー労力を「変更された行の総数」として計算し、総労力の一定割合(例えば20\%)を使用してレビューできるコミット数を評価している。

しかし、この手法には重要な単純化が含まれている。具体的には、レビュー労力を変更行数のみで計算しているため、変更の複雑さや影響範囲の広さが考慮されていない。実際のレビュー労力は、変更されたファイル数や変更の分散度といった要因にも依存する。例えば、10個のファイルに分散した100行の変更は、1個のファイルに集中した100行の変更よりもレビュー労力が大きいと考えられる。

さらに重要な問題として、従来手法では「レビューに必要な総労力」を全コミットのレビュー労力の和として設定している場合が多い。しかし、極端に大きなレビュー労力を要するコミット(例えば、数千行の変更を含むコミット)が存在する場合、この設定は現実的ではない。実際の開発現場では、レビューに使える労力には上限があり、その上限を超える巨大なコミットは分割されるか、別の品質保証プロセスが適用される。