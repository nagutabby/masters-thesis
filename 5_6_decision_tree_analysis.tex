決定木の構造を可視化した結果、提案手法(ステップ3)では、ルートノード付近で\texttt{lines\_added}や\texttt{tokens\_change}などの時系列変化メトリクスが頻繁に使用されていることが分かった。これにより、静的メトリクスのみを用いたベースラインと比較して、より高い確信度で欠陥の有無を分類できていることが確認された。