ランダムフォレストを構成する決定木を可視化し、モデルがどのように欠陥の有無を予測しているかを分析した。決定木の可視化により、特徴量重要度だけでは把握できない具体的な分類条件を抽出可能である。

提案手法(ステップ3)の決定木では、ルートノード付近で\texttt{lines\_added}(追加行数)や\texttt{tokens\_change}(トークン数の変化量)といった変更メトリクスが頻繁に使用されていた。これらは、コミットの規模やメソッドの変化量という時系列的な情報を表す特徴量である。決定木の浅い階層でこれらの特徴量が選択されることは、コードの構造的な変化が欠陥予測において重要な特徴となっていることを示している。