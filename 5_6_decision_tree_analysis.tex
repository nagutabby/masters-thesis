決定木による分類条件の可視化により、提案手法がバグ混入リスクをどのように判断しているかを明らかにする。ここでは、ベースラインとステップ3(提案手法)における決定木の構造を比較し、時系列変化メトリクスの導入が分類ルールにどのような影響を与えたかを分析する。決定木の最大深さは3に制限し、解釈可能性を重視した。

\paragraph{陽性クラスに分類するノード数の変化}
表\ref{tab:leaf_nodes}に、ベースラインとステップ3における陽性クラスに分類するリーフノード数と、高確信度(陽性割合0.8以上)で分類するノード数を示す。

\begin{table}[ht]
\centering
\caption{陽性クラスに分類するリーフノード数の比較}
\label{tab:leaf_nodes}
\begin{tabular}{|l|r|r|r|r|}
\hline
プロジェクト & ベースライン & ステップ3 & ベースライン & ステップ3 \\
 &  &  & (確信度≥0.8) & (確信度≥0.8) \\
\hline
Elasticsearch & 4 & 2 & 0 & 0 \\
Hazelcast & 3 & 4 & 1 (0.847) & 1 (0.862) \\
Neo4j & 4 & 3 & 1 (0.808) & 2 (0.952, 0.907) \\
Netty & 3 & 3 & 2 (0.852, 0.850) & 1 (0.849) \\
OrientDB & 2 & 5 & 0 & 1 (0.911) \\
\hline
\end{tabular}
\end{table}

ベースラインでは、5つのプロジェクト全体で陽性に分類するリーフノードは平均3.2個であったのに対し、ステップ3では平均3.4個となった。より重要な変化は、高確信度で陽性に分類するノードの増加である。ベースラインでは、確信度0.8以上で陽性に分類するノードは5プロジェクト中4個(Hazelcastで1個、Neo4jで1個、Nettyで2個)であったが、ステップ3では5個に増加した。特にNeo4jでは、陽性割合0.952と0.907という非常に高い確信度のノードが2個出現し、OrientDBでは最高0.911の確信度を達成した。Hazelcastでも確信度が0.847から0.862に向上している。これは、時系列変化メトリクスの導入により、バグ混入リスクの高い変更をより明確に識別できるようになったことを示している。

\paragraph{陰性クラス分類における確信度の向上}
陰性クラスの分類においても、確信度の向上が観察された。ベースラインでは、確信度1.000(陰性割合1.000)のノードは、Elasticsearch、Hazelcast、Netty、OrientDBの4プロジェクトで合計6個存在した。ステップ3では、この数は1個に減少したが、これは分類ルールがより複雑になったためである。一方で、確信度0.9以上のノードを見ると、ベースラインでは2個(Nettyの0.965、Neo4jの0.917)であったのに対し、ステップ3では8個に増加した。特に、Elasticsearchでは0.973と1.000、Hazelcastでは0.965、Neo4jでは0.952、Nettyでは0.975と0.963と0.912、OrientDBでは0.937という高確信度のノードが出現した。

この変化は、時系列変化メトリクスにより、バグが混入していない変更の特徴をより正確に捉えられるようになったことを示唆している。ベースラインでは静的な特徴量のみを用いていたため、完全な確信度(1.000)でしか陰性に分類できない場合があったが、ステップ3では変更メトリクスという追加情報により、0.9以上の高い確信度で多様な陰性パターンを識別できるようになった。

\paragraph{分類条件と判断基準の可視化}
決定木の分類ルールを分析すると、ベースラインとステップ3で使用される特徴量が大きく異なることが明らかになった。表\ref{tab:decision_tree_features}に、各プロジェクトで使用された主要な特徴量を示す。

\begin{table}[ht]
\centering
\caption{決定木で使用された主要な特徴量}
\label{tab:decision_tree_features}
\begin{tabular}{|l|p{5cm}|p{6cm}|}
\hline
プロジェクト & ベースライン & ステップ3 \\
\hline
Elasticsearch & MI, TF-IDF, NII & tokens\_change, lines\_added, operation\_type\_added, HNDB, lines\_deleted \\
\hline
Hazelcast & NOI, TF-IDF, HTRP, NII, MISM & NOI, NII, tokens\_change, HNDB, lines\_added, num\_files \\
\hline
Neo4j & NII, TF-IDF, HCPL, MIMS, HDIF, NOI, HVOL & lines\_deleted, lines\_added, operation\_type\_NaN, TF-IDF, tokens\_change, entropy \\
\hline
Netty & TF-IDF, NII, TLOC & tokens\_change, operation\_type\_NaN, lines\_deleted, TF-IDF, lines\_added, HNDB \\
\hline
OrientDB & Brace Rules, TF-IDF, TLLOC, WarningMinor, HVOL, MI & NII, MISEI, lines\_deleted, lines\_added, HVOL, HNDB \\
\hline
\end{tabular}
\end{table}

ベースラインでは、TF-IDFやMI、NIIといった静的コードメトリクスが主に使用されていた。一方、ステップ3では、全てのプロジェクトでtokens\_change、lines\_added、lines\_deletedといったコミット間の変化量メトリクスが上位の分割条件として採用された。特に、tokens\_changeとoperation\_type\_addedというメソッド単位の変化量メトリクスが、多くのプロジェクトで重要な分割条件となっている。

具体的な分類ルールを見ると、ステップ3では時系列変化を中心とした直感的な条件が多く出現した。例えば、Elasticsearchでは「lines\_added <= 103.5 AND operation\_type\_added <= 0.5 AND tokens\_change <= 0.5」という条件で陽性割合0.773となり、小規模な変更で既存メソッドのみを修正した場合にバグ混入リスクが高いことを示している。逆に、「lines\_added <= 103.5 AND operation\_type\_added > 0.5」という新規メソッド追加を含む条件では、陰性割合1.000となり、新規メソッドの追加がバグ混入リスクを低下させることが明確に示された。

Neo4jでは、「lines\_added <= 91.5 AND lines\_added <= 24.5」という小規模変更が陽性割合0.907、「lines\_added <= 91.5 AND lines\_added > 24.5 AND entropy > 0.954」という中規模で分散度の高い変更が陽性割合0.952となった。一方、「lines\_added > 91.5」という大規模変更では、tokens\_changeやlines\_deletedといった追加条件により陰性に分類されており、大規模変更が計画的に実施される傾向を反映している。

Nettyでは、tokens\_changeが最も重要な分割条件となり、「tokens\_change <= 0.5 AND lines\_deleted <= 54.5 AND tokens\_change > -0.5」という条件で陽性割合0.849を達成した。これは、メソッドのトークン数がわずかに変化した場合にバグ混入リスクが高いことを示している。逆に、「tokens\_change <= 0.5 AND lines\_deleted <= 54.5 AND tokens\_change <= -0.5」ではトークン数が減少(リファクタリングによる簡素化)した場合に陰性割合0.975となり、コードの整理がバグ混入リスクを低下させることが示された。

OrientDBでは、ステップ3で陽性リーフノードが2個から5個に増加し、より多様なバグ混入パターンを識別できるようになった。「lines\_added <= 104.5 AND HNDB > 30.887 AND lines\_deleted <= 14.5」という条件では陽性割合0.774となり、小規模な変更で複雑度が高い場合のリスクを捉えている。また、「lines\_added > 104.5 AND HVOL > 256.864 AND lines\_deleted > 1840.5」という大規模な変更でも陽性割合0.911となる条件が見つかり、大規模かつ複雑な変更にはリスクが伴うことが示された。

図\ref{fig:decision_tree_vis}に、代表例としてElasticsearchプロジェクトにおけるステップ3の決定木を示す。

\begin{figure}[ht]
\centering
\includegraphics[width=0.9\textwidth]{figures/elasticsearch/decision_tree_visualization.png}
\caption{決定木による分類ルールの可視化(Elasticsearch、ステップ3)}
\label{fig:decision_tree_vis}
\end{figure}