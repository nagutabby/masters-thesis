提案手法のレビュー労力削減効果を評価するため、Cost-Benefit Curveを用いる。Cost-Benefit Curveは、横軸に投入したレビュー労力、縦軸に発見したバグ数をプロットしたグラフである。

提案手法(労力考慮型モデル)では、密度の高い順にコミットを選択してレビューする。各コミットをレビューするごとに、累積レビュー労力と累積発見バグ数を記録し、曲線を描く。

比較対象として、ベースラインモデル(新たな特徴量を追加する前のデータセットで学習したモデル)を用いる。ベースラインモデルでも同様に密度に基づく貪欲法を適用し、レビュー対象を選択する。すなわち、ベースラインモデルが予測したバグ混入確率 $\hat{y}_i^{\text{baseline}}$ を用いて密度を計算し、密度の高い順にコミットを選択してレビューする。

この比較により、労力を考慮したコードレビューモデルという統一的な評価枠組みの中で、特徴量エンジニアリング(メソッド単位とコミット単位の変更メトリクスの追加)の効果を測定できる。提案手法がベースライン手法よりも左上に位置する曲線を描く場合、同じレビュー労力でより多くのバグを発見できることを意味し、提案した特徴量の有効性が実証される。