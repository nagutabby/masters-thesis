静的解析は、ソフトウェアを実行することなくソースコードを検査し、潜在的な問題を検出する手法である。これは、プログラムを実際に動作させて挙動を観察する動的解析と対比される。静的解析ツールは、コードの構造、複雑度、スタイル、潜在的な欠陥パターンなどを自動的に分析し、JIS X 25010:2013で定義される保守性(特に解析性と修正性)の向上に貢献する。近年では、SonarQube\cite{sonarqube}やCheckstyle\cite{checkstyle}などの静的解析ツールが開発環境に統合され、コーディング中にリアルタイムでコードの問題を検出できるようになっている。

Romanoら\cite{romano2022}は、テスト駆動開発(TDD)において静的解析ツールを使用することがソフトウェア品質に与える影響を実証的に調査した。TDDは、レッドフェーズ(テストを書いて失敗させる)、グリーンフェーズ(テストに合格する最小限のコードを書く)、リファクタリングフェーズ(コードを改善する)という3つのフェーズを繰り返す開発手法である。理論的には、リファクタリングフェーズで\textbf{技術的負債}(将来の保守コストを増加させる設計上の妥協や実装上の問題)を返済し、品質を継続的に向上させることが期待される。

しかし、実際にはこのフェーズがしばしばスキップされることが観察されていた。リファクタリングがスキップされる主な理由は、以下の3点である。第一に、開発スケジュールの時間的プレッシャーにより、機能実装を優先し、コード品質の改善が後回しにされる。第二に、リファクタリングは即座にユーザーに見える価値を提供しないため、優先度が低く見なされる。第三に、技術的負債は可視化されにくく、どの程度の問題があるのかを開発者が認識しづらい。Romanoらは、静的解析ツールを用いることでこれらの問題に対処できるのではないかという仮説を立てた。

Romanoらは、静的解析ツールであるSonarLint(SonarQube for IDE)を使用する群と使用しない群に分けて実験を実施した。ソフトウェア品質を定量化するため、将来の不具合を招く恐れのある不適切な構造の数(Smell)、技術的負債の推定値(SqaleIndex)、循環的複雑度(WMC)、コードの分かりやすさ(CognCompl)、読みやすさ(BW)などの指標を測定した。

実験の結果、SonarLintを使用した群は、使用しない群と比較して、Smell、SqaleIndex、CognComplにおいて統計的に有意な改善が見られた。すなわち、静的解析ツールの使用は、不適切な構造の削減やコードの分かりやすさの向上に寄与することが示された。

静的解析ツールがこのような改善をもたらす理由は、以下の3点にある。第一に、問題の可視化により、開発者が技術的負債の存在に気づくことができる。漠然とした「コードが良くない」という感覚ではなく、具体的な指標として提示されることで、改善の動機が生まれる。第二に、コーディング中の即座のフィードバックにより、問題が小さいうちに修正できる。後から大規模なリファクタリングを行うよりも、その場で修正する方がコストが低い。第三に、ツールが提供する客観的基準により、コードレビューでの議論が建設的になり、チーム内で品質基準が共有される。これらの改善は、JIS X 25010:2013で定義される保守性の副特性、特に解析性(コードの問題点を識別する容易さ)と修正性(欠陥を取り込まずに変更を行う容易さ)の向上に貢献する。

一方で、実験後のアンケートでは、参加者はSonarLintを使用するとTDDがより困難になると認識しており、ツールの使用が追加の認知的負荷をもたらす可能性が示唆された。TDDが困難になる理由として、以下の要因が考えられる。第一に、静的解析ツールの警告に対応するための追加作業が発生し、本来のTDDサイクルが中断される。第二に、グリーンフェーズでは「テストに合格する最小限のコード」を書くことが求められるが、静的解析ツールが即座に品質問題を指摘するため、最小限のコードと品質の高いコードの間で葛藤が生じる。第三に、ツールの誤検出(偽陽性)に対処する必要があり、それが開発者の集中力を削ぐ。

この研究から明らかになる静的解析の課題は、文脈依存のしきい値の未検証である。SonarLintのようなツールは、事前に定義されたルールに基づいて不適切な構造を検出するが、プロジェクトの特性や開発フェーズに応じて適切なしきい値は異なる可能性がある。Romanoらの研究では、このような文脈依存性については検証されておらず、全てのプロジェクトに対して同一の設定が適用されている。

これが問題である理由は、以下の3点にある。第一に、偽陽性(実際には問題でないのに警告される)により、開発者が警告を無視するようになり、ツールへの信頼性が低下する。第二に、プロジェクトの初期段階では厳密な品質基準を適用することが非現実的な場合があり、過度な警告が開発速度を低下させる。第三に、本当に重要な問題が大量の警告に埋もれてしまい、見逃される可能性がある。そのため、静的解析ツールが検出する問題の中には、実際にはプロジェクトの文脈では問題とならない指摘が含まれている可能性がある。