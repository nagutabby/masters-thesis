静的解析は、ソフトウェアを実行することなくソースコードを検査し、潜在的な問題を検出する手法である。これは、プログラムを実際に動作させて挙動を観察する動的解析と対比される。静的解析ツールは、コードの構造、複雑度、スタイル、潜在的な欠陥パターンなどを自動的に分析し、JIS X 25010:2013で定義される保守性(特に解析性と修正性)の向上に貢献する。近年では、SonarQube\cite{sonarqube}やCheckstyle\cite{checkstyle}などの静的解析ツールが開発環境に統合され、コーディング中にリアルタイムでコードの問題を検出できるようになっている。

Romanoら\cite{romano2022}は、テスト駆動開発(TDD)において静的解析ツールを使用することがソフトウェア品質に与える影響を実証的に調査した。TDDは、レッドフェーズ(テストを書いて失敗させる)、グリーンフェーズ(テストに合格する最小限のコードを書く)、リファクタリングフェーズ(コードを改善する)という3つのフェーズを繰り返す開発手法である。理論的には、リファクタリングフェーズで\textbf{技術的負債}(将来の保守コストを増加させる設計上の妥協や実装上の問題)を返済し、品質を継続的に向上させることが期待されるが、実際にはこのフェーズがしばしばスキップされることが観察されていた。

Romanoらは、静的解析ツールであるSonarLint(SonarQube for IDE)を使用する群と使用しない群に分けて実験を実施した。ソフトウェア品質を定量化するため、将来の不具合を招く恐れのある不適切な構造の数(Smell)、技術的負債の推定値(SqaleIndex)、循環的複雑度(WMC)、コードの分かりやすさ(CognCompl)、読みやすさ(BW)などの指標を測定した。

実験の結果、SonarLintを使用した群は、使用しない群と比較して、Smell、SqaleIndex、CognComplにおいて統計的に有意な改善が見られた。すなわち、静的解析ツールの使用は、不適切な構造の削減やコードの分かりやすさの向上に寄与することが示された。これらの改善は、JIS X 25010:2013で定義される保守性の副特性、特に解析性(コードの問題点を識別する容易さ)と修正性(欠陥を取り込まずに変更を行う容易さ)の向上に直接貢献する。一方で、実験後のアンケートでは、参加者はSonarLintを使用するとTDDがより困難になると認識しており、ツールの使用が追加の認知的負荷をもたらす可能性が示唆された。

この研究から明らかになる静的解析の課題は、文脈依存のしきい値の未検証である。SonarLintのようなツールは、事前に定義されたルールに基づいて不適切な構造を検出するが、プロジェクトの特性や開発フェーズに応じて適切なしきい値は異なる可能性がある。Romanoらの研究では、このような文脈依存性については検証されておらず、全てのプロジェクトに対して同一の設定が適用されている。そのため、静的解析ツールが検出する問題の中には、実際にはプロジェクトの文脈では問題とならない偽陽性が含まれている可能性がある。