%%概要

システムを実装する目的は各ステークホルダーがシステムに対して操作を行う際に
必要な情報が全て与えられているかを確認すること、
操作に対して必要な情報が正しく返されるかを確認することである。
このため入出力は全て文字情報(CUI)とする。
また認証機能なども省略する。

システムは以下の部分から構成される。

\begin{enumerate}

\item 学習管理部

ユーザが機能を選択した時に、必要な情報の入力と出力を対話的に実現する。
選択可能な項目は文字として表示され、
ユーザは番号を入力することで対応する項目を選択する。
ナビゲーションは「進む」「戻る」などを番号で選択する。
試験の際の選択肢の出力と解答の入力も同じ仕組みを用いる。

\item 情報保持部

最初に教材関連情報(課程、教材、試験など)を入力して登録する。
選択肢が発生する場合は必要に応じて
学習管理部のメニュー選択機能を利用する。

\item 情報分析部

学習が可能(まだ完了試験に合格していない)な教材を示し、
その中から学習する教材を選択する仕組みを持つ。


\end{enumerate}


\clearpage



\section{構成}

システムの構成要素である学習管理部のクラス図を図\ref{fig:学習管理部}に、
情報保持部のクラス図を図\ref{fig:情報保持部}に、
情報分析部のクラス図を図ref{fig:情報分析部}に示す。

学習管理部ではランナー(\verb|Runner|)が起動クラスとなりシステムを制御する。
他にはユーザ(\verb|User|)、ライブラリ(\verb|Library|)とスクリーン(\verb|Screen|)のクラスがある。
ユーザは課程設計者(\verb|Designer|)、教育担当者(\verb|Professor|)、学習者(\verb|Student|)をサブクラスにもつ。


\begin{figure}[H]
  \vspace{0.5cm}
  \hspace*{0.5cm}
  % \includegraphics[width=13cm]{../UI.pdf}
  \caption{学習管理部}
  \label{fig:学習管理部}
\end{figure}

情報保持部で最大のものが課程(\verb|Course|)である。これは達成目標を表す文書(\verb|text|)を含む。
以下すべてのクラスは文書を持ち、必要に応じて画面に表示する。

課程は複数の教材(\verb|Book|)と修了試験(\verb|QualificationExam|)から構成される。
教材にはそれを学習できる制限を表すランク(\verb|rank|)が付随する。
学習者も同様にランクを持っており、自分のランクより高い教材を選択することができない。

\begin{figure}[H]
  \vspace{0.5cm}
  \hspace*{0.5cm}
  % \includegraphics[width=13cm]{../教材関連.pdf}
  \caption{情報保持部}
  \label{fig:情報保持部}
\end{figure}

教材は複数の学習単位(\verb|Chapter|)と完了試験(\verb|CompletionExam|)から構成される。
学習単位は複数の学習項目(\verb|Section|)と確認試験(\verb|ConfirmationExam|)から構成される。
学習単位は理解するまで何度でも繰り返し学習ができる。
これは前の学習(\verb|prev|)と次の学習単位(\verb|next|)をリンクで保持することで実現される。

試験はすべて\verb|Exam|のサブクラスとして実現する。合格基準(\verb|border|)を保持し、
受験者がこれ以上の得点を得た場合に合格とする。
試験は複数の設問(\verb|Question|)から構成される。
設問はその内容を文書及び図として含む\verb|Box|を保持する。
解答(\verb|Answer|)は複数存在し、それぞれに対して得点(\verb|point|)が決められている。
正解/不正解の他に部分的正解として部分点を与えることが可能である。

このように試験を通じて、学習者が学習内容の理解度を確認し、
次の学習内容へ進む条件を満たすかどうかを判断する。
理解度が不足する場合は、学習者まだ把握していない内容を確定し、再学習ができる。
教育担当者側も、説明が不適切な内容を確定し修正できる。


\begin{figure}[H]
  \vspace{0.5cm}
  \hspace*{0.5cm}
  % \includegraphics[width=13cm]{../履歴関連.pdf}
  \caption{情報分析部}
  \label{fig:情報分析部}
\end{figure}


情報分析部で最大のものが教材改善進捗記録(\verb|ScoreHistory|)である。
これは学習した人数を表す数字(\verb|count|)と正答率及び合格率の上昇率(\verb|Improvment|)を含む。
教材改善進捗記録は課程設計者(\verb|Designer|)に利用され、課程(\verb|Course|)の改善に役に立つ。

教材改善進捗記録は複数の学習進捗記録(\verb|ScoreBook|)から構成される。
学習進捗記録にはそれを学習した人数を表す数字(\verb|count|)と正答率及び正答率(\verb|ratios|)が付随する。
学習進捗記録は教育担当者(\verb|Professor|)に利用され、
改善された教材はバージョン(\verb|version|)を更新された上で課程に登録される。
学習者は常に最新のバージョンのみ閲覧できる。

学習進捗記録は複数の学習記録(\verb|ScoreRec|)から構成される。
学習記録は試験内容(\verb|mark|)学習時間(\verb|date|)、獲得した点数(\verb|Score|)を含む、
学習者(\verb|Student|)から利用する。

このような履歴内容による、学習者の学習状況を把握できる。
教育担当者と課程設計者も教材と課程の修正に役に立つデータを獲得できる。


\clearpage


\section{挙動}


複数の学習者が
ユースケース「学習」を起動した後で、
教育担当者が「教材改善のための分析」を起動し、
すべての学習者の正答率を取得するまでの
呼び出しと戻り値を図\ref{fig:シーケンス図}に示す。

ユースケース「学習」の最初に学習記録が作成される。
完了時に完了試験の成績が学習記録に追加される。

教育担当者は教材に対応する学習進捗記録に新しい成績が追加される度に、
正答率が低い学習項目があるかを調べる。

\begin{figure}[H]
  \vspace{0.5cm}
  \hspace*{0.5cm}
  % \includegraphics[width=13cm]{../シーケンス図0.pdf}
  \caption{シーケンス図}
  \label{fig:シーケンス図}
\end{figure}

項目が発見されたら、その項目を対象に
記述が簡潔で分かりやすく改善した教材を用意し、
それをユースケース「教材の登録」により情報保持部に追加する。





