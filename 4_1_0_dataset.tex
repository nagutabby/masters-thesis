本研究では、Ferencらが構築したBugHunter Datasetを基盤として使用する。このデータセットは、GitHubでホストされている15のJavaプロジェクトから自動的に収集されたバグ情報と、各コミットにおけるソースコード要素(ファイル、クラス、メソッド)のメトリクスを含んでいる。


BugHunterデータセットの特徴は、従来の研究が特定のリリースバージョンにおける全てのソースコード要素を収集していたのに対し、バグ混入コミットとバグ修正コミットという、バグの存在を特定できる最も狭い期間におけるコードの状態を捉えている点である。これにより、バグが混入した時点と修正された時点でのコードメトリクスの変化を分析できる。