表\ref{tab:leaf_nodes}に、ベースラインとステップ3における陽性クラスに分類するリーフノード数と、高確信度(陽性割合0.8以上)で分類するノード数を示す。

\begin{table}[h]
\centering
\caption{陽性クラスに分類するリーフノード数の比較}
\label{tab:leaf_nodes}
\begin{tabular}{|l|r|r|r|r|}
\hline
プロジェクト & ベースライン & ステップ3 & ベースライン & ステップ3 \\
 &  &  & (確信度≥0.8) & (確信度≥0.8) \\
\hline
Elasticsearch & 4 & 2 & 0 & 0 \\
Hazelcast & 3 & 4 & 1 (0.847) & 1 (0.862) \\
Neo4j & 4 & 3 & 1 (0.808) & 2 (0.952, 0.907) \\
Netty & 3 & 3 & 2 (0.852, 0.850) & 1 (0.849) \\
OrientDB & 2 & 5 & 0 & 1 (0.911) \\
\hline
\end{tabular}
\end{table}

ベースラインでは、5つのプロジェクト全体で陽性に分類するリーフノードは平均3.2個であったのに対し、ステップ3では平均3.4個となった。より重要な変化は、高確信度で陽性に分類するノードの増加である。ベースラインでは、確信度0.8以上で陽性に分類するノードは5プロジェクト中4個(Hazelcastで1個、Neo4jで1個、Nettyで2個)であったが、ステップ3では5個に増加した。特にNeo4jでは、陽性割合0.952と0.907という非常に高い確信度のノードが2個出現し、OrientDBでは最高0.911の確信度を達成した。Hazelcastでも確信度が0.847から0.862に向上している。これは、時系列変化メトリクスの導入により、バグ混入リスクの高い変更をより明確に識別できるようになったことを示している。