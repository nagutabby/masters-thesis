本研究の主要な貢献は、以下の3点である。

第一に、コードの時系列変化を点過程データとして整理し、従来の時系列分析との差異を明確化したことである。ソフトウェア開発におけるコミットは、株価や気温のような等間隔で観測されるデータとは異なり、不規則な時間間隔で発生するイベントデータである。特にOSSプロジェクトでは、開発者の生活パターンや活動期・休止期の存在により、コミットの発生タイミングは予測困難である。本研究では、こうした不規則性を持つデータを点過程データとして位置付け、直前のコミットとの差分に基づく変化量を特徴量として用いることで、時系列呪法を欠陥予測に活用する枠組みを提示した。この整理により、従来の等間隔な時系列分析手法との違いが明確になり、ソフトウェア欠陥予測における時系列情報活用の方向性が示された。

第二に、メソッド単位とコミット単位という異なる粒度でのメトリクスを設計し、両者を統合することで予測性能を向上させたことである。メソッド単位では、コード行数、トークン数、循環的複雑度の変化量を用い、個々のメソッドがどの程度変更されたかというミクロ的視点を捉える。一方、コミット単位では、変更されたファイル数、追加・削除行数の割合、ファイル平均行数といった変化率を用い、コミット全体の影響範囲や変更の性質というマクロ的視点を捉える。さらに、メソッド単位では変化量、コミット単位では
変化率を採用する理論的根拠を提示した。メソッドは比較的スケールが均一であるため変化量が適しており、コミットは複数のファイルを含み変更規模が多様であるため変化率が適している。このメトリクス設計により、変更の特性を多角的に捉えることが可能になった。

第三に、実際の5つのOSSプロジェクトを用いた実験により、提案手法の有効性を実証したことである。Elasticsearch、Hazelcast、Netty、OrientDB、Neo4jという規模や特性の異なるプロジェクトにおいて、F1スコアが向上することを確認した。また、有意性検定により、この改善が統計的に有意であることを検証した。さらに、特徴量の寄与度、Partial Dependence Plot、決定木といった複数の手法を用いてモデルの解釈性を高め、どの特徴量がバグ予測に寄与しているかを明らかにした。これにより、提案手法が実際のプロジェクトで適用可能であり、実務での意思決定を支援できることが示された。