本研究の主要な貢献は以下の3点である。

\textbf{貢献1: 不規則な時系列データとしての分析}

コミットを不規則なタイミングで発生するイベントとして扱い、等間隔データを前提とする従来の時系列分析との違いを明確化した。この視点により、コミット間隔を考慮した特徴量設計が可能となった。

\textbf{貢献2: 構成要素が異なる変更メトリクスの統合}

メソッド単位とコミット単位の変更メトリクスを統合し、局所的な変化とプロジェクト全体の動向を同時に捉える手法を提示した。

\textbf{貢献3: 実際のプロジェクトでの検証}

5つのOSSプロジェクトで欠陥発見率の向上とその統計的有意性を確認し、特徴量の分析により実用可能性を示した。