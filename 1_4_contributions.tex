本研究の主要な貢献は以下の3点である。

\textbf{貢献1: 不規則な時系列データとしての分析}

開発者がいつコードを書き換えるか分からない「不規則なタイミングで発生するイベント」としてコミットを捉え直し、決まった間隔で記録される一般的な時系列データとの違いを整理した。従来の時系列分析手法は等間隔データを前提としており、コミットの不規則性を適切に扱えていなかった。本視点により、コミット間隔を考慮した特徴量設計が可能となり、より実態に即した欠陥予測が実現できる。

\textbf{貢献2: メトリクス設計}

メソッド・コミット単位の変更メトリクスを統合し、理論的根拠とともに提示した。単一レベルのメトリクスでは捉えられない、コードの局所的変化とプロジェクト全体の開発動向という複数のリスク要因を統合することで、予測精度の向上を実現した。

\textbf{貢献3: 実プロジェクトでの実証}

5つのOSSプロジェクトでF1スコア向上と統計的有意性を確認し、特徴量重要度の分析により実用可能性を示した。