各ステークホルダーが必要とする機能を実現するために決定したユースケースを以下に示す。

\begin{enumerate}
\item 教材の登録

\medskip
\begin{tabular}{|l|} \hline
起動アクター:教育担当者\\%
教育担当者は作成または修正した教材を登録する         \\%
これは失敗しない\\ \hline
\end{tabular}
\medskip

\item 学習

\medskip
\begin{tabular}{|l|} \hline
起動アクター:学習者\\%
成功時:\\%
\; 学習者は学習する課程を選択する\\%
\; 学習者が学習目標による選択した課程の中で、\\%
\; 自分のランクに合った教材を閲覧して学習する\\%
\; 教材のランクが学習者のランクより大きい場合に失敗する    \\%
失敗時:\\%
\; 学習者にメッセージを返す\\ \hline
\end{tabular}
\medskip

\item 完了試験の受験

\medskip
\begin{tabular}{|l|} \hline
起動アクター:学習者\\%
事前条件:学習者が教材の学習を完了している          \\%
システムは学習者の受験結果を成績として得る\\%
成績が合格基準以上:\\%
\; システムは合格証明書を発行する\\%
成績が合格基準未満:\\%
\; 学習者は理解が不足している箇所を再学習する\\ \hline
\end{tabular}
\medskip

\clearpage

\item 修了試験の受験

\medskip
\begin{tabular}{|l|} \hline
起動アクター:学習者\\%
事前条件:学習者は課程の全て教材の合格証明書を得ている    \\%
システムは学習者の受験結果を成績として得る\\%
成績が合格基準以上:\\%
\; システムは修了証明書を発行する\\%
成績が合格基準未満:\\%
学習者は理解が不足している箇所を再学習する\\ \hline
\end{tabular}
\medskip


\item 教材改善のための分析

\medskip
\begin{tabular}{|l|} \hline
起動アクター:教育担当者\\%
システムは教材に対応する学習進捗記録を取得する        \\%
システムは正答率を計算する\\%
これは失敗しない\\ \hline
\end{tabular}
\medskip


\item 課程改善のための分析

\medskip
\begin{tabular}{|l|} \hline
起動アクター:課程設計者\\%
システムは課程に対応する教材改善進捗記録を取得する      \\%
システムは教材改善前後の正答率の増加を計算する\\%
増加が基準値以上:\\%
\; システムは改善不要を通知する\\%
増加が基準値未満:\\%
\; システムは改善の必要があると提示する\\ \hline
\end{tabular}
\medskip


\end{enumerate}
\clearpage