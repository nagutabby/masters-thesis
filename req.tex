機能要求は各ステークホルダーが必要とする情報の提供及び分析の支援である。
図\ref{fig:システムが保持する情報}に保持すべき情報を示す。
これらの作成と利用の際にシステムが提供する機能を情報ごとに列挙する。

\begin{itemize}

\item 学習記録

学習記録は学習者が教材を学習者する度に作成され、教材ごとに履歴として保持される。


\item 学習進捗状況

教育担当者が利用できるように正答率及び完了試験の合格率を計算する。
その時に全体集合を履歴中特定の条件を満たすものとして与えることを可能にする。

\item 教材改善進捗状況

教材が改善される度に、その教材を新たに学習した学習者に対する学習進捗状況として作成され、
教材ごとに履歴として保持される。
課程設計者が利用できるように正答率及び合格率の改善の効果を計算する。
全体集合は同様に特定の条件により選択可能にする。

\end{itemize}


\begin{figure}[H]
  \vspace{0.5cm}
  \hspace*{0.5cm}
  % \includegraphics[width=12cm,bb=0 150 800 500]{../structure.pdf}
  \caption{システムが保持する情報}
  \label{fig:システムが保持する情報}
\end{figure}


\clearpage
