コミット単位では、一つのコミット全体での変更の影響範囲や性質を捉えるため、以下の変化率を特徴量として用いる。

\begin{itemize}
    \item 変更されたファイル数
    \begin{itemize}
        \item 1つのコミットで変更されたファイルの総数
    \end{itemize}
    \item 追加行数の割合
    \begin{itemize}
        \item 変更前の総行数に対する追加されたコード行数の割合
    \end{itemize}
    \item 削除行数の割合
    \begin{itemize}
        \item 変更前の総行数に対する削除されたコード行数の割合
    \end{itemize}
    \item 1ファイル当たりの平均行数
    \begin{itemize}
        \item 変更対象となったファイルの平均規模
    \end{itemize}
\end{itemize}

これらのメトリクスで変化率を採用する理論的根拠は以下の通りである。第一に、変更規模の多様性への対応である。コミットには、単一のファイルのみを変更する小規模なものから、数十のファイルを変更する大規模なものまで、多様な規模が存在する。変化率を用いることで、異なる規模のコミット間での比較が可能になる。第二に、ファイルに対する影響度を測定できる。追加行数の割合や削除行数の割合は、ファイルサイズに対してどの程度の変更が加えられたかを示し、大規模な機能追加やリファクタリングを示唆する指標となる。第三に、予測基準のバランスが向上する。変化量のみを用いると、大規模なコミットほどバグ修正コミットであると判定されやすくなるが、変化率を用いることで、小規模だが重要な欠陥の混入や修正を検出しやすくなる。

メソッド単位の変更メトリクスは、変更されたメソッド自体の特性を捉える。例えば、あるメソッドの循環的複雑度が5から15に増加した場合、そのメソッド内部の制御構造が複雑化したことを示す。一方、コミット単位の変更メトリクスは、同じコミットで変更された他のコードの特性も捉える。例えば、変更されたファイル数が多い場合、その変更が複数のモジュールに影響を与えていることを示す。

これらのメトリクスを組み合わせることで、コード変更をミクロとマクロの両面から多角的に評価できる。メソッド単位のメトリクスが局所的な変更の性質を捉え、コミット単位のメトリクスが変更の全体的な影響範囲を捉えることで、より精度の高い欠陥予測が可能になる。