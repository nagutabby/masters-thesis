提案手法の有効性を評価するため、レビュー労力に対する欠陥発見数を分析した。表\ref{tab:cost_benefit_summary}に、レビュー労力を20\%および40\%に制限した場合の欠陥発見率を示す。

\begin{table}[ht]
\centering
\caption{レビュー労力20\%および40\%時点での欠陥発見率}
\label{tab:cost_benefit_summary}
\small
\begin{tabular}{|l|r|r|r|r|}
\hline
 & \multicolumn{2}{c|}{レビュー労力 20\%時} & \multicolumn{2}{c|}{レビュー労力 40\%時} \\
\cline{2-5}
プロジェクト & ベースライン & \textbf{提案手法} & ベースライン & \textbf{提案手法} \\
\hline
Elasticsearch & 64.7\% & \textbf{68.1\%} & 78.8\% & \textbf{86.6\%} \\
Hazelcast & 56.1\% & \textbf{56.1\%} & 77.6\% & \textbf{80.0\%} \\
Neo4j & 53.0\% & \textbf{68.8\%} & 72.5\% & \textbf{89.5\%} \\
Netty & 55.8\% & \textbf{75.6\%} & 77.4\% & \textbf{93.5\%} \\
OrientDB & 54.0\% & \textbf{65.8\%} & 72.6\% & \textbf{85.7\%} \\
\hline
\end{tabular}
\end{table}

全プロジェクトで提案手法がベースラインを上回り、同一のレビュー労力における欠陥発見率の増加が確認された。レビュー労力20\%の時点では、ベースラインが平均56.7\%の欠陥を発見するのに対し、提案手法では66.9\%を発見し、10.2ポイントの改善を達成した。レビュー労力40\%では、ベースラインの75.8\%に対し提案手法は87.0\%となり、11.3ポイントの改善が見られた。

特にNeo4jとNettyでは顕著な効果が確認された。Neo4jではレビュー労力20\%時点で15.8\%、40\%時点で17.0\%の改善を示した。Nettyでは20\%時点で19.8\%、40\%時点で16.1\%の改善となり、わずか20\%の労力で全体の75\%以上の欠陥を検出できるようになった。一方、Hazelcastでは20\%時点で改善が見られなかったものの、40\%時点では2.4\%の改善が確認された。

図\ref{fig:comparison_cost_benefit_neo4j}に、Neo4jにおけるレビュー労力に対する欠陥発見数の累積曲線を示す。提案手法の曲線がベースラインより左上に位置しており、同一労力でより多くの欠陥を発見できることが視覚的に確認できる。

\begin{figure}[ht]
\centering
\includegraphics[width=0.7\textwidth]{figures/neo4j/comparison_cost_benefit_curve.png}
\caption{レビュー労力に対する欠陥発見数の比較(代表例: Neo4j)}
\label{fig:comparison_cost_benefit_neo4j}
\end{figure}

\paragraph{統計的有意性の検証}
ウィルコクソンの符号順位検定により、改善の統計的有意性を検証した。表\ref{tab:wilcoxon_results}に結果を示す。

\begin{table}[ht]
\centering
\caption{欠陥発見率に対するウィルコクソンの符号順位検定の結果}
\label{tab:wilcoxon_results}
\begin{tabular}{|l|r|r|r|l|}
\hline
労力 & ベースライン & 提案手法 & 平均改善幅 & p値 \\
\hline
20\% & 56.7\% & 66.9\% & +10.2\% & 0.0625 \\
40\% & 75.8\% & 87.0\% & +11.3\% & 0.0313 \\
\hline
\end{tabular}
\end{table}

レビュー労力40\%の時点で統計的に有意な改善が確認された($p=0.0313$)。一方、20\%の時点では有意差は認められなかった($p=0.0625$)。これは、20\%時点ではプロジェクト間のばらつきが大きく、Hazelcastで改善が見られなかった一方でNettyやNeo4jでは大幅な改善が見られたためである。