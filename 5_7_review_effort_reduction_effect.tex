提案手法の実践的な有効性を評価するため、Cost-Benefit Curveを用いてレビュー効率を分析した。表\ref{tab:cost_benefit_summary}に、レビュー労力を20\%および40\%に制限した場合のバグ検出率を示す。

\begin{table}[ht]
\centering
\caption{レビュー労力20\%および40\%時点でのバグ検出率(提案手法の効果)}
\label{tab:cost_benefit_summary}
\small
\begin{tabular}{|l|r|r|r|r|}
\hline
 & \multicolumn{2}{c|}{レビュー労力 20\%時} & \multicolumn{2}{c|}{レビュー労力 40\%時} \\
\cline{2-5}
プロジェクト & ベースライン & \textbf{提案手法} & ベースライン & \textbf{提案手法} \\
\hline
Elasticsearch & 64.7\% & \textbf{68.1\%} & 78.8\% & \textbf{86.6\%} \\
Hazelcast & 56.1\% & \textbf{56.1\%} & 77.6\% & \textbf{80.0\%} \\
Neo4j & 53.0\% & \textbf{68.8\%} & 72.5\% & \textbf{89.5\%} \\
Netty & 55.8\% & \textbf{75.6\%} & 77.4\% & \textbf{93.5\%} \\
OrientDB & 54.0\% & \textbf{65.8\%} & 72.6\% & \textbf{85.7\%} \\
\hline
\end{tabular}
\end{table}

特にNeo4jやNettyでは、提案手法を用いることで、わずか20\%の労力で全体の約70〜75\%のバグを検出可能であり、ベースラインと比較して15ポイント以上の大幅な効率化を達成した。図\ref{fig:comparison_cost_benefit_neo4j}に、Neo4jにおけるCost-Benefit Curveを示す。提案手法(青線)がベースライン(赤線)よりも上側に位置しており、少ない労力で多くのバグを発見できていることが分かる。

\begin{figure}[ht]
\centering
\includegraphics[width=0.7\textwidth]{figures/neo4j/comparison_cost_benefit_curve.png}
\caption{Cost-Benefit Curveの比較(代表例: Neo4j)}
\label{fig:comparison_cost_benefit_neo4j}
\end{figure}