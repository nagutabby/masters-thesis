提案手法の実践的な有効性を評価するため、レビュー労力に対する欠陥発見数を分析した。表\ref{tab:cost_benefit_summary}に、レビュー労力を20\%および40\%に制限した場合の欠陥発見率を示す。

\begin{table}[ht]
\centering
\caption{レビュー労力20\%および40\%時点での欠陥発見率(提案手法の効果)}
\label{tab:cost_benefit_summary}
\small
\begin{tabular}{|l|r|r|r|r|}
\hline
 & \multicolumn{2}{c|}{レビュー労力 20\%時} & \multicolumn{2}{c|}{レビュー労力 40\%時} \\
\cline{2-5}
プロジェクト & ベースライン & \textbf{提案手法} & ベースライン & \textbf{提案手法} \\
\hline
Elasticsearch & 64.7\% & \textbf{68.1\%} & 78.8\% & \textbf{86.6\%} \\
Hazelcast & 56.1\% & \textbf{56.1\%} & 77.6\% & \textbf{80.0\%} \\
Neo4j & 53.0\% & \textbf{68.8\%} & 72.5\% & \textbf{89.5\%} \\
Netty & 55.8\% & \textbf{75.6\%} & 77.4\% & \textbf{93.5\%} \\
OrientDB & 54.0\% & \textbf{65.8\%} & 72.6\% & \textbf{85.7\%} \\
\hline
\end{tabular}
\end{table}

特にNeo4jやNettyでは、提案手法を用いることで、わずか20\%の労力で全体の約70〜75\%の欠陥を検出可能であり、ベースラインと比較して15ポイント以上の大幅な効率化を達成した。図\ref{fig:comparison_cost_benefit_neo4j}に、Neo4jにおけるコードレビュー労力に対する欠陥発見数の累積積み上げグラフを示す。提案手法(青線)がベースライン(赤線)よりも上側に位置しており、少ない労力で多くの欠陥を発見できていることが分かる。

\begin{figure}[ht]
\centering
\includegraphics[width=0.7\textwidth]{figures/neo4j/comparison_cost_benefit_curve.png}
\caption{レビュー労力に対する欠陥発見数の比較(代表例: Neo4j)}
\label{fig:comparison_cost_benefit_neo4j}
\end{figure}

\paragraph{統計的有意性の検証}
レビュー労力に対する欠陥発見率の改善が統計的に有意であるかを検証するため、ウィルコクソンの符号順位検定を実施した。表\ref{tab:wilcoxon_results}に、レビュー労力20%と40%の時点における検定結果を示す。

\begin{table}[ht]
\centering
\caption{レビュー労力に対する欠陥発見率のウィルコクソンの符号順位検定結果}
\label{tab:wilcoxon_results}
\begin{tabular}{|l|r|r|r|l|}
\hline
労力 & ベースライン & 提案手法 & 平均改善幅 & p値 \\
\hline
20\% & 56.7\% & 66.9\% & +10.2\% & 0.0625 \\
40\% & 75.8\% & 87.0\% & +11.3\% & 0.0313 \\
\hline
\end{tabular}
\end{table}

レビュー労力20\%の時点では、平均10.2ポイントの改善が見られたものの、統計的に有意な差は認められなかった(p=0.0625)。一方、レビュー労力40\%の時点では、平均11.3ポイントの改善が確認され、統計的に有意な差が認められた(p=0.0313)。

この結果は、提案手法が複数のプロジェクトにわたって一貫してベースライン手法を上回ることを統計的に裏付けている。特に、レビュー労力40\%という実用的な範囲において、提案手法の優位性が統計的に保証されることは、実際の開発現場での適用可能性を示唆する重要な知見である。

レビュー労力20\%の時点で有意差が認められなかった理由として、プロジェクト間のばらつきが考えられる。表に示したように、Hazelcastでは改善幅が0.0\%であり、Elasticsearchでも3.4\%にとどまった一方、NettyとNeo4jでは15\%以上の大きな改善が見られた。このばらつきにより、サンプルサイズ5という制約の下では統計的有意性の基準を満たさなかったと考えられる。しかし、レビュー労力40\%では、全5プロジェクトで2.4\%から17.0\%の改善が一貫して確認され、統計的有意性が認められた。