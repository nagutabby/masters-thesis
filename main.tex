\documentclass[12pt,a4paper]{ltjsreport}
\usepackage{graphicx}
\usepackage{fontspec}
\usepackage{url}
\usepackage{colortbl,array,xcolor}
\usepackage{here}
\usepackage{amsmath}
\usepackage{lmodern}
\usepackage{silence}
\WarningFilter{caption}{Unknown document class (or package)}
\usepackage{subcaption}

\begin{document}
\thispagestyle{empty}
\begin{center}                                                        
コードの時系列変化を考慮した\\
保守性低下の要因分析と改善
\vfill
\vfill
2410064 笹川 尋翔\\
\vfill
主指導教員  鈴木 正人\\
\vfill
北陸先端科学技術大学院大学\\
先端科学技術研究科\\
情報科学\\ 
\vfill
令和8年3月\\ % 学位授与年月
\vfill
\end{center}

% 目次の出力
\tableofcontents

\clearpage
\centerline{概要}
In software development, continuous code changes bring about defect risk. Existing defect prediction methods use structural metrics. However, these metrics cannot capture the characteristics of changes. Recent studies have investigated change metrics from the version control history. They have shown that change-related features are more strongly correlated with defects than structural metrics. However, these approaches have two problems. First, they do not consider irregularly occurring change events. Instead, they collect change information at specific versions. Second, they do not combine the characteristics of different components. Instead, they focus on the metrics of specific components.

In this work, we propose a defect prediction method that differs from existing approaches in three aspects. First, traditional methods partition the state at regular intervals. In contrast, we treat commits as irregularly occurring events and extract features from the differences between consecutive commits. Thus, we capture the irregular development processes in software projects. Second, existing studies focus on single components. In contrast, our method combines per-method metrics that reflect local changes and per-commit metrics that reflect global changes. This allows us to capture both local and global trends simultaneously. Third, our method selects commits for review by considering both the review effort and the probability of defect introduction. This consideration is based on the scale of the changes, the complexity of the changes, and the spread of the changes. Consequently, this provides a more effective review prioritization method under review effort constraints. 

Experiments on five open-source projects validate the effectiveness of our method. We confirm that the prediction accuracy has been improved in all projects. Moreover, the proposed method outperforms existing methods using only structural metrics. The defect discovery rate is improved by the review prioritization method considering more features.
\clearpage

\chapter{はじめに}
\section{背景}
ソフトウェア開発では継続的な変更が不可避であり、各変更はバグ混入のリスクを伴う。変更により既存コードとの整合性が崩れ、副作用が発生しやすくなるためである。Microsoft Researchの調査では76\%の開発者がリファクタリングによるバグ混入を懸念している\cite{kim2014}。

従来の品質保証手法には限界がある。静的解析はコードの時系列変化を捉えられないため、頻繁に変更される不安定な箇所や、変更が集中する高リスク領域を識別できない。コードレビューでは全ての変更を詳細に検査する時間的な余裕がなく、どの変更を優先的にレビューすべきかの判断基準が不明確である。特に大規模プロジェクトでは、限られたレビューリソースの効率的配分が重要な課題となっている。

機械学習による欠陥予測の研究が進められているが、既存手法はコードの静的特徴量に依存している。静的特徴量では変更頻度や変更パターンといった動的な品質リスク要因を捉えられないため、予測精度に限界がある。コードがどのように変化してきたかという時系列情報は、開発プロセスの動的な側面を反映し、欠陥との強い相関が期待されるが、十分に活用されていない。
\section{動機}
従来のソフトウェア品質保証のアプローチは、主に事後対応型である。すなわち、バグが発生してから検出し、修正するという流れが一般的であった。しかし、バグが本番環境で発見された場合、その影響はユーザー体験の低下やシステムの停止など、組織にとって大きな損失につながる可能性がある。さらに、開発が進んだ後の段階でバグを修正するコストは、開発初期段階での修正に比べて桁違いに高くなることが知られている。そのため、バグが混入する前に、あるいは混入直後に検出し対処する事前予防型のアプローチへの転換が求められている。

事前予防型のアプローチを実現するためには、コード変更時点でバグ混入リスクを評価し、リスクの高い偏光を早期に特定する必要がある。これにより、開発者は変更を本番環境に反映する前に、より慎重なレビューやテストを実施できる。また、問題が小規模なうちに対処することで、後の段階での大規模な修正を回避し、開発コストを抑制できる可能性がある。

しかし、現実の開発現場では、全ての変更に対して十分な時間をかけてレビューやテストを実施することは困難である。特に、アジャイル開発やDevOpsのような短いサイクルでの開発が主流となる中、レビューに割ける時間やリソースは限られている。そのため、限られたレビューリソースをどのように配分するかという問題が重要になる。バグ混入リスクの高い変更を優先的にレビューすることで、同じリソースでより多くの欠陥を発見できれば、開発効率と品質の両立が可能になる。

このような効率的なレビュー優先度付けを実現するためには、コード変更の特性に基づいてバグ混入リスクを定量的に評価する手法が必要である。ここで重要なのは、単に現時点でのコードの静的な特性を見るだけでなく、コードがどのように変化してきたかという時系列的な情報に着目することである。例えば、短期間に頻繁に変更されているコード、あるいは大規模な変更が加えられたコードは、バグ混入リスクが高い可能性がある。また、過去にバグが多く発見された箇所への変更も、同様にリスクが高いと考えられる。

コードの変化過程に着目した時系列情報の活用は、静的解析では捉えきれない動的なリスク要因を明らかにする可能性がある。変更の頻度、変更の規模、変更されたファイル間の関連性など、時系列的な観点から抽出できる特徴量は多数存在する。これらの情報を機械学習モデルに組み込むことで、より精度の高い欠陥予測が可能になると期待される。

本研究は、こうした背景から、コードの時系列変化に着目したソフトウェア欠陥予測手法の確立を目指す。時系列情報を活用することで、バグ混入リスクの事前予測精度を向上させ、限られたレビューリソースの効率的な配分を支援する実用的な手法を提供することが本研究の動機である。
\section{目的}
本研究の目的は、コードの時系列変化を考慮した欠陥予測により、限られたレビューリソースの効率的配分を支援することである。具体的には以下を達成する。

\textbf{目的1: 時系列メトリクスの設計}

メソッド単位の変化量とコミット単位の変化率を組み合わせ、ミクロとマクロの両視点から変更特性を捉える。

\textbf{目的2: 高精度予測モデルの構築}

時系列特徴量を用いた機械学習モデルにより、従来手法を上回る予測性能を実証する。

\textbf{目的3: レビュー労力削減効果の定量評価}

ナップサック問題としての定式化により、同一労力でのバグ発見数向上をCost-Benefit Curveで検証する。
\section{貢献}
本研究の主要な貢献は以下の3点である。

\textbf{貢献1: 不規則な時系列データとしての分析}

開発者がいつコードを書き換えるか分からない「不規則なタイミングで発生するイベント」としてコミットを捉え直し、決まった間隔で記録される一般的な時系列データとの違いを整理した。従来の時系列分析手法は等間隔データを前提としており、コミットの不規則性を適切に扱えていなかった。本視点により、コミット間隔を考慮した特徴量設計が可能となり、より実態に即した欠陥予測が実現できる。

\textbf{貢献2: メトリクス設計}

メソッド・コミット単位の変更メトリクスを統合し、理論的根拠とともに提示した。単一レベルのメトリクスでは捉えられない、コードの局所的変化とプロジェクト全体の開発動向という複数のリスク要因を統合することで、予測精度の向上を実現した。

\textbf{貢献3: 実プロジェクトでの実証}

5つのOSSプロジェクトでF1スコア向上と統計的有意性を確認し、特徴量重要度の分析により実用可能性を示した。
\section{構成}
本論文の構成は以下の通りである。

第2章では、ソフトウェア欠陥予測に関する関連研究を概観する。Just-In-Time品質保証の分野におけるKameiらの研究を取り上げ、14個の変更メトリクスとその問題点におついて詳述する。また、Ferencらのコミットベースの欠陥予測手法、Hanらのコードレビューテキスト分析、Romanoらの静的解析による品質改善研究を紹介する。これらの既存研究を整理することで、コミット間変化量の考慮不足や特性値間の関連性分析の欠如といった問題点を明らかにする。

第3章では、本研究で提案する時系列変化を考慮した欠陥予測手法について説明する。まず、コミットを点過程データとして位置付け、従来の等間隔時系列分析との違いを明確化する。次に、メソッド単位での変化量メトリクスとコミット単位での変更率メトリクスの設計方針を述べ、それぞれのメトリクスで変化量と変化率を使い分ける理論的根拠を示す。最後に、機械学習モデルの選定理由、モデルの評価指標、有意性検定について説明する。

第4章では、実験設定について詳述する。データセットにおける正解ラベルの定義方法、機械学習モデルの学習における追加の制約、プロジェクト選定基準を説明する。対象とする5つのプロジェクト(Elasticsearch、Hazelcast、Netty、OrientDB、Neo4j)の特性を示し、直前コミットとの差分に基づく特徴量生成方法と前処理手順を述べる。さらに、ベースライン、変化量メトリクスの追加後、変化率メトリクスの追加後の3段階での評価を行う実験手順と、モデル解釈性分析の方法について説明する。

第5章では、実験結果を報告する。5つのプロジェクトにおけるF1スコアなどの評価指標の比較表を示し、提案手法による予測性能の改善を定量的に示す。特徴量の寄与度の分析により、どのような特徴量が予測に貢献しているのかを明らかにする。また、特徴量の分布、Partial Dependence Plotによる各特徴量と陽性確率の関係、決定木による分類条件の可視化、Cost-Benefit Curveによるレビュー労力削減効果を示す。

第6章では、実験結果に基づく考察を行う。予測性能向上のメカニズムを分析し、各特徴量の意味と予測への寄与について解釈する。プロジェクト間での特徴量の影響の違いを考察し、その背景にある開発フェーズやコーディング文化の違いを議論する。また、クラス不均衡の影響やモデルの予測特性、レビュー労力モデルの単純化といった本研究の限界を明示し、今後の研究課題を提示する。

第7章では、本研究の成果を総括し、主要な貢献を再確認する。主要な時系列分析手法との違い、メトリクス設計、実プロジェクトでの性能検証という3つの貢献を振り返り、実用的な意義と今後の展望について述べる。
\clearpage

\chapter{コード品質と関連研究}
本章では、ソフトウェア欠陥予測に関連する研究を概観する前に、本研究で用いる主要な概念と用語について説明する。

ソフトウェアの品質を体系的に評価するため、JIS X 25010:2013\cite{jisx25010}では、製品品質モデルと利用時の品質モデルという2つの品質モデルが定義されている。製品品質モデルは、「ソフトウェアの静的特徴及びコンピュータシステムの動的特徴」に関する品質を8つの特性で分類する。これらの特性は、機能適合性、性能効率性、互換性、使用性、信頼性、セキュリティ、保守性、移植性である。各特性は更に副特性に分割され、より詳細な品質評価を可能にする。

本研究が特に注目するのは、\textbf{保守性}と\textbf{信頼性}である。保守性は「意図した保守者によって、製品又はシステムが修正することができる有効性及び効率性の度合い」と定義され、5つの副特性を持つ。\textbf{モジュール性}(一つの構成要素に対する変更が他の構成要素に与える影響が最小になる度合い)、\textbf{再利用性}(一つ以上のシステムに資産を使用することができる度合い)、\textbf{解析性}(変更の影響を総合評価すること、欠陥若しくは故障の原因を診断すること、又は修正しなければならない部分を識別することが可能であることについての有効性及び効率性の度合い)、\textbf{修正性}(欠陥の取込みも既存の製品品質の低下もなく、有効的に、かつ、効率的に製品又はシステムを修正することができる度合い)、\textbf{試験性}(試験基準を確立し、その基準が満たされているかどうかを決定するために試験を実行することができる有効性及び効率性の度合い)である。

信頼性は「明示された時間帯で、明示された条件下に、システム、製品又は構成要素が明示された機能を実行する度合い」と定義され、4つの副特性を持つ。\textbf{成熟性}(通常の運用操作の下で、システム、製品又は構成要素が信頼性に対するニーズに合致している度合い)、\textbf{可用性}(使用することを要求されたとき、システム、製品又は構成要素が運用操作可能及びアクセス可能な度合い)、\textbf{障害許容性}(ハードウェア又はソフトウェア障害にもかかわらず、システム、製品又は構成要素が意図したように運用操作できる度合い)、\textbf{回復性}(中断時又は故障時に、製品又はシステムが直接的に影響を受けたデータを回復し、システムを希望する状態に復元することができる度合い)である。

本研究では、コードの変更がこれらの品質特性、特に保守性の解析性・修正性と信頼性の成熟性にどのように影響を与えるかを分析する。コードの変化の過程を捉えることが重要である理由は、静的なスナップショットだけでは変更の勢いや不安定性を把握できないためである。例えば、ある時点でのコードメトリクスは問題なくても、短期間に頻繁な変更が繰り返されていれば、それは設計の不安定性や開発者の理解不足を示唆している可能性がある。時系列変化を分析することで、保守性を低下させる要因や、信頼性を脅かす欠陥の混入パターンを明らかにすることを目指す。

ソフトウェアの品質を定量的に評価するため、様々なコードメトリクスが提案されてきた。これらのメトリクスは、大きく3つのカテゴリーに分類できる。

\textbf{複雑度メトリクス}は、コードの構造的な複雑さを測定する。代表的なものとして、McCabeの循環的複雑度(Cyclomatic Complexity)\cite{mccabe1976}がある。これは、プログラムの制御フローグラフにおける独立したパスの数を表し、$V(G) = E - N + 2P$($E$は辺の数、$N$は節の数、$P$は連結成分の数)として計算される。循環的複雑度が高いほど、コードの理解が困難になり、テストすべきパスが増加する。これが欠陥混入リスクを高める理由は、人間の認知的限界により、開発者が複雑なコードの全ての振る舞いを把握することが困難になり、また、テストパスの増加によってテストの網羅性が低下するためである。その結果、保守性の解析性と修正性を低下させる。

\textbf{規模メトリクス}は、コードの量的な大きさを測定する。最も基本的なメトリクスは、LOC(Lines of Code、総コード行数)である。変更規模を示すメトリクスとしては、追加されたコード行数(LA)、削除されたコード行数(LD)、変更前のファイルのコード行数(LT)などがある。規模が大きいほど、より多くのコードの変更や実装が必要となるため、欠陥が発生する可能性が高くなる。

\textbf{結合度・凝集度メトリクス}は、オブジェクト指向プログラムの構造的品質を測定する。Chidamber \& Kemerer\cite{chidamber1994}が提案したCKメトリクスは、オブジェクト指向設計の品質を評価するための標準的な指標である。CBO(Coupling Between Objects、オブジェクト間結合度)は、あるクラスが他のクラスに依存している度合いを示す。RFC(Response For Class、クラスの応答数)は、クラスが呼び出す可能性のあるメソッドの総数を示す。LCOM(Lack of Cohesion of Methods、メソッド凝集度の欠如)は、クラス内のメソッド間の凝集性の欠如を測定する。これらのメトリクスは、保守性のモジュール性や再利用性に直接影響を与える。

本研究では、これらの既存メトリクスに加えて、メソッドレベルでの時系列変化を捉えるための新しいメトリクスを提案する。

ソフトウェア工学における欠陥関連の用語は、ISO/IEC/IEEE 24765\cite{iso24765}において厳密に定義されている。\textbf{欠陥(Defect)}は、成果物が要件または仕様を満たさず、修理や交換が必要となることである。欠陥は、コード内に存在するが、必ずしも実行時に顕在化するとは限らない。\textbf{故障(Failure)}は、システムが要求された機能を実行できない事象のことである。故障は、システムの実行時に観測される異常な振る舞いであり、欠陥が顕在化した結果である。\textbf{誤り(Error)}は、人間が犯す間違いのことであり、欠陥の原因となる。

本研究では、「バグ」という用語を一般的な意味で使用するが、厳密な議論では「欠陥」という用語を用いる。また、「バグ混入コミット」または「欠陥誘発コミット」とは、後に修正が必要となる欠陥を含むコードをリポジトリに追加したコミットを指す。

欠陥混入コミットの特定には、SZZアルゴリズム\cite{sliwerski2005}を用いる。SZZアルゴリズムは、Śliwerski、Zimmermann、Zellerによって提案された手法であり、バグ修正コミットから遡って、そのバグを最初に混入させたコミットを特定する。具体的な手順は以下の通りである。まず、バグレポートシステム(例:JIRA、Bugzilla)とコミットメッセージを関連付けることで、バグ修正コミットを特定する。次に、バグ修正コミットにおいて変更された行を特定し、Gitの\texttt{git blame}機能などを用いて、それらの行が最後に修正されたコミットを遡って追跡する。この追跡によって特定されたコミットが、バグを混入させたと推定される。

本研究では、Gitなどのバージョン管理システム(VCS: Version Control System)に記録された変更履歴を分析対象とする。\textbf{コミット(Commit)}は、ソースコードへの変更をリポジトリに記録する単位である。各コミットには、変更されたファイル、追加・削除された行、コミットメッセージ、作成者、作成日時などの情報が含まれる。\textbf{差分(Diff)}は、コミット間でのファイルの変更内容を表現する方法であり、どの行が追加・削除・変更されたかを示す。

本研究がコミット単位を分析単位として選択した理由は、コミットが開発者の論理的な作業単位を表し、変更の意図やコンテキストが比較的明確であるためである。変更の文脈が明確であることは、欠陥予測において重要な意味を持つ。なぜなら、欠陥の混入は単にコード量の増加だけでなく、変更の目的や背景と密接に関係しているからである。例えば、バグ修正のための変更は新機能追加よりも欠陥を誘発しやすい傾向がある。また、レビュー時には変更の意図を理解することで、より効果的なレビューが可能になる。ファイルやパッケージ単位での予測と比較して、コミット単位での予測は以下の3つの利点がある。第一に、予測単位の粒度が小さいため、開発者は具体的にどのコード断片をレビューすべきかを特定しやすい。第二に、変更を行った開発者が明確であるため、品質保証活動の担当者を容易に割り当てることができる。第三に、開発サイクルの早い段階で予測が行われるため、開発者の記憶が新しいうちにレビューやテストを実施でき、修正コストを低減できる。

欠陥予測モデルの性能を評価するため、本研究では以下の指標を使用する。これらの指標は、\textbf{混同行列(Confusion Matrix)}に基づいて計算される。混同行列は、予測結果と実際の結果を4つのカテゴリーに分類する。\textbf{TP(True Positive、真陽性)}は、バグありと予測し、実際にバグがあったケース、\textbf{TN(True Negative、真陰性)}は、バグなしと予測し、実際にバグがなかったケース、\textbf{FP(False Positive、偽陽性)}は、バグありと予測したが、実際にはバグがなかったケース、\textbf{FN(False Negative、偽陰性)}は、バグなしと予測したが、実際にはバグがあったケースである。

\textbf{適合率(Precision)}は、バグありと予測したもののうち、実際にバグがあった割合を示す。$Precision = \frac{TP}{TP + FP}$として計算される。適合率が高いほど、誤検出(偽陽性)が少ないことを意味する。\textbf{再現率(Recall)}は、実際にバグがあったもののうち、バグありと予測できた割合を示す。$Recall = \frac{TP}{TP + FN}$として計算される。再現率が高いほど、見逃し(偽陰性)が少ないことを意味する。

\textbf{F1スコア(F1-score)}は、適合率と再現率の調和平均であり、$F1 = 2 \times \frac{Precision \times Recall}{Precision + Recall}$として計算される。F1スコアは、適合率と再現率のバランスを評価する指標であり、両者が共に高い値を取るときに高くなる。本研究では、F1スコアを主要な評価指標として使用する。

ソフトウェアの変更履歴では、「バグが含まれるケース」が「正常なケース」に比べて圧倒的に少ないという、データの極端な偏り(クラス不均衡問題)がある。クラス不均衡が問題となる理由は、機械学習モデルが多数クラス(バグなし)に偏った予測を行い、少数クラス(バグあり)の検出精度が著しく低下するためである。この問題に対処するため、\textbf{オーバーサンプリング}(少数クラスのデータを人工的に増やす手法。例: SMOTE)や\textbf{アンダーサンプリング}(多数クラスのデータを削減する手法)といった手法が用いられる。本研究では、ランダムアンダーサンプリングを用いる。この手法を選択した理由は、オーバーサンプリングによる人工データの生成は過学習のリスクを高める可能性があるのに対し、アンダーサンプリングは実データのみを使用するため、モデルの汎化性能が安定しやすいためである。具体的には、バグありクラスとバグなしクラスから同数のデータを取得する。
\section{Just-In-Time品質保証}
従来のソフトウェア欠陥予測手法では、ファイルやパッケージ単位での予測が主流であった。このアプローチでは、McCabeの循環的複雑度やCKメトリクス、コード行数といったコードメトリクス、あるいは過去の欠陥数や変更回数といったプロセスメトリクスを用いて、欠陥が発生しやすいモジュールを特定する。しかし、このアプローチには以下の3つの問題点がある。第一に、予測単位の粒度が大きいため、重要なファイルが特定された後も、開発者はレビューやテストの対象となる具体的な関数やコードの断片を見つけるのにかなりの時間を費やす必要がある。第二に、予測がモジュール単位で行われるため、誰がその品質保証活動を担当すべきかが明確でない。第三に、予測が開発サイクルの遅い段階で行われるため、問題が発見された時点では既に多くのコードが書かれており、修正コストが高くなる。

これらの問題に対処するため、Kameiら\cite{kamei2013}は、ファイルやパッケージではなく、コミット単位での欠陥予測を行う「Just-In-Time品質保証」というアプローチを提案した。このアプローチでは、開発者がコードをリポジトリにコミットした直後に、その変更が欠陥を誘発するリスクを予測する。これにより、開発者は記憶が新しいうちにリスクの高い変更をレビューやテストできるため、より効率的かつ効果的な品質保証活動が可能になる。変更単位での予測には、予測単位の粒度が小さいこと、変更を行った開発者への具体的な作業割り当てとして予測を表現できること、開発サイクルの早い段階で予測が行われることという3つの利点がある。

Kameiらは、コード変更から抽出した特徴量を5つのカテゴリーに分類し、合計14個の変更メトリクスを提案した。これらのメトリクスは以下の通りである。

\begin{itemize}
    \item Diffusion
    \begin{itemize}
        \item 変更の拡散範囲を測定するメトリクスであり、NS(変更されたサブシステムの数)、ND(変更されたディレクトリの数)、NF(変更されたファイルの数)、Entropy(各ファイル間の変更されたコードの分散度)が含まれる。広範囲に分散した変更は理解が複雑になり、全ての変更箇所を追跡する必要があるため、欠陥発生リスクが高いと考えられる。
    \end{itemize}
    \item Size
    \begin{itemize}
        \item 変更の規模を測定するメトリクスであり、LA(追加されたコード行数)、LD(削除されたコード行数)、LT(変更前のファイルのコード行数)が含まれる。変更が大きいほど、より多くのコードの変更や実装が必要となるため、欠陥が発生する可能性が高くなる。
    \end{itemize}
    \item Purpose
    \begin{itemize}
        \item 変更の目的を示すメトリクスであり、FIX(変更が欠陥修正であるかどうか)が含まれる。欠陥を修正する変更は、以前の実装で欠陥が混入したことを意味し、その箇所に再び欠陥が混入しやすい可能性がある。
    \end{itemize}
    \item History
    \begin{itemize}
        \item ファイルの変更履歴を測定するメトリクスであり、NDEV(変更されたファイルを変更した開発者数)、AGE(直前の変更と現在の変更の間の平均時間間隔)、NUC(変更されたファイルへのユニークな変更回数)が含まれる。過去の研究では、ファイルに対する過去の変更回数と欠陥修正回数は、ファイルのバグの多さを示す良い指標であることが示されている。
    \end{itemize}
    \item Experience
    \begin{itemize}
        \item 開発者の経験を測定するメトリクスであり、EXP(開発者の全体的な経験)、REXP(開発者の最近の経験)、SEXP(サブシステムにおける開発者の経験)が含まれる。経験が豊富な開発者ほど欠陥を導入しにくいと考えられる。
    \end{itemize}
\end{itemize}

Kameiらは、6つのOSSプロジェクトと5つの商用プロジェクトを対象とした大規模な実証研究を実施した。多重共線性に対処するため、高い相関関係にある因子を除去し、NDとREXPをモデルから除外した。また、LAとLDをLTで割って正規化し(LA/LT、LD/LT)、LTとNUCをNFで割って正規化した(LT/NF、NUC/NF)。その結果、最終的に12個のメトリクスを用いてロジスティック回帰モデルを構築した。10分割交差検証による評価の結果、OSSプロジェクトでは平均適合率37\%、平均再現率67\%、商用プロジェクトでは平均適合率32\%、平均再現率62\%を達成した。

さらに、レビュー労力削減効果を評価するため、変更された行の総数を用いて労力を計算し、総労力の20\%をレビューに使用できると仮定した。予測されたロジスティック確率に基づいて変更を優先順位付けし、労力考慮型モデル(EALR)を構築した。その結果、OSSプロジェクトでは平均28\%、商用プロジェクトでは平均43\%の欠陥を誘発する変更を、総労力の20\%で特定できることが示された。これは、Just-In-Time品質保証が最もリスクの高い変更に集中するための効果的な手法であることを示している。しかし、この手法には重要な単純化が含まれている。レビュー労力を「変更された行の総数」のみで計算しているため、変更の複雑さや影響範囲の広さが考慮されていない。実際のレビュー労力は、変更されたファイル数や変更の分散度(Entropy)といった要因にも依存すると考えられるが、Kameiらの手法ではこれらの要因が労力計算に反映されていない。

特徴量の重要性を分析した結果、OSSプロジェクトではNF、LA/LT、LT/NF、FIXが、商用プロジェクトではNF、LA/LT、LT/NF、NDEV、AGEが、欠陥リスクを増加させる最も重要な要因であることが明らかになった。ただし、これらのメトリクスにはいくつかの問題点が存在する。FIXは特定のコミットが欠陥修正であるかどうかを示すフラグであり、どのメソッドやクラスが欠陥を誘発しやすいかという具体的な情報を提供しない。AGEは商用プロジェクトでは有効だが、OSSプロジェクトではボランティアベースで開発が行われるため、開発頻度が不定期であり効果的ではない。NDEVも商用プロジェクトではチームでの共同作業が多いため有効だが、OSSプロジェクトでは1つの変更は基本的に1人の開発者が担当するため効果的ではない。NSはサブシステムの総数がプロジェクトによって異なるため、プロジェクト間での比較が困難である。Entropyは適切なモジュール分割を行った場合でも、変更が複数のファイルに分散していれば欠陥発生リスクが高いと判断されてしまう。
\section{コミットベースの欠陥予測}
コミット単位での欠陥予測は、バージョン管理システム(VCS: Version Control System)に記録された変更履歴を活用し、コードの変化と欠陥の関係を分析する手法である。VCSには、各コミットに対して、変更されたファイル、追加・削除された行、コミットメッセージ、作成者、作成日時などの情報が記録される。これらの情報を活用することで、開発者がコミットした直後に、その変更が欠陥を誘発するリスクを予測することが可能になる。この分野では、どのコミットが欠陥を混入させたかを特定するための手法と、コミット時点でのコードの特性を捉えるためのデータセット構築が重要な課題となっている。

Ferencら\cite{ferenc2020}は、GitHubからバグ情報を自動的に収集し、コミットごと、ソフトウェアの構成要素(メソッド、クラス)ごとのメトリクスを含む「BugHunter Dataset」を構築した。このデータセットの特徴は、従来の研究が特定のリリースバージョンにおける全てのソースコード要素の特性を収集していたのに対し、バグ混入コミットとバグ修正コミットという、バグの存在を特定できる最も狭い期間において、同じソースコード要素のバグあり状態と修正済み状態の両方を捉えることである。

データセット構築において、Ferencらは15のJavaプロジェクトを対象とし、欠陥が混入したコミットを特定するためにSZZアルゴリズムを用いた。この手法により、欠陥混入時と修正時のコードメトリクスを比較することが可能になる。

データセット構築の過程で、同じメトリクス値を持ちながら異なる数のバグが割り当てられているエントリーが存在するという問題に直面した。これは機械学習による欠陥予測の精度に悪影響を与える冗長性を生み出すため、Removal法、Subtract法、Single法、GCF法という4つのフィルタリング手法を比較検証した。その結果、より大きなエントリー数を持つクラスのエントリーを保持するRemoval法が最も高いF1スコアを達成した。また、クラス不均衡問題に対処するため、ランダムアンダーサンプリングを用いて、バグありクラスとバグなしクラスから同数のデータを取得した。

構築されたデータセットを用いた欠陥予測実験では、ナイーブベイズ、ロジスティック回帰、C4.5、ランダムフォレストなど11種類の機械学習アルゴリズムを比較評価した。ファイル、クラス、メソッドという3つのレベルで予測を実施した結果、メソッドレベルではランダムフォレストが最も高い性能を示し、平均F1スコアは約0.63であった。クラスレベルでは単純ロジスティック回帰が最も高く、平均F1スコアは約0.57、ファイルレベルではランダムツリーが最も高く、平均F1スコアは0.55であった。これらの結果は、より細かい粒度(メソッドレベル)での予測が、より高い精度を達成できることを示している。

一方、Hanらは、コードレビューを通じた欠陥検出の実態を調査した。OpenStackプロジェクトのNovaとNeutronを対象に、19,146件のレビューコメントを手動で分析し、将来の不具合を招く恐れのある不適切な構造がどの程度特定されるかを調査した。こうした不適切な構造とは、Martin Fowler が提唱した概念であり、コード内の潜在的な問題を示す特徴的なパターン(例: 肥大化したメソッド、重複コード、過度に複雑なクラス構造など)を指す。これらは必ずしも欠陥ではないが、保守性を低下させ、将来的に欠陥を誘発しやすくする要因となる。その結果、コードレビューでこうした不適切な構造が特定されることは一般的ではないことが明らかになった。約1,200件のレビューのうち、不適切な構造が明示的に指摘されたのは限られた数であった。さらに、レビューの大部分(70\%)では、特定された不適切な構造について説明が提供されておらず、レビュアーは単に問題を指摘するだけで、その理由を詳しく説明していなかった。

これらの研究から、コミットベースの欠陥予測における重要な課題が明らかになる。第一に、Ferencらの研究では各コミットの特性値を用いた予測が行われているが、コミット間の変化は対象外である。すなわち、あるコミット時点でのコードメトリクスは測定されているが、直前のコミットからどのように変化したかという時系列的な情報は活用されていない。第二に、Hanらの研究が示すように、コードレビューでは70\%のケースで欠陥の原因が明示されないため、レビューテキストのみから欠陥を予測することには限界がある。これらの課題は、コミット間の変化を考慮し、特性値同士の関連性を分析する新たなアプローチの必要性を示唆している。
\section{静的解析による品質改善}
静的解析は、ソフトウェアを実行することなくソースコードを検査し、潜在的な問題を検出する手法である。これは、プログラムを実際に動作させて挙動を観察する動的解析と対比される。静的解析ツールは、コードの構造、複雑度、スタイル、潜在的な欠陥パターンなどを自動的に分析し、JIS X 25010:2013で定義される保守性(特に解析性と修正性)の向上に貢献する。近年では、SonarQube\cite{sonarqube}やCheckstyle\cite{checkstyle}などの静的解析ツールが開発環境に統合され、コーディング中にリアルタイムでコードの問題を検出できるようになっている。

Romanoら\cite{romano2022}は、テスト駆動開発(TDD)において静的解析ツールを使用することがソフトウェア品質に与える影響を実証的に調査した。TDDは、レッドフェーズ(テストを書いて失敗させる)、グリーンフェーズ(テストに合格する最小限のコードを書く)、リファクタリングフェーズ(コードを改善する)という3つのフェーズを繰り返す開発手法である。理論的には、リファクタリングフェーズで\textbf{技術的負債}(将来の保守コストを増加させる設計上の妥協や実装上の問題)を返済し、品質を継続的に向上させることが期待される。

しかし、実際にはこのフェーズがしばしばスキップされることが観察されていた。リファクタリングがスキップされる主な理由は、以下の3点である。第一に、開発スケジュールの時間的プレッシャーにより、機能実装を優先し、コード品質の改善が後回しにされる。第二に、リファクタリングは即座にユーザーに見える価値を提供しないため、優先度が低く見なされる。第三に、技術的負債は可視化されにくく、どの程度の問題があるのかを開発者が認識しづらい。Romanoらは、静的解析ツールを用いることでこれらの問題に対処できるのではないかという仮説を立てた。

Romanoらは、静的解析ツールであるSonarLint(SonarQube for IDE)を使用する群と使用しない群に分けて実験を実施した。ソフトウェア品質を定量化するため、将来の不具合を招く恐れのある不適切な構造の数(Smell)、技術的負債の推定値(SqaleIndex)、循環的複雑度(WMC)、コードの分かりやすさ(CognCompl)、読みやすさ(BW)などの指標を測定した。

実験の結果、SonarLintを使用した群は、使用しない群と比較して、Smell、SqaleIndex、CognComplにおいて統計的に有意な改善が見られた。すなわち、静的解析ツールの使用は、不適切な構造の削減やコードの分かりやすさの向上に寄与することが示された。

静的解析ツールがこのような改善をもたらす理由は、以下の3点にある。第一に、問題の可視化により、開発者が技術的負債の存在に気づくことができる。漠然とした「コードが良くない」という感覚ではなく、具体的な指標として提示されることで、改善の動機が生まれる。第二に、コーディング中の即座のフィードバックにより、問題が小さいうちに修正できる。後から大規模なリファクタリングを行うよりも、その場で修正する方がコストが低い。第三に、ツールが提供する客観的基準により、コードレビューでの議論が建設的になり、チーム内で品質基準が共有される。これらの改善は、JIS X 25010:2013で定義される保守性の副特性、特に解析性(コードの問題点を識別する容易さ)と修正性(欠陥を取り込まずに変更を行う容易さ)の向上に貢献する。

一方で、実験後のアンケートでは、参加者はSonarLintを使用するとTDDがより困難になると認識しており、ツールの使用が追加の認知的負荷をもたらす可能性が示唆された。TDDが困難になる理由として、以下の要因が考えられる。第一に、静的解析ツールの警告に対応するための追加作業が発生し、本来のTDDサイクルが中断される。第二に、グリーンフェーズでは「テストに合格する最小限のコード」を書くことが求められるが、静的解析ツールが即座に品質問題を指摘するため、最小限のコードと品質の高いコードの間で葛藤が生じる。第三に、ツールの誤検出(偽陽性)に対処する必要があり、それが開発者の集中力を削ぐ。

この研究から明らかになる静的解析の課題は、文脈依存のしきい値の未検証である。SonarLintのようなツールは、事前に定義されたルールに基づいて不適切な構造を検出するが、プロジェクトの特性や開発フェーズに応じて適切なしきい値は異なる可能性がある。Romanoらの研究では、このような文脈依存性については検証されておらず、全てのプロジェクトに対して同一の設定が適用されている。

これが問題である理由は、以下の3点にある。第一に、偽陽性(実際には問題でないのに警告される)により、開発者が警告を無視するようになり、ツールへの信頼性が低下する。第二に、プロジェクトの初期段階では厳密な品質基準を適用することが非現実的な場合があり、過度な警告が開発速度を低下させる。第三に、本当に重要な問題が大量の警告に埋もれてしまい、見逃される可能性がある。そのため、静的解析ツールが検出する問題の中には、実際にはプロジェクトの文脈では問題とならない指摘が含まれている可能性がある。
\section{既存研究の問題点}
前述した関連研究は、ソフトウェア欠陥予測において重要な知見を提供してきたが、いくつかの問題点が残されている。

第一に、Kameiらの研究で提案された変更メトリクスには、プロジェクト特性や開発体制への依存性という問題がある。例えば、AGE(変更間隔)は商用プロジェクトでは有効だが、OSSプロジェクトではボランティアベースで開発が行われるため、開発頻度が不定期であり効果的ではない。NDEV(開発者数)も、商用プロジェクトではチームでの共同作業が多いため有効だが、OSSプロジェクトでは1つの変更は基本的に1人の開発者が担当するため効果的ではない。NS(サブシステム数)はプロジェクトによってサブシステムの総数が異なるため、プロジェクト間での比較が困難である。これらのメトリクスは、特定の開発体制やプロジェクト特性に依存しており、汎用性に欠ける。

第二に、最も重要な問題点として、既存研究ではコミット間の変化量が十分に考慮されていないことが挙げられる。Kameiらの研究では、コミット単位での特徴量(変更されたファイル数、追加・削除行数など)を用いているが、これらは単一のコミットの特性を表すものであり、前回のコミットからどのように変化したかという時系列的な情報は含まれていない。同様に、Ferencらの研究でも、各コミット時点でのコードメトリクスは測定されているが、コミット間の変化量は対象外である。コードの変化の過程には、バグ混入リスクを示す重要な情報が含まれている可能性があるが、この観点からの体系的な分析は不足している。例えば、循環的複雑度が短期間に大きく増加したメソッドや、頻繁にコード行数が変動するメソッドは、バグ混入リスクが高い可能性があるが、こうした変化のパターンは従来の研究では捉えられていない。

第三に、特性値同士の関連性分析が欠如している。既存研究では、個々のメトリクスが独立して評価されることが多く、複数のメトリクスがどのように相互作用してバグ混入リスクに影響を与えるかについての分析は限られている。例えば、変更されたファイル数が多く、かつ各ファイルの変更規模も大きい場合、これらの特徴量の組み合わせがバグ混入リスクにどのような影響を与えるかは明らかになっていない。Kameiらの研究では多重共線性への対処として高相関の因子を除去しているが、これは統計的な問題を回避するための処理であり、メトリクス間の本質的な関連性を分析するものではない。

第四に、Hanらの研究が示すように、コードレビューでは70\%のケースで欠陥の原因が明示されないため、レビューテキストのみから欠陥を予測することには限界がある。レビュアーは単に問題を指摘するだけで、その理由を詳しく説明していないことが多い。このため、自然言語処理によるレビューテキスト分析だけでは、バグ混入の原因を特定することは困難である。

これらの問題点は、コミット間の変化量を考慮し、異なる粒度でのメトリクスを統合し、メトリクス間の関連性を分析する新たなアプローチの必要性を示唆している。本研究では、これらの課題に対処するため、メソッド単位とコミット単位という異なる粒度での時系列変化を捉える手法を提案する。
\clearpage

\chapter{提案手法}
\section{研究のアプローチ}
本研究では、プログラムの変更(コミット)が「いつ起こるか分からない不規則なイベント」である点に着目し、欠陥予測を行う手法を採用する。

従来の時系列分析(株価や気温の予測など)は、一定の間隔で測られたデータを対象としている。例えば、1時間ごとの気温の変化から将来を予測する手法(ARIMAやLSTMなど)が一般的だが、これらはデータが規則正しく並んでいることを前提としている。

しかし、ソフトウェア開発におけるコミットは、決まった間隔で行われるものではない。開発者の作業ペースやプロジェクトの状況によって、頻繁に更新される時期もあれば、長期間動きがない時期もある。このように、いつ起こるか分からない出来事の記録は、点過程データと呼ばれる 。本研究では、この不規則なタイミングそのものに開発のリズムやリスクが隠れていると考え、分析の基盤とする。

具体的なアプローチとして、前回の作業から「何が、どれくらい変わったか」という変化に注目する。例えば、ある関数が書き換えられた際、単に現在の行数を測るのではなく、前回の状態から「何行増えたか」「構造がどれほど複雑になったか」といった時系列の変化を数値化する。これにより、短期間での急激な複雑化といった、静的な解析では見落とされがちな「不具合の予兆」を捉えることが可能になる。

また、本研究では「将来の数値を当てる(回帰)」のではなく、「その変更に欠陥が含まれるか、含まれないか」という「二値分類」の問題として予測を行う。これは、最終的な目的が「限られた確認作業(レビュー)の時間を、リスクの高い箇所に集中させること」にあるためである。

研究の全体像を図\ref{fig:research_approach_overview}に示す。不規則なコミットを点過程データとして整理し、その変化の内容から機械学習(ランダムフォレストなど)を用いて、欠陥が混入している確率を導き出す。

\begin{figure}[htbp]
  \centering
  \includegraphics[width=0.95\textwidth]{figures/research_approach_overview.pdf}
  \caption{研究アプローチ全体の概念図}
  \label{fig:research_approach_overview}
\end{figure}
\clearpage
\section{メトリクス設計}
コードの時系列変化を捉えるため、メソッド単位とコミット単位という異なる粒度での特徴量を設計する。これらは、それぞれミクロ的視点とマクロ的視点からコード変更の特性を捉える。
\subsection{メソッド単位の変更メトリクス}
メソッド単位では、個々のメソッドがどの程度変更されたかを捉えるため、以下の変化量を特徴量として用いる。

\begin{itemize}
    \item コード行数の変化量
    \begin{itemize}
        \item 直前のコミットと比較したときのコード行数の増減
    \end{itemize}
    \item トークン数の変化量
    \begin{itemize}
        \item 直前のコミットと比較したときのトークン数の増減
    \end{itemize}
    \item 循環的複雑度の変化量
    \begin{itemize}
        \item 直前のコミットと比較したときの循環的複雑度の増減
    \end{itemize}
\end{itemize}

これらのメトリクスで変化量を採用する理論的根拠は以下の通りである。第一に、メソッドは比較的スケールが均一である。個々のメソッドは通常、特定の機能を実装するために設計されるため、極端に大きなものや小さなものが混在することは少ない。第二に、小規模メソッドでの変化率の不安定性を回避できる。例えば、10行のメソッドに2行追加した場合、変化率は20\%となるが、100行のメソッドに2行追加した場合、変化率は2\%となる。このように、変化率を用いると小規模なメソッドでは値が極端に大きくなり、予測モデルにノイズをもたらす可能性がある。第三に、実際の作業負荷との対応である。「2行の追加」と「50行の追加」では、明らかに後者の方がレビュイーとレビュアーの双方にとって作業負荷が高く、欠陥発生率が高くなる傾向にある。変化量はこうした実際の作業負荷を直接的に反映する。
\clearpage
\subsection{コミット単位の変更メトリクス}
コミット単位では、一つのコミット全体での変更の影響範囲や性質を捉えるため、以下の変化率を特徴量として用いる。

\begin{itemize}
    \item 変更されたファイル数
    \begin{itemize}
        \item 1つのコミットで変更されたファイルの総数
    \end{itemize}
    \item 追加行数の割合
    \begin{itemize}
        \item 変更前の総行数に対する追加されたコード行数の割合
    \end{itemize}
    \item 削除行数の割合
    \begin{itemize}
        \item 変更前の総行数に対する削除されたコード行数の割合
    \end{itemize}
    \item 1ファイル当たりの平均行数
    \begin{itemize}
        \item 変更対象となったファイルの平均規模
    \end{itemize}
\end{itemize}

これらのメトリクスで変化率を採用する理論的根拠は以下の通りである。第一に、変更規模の多様性への対応である。コミットには、単一のファイルのみを変更する小規模なものから、数十のファイルを変更する大規模なものまで、多様な規模が存在する。変化率を用いることで、異なる規模のコミット間での比較が可能になる。第二に、ファイルに対する影響度を測定できる。追加行数の割合や削除行数の割合は、ファイルサイズに対してどの程度の変更が加えられたかを示し、大規模な機能追加やリファクタリングを示唆する指標となる。第三に、予測基準のバランスが向上する。変化量のみを用いると、大規模なコミットほどバグ修正コミットであると判定されやすくなるが、変化率を用いることで、小規模だが重要な欠陥の混入や修正を検出しやすくなる。

メソッド単位の変更メトリクスは、変更されたメソッド自体の特性を捉える。例えば、あるメソッドの循環的複雑度が5から15に増加した場合、そのメソッド内部の制御構造が複雑化したことを示す。一方、コミット単位の変更メトリクスは、同じコミットで変更された他のコードの特性も捉える。例えば、変更されたファイル数が多い場合、その変更が複数のモジュールに影響を与えていることを示す。

これらのメトリクスを組み合わせることで、コード変更をミクロとマクロの両面から多角的に評価できる。メソッド単位のメトリクスが局所的な変更の性質を捉え、コミット単位のメトリクスが変更の全体的な影響範囲を捉えることで、より精度の高い欠陥予測が可能になる。
\clearpage
\section{機械学習モデル}
本研究では、3.1節と3.2節で設計した時系列変化を考慮したメトリクスを用いて欠陥予測モデルを構築する。機械学習アルゴリズムとしてランダムフォレストを採用し、モデルの性能を評価する。

\paragraph{ランダムフォレストの採用}
ランダムフォレストを選定した理由は以下の通りである。

第一に、非線形な関係を捉える能力が高い。ソフトウェア欠陥予測において、特徴量と欠陥の有無の関係は複雑であり、単純な線形モデルでは表現できない相互作用が存在する。例えば、変更行数が少なくても変更ファイル数が多い場合は欠陥リスクが高まるといった、複数の特徴量の組み合わせによる非線形な効果を捉える必要がある。

第二に、アンサンブル学習による予測精度の安定性である。ランダムフォレストは複数の決定木の予測を集約するため、単一の決定木と比較して過学習に強く、安定した予測性能を示す。

第三に、モデルの解釈性である。欠陥予測モデルを実際の開発現場で活用するには、なぜそのコミットが欠陥を含むと予測されたのかを理解できる必要がある。ランダムフォレストは、特徴量重要度により各特徴量の予測への寄与度を定量化でき、Partial Dependence Plot(PDP)により特定の特徴量と予測結果の関係を可視化できる。これらの解釈性ツールは、予測モデルの振る舞いを理解し、開発者が予測結果を信頼できるかを判断する上で重要である。

\paragraph{評価指標}
モデルの性能を評価するため、本研究では主にF1スコアを用いる。F1スコアは、適合率と再現率の調和平均として定義される。適合率は欠陥を含むと予測したデータのうち実際に欠陥を含むデータの割合、再現率は実際に欠陥を含むデータのうち欠陥を含むと予測できた割合を表す。F1スコアは両者のバランスを評価するため、不均衡データにおいても適切な評価が可能である。F1スコアを採用する理由は、欠陥予測のために使用するデータが不均衡であるためである。一般的に、欠陥を含むメソッドやコミットの数は、欠陥を含まないメソッドやコミットの数よりも少ない。このような不均衡データでは、正解率のみでは適切な評価ができない。例えば、全てのデータを欠陥なしと予測しても、高い正確率が得られる可能性がある。

\paragraph{統計的仮説検定}
提案手法がベースライン手法と比較して統計的に有意な改善をもたらしているかを検証するため、マクネマー検定を実施する。マクネマー検定は、同じデータセットに対する2つの分類器の予測結果を比較するための検定手法である。この検定により、提案手法による予測性能の改善が偶然ではなく、統計的に有意であることを確認できる。

\paragraph{交差検証}
モデルの汎化性能を評価するため、10分割交差検証を用いる。データセットを10個のサブセットに分割し、そのうち9個を訓練データ、1個をテストデータとして使用する。この過程を10回繰り返し、各サブセットが1度だけテストデータとして使用されるようにする。これにより、特定のデータ分割に依存しない頑健な性能評価が可能となる。

具体的な実装と評価手順については第4章で述べる。
\clearpage
\section{レビュー労力の測定手法}
\subsection{従来手法の問題点}
従来のレビュー優先度付け手法では、欠陥予測モデルが出力したバグ混入確率に基づいてコミットを順位付けし、上位からレビューを行うアプローチが一般的であった。Kameiらの研究では、レビュー労力を「変更された行の総数」として計算し、総労力の一定割合(例えば20\%)を使用してレビューできるコミット数を評価している。

しかし、この手法には重要な単純化が含まれている。具体的には、レビュー労力を変更行数のみで計算しているため、変更の複雑さや影響範囲の広さが考慮されていない。実際のレビュー労力は、変更されたファイル数や変更の分散度といった要因にも依存する。例えば、10個のファイルに分散した100行の変更は、1個のファイルに集中した100行の変更よりもレビュー労力が大きいと考えられる。

さらに重要な問題として、従来手法では「レビューに必要な総労力」を全コミットのレビュー労力の和として設定している場合が多い。しかし、極端に大きなレビュー労力を要するコミット(例えば、数千行の変更を含むコミット)が存在する場合、この設定は現実的ではない。実際の開発現場では、レビューに使える労力には上限があり、その上限を超える巨大なコミットは分割されるか、別の品質保証プロセスが適用される。
\clearpage
\subsection{ナップサック問題への定式化}
本研究では、レビュー対象コミットの選択をナップサック問題として定式化する。ナップサック問題は、容量制約のある袋(ナップサック)に、価値と重さを持つ複数のアイテムを入れるとき、総重量が容量を超えないように、総価値を最大化するアイテムの組み合わせを求める最適化問題である。

レビュー対象の選択問題をナップサック問題に対応付けると、以下のようになる。

\begin{itemize}
    \item アイテム: レビュー待ちの各コミット $i$($i = 1, 2, ..., N$)
    \item アイテム数 $N$: レビュー待ちの全てのコミット数
    \item アイテムの重さ $W_i$: コミット $i$ のレビューに必要な労力
    \item アイテムの価値 $V_i$: コミット $i$ のレビューによるバグ発見期待値(モデルが予測したバグ混入確率 $\hat{y}_i$)
    \item ナップサックの容量 $C_{total}$: レビューに使える総労力
    \item 目的: レビュー労力の合計が $C_{total}$ を超えないように、レビューするコミットの組み合わせを選び、バグ発見期待値の合計 $V_{total}$ を最大化
\end{itemize}

数式で表現すると、以下の最適化問題となる。

\begin{align}
\text{maximize} \quad & \sum_{i=1}^{N} V_i x_i \\
\text{subject to} \quad & \sum_{i=1}^{N} W_i x_i \leq C_{total} \\
& x_i \in \{0, 1\}
\end{align}

ここで、$x_i$ は二値変数であり、コミット $i$ をレビューする場合は $x_i = 1$、レビューしない場合は $x_i = 0$ となる。
\clearpage
\subsection{レビュー労力の計算式}
各コミット $i$ のレビュー労力 $W_i$ を計算するため、以下の要素を考慮する。

\begin{itemize}
    \item コードチャーン $C_i$: コミット $i$ における追加行数と削除行数の合計
\end{itemize}

\[
C_i = LA_i + LD_i
\]

ここで、$LA_i$ は追加されたコード行数、$LD_i$ は削除されたコード行数である。
\begin{itemize}
    \item 変更ファイル数 $N_i$: コミット $i$ で変更されたファイルの総数
    \item Entropy $H_i$: コミット $i$ における変更の分散度を表す指標
\end{itemize}

\[
H_i = -\sum_{k=1}^{n_i} p_k \log_2 p_k
\]

ここで、$n_i$ はコミット $i$ で変更されたファイル数、$p_k$ はファイル $k$ が変更全体に占める割合である。

\[
p_k = \frac{\text{file}_k \text{の変更行数}}{\text{全変更行数}}
\]

Entropyを正規化するため、以下の式を用いる。

\[
H_i^{\text{norm}} = \frac{H_i}{\log_2 n_i}
\]

この正規化により、Entropyは0から1の範囲に収まり、ファイル数が異なるコミット間での比較が可能になる。

これらの要素を組み合わせて、ベース労力 $E_{\text{raw}, i}$ を以下のように計算する。

\[
E_{\text{raw}, i} = C_i \times N_i^{H_i^{\text{norm}}}
\]

この式は、変更の規模(コードチャーン)と、変更の複雑さ(変更ファイル数とEntropy)の両方を考慮している。変更ファイル数が多いほど、またEntropyが高い(変更が複数のファイルに分散している)ほど、レビュー労力が増加する。

次に、極端に大きな値の影響を緩和するため、対数変換を適用する。

\[
W_i = E_{\text{adj}, i} = \ln(E_{\text{raw}, i} + 1)
\]

数千行の変更を含む巨大なコミットが存在する場合、ベース労力が極端に大きくなり、他のコミットの労力が相対的に無視できるほど小さくなってしまう。対数変換により、この影響を緩和し、中小規模のコミットも適切に評価できる。
\clearpage
\subsection{貪欲法}
ナップサック問題を解くための代表的なアルゴリズムとして、動的計画法と貪欲法がある。

動的計画法は最適解を求めることができるが、計算量が $O(NC_{total})$ であり、容量 $C_{total}$ が大きいほど計算時間が増加する。レビュー労力の場合、$C_{total}$ は全コミットの補正済み労力の和(あるいはその一部)となるため、非常に大きな値になる可能性がある。実際に動的計画法を実装して実験を試みたところ、メモリ容量が不足し、プログラムが終了してしまった。

貪欲法は近似解を求める手法であり、計算量は $O(N \log N)$(ソートが必要な場合)である。貪欲法では、アイテムを「価値と重さの比(密度)」の降順にソートし、密度が高いものから順にナップサックに入れていく。最適解は保証されないが、計算時間が容量に依存せず、実用的な時間で解を得られる。実験では、貪欲法により数秒で処理が完了した。

本研究では、計算時間とメモリ効率を考慮し、貪欲法を採用する。

各コミット $i$ の密度 $D_i$ を以下のように定義する。

\[
D_i = \frac{V_i}{W_i} = \frac{\hat{y}_i}{E_{\text{adj}, i}}
\]

ここで、$\hat{y}_i$ はモデルが予測したコミット $i$ のバグ混入確率、$E_{\text{adj}, i}$ はコミット $i$ の補正済みレビュー労力である。

貪欲法のアルゴリズムは以下の通りである。

\begin{enumerate}
    \item 全てのコミットについて密度 $D_i$ を計算する
    \item 密度の降順にコミットをソートする
    \item 累積労力 $W_{\text{累積}} = 0$ とする
    \item ソートされた順にコミットを選択し、以下を実行する:
    \begin{itemize}
        \item $W_{\text{累積}} + W_i \leq C_{total}$ であれば、コミット $i$ をレビュー対象に追加し、$W_{\text{累積}} \leftarrow W_{\text{累積}} + W_i$ とする
        \item そうでなければ、コミット $i$ をスキップする
    \end{itemize}
    \item 累積労力が容量を超えるまで、または全てのコミットを検討するまで繰り返す
\end{enumerate}

この貪欲法により、限られたレビュー労力の中で、バグ発見期待値を効率的に最大化できる。
\clearpage
\subsection{レビューに使える総労力}
ナップサックの容量 $C_{total}$(レビューに使える総労力)の設定には注意が必要である。

素朴なアプローチとして、全コミットの労力の和を $C_{total}$ とする方法が考えられる。しかし、極端に大きな労力を要するコミット(例えば、数千行の変更を含むコミット)が存在する場合、この設定は不適切である。実際の開発現場では、レビューに使える労力には上限があり、巨大なコミットは分割されるか、別の品質保証プロセスが適用される。

本研究では、以下の手順で $C_{total}$ を設定する。

\begin{enumerate}
    \item 全コミットをレビュー労力 $W_i$ の昇順にソートする
    \item 上位80\%のコミット(レビュー労力が小さい方から80\%)を選択する
    \item 選択されたコミットのレビュー労力の和を $C_{total}$ とする
\end{enumerate}

\[
C_{total} = \sum_{i \in S_{80\%}} W_i
\]

ここで、$S_{80\%}$ は労力の小さい順に並べた上位80\%のコミットの集合である。

この設定により、極端に大きな労力を要するコミット(上位20\%)の影響を除外し、より現実的なレビュー労力の制約をモデル化できる。80\%というしきい値は、実験的に決定されたものであり、プロジェクトの特性に応じて調整可能である。
\clearpage
\subsection{Cost-Benefit Curveによる評価}
提案手法のレビュー労力削減効果を評価するため、Cost-Benefit Curveを用いる。Cost-Benefit Curveは、横軸に投入したレビュー労力、縦軸に発見したバグ数をプロットしたグラフである。

提案手法(労力考慮型モデル)では、密度の高い順にコミットを選択してレビューする。各コミットをレビューするごとに、累積レビュー労力と累積発見バグ数を記録し、曲線を描く。

比較対象として、ベースラインモデル(新たな特徴量を追加する前のデータセットで学習したモデル)を用いる。ベースラインモデルでも同様に密度に基づく貪欲法を適用し、レビュー対象を選択する。すなわち、ベースラインモデルが予測したバグ混入確率 $\hat{y}_i^{\text{baseline}}$ を用いて密度を計算し、密度の高い順にコミットを選択してレビューする。

この比較により、労力を考慮したコードレビューモデルという統一的な評価枠組みの中で、特徴量エンジニアリング(メソッド単位とコミット単位の変更メトリクスの追加)の効果を測定できる。提案手法がベースライン手法よりも左上に位置する曲線を描く場合、同じレビュー労力でより多くのバグを発見できることを意味し、提案した特徴量の有効性が実証される。
\clearpage

\chapter{実験}
\section{データセット}
本研究では、Ferencらが構築したBugHunter Datasetを基盤として使用する。このデータセットは、GitHubでホストされている15のJavaプロジェクトから自動的に収集されたバグ情報と、各コミットにおけるソースコード要素(ファイル、クラス、メソッド)のメトリクスを含んでいる。


BugHunterデータセットの特徴は、従来の研究が特定のリリースバージョンにおける全てのソースコード要素を収集していたのに対し、バグ混入コミットとバグ修正コミットという、バグの存在を特定できる最も狭い期間におけるコードの状態を捉えている点である。これにより、バグが混入した時点と修正された時点でのコードメトリクスの変化を分析できる。
\subsection{正解ラベルの定義}
本研究では、メソッドレベルでの欠陥予測を行うため、各メソッドに含まれるバグの数を二値化して正解ラベルとする。具体的には、バグの数が0のメソッドを「バグなし」クラス、バグの数が1以上のメソッドを「バグあり」クラスに分類する。この二値分類により、ランダムフォレストのような分類アルゴリズムを適用できる形式にデータを整形する。
\clearpage
\subsection{データの前処理}
第一に、値が全て同じであるカラムを削除する。これらのカラムは予測に寄与しないため、モデルの複雑性を低減するために除外する。

第二に、メソッドの識別子をベクトルに変換する。メソッド名やクラス名といった識別子は、そのままでは機械学習モデルに入力できないため、トークン分割などの手法を用いて数値ベクトルに変換する。
\subsection{使用するレコード数の上限}
交差検証時の評価指標の安定化を図るため、各プロジェクトのデータセットから抽出するデータポイント数の上限を5,000件とする。当初は3,000件を上限としていたが、この設定では交差検証のフェーズごとにF1スコアが大きく変動し、最悪と最良のフェーズを比較すると0.1前後の誤差が生じていた。データポイント数を5,000件に増やすことで、この変動を抑制し、より安定した性能評価が可能になった。ただし、この変動は完全には解消されておらず、データの特性に起因する本質的な課題として残っている。

各プロジェクトのデータセットから、利用可能なデータが5,000件以下の場合は全データを使用し、5,000件を超える場合は最初の5,000件を使用する。データセット内のレコードは必ずしも時系列順に並んでいるわけではないが、同一メソッドの複数のバージョン(バグ混入時と修正時)が含まれており、これらの情報を用いて変化量を計算することが可能である。
\clearpage
\section{対象プロジェクト}
有意性検定で帰無仮説が棄却されない問題を回避するため、BugHunterデータセットに含まれる15のプロジェクトの中から、データセットのレコード数が多いプロジェクトを優先的に選定する。


データセットの構築にはSZZアルゴリズムが用いられているため、解決済みのバグレポートの数とデータセットのレコード数の大きさには相関関係がある。そのため、解決済みのバグレポート数を基準として、データセットの相対的な大きさを推測できる。この基準に基づき、解決済みのバグレポート数が多い上位5つのプロジェクトを対象として選定した。

\begin{table}[h]
\centering
\caption{各プロジェクトのバグレポート数}
\begin{tabular}{|l|r|}
\hline
プロジェクト & バグレポート数 \\
\hline
Elasticsearch & 4,287 \\
Hazelcast & 3,762 \\
Netty & 2,207 \\
OrientDB & 1,272 \\
Neo4j & 1,152 \\
\hline
\end{tabular}
\end{table}

Elasticsearchは、分散検索・分析エンジンであり、大規模なログデータやテキストデータの検索に広く使用されている。
Hazelcastは、インメモリデータグリッドを提供する分散コンピューティングプラットフォームである。
Nettyは、高性能な非同期イベント駆動型のネットワークアプリケーションフレームワークである。
OrientDBは、マルチモデルデータベースであり、グラフデータベースとドキュメントデータベースの機能を併せ持つ。
Neo4jは、グラフデータベースの代表的な実装の一つであり、ソーシャルネットワーク分析や推薦システムなど、関係性を重視するアプリケーションで広く使用されている。

これらのプロジェクトは、いずれもJavaで記述されており、GitHubでホストされている活発なOSSプロジェクトである。ドメインも検索エンジン、分散システム、ネットワークフレームワーク、データベースと多岐にわたり、異なる開発特性を持つ。このような多様性により、提案手法が特定のプロジェクトタイプに依存せず、広範なソフトウェアシステムに適用可能であることを検証できる。
\clearpage
\section{特徴量生成}
コードの時系列変化を捉えるため、3.2節で述べたメソッド単位の変更メトリクスとコミット単位の変更メトリクスを生成する。
\subsection{変更メトリクスの計算}
メソッド単位の変更メトリクスとして、コード行数の変化量、トークン数の変化量、循環的複雑度の変化量を算出する。これらは、同一メソッドのバグ混入時とその1つ前のコミットの値の差分として計算される。

コミット単位の変更メトリクスとして、変更されたファイル数(NF)、追加行数の割合(LA/LT)、削除行数の割合(LD/LT)、1ファイル当たりの平均行数(LT/NF)を生成する。これらは、BugHunterデータセットに含まれるコミット情報から計算により導出される。
\subsection{メソッドの操作タイプのラベル付与}
各メソッドに対して、そのメソッドが追加、変更、削除のいずれの操作を受けたかを示すラベルを付与する。これらの操作タイプは、バグ混入コミットあるいはバグ修正コミットと、その直前のコミットを比較することで判定される。

\begin{itemize}
    \item 追加(Add)
    \begin{itemize}
        \item 直前のコミットには存在せず、当該コミットで新たに追加されたメソッド
    \end{itemize}
    \item 変更(Modify)
    \begin{itemize}
        \item 直前のコミットと当該コミットの両方に存在し、内容が変更されたメソッド
    \end{itemize}
    \item 削除(Delete)
    \begin{itemize}
        \item 直前のコミットには存在したが、当該コミットで削除されたメソッド
    \end{itemize}
\end{itemize}

これらのラベルは、カテゴリカル変数として扱い、One-Hotエンコーディングを用いて数値ベクトルに変換する。
\clearpage
\subsection{メソッド識別子の処理}
メソッドの完全修飾名(例: org.elasticsearch.index.fielddata.plain.GeoPointDoubleArrayAtomicFieldData\$Empty.<init>()V)は、そのままでは機械学習モデルに入力できないため、トークン分割を行う。

具体的には、以下の手順で処理する。第一に、正規表現を用いてメソッドシグネチャからパッケージ名、クラス名、メソッド名を抽出する。第二に、パッケージ名を.で分割し、各部分をトークンとする。第三に、クラス名を\$で分割し、各部分をキャメルケース分割する。例えば、"GeoPointDoubleArrayAtomicFieldData"は["Geo", "Point", "Double", "Array", "Atomic", "Field", "Data"]に分割される。第四に、メソッド名をスネークケース(\_や-)で分割した後、さらにキャメルケース分割する。ただし、<init>や<clinit>などの特殊なメソッド名は"constructor"というトークンに変換する。第五に、全てのトークンを小文字に変換する。

さらに、"java"、"util"、"get"、"set"などのJava言語における一般的な単語や、最小トークン長(3文字)未満のトークンをストップワードとして除外する。これにより、メソッドやクラスの本質的な意味を表す有用なトークンのみが抽出される。
抽出されたトークンは、語彙辞書を用いて数値インデックスに変換され、さらにベクトル表現に変換される。この処理により、メソッドが属するパッケージやクラスの名前、メソッド名自体が持つ意味的な情報を特徴量として活用できる。
\clearpage
\section{実験手順}
\subsection{モデルの段階的評価}
提案手法の各要素がどの程度予測性能に寄与するかを明らかにするため、3段階の比較評価を行う。

ステップ1: はじめに、変更メトリクスを追加する前のモデルを構築する。このモデルは、BugHunter Datasetに元々含まれているコードメトリクスのみを特徴量として使用する。このモデルをベースラインとし、後続のモデルとの比較基準とする。

ステップ2: ベースラインモデルに対して、メソッド単位の変更メトリクスを追加したモデルを構築する。これにより、メソッド単位の時系列変化を考慮することの効果を評価する。

ステップ3: ステップ2のモデルに対して、さらにコミット単位の変更メトリクスを追加したモデルを構築する。これにより、ミクロ的視点とマクロ的視点を統合することの効果を評価する。
\subsection{モデルの訓練と評価}
各モデルに対して、以下の手順で訓練と評価を行う。

データ分割: 各プロジェクトのデータセットを訓練データとテストデータに分割する。訓練データはモデルの学習に使用し、テストデータは最終的な性能評価に使用する。

10分割交差検証: 訓練データに対して10分割交差検証を実施し、モデルの性能の誤差を測定する。データセットを10個のサブセットに分割し、そのうち9個を訓練データ、1個を検証データとして使用する。この過程を10回繰り返し、各サブセットが1度だけ検証データとして使用されるようにする。各フェーズでF1スコア、適合率、再現率を計算し、性能のばらつきを評価する。

最終評価: 訓練されたモデルをテストデータに適用し、F1スコア、適合率、再現率、正解率、ROC-AUCを算出する。この値を各モデルの最終的な性能指標とする。

統計的有意性の検証: 提案手法(ステップ3)とベースライン(ステップ1)の予測結果に統計的に有意な差があるかを検証するため、マクネマー検定を実施する。マクネマー検定は、同じテストデータに対する2つの分類器の予測結果を比較するための統計的検定手法であり、p値が0.05未満の場合、有意な差があると判断する。

レビュー労力削減効果の評価: 提案手法によるレビュー労力削減効果を評価するため、Cost-Benefit Curveを作成する。評価手順は以下の通りである。

\begin{enumerate}
    \item 3.4節で述べた手法に従い、各コミットのレビュー労力 $W_i$ を計算する。コードチャーン、変更ファイル数、Entropyから計算したベース労力に対数変換を適用し、補正済み労力を得る。
    \item 全コミットをレビュー労力の昇順にソートし、上位80\%のコミットの労力の和を、レビューに使える総労力 $C_{total}$ として設定する。
    \item 各モデル(ベースライン、ステップ2、ステップ3)について、予測したバグ混入確率 $\hat{y}_i$ と補正済み労力 $E_{\text{adj}, i}$ から密度 $D_i = \frac{\hat{y}_i}{E_{\text{adj}, i}}$ を計算する。
    \item 密度の降順にコミットをソートし、貪欲法により累積労力が $C_{total}$ を超えない範囲でレビュー対象を選択する。各コミットをレビューするごとに、累積レビュー労力と累積発見バグ数を記録する
    \item 横軸を累積レビュー労力、縦軸を累積発見バグ数としてCost-Benefit Curveを描画する。ベースラインモデルと提案手法(ステップ2、ステップ3)のCurveを同一グラフ上に描き、視覚的に比較する
\end{enumerate}

\clearpage
\subsection{モデル解釈性の分析}
構築したモデルがどのような判断基準で予測を行っているかを理解するため、以下の分析を実施する。

特徴量の寄与度の算出: ランダムフォレストが提供するFeature Importanceを計算し、どの特徴量が予測に最も寄与しているかを明らかにする。Feature Importanceは、各特徴量が決定木の分岐においてどの程度情報利得をもたらしたかを示す指標である。この分析により、メソッド単位の変更メトリクスとコミット単位の変更メトリクスのうち、どの特徴量がバグ予測に重要であるかを定量的に評価できる。

Partial Dependence Plot(PDP)の生成: 各特徴量と陽性予測確率の関係を可視化するため、PDPを生成する。PDPは、特定の特徴量の値を変化させたときに予測確率がどのように変化するかを示すグラフであり、特徴量と予測結果の関係を直感的に理解できる。

決定木の可視化: ランダムフォレストを構成する決定木の一つを可視化し、どのような分類条件でバグの有無を判断しているかを確認する。決定木の可視化により、モデルの判断基準と確信度を具体的に把握できる。

これらの実験手順により、提案手法の有効性を多角的に評価し、時系列変化を考慮することがソフトウェア欠陥予測にどのような効果をもたらすかを明らかにする。
\clearpage

\chapter{評価}
\section{予測性能}
本節では、提案手法の予測性能を評価した結果を報告する。評価は、ベースライン(BugHunter Datasetに前処理を適用したもの)、ステップ2(メソッド単位の変化量メトリクスを追加)、ステップ3(コミット単位の変更率メトリクスをさらに追加)の3段階で実施した。各段階でランダムフォレストを用いてモデルを構築し、10分割交差検証による性能評価とテストデータによる最終評価を行った。
\subsection{テストデータによる評価}
表\ref{tab:evaluation}に、5つのプロジェクトにおける各段階のF1スコア、適合率、再現率、正解率、ROC-AUCを示す。ベースラインと比較して、ステップ2では全てのプロジェクトでF1スコアが向上し、ステップ3ではさらなる性能向上が確認された。

\begin{table}[ht]
\centering
\caption{テストデータによる評価}
\label{tab:evaluation}
\begin{tabular}{|l|l|r|r|r|r|r|}
\hline
プロジェクト & モデル & F1スコア & 適合率 & 再現率 & 正解率 & ROC-AUC \\
\hline
Elasticsearch & ベースライン & 0.5754 & 0.4576 & 0.7750 & 0.6340 & 0.7746 \\
 & ステップ2 & 0.7075 & 0.5983 & 0.8656 & 0.7710 & 0.8802 \\
 & ステップ3 & 0.7669 & 0.6770 & 0.8844 & 0.8280 & 0.9254 \\
\hline
Hazelcast & ベースライン & 0.6783 & 0.5824 & 0.8119 & 0.7420 & 0.8747 \\
 & ステップ2 & 0.7332 & 0.6549 & 0.8328 & 0.7970 & 0.8998 \\
 & ステップ3 & 0.7901 & 0.7014 & 0.9045 & 0.8390 & 0.9316 \\
\hline
Neo4j & ベースライン & 0.4782 & 0.3664 & 0.6883 & 0.6290 & 0.7130 \\
 & ステップ2 & 0.6158 & 0.4882 & 0.8340 & 0.7430 & 0.8296 \\
 & ステップ3 & 0.7420 & 0.6358 & 0.8907 & 0.8470 & 0.9382 \\
\hline
Netty & ベースライン & 0.4548 & 0.3279 & 0.7419 & 0.6140 & 0.7054 \\
 & ステップ2 & 0.6607 & 0.5394 & 0.8525 & 0.8100 & 0.8815 \\
 & ステップ3 & 0.7466 & 0.6433 & 0.8894 & 0.8690 & 0.9337 \\
\hline
OrientDB & ベースライン & 0.4833 & 0.3602 & 0.7342 & 0.6280 & 0.7376 \\
 & ステップ2 & 0.5241 & 0.3945 & 0.7806 & 0.6640 & 0.7647 \\
 & ステップ3 & 0.7012 & 0.5801 & 0.8861 & 0.8210 & 0.9141 \\
\hline
\end{tabular}
\end{table}

Elasticsearchでは、F1スコアがベースラインの0.5754からステップ2で0.7075、ステップ3で0.7669へと段階的に向上した。ベースラインと比較したステップ3の改善幅は0.1915ポイントである。適合率は0.4576から0.6770へ0.2194ポイント向上し、再現率は0.7750から0.8844へ0.1094ポイント改善した。正解率も0.6340から0.8280へ0.1940ポイント向上した。ROC-AUCは0.7746から0.9254へ0.1508ポイント向上し、モデルの識別能力が大きく改善されたことが示された。

Hazelcastでは、F1スコアがベースラインの0.6783からステップ2で0.7332、ステップ3で0.7901へと向上した。ベースラインと比較したステップ3の改善幅は0.1118ポイントである。適合率は0.5824から0.7014へ0.1190ポイント向上し、再現率は0.8119から0.9045へ0.0926ポイント改善した。正解率は0.7420から0.8390へ0.0970ポイント向上し、ROC-AUCは0.8747から0.9316へ0.0569ポイント改善された。

Neo4jでは、F1スコアがベースラインの0.4782からステップ2で0.6158、ステップ3で0.7420へと大幅に向上した。ベースラインと比較したステップ3の改善幅は0.2638ポイントであり、5つのプロジェクトの中で最も大きな改善幅である。適合率は0.3664から0.6358へ0.2694ポイント向上し、再現率は0.6883から0.8907へ0.2024ポイント改善した。正解率は0.6290から0.8470へ0.2180ポイント向上し、ROC-AUCは0.7130から0.9382へ0.2252ポイント大幅に改善された。

Nettyでは、F1スコアがベースラインの0.4548からステップ2で0.6607、ステップ3で0.7466へと向上した。ベースラインと比較したステップ3の改善幅は0.2918ポイントであり、5つのプロジェクトの中で最も大きな改善幅である。適合率は0.3279から0.6433へ0.3154ポイント向上し、再現率は0.7419から0.8894へ0.1475ポイント改善した。正解率は0.6140から0.8690へ0.2550ポイント向上し、ROC-AUCは0.7054から0.9337へ0.2283ポイント大幅に改善された。

OrientDBでは、F1スコアがベースラインの0.4833からステップ2で0.5241、ステップ3で0.7012へと向上した。ベースラインと比較したステップ3の改善幅は0.2179ポイントである。適合率は0.3602から0.5801へ0.2199ポイント向上し、再現率は0.7342から0.8861へ0.1519ポイント改善した。正解率は0.6280から0.8210へ0.1930ポイント向上し、ROC-AUCは0.7376から0.9141へ0.1765ポイント改善された。
\clearpage
\subsection{統計的有意性の検証}
ベースラインとステップ3の予測結果に統計的に有意な差があるかを検証するため、マクネマー検定を実施した。マクネマー検定は、同じテストデータに対する2つの分類器の予測結果を比較するための統計的検定手法である。有意水準は0.01とし、p値が0.01未満の場合に有意な差があると判断する。

検定の結果、全てのプロジェクトにおいてp値は0.000000となり、有意水準0.01を大きく下回った。これは、ベースラインとステップ3の予測結果の差が偶然ではなく、統計的に有意であることを示している。すなわち、メソッド・コミット単位の変更メトリクスを追加することで、予測性能が統計的に有意に向上したことが実証された。
\subsection{交差検証による性能変動の評価}
10分割交差検証の結果を表\ref{tab:crossvalidation}に示す。各プロジェクトにおいて、ベースライン、ステップ2、ステップ3の順にF1スコアとROC-AUCの平均値と標準偏差を報告する。

\begin{table}[ht]
\centering
\caption{10分割交差検証の結果(平均値 ± 標準偏差)}
\label{tab:crossvalidation}
\begin{tabular}{|l|l|r|r|}
\hline
プロジェクト & モデル & F1スコア & ROC-AUC \\
\hline
Elasticsearch & ベースライン & 0.6016 ± 0.0152 & 0.7988 ± 0.0179 \\
 & ステップ2 & 0.7165 ± 0.0242 & 0.8918 ± 0.0136 \\
 & ステップ3 & 0.7910 ± 0.0242 & 0.9313 ± 0.0161 \\
\hline
Hazelcast & ベースライン & 0.6934 ± 0.0234 & 0.8919 ± 0.0160 \\
 & ステップ2 & 0.7489 ± 0.0250 & 0.9215 ± 0.0145 \\
 & ステップ3 & 0.7833 ± 0.0185 & 0.9314 ± 0.0111 \\
\hline
Neo4j & ベースライン & 0.4472 ± 0.0244 & 0.6775 ± 0.0186 \\
 & ステップ2 & 0.5765 ± 0.0232 & 0.8101 ± 0.0136 \\
 & ステップ3 & 0.7393 ± 0.0329 & 0.9360 ± 0.0111 \\
\hline
Netty & ベースライン & 0.4257 ± 0.0280 & 0.6864 ± 0.0312 \\
 & ステップ2 & 0.6212 ± 0.0331 & 0.8707 ± 0.0290 \\
 & ステップ3 & 0.7076 ± 0.0275 & 0.9252 ± 0.0160 \\
\hline
OrientDB & ベースライン & 0.4734 ± 0.0392 & 0.7246 ± 0.0364 \\
 & ステップ2 & 0.5086 ± 0.0402 & 0.7535 ± 0.0326 \\
 & ステップ3 & 0.6636 ± 0.0251 & 0.8985 ± 0.0192 \\
\hline
\end{tabular}
\end{table}

Elasticsearchでは、ベースラインのF1スコアが0.6016±0.0152、ステップ2で0.7165±0.0242、ステップ3で0.7910±0.0242となった。ROC-AUCはベースラインの0.7988±0.0179からステップ2で0.8918±0.0136、ステップ3で0.9313±0.0161へと向上した。標準偏差は全ての段階で0.02前後と比較的小さく、性能が安定していることが確認された。

Hazelcastでは、F1スコアがベースラインの0.6934±0.0234からステップ2で0.7489±0.0250、ステップ3で0.7833±0.0185へと向上した。ROC-AUCはベースラインの0.8919±0.0160からステップ2で0.9215±0.0145、ステップ3で0.9314±0.0111へと改善された。特にステップ3では標準偏差が0.0111と最も小さく、性能の安定性が向上したことが示された。

Neo4jでは、F1スコアがベースラインの0.4472±0.0244からステップ2で0.5765±0.0232、ステップ3で0.7393±0.0329へと大幅に向上した。ROC-AUCはベースラインの0.6775±0.0186からステップ2で0.8101±0.0136、ステップ3で0.9360±0.0111へと改善された。ステップ3では標準偏差が0.0329とやや大きいが、これはプロジェクトの特性やデータの偏りに起因する可能性がある。

Nettyでは、F1スコアがベースラインの0.4257±0.0280からステップ2で0.6212±0.0331、ステップ3で0.7076±0.0275へと向上した。ROC-AUCはベースラインの0.6864±0.0312からステップ2で0.8707±0.0290、ステップ3で0.9252±0.0160へと改善された。ステップ3では標準偏差が0.0160と小さくなり、性能の安定性が向上した。

OrientDBでは、F1スコアがベースラインの0.4734±0.0392からステップ2で0.5086±0.0402、ステップ3で0.6636±0.0251へと向上した。ROC-AUCはベースラインの0.7246±0.0364からステップ2で0.7535±0.0326、ステップ3で0.8985±0.0192へと改善された。ステップ3では標準偏差が0.0251と小さくなり、性能の安定性が向上した。

全てのプロジェクトにおいて、交差検証の標準偏差は0.04以下と比較的小さく、提案手法が異なるデータ分割に対しても安定した性能を発揮することが確認された。特に、ステップ3ではROC-AUCの標準偏差が全プロジェクトで0.02以下となり、モデルの識別能力が安定していることが示された。
\clearpage
\section{特徴量の寄与度}
\input{5_2_feature_contribution.tex}
\clearpage
\section{特徴量の分布}
提案手法で導入した特徴量がどのような値の範囲と分布を持つかを把握するため、Feature Importance上位10個の特徴量について統計情報を分析した。各特徴量の最小値、最大値、中央値を算出し、特徴量の分布特性を明らかにする。表\ref{tab:feature_statistics}に、5つのプロジェクトにおける主要な特徴量の統計情報を示す。

\begin{table}[ht]
\centering
\caption{Feature Importance上位10個の特徴量の統計情報(抜粋)}
\label{tab:feature_statistics}
\small
\begin{tabular}{|l|l|r|r|r|}
\hline
プロジェクト & 特徴量 & 最小値 & 最大値 & 中央値 \\
\hline
\textbf{Elasticsearch} & lines\_added & 0.00 & 203,237.00 & 104.00 \\
 & entropy & 0.00 & 1.00 & 0.73 \\
 & num\_files & 0.00 & 3,651.00 & 5.00 \\
 & lines\_deleted & 0.00 & 325,455.00 & 33.00 \\
 & tokens\_change & -395.00 & 357.00 & 0.00 \\
 & length\_change & -99.00 & 35.00 & 0.00 \\
 & operation\_type\_added & 0.00 & 1.00 & 0.00 \\
\hline
\textbf{Hazelcast} & lines\_added & 0.00 & 13,739.00 & 139.00 \\
 & num\_files & 0.00 & 549.00 & 5.00 \\
 & lines\_deleted & 0.00 & 8,368.00 & 32.00 \\
 & entropy & 0.00 & 1.00 & 0.73 \\
 & tokens\_change & -209.00 & 158.00 & 0.00 \\
\hline
\textbf{Neo4j} & lines\_added & 0.00 & 3,893.00 & 203.00 \\
 & lines\_deleted & 0.00 & 149,409.00 & 53.00 \\
 & entropy & 0.00 & 1.00 & 0.79 \\
 & num\_files & 0.00 & 1,009.00 & 8.00 \\
 & tokens\_change & -130.00 & 180.00 & 0.00 \\
 & length\_change & -37.00 & 60.00 & 0.00 \\
 & operation\_type\_added & 0.00 & 1.00 & 0.00 \\
\hline
\textbf{Netty} & tokens\_change & -529.00 & 190.00 & 0.00 \\
 & lines\_deleted & 0.00 & 4,202.00 & 34.00 \\
 & lines\_added & 0.00 & 3,327.00 & 119.00 \\
 & length\_change & -83.00 & 46.00 & 0.00 \\
 & num\_files & 1.00 & 780.00 & 4.00 \\
 & operation\_type\_added & 0.00 & 1.00 & 0.00 \\
 & ccn\_change & -22.00 & 12.00 & 0.00 \\
 & entropy & 0.00 & 1.00 & 0.76 \\
\hline
\textbf{OrientDB} & lines\_added & 0.00 & 11,029.00 & 102.00 \\
 & lines\_deleted & 0.00 & 8,234.00 & 29.00 \\
 & num\_files & 0.00 & 124.00 & 3.00 \\
 & entropy & 0.00 & 1.00 & 0.59 \\
\hline
\end{tabular}
\end{table}

コミット単位の変更メトリクスであるlines\_addedとlines\_deletedは、全てのプロジェクトで広い値の範囲を持つことが確認された。lines\_addedの最大値は、Elasticsearchで203,237行、Hazelcastで13,739行、Neo4jで3,893行、Nettyで3,327行、OrientDBで11,029行である。一方、中央値はそれぞれ104行、139行、203行、119行、102行と比較的小さい。これは、大部分のコミットは小規模な変更であるが、一部に非常に大規模な変更を含むコミットが存在することを示している。lines\_deletedも同様の傾向を示しており、最大値はElasticsearchで325,455行と極めて大きいが、中央値は33行と小さい。この分布の非対称性は、ソフトウェア開発において大規模なリファクタリングやモジュール削除が稀に発生することを反映している。

num\_files(変更ファイル数)も同様に広い値の範囲を持つ。最大値はElasticsearchで3,651ファイル、Hazelcastで549ファイル、Neo4jで1,009ファイル、Nettyで780ファイル、OrientDBで124ファイルである。中央値は3から8ファイルの範囲にあり、多くのコミットは少数のファイルを変更するが、一部のコミットでは非常に多くのファイルにまたがる変更が行われることを示している。特にElasticsearchでは、最大値と中央値の差が極めて大きく、プロジェクト全体に影響を与える大規模な変更が存在することが分かる。

entropy(変更の分散度)は、0から1の範囲で正規化された値を取る。全てのプロジェクトで最小値は0、最大値は1である。中央値は0.59から0.79の範囲にあり、多くのコミットで変更が複数のファイルに分散していることを示している。特にNeo4jでは中央値が0.79と高く、変更が広範囲に分散する傾向が強いことが分かる。

メソッド単位の変更メトリクスであるtokens\_change(トークン数の変化量)は、全てのプロジェクトで中央値が0である。これは、多くのメソッドが変更されないか、変更されても小規模であることを意味する。最小値は負の値を取り、Elasticsearchで-395、Hazelcastで-209、Neo4jで-130、Nettyで-529である。負の値は、メソッドからトークンが削除されたことを示している。最大値はElasticsearchで357、Hazelcastで158、Neo4jで180、Nettyで190であり、一部のメソッドでは大幅にトークン数が増加していることが分かる。

length\_change(コード行数の変化量)も、tokens\_changeと同様に中央値が0である。最小値は負の値を取り、Elasticsearchで-99、Neo4jで-37、Nettyで-83である。最大値はElasticsearchで35、Neo4jで60、Nettyで46であり、メソッドのコード行数の変化は比較的小規模であることが示されている。

operation\_type\_added(追加操作のフラグ)は、0または1の二値を取るバイナリ特徴量である。中央値は全てのプロジェクトで0であり、メソッドが新規追加されないコミットの方が多いことが分かる。

ccn\_change(McCabeの循環的複雑度の変化量)は、Nettyでのみ上位10位以内に含まれている。最小値は-22、最大値は12、中央値は0である。この分布は、大部分のメソッドで循環的複雑度が変化しないか、変化しても小規模であることを示している。

全体として、コミット単位の変更メトリクスは広い値の範囲を持ち、最大値と中央値の差が大きいことから、分布が右に歪んでいることが確認された。これは、少数の大規模なコミットが全体の統計に大きく影響を与えていることを示している。一方、メソッド単位の変更メトリクスは、中央値が0であり、大部分のメソッドが変更されないか小規模な変更にとどまることが示された。この分布特性は、機械学習モデルが少数の大規模変更や特定のメソッドの変化パターンを捉えることで、バグ混入リスクを予測していることを示唆している。
\clearpage
\section{Partial Dependence Plot}
\input{5_4_partial_dependence_plot.tex}
\clearpage
\section{決定木}
決定木による分類条件の可視化により、提案手法がバグ混入リスクをどのように判断しているかを明らかにする。ここでは、ベースラインとステップ3(提案手法)における決定木の構造を比較し、時系列変化メトリクスの導入が分類ルールにどのような影響を与えたかを分析する。決定木の最大深さは3に制限し、解釈可能性を重視した。

\subsection{陽性クラスに分類するノード数の変化}
表\ref{tab:leaf_nodes}に、ベースラインとステップ3における陽性クラスに分類するリーフノード数と、高確信度(陽性割合0.8以上)で分類するノード数を示す。

\begin{table}[h]
\centering
\caption{陽性クラスに分類するリーフノード数の比較}
\label{tab:leaf_nodes}
\begin{tabular}{|l|r|r|r|r|}
\hline
プロジェクト & ベースライン & ステップ3 & ベースライン & ステップ3 \\
 &  &  & (確信度≥0.8) & (確信度≥0.8) \\
\hline
Elasticsearch & 4 & 2 & 0 & 0 \\
Hazelcast & 3 & 4 & 1 (0.847) & 1 (0.862) \\
Neo4j & 4 & 3 & 1 (0.808) & 2 (0.952, 0.907) \\
Netty & 3 & 3 & 2 (0.852, 0.850) & 1 (0.849) \\
OrientDB & 2 & 5 & 0 & 1 (0.911) \\
\hline
\end{tabular}
\end{table}

ベースラインでは、5つのプロジェクト全体で陽性に分類するリーフノードは平均3.2個であったのに対し、ステップ3では平均3.4個となった。より重要な変化は、高確信度で陽性に分類するノードの増加である。ベースラインでは、確信度0.8以上で陽性に分類するノードは5プロジェクト中4個(Hazelcastで1個、Neo4jで1個、Nettyで2個)であったが、ステップ3では5個に増加した。特にNeo4jでは、陽性割合0.952と0.907という非常に高い確信度のノードが2個出現し、OrientDBでは最高0.911の確信度を達成した。Hazelcastでも確信度が0.847から0.862に向上している。これは、時系列変化メトリクスの導入により、バグ混入リスクの高い変更をより明確に識別できるようになったことを示している。
\clearpage
\subsection{陰性クラス分類における確信度の向上}
陰性クラスの分類においても、確信度の向上が観察された。ベースラインでは、確信度1.000(陰性割合1.000)のノードは、Elasticsearch、Hazelcast、Netty、OrientDBの4プロジェクトで合計6個存在した。ステップ3では、この数は1個に減少したが、これは分類ルールがより複雑になったためである。一方で、確信度0.9以上のノードを見ると、ベースラインでは2個(Nettyの0.965、Neo4jの0.917)であったのに対し、ステップ3では8個に増加した。特に、Elasticsearchでは0.973と1.000、Hazelcastでは0.965、Neo4jでは0.952、Nettyでは0.975と0.963と0.912、OrientDBでは0.937という高確信度のノードが出現した。

この変化は、時系列変化メトリクスにより、バグが混入していない変更の特徴をより正確に捉えられるようになったことを示唆している。ベースラインでは静的な特徴量のみを用いていたため、完全な確信度(1.000)でしか陰性に分類できない場合があったが、ステップ3では変更メトリクスという追加情報により、0.9以上の高い確信度で多様な陰性パターンを識別できるようになった。
\subsection{分類条件と判断基準の可視化}
決定木の分類ルールを分析すると、ベースラインとステップ3で使用される特徴量が大きく異なることが明らかになった。表\ref{tab:decision_tree_features}に、各プロジェクトで使用された主要な特徴量を示す。

\begin{table}[ht]
\centering
\caption{決定木で使用された主要な特徴量}
\label{tab:decision_tree_features}
\begin{tabular}{|l|p{5cm}|p{6cm}|}
\hline
プロジェクト & ベースライン & ステップ3 \\
\hline
Elasticsearch & MI, TF-IDF, NII & tokens\_change, lines\_added, operation\_type\_added, HNDB, lines\_deleted \\
\hline
Hazelcast & NOI, TF-IDF, HTRP, NII, MISM & NOI, NII, tokens\_change, HNDB, lines\_added, num\_files \\
\hline
Neo4j & NII, TF-IDF, HCPL, MIMS, HDIF, NOI, HVOL & lines\_deleted, lines\_added, operation\_type\_NaN, TF-IDF, tokens\_change, entropy \\
\hline
Netty & TF-IDF, NII, TLOC & tokens\_change, operation\_type\_NaN, lines\_deleted, TF-IDF, lines\_added, HNDB \\
\hline
OrientDB & Brace Rules, TF-IDF, TLLOC, WarningMinor, HVOL, MI & NII, MISEI, lines\_deleted, lines\_added, HVOL, HNDB \\
\hline
\end{tabular}
\end{table}

ベースラインでは、TF-IDFやMI、NIIといった静的コードメトリクスが主に使用されていた。一方、ステップ3では、全てのプロジェクトでtokens\_change、lines\_added、lines\_deletedといったコミット間の変化量メトリクスが上位の分割条件として採用された。特に、tokens\_changeとoperation\_type\_addedというメソッド単位の変化量メトリクスが、多くのプロジェクトで重要な分割条件となっている。

具体的な分類ルールを見ると、ステップ3では時系列変化を中心とした直感的な条件が多く出現した。例えば、Elasticsearchでは「lines\_added <= 103.5 AND operation\_type\_added <= 0.5 AND tokens\_change <= 0.5」という条件で陽性割合0.773となり、小規模な変更で既存メソッドのみを修正した場合にバグ混入リスクが高いことを示している。逆に、「lines\_added <= 103.5 AND operation\_type\_added > 0.5」という新規メソッド追加を含む条件では、陰性割合1.000となり、新規メソッドの追加がバグ混入リスクを低下させることが明確に示された。

Neo4jでは、「lines\_added <= 91.5 AND lines\_added <= 24.5」という小規模変更が陽性割合0.907、「lines\_added <= 91.5 AND lines\_added > 24.5 AND entropy > 0.954」という中規模で分散度の高い変更が陽性割合0.952となった。一方、「lines\_added > 91.5」という大規模変更では、tokens\_changeやlines\_deletedといった追加条件により陰性に分類されており、大規模変更が計画的に実施される傾向を反映している。

Nettyでは、tokens\_changeが最も重要な分割条件となり、「tokens\_change <= 0.5 AND lines\_deleted <= 54.5 AND tokens\_change > -0.5」という条件で陽性割合0.849を達成した。これは、メソッドのトークン数がわずかに変化した場合にバグ混入リスクが高いことを示している。逆に、「tokens\_change <= 0.5 AND lines\_deleted <= 54.5 AND tokens\_change <= -0.5」ではトークン数が減少(リファクタリングによる簡素化)した場合に陰性割合0.975となり、コードの整理がバグ混入リスクを低下させることが示された。

OrientDBでは、ステップ3で陽性リーフノードが2個から5個に増加し、より多様なバグ混入パターンを識別できるようになった。「lines\_added <= 104.5 AND HNDB > 30.887 AND lines\_deleted <= 14.5」という条件では陽性割合0.774となり、小規模な変更で複雑度が高い場合のリスクを捉えている。また、「lines\_added > 104.5 AND HVOL > 256.864 AND lines\_deleted > 1840.5」という大規模な変更でも陽性割合0.911となる条件が見つかり、大規模かつ複雑な変更にはリスクが伴うことが示された。

図\ref{fig:decision_tree_vis}に、代表例としてElasticsearchプロジェクトにおけるステップ3の決定木を示す。

\begin{figure}[ht]
\centering
\includegraphics[width=0.9\textwidth]{figures/elasticsearch/decision_tree_visualization.png}
\caption{決定木による分類ルールの可視化(Elasticsearch、ステップ3)}
\label{fig:decision_tree_vis}
\end{figure}
\clearpage
\section{レビュー労力の削減効果}
\input{5_6_review_effort_reduction_effect.tex}

\chapter{考察}
\section{研究成果}
本研究では、Cost-Benefit Curveを用いた効率的なコードレビュー戦略の提案と評価を行った。実験結果から、提案手法は従来手法と比較して、限られたレビューリソースの下でより多くの欠陥を検出できることが示された。

提案手法が高い性能を示した要因として、以下の2点が考えられる。第一に、メトリクスの選定において、コミットレベルとメソッドレベルの両方の情報を活用したことで、欠陥の予測精度が向上した。第二に、労力考慮型アプローチの採用により、実際の開発現場におけるレビュー工数の制約を適切にモデル化した。

一方で、プロジェクトによって性能にばらつきが見られた点は重要な知見である。これは、プロジェクトごとに開発プロセスやコーディング規約が異なることに起因すると考えられる。

これらの結果は、提案手法が理論的な有効性だけでなく、実際のソフトウェア開発環境においても適用可能であることを示している。特に、レビューリソースが制約される状況下での品質保証活動において、提案手法は有用な選択肢となりうる。
\clearpage
\section{提案手法の適用可能性}
本研究の成果は、ソフトウェア開発プロセスにおいて以下の適用可能性を持つ。

まず、限られたレビューリソースを最大限に活用するための具体的な戦略を提供できる。多くの開発組織では、すべてのコード変更を詳細にレビューする時間的余裕がない。提案手法を用いることで、欠陥を含む可能性が高いコード変更を優先的にレビューし、全体としての品質向上とリソース効率化を両立できる。

次に、コストベネフィット分析の枠組みは、レビュープロセスの継続的改善に活用できる。プロジェクトの進行に伴い、レビューにかけられる工数やプロジェクトの品質要求は変化する。本研究で提案した分析手法により、現状のレビュー戦略が適切かどうかを定量的に評価し、必要に応じて調整することが可能となる。

さらに、メトリクスに基づく欠陥予測は、開発者への教育的なフィードバックとしても機能しうる。どのような特徴を持つコード変更が欠陥を引き起こしやすいかを可視化することで、開発者はより品質の高いコードを書くための指針を得られる。
\section{本研究の限界}
本研究にはいくつかの限界が存在する。

第一に、評価に用いたデータセットが特定のオープンソースプロジェクトに限定されている点である。企業の商用プロジェクトでは、開発プロセスやチーム構成が異なるため、同様の結果が得られるとは限らない。特に、ドメイン固有の知識が重要となるプロジェクトでは、汎用的なメトリクスだけでは不十分な可能性がある。

第二に、レビューコストのモデル化が簡略化されている点である。実際のレビュープロセスでは、コードの行数以外にも、変更の複雑さ、レビュアーの専門性、コミュニケーションコストなど、多様な要因がレビュー工数に影響する。より精緻なコストモデルの構築は今後の課題である。

第三に、予測モデルの解釈可能性が十分に検討されていない点である。機械学習モデルは高い予測精度を達成できるが、なぜそのような予測をしたのかを説明することが難しい場合がある。開発現場で受け入れられるためには、予測の根拠を示すことが重要である。

第四に、時間的な要因を考慮していない点である。本研究では静的なデータセットを用いた評価を行ったが、実際のプロジェクトでは時間経過とともにコードベースや開発チームが変化する。モデルの経年劣化や再学習の戦略については、さらなる検討が必要である。
\clearpage
\section{今後の研究課題}
本研究の成果を発展させるため、以下の研究課題が考えられる。

まず、より多様なプロジェクトでの評価が必要である。特に、異なるプログラミング言語、開発規模、ドメインにおける提案手法の有効性を検証することで、汎用性を高められる。また、企業との共同研究により、商用プロジェクトでの実証実験を行うことも重要である。

次に、動的な環境への適応が課題となる。プロジェクトの進行に伴うコードベースの変化や開発チームの学習効果を考慮し、予測モデルを継続的に更新する仕組みが求められる。オンライン学習やアクティブラーニングの手法を取り入れることで、この課題に対処できる可能性がある。

さらに、レビュープロセス全体の最適化に向けた拡張が考えられる。本研究ではレビュー対象の選定に焦点を当てたが、レビュアーの割り当てやレビューの深さの調整なども含めた総合的な最適化が、より大きな効果をもたらすだろう。

最後に、他の品質保証活動との統合も重要な研究方向である。コードレビューは品質保証活動の一部であり、自動テストや静的解析など他の活動と組み合わせることで、さらなる効率化が期待できる。複数の品質保証活動を統合的に最適化する枠組みの構築は、実務上の価値が高い研究テーマである。

\chapter{おわりに}
本研究では、ソフトウェア開発におけるコードレビューの効率化を目的として、コミット単位とメソッド単位の変更メトリクスを活用した欠陥予測手法を提案した。提案手法では、コミットを不規則なタイミングで発生するイベントとして扱い、直前の状態からの差分を特徴量として学習する。さらに、新たなレビュー労力の計算式とレビュー対象コミットの選択基準を設けることで、少ないレビュー労力で高い欠陥発見率を実現する。

5つのオープンソースプロジェクトを対象とした実験により、提案手法の有効性が示された。全てのプロジェクトにおいて、構造的メトリクスのみのベースラインと比較してF1スコアが向上し、平均改善幅は0.21、最終的なF1スコアは0.70以上を達成した。ROC-AUCはすべてのプロジェクトで0.91を超え、高い識別性能を示した。レビュー労力の分析では、20\%のレビュー労力で、全ての欠陥のうち平均66.9\%を検出できることが示され、提案手法がリソース配分の効率化に有効であることが確認された。

今後の課題として、より多様なプログラミング言語やドメインへの適用が挙げられる。本研究はJavaプロジェクトを対象としたが、型システムやメモリ管理が異なる言語では、メトリクスの調整が必要となる可能性がある。また、本研究で対象としていない金融システムや医療システムでは、より厳格な品質要求や規制要件が存在する。これらのドメインでは、欠陥の特徴が異なる可能性があり、ドメイン固有の調整が必要となる。さらに、プロジェクトの進行に伴うコードベースの変化に対応するため、欠陥予測モデルを継続的に改善する仕組みが求められる。


\renewcommand{\bibname}{参考文献}
\begin{thebibliography}{99}
\bibitem{kim2014}
Miryung Kim, Thomas Zimmermann, and Nachiappan Nagappan,
``An Empirical Study of Refactoring Challenges and Benefits at Microsoft,''
\textit{IEEE Transactions on Software Engineering},
vol. 40, no. 7, pp. 633--649, July 2014.

\bibitem{kamei2013}
Yasutaka Kamei, Emad Shihab, Bram Adams, Ahmed E. Hassan, Audris Mockus, and Anand Sinha,
``A Large-Scale Empirical Study of Just-in-Time Quality Assurance,''
\textit{IEEE Transactions on Software Engineering},
vol. 39, no. 6, pp. 757--773, June 2013.

\bibitem{ferenc2020}
Rudolf Ferenc, P\'{e}ter Gyimesi, G\'{a}bor Gyimesi, Zolt\'{a}n T\'{o}th, and Tibor Gyim\'{o}thy,
``An Automatically Created Novel Bug Dataset and Its Validation in Bug Prediction,''
\textit{Journal of Systems and Software},
vol. 169, p. 110691, November 2020.

\bibitem{han2021}
Xiaofeng Han, Amjed Tahir, Peng Liang, Steve Counsell, and Yajing Luo,
``Understanding Code Smell Detection via Code Review: A Study of the OpenStack Community,''
\textit{Proc. 2021 IEEE/ACM 29th International Conference on Program Comprehension (ICPC)},
pp. 323--334, May 2021.

\bibitem{romano2022}
Simone Romano, Fiorella Zampetti, Maria Teresa Baldassarre, Massimiliano Di Penta, and Giuseppe Scanniello,
``Do Static Analysis Tools Affect Software Quality when Using Test-driven Development?''
\textit{Proc. 16th ACM/IEEE International Symposium on Empirical Software Engineering and Measurement (ESEM)},
pp. 80--91, September 2022.

\bibitem{jisx25010}
日本産業標準調査会,
``JIS X 25010: 2013 システム及びソフトウェア製品の品質要求及び評価(SQuaRE)-システム及びソフトウェア品質モデル,''
日本規格協会,2013年.

\bibitem{mccabe1976}
Thomas J. McCabe,
``A Complexity Measure,''
\textit{IEEE Transactions on Software Engineering},
Vol. SE-2, No. 4, pp. 308--320, December 1976.

\bibitem{chidamber1994}
Shyam R. Chidamber and Chris F. Kemerer,
``A Metrics Suite for Object Oriented Design,''
\textit{IEEE Transactions on Software Engineering},
Vol. 20, No. 6, pp. 476--493, June 1994.

\bibitem{iso24765}
ISO/IEC/IEEE 24765:2017,
``Systems and Software Engineering -- Vocabulary,''
International Organization for Standardization/International Electrotechnical Commission/Institute of Electrical and Electronics Engineers, 2017.

\bibitem{sliwerski2005}
Jacek \'Sliwerski, Thomas Zimmermann, and Andreas Zeller,
``When Do Changes Induce Fixes?''
\textit{Proc. 2005 International Workshop on Mining Software Repositories (MSR)},
pp. 1--5, May 2005.

\bibitem{fowler1999}
Martin Fowler,
\textit{Refactoring: Improving the Design of Existing Code},
Addison-Wesley Professional, 1999.

\end{thebibliography}
\end{document}
