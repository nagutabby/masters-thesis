\documentclass[12pt,a4paper]{ltjsreport}
\usepackage{graphicx}
\usepackage{fontspec}
\usepackage{url}
\usepackage{colortbl,array,xcolor}
\usepackage{here}
\usepackage{amsmath}
\usepackage{lmodern}
\usepackage{silence}
\WarningFilter{caption}{Unknown document class (or package)}
\usepackage{subcaption}

\begin{document}
\thispagestyle{empty}
\begin{center}                                                        
コードの時系列変化を考慮した\\
保守性低下の要因分析と改善
\vfill
\vfill
2410064 笹川 尋翔\\
\vfill
主指導教員  鈴木 正人\\
\vfill
北陸先端科学技術大学院大学\\
先端科学技術研究科\\
情報科学\\ 
\vfill
令和8年3月\\ % 学位授与年月
\vfill
\end{center}

% 目次の出力
\tableofcontents

\clearpage
\centerline{概要}
In software development, continuous code changes bring about defect risk. Existing defect prediction methods use structural metrics. However, these metrics cannot capture the characteristics of changes. Recent studies have investigated change metrics from the version control history. They have shown that change-related features are more strongly correlated with defects than structural metrics. However, these approaches have two problems. First, they do not consider irregularly occurring change events. Instead, they collect change information at specific versions. Second, they do not combine the characteristics of different components. Instead, they focus on the metrics of specific components.

In this work, we propose a defect prediction method that differs from existing approaches in three aspects. First, traditional methods partition the state at regular intervals. In contrast, we treat commits as irregularly occurring events and extract features from the differences between consecutive commits. Thus, we capture the irregular development processes in software projects. Second, existing studies focus on single components. In contrast, our method combines per-method metrics that reflect local changes and per-commit metrics that reflect global changes. This allows us to capture both local and global trends simultaneously. Third, our method selects commits for review by considering both the review effort and the probability of defect introduction. This consideration is based on the scale of the changes, the complexity of the changes, and the spread of the changes. Consequently, this provides a more effective review prioritization method under review effort constraints. 

Experiments on five open-source projects validate the effectiveness of our method. We confirm that the prediction accuracy has been improved in all projects. Moreover, the proposed method outperforms existing methods using only structural metrics. The defect discovery rate is improved by the review prioritization method considering more features.
\clearpage

\chapter{はじめに}
\section{背景}
ソフトウェア開発では継続的な変更が不可避であり、各変更はバグ混入のリスクを伴う。変更により既存コードとの整合性が崩れ、副作用が発生しやすくなるためである。Microsoft Researchの調査では76\%の開発者がリファクタリングによるバグ混入を懸念している\cite{kim2014}。

従来の品質保証手法には限界がある。静的解析はコードの時系列変化を捉えられないため、頻繁に変更される不安定な箇所や、変更が集中する高リスク領域を識別できない。コードレビューでは全ての変更を詳細に検査する時間的な余裕がなく、どの変更を優先的にレビューすべきかの判断基準が不明確である。特に大規模プロジェクトでは、限られたレビューリソースの効率的配分が重要な課題となっている。

機械学習による欠陥予測の研究が進められているが、既存手法はコードの静的特徴量に依存している。静的特徴量では変更頻度や変更パターンといった動的な品質リスク要因を捉えられないため、予測精度に限界がある。コードがどのように変化してきたかという時系列情報は、開発プロセスの動的な側面を反映し、欠陥との強い相関が期待されるが、十分に活用されていない。
\section{動機}
従来のソフトウェア品質保証のアプローチは、主に事後対応型である。すなわち、バグが発生してから検出し、修正するという流れが一般的であった。しかし、バグが本番環境で発見された場合、その影響はユーザー体験の低下やシステムの停止など、組織にとって大きな損失につながる可能性がある。さらに、開発が進んだ後の段階でバグを修正するコストは、開発初期段階での修正に比べて桁違いに高くなることが知られている。そのため、バグが混入する前に、あるいは混入直後に検出し対処する事前予防型のアプローチへの転換が求められている。

事前予防型のアプローチを実現するためには、コード変更時点でバグ混入リスクを評価し、リスクの高い偏光を早期に特定する必要がある。これにより、開発者は変更を本番環境に反映する前に、より慎重なレビューやテストを実施できる。また、問題が小規模なうちに対処することで、後の段階での大規模な修正を回避し、開発コストを抑制できる可能性がある。

しかし、現実の開発現場では、全ての変更に対して十分な時間をかけてレビューやテストを実施することは困難である。特に、アジャイル開発やDevOpsのような短いサイクルでの開発が主流となる中、レビューに割ける時間やリソースは限られている。そのため、限られたレビューリソースをどのように配分するかという問題が重要になる。バグ混入リスクの高い変更を優先的にレビューすることで、同じリソースでより多くの欠陥を発見できれば、開発効率と品質の両立が可能になる。

このような効率的なレビュー優先度付けを実現するためには、コード変更の特性に基づいてバグ混入リスクを定量的に評価する手法が必要である。ここで重要なのは、単に現時点でのコードの静的な特性を見るだけでなく、コードがどのように変化してきたかという時系列的な情報に着目することである。例えば、短期間に頻繁に変更されているコード、あるいは大規模な変更が加えられたコードは、バグ混入リスクが高い可能性がある。また、過去にバグが多く発見された箇所への変更も、同様にリスクが高いと考えられる。

コードの変化過程に着目した時系列情報の活用は、静的解析では捉えきれない動的なリスク要因を明らかにする可能性がある。変更の頻度、変更の規模、変更されたファイル間の関連性など、時系列的な観点から抽出できる特徴量は多数存在する。これらの情報を機械学習モデルに組み込むことで、より精度の高い欠陥予測が可能になると期待される。

本研究は、こうした背景から、コードの時系列変化に着目したソフトウェア欠陥予測手法の確立を目指す。時系列情報を活用することで、バグ混入リスクの事前予測精度を向上させ、限られたレビューリソースの効率的な配分を支援する実用的な手法を提供することが本研究の動機である。
\section{目的}
本研究の目的は、コードの時系列変化を考慮した欠陥予測により、限られたレビューリソースの効率的配分を支援することである。具体的には以下を達成する。

\textbf{目的1: 時系列メトリクスの設計}

メソッド単位の変化量とコミット単位の変化率を組み合わせ、ミクロとマクロの両視点から変更特性を捉える。

\textbf{目的2: 高精度予測モデルの構築}

時系列特徴量を用いた機械学習モデルにより、従来手法を上回る予測性能を実証する。

\textbf{目的3: レビュー労力削減効果の定量評価}

ナップサック問題としての定式化により、同一労力でのバグ発見数向上をCost-Benefit Curveで検証する。
\section{貢献}
本研究の主要な貢献は以下の3点である。

\textbf{貢献1: 不規則な時系列データとしての分析}

開発者がいつコードを書き換えるか分からない「不規則なタイミングで発生するイベント」としてコミットを捉え直し、決まった間隔で記録される一般的な時系列データとの違いを整理した。従来の時系列分析手法は等間隔データを前提としており、コミットの不規則性を適切に扱えていなかった。本視点により、コミット間隔を考慮した特徴量設計が可能となり、より実態に即した欠陥予測が実現できる。

\textbf{貢献2: メトリクス設計}

メソッド・コミット単位の変更メトリクスを統合し、理論的根拠とともに提示した。単一レベルのメトリクスでは捉えられない、コードの局所的変化とプロジェクト全体の開発動向という複数のリスク要因を統合することで、予測精度の向上を実現した。

\textbf{貢献3: 実プロジェクトでの実証}

5つのOSSプロジェクトでF1スコア向上と統計的有意性を確認し、特徴量重要度の分析により実用可能性を示した。
\section{構成}
本論文の構成は以下の通りである。

第2章では、ソフトウェア欠陥予測に関する関連研究を概観する。Just-In-Time品質保証の分野におけるKameiらの研究を取り上げ、14個の変更メトリクスとその問題点におついて詳述する。また、Ferencらのコミットベースの欠陥予測手法、Hanらのコードレビューテキスト分析、Romanoらの静的解析による品質改善研究を紹介する。これらの既存研究を整理することで、コミット間変化量の考慮不足や特性値間の関連性分析の欠如といった問題点を明らかにする。

第3章では、本研究で提案する時系列変化を考慮した欠陥予測手法について説明する。まず、コミットを点過程データとして位置付け、従来の等間隔時系列分析との違いを明確化する。次に、メソッド単位での変化量メトリクスとコミット単位での変更率メトリクスの設計方針を述べ、それぞれのメトリクスで変化量と変化率を使い分ける理論的根拠を示す。最後に、機械学習モデルの選定理由、モデルの評価指標、有意性検定について説明する。

第4章では、実験設定について詳述する。データセットにおける正解ラベルの定義方法、機械学習モデルの学習における追加の制約、プロジェクト選定基準を説明する。対象とする5つのプロジェクト(Elasticsearch、Hazelcast、Netty、OrientDB、Neo4j)の特性を示し、直前コミットとの差分に基づく特徴量生成方法と前処理手順を述べる。さらに、ベースライン、変化量メトリクスの追加後、変化率メトリクスの追加後の3段階での評価を行う実験手順と、モデル解釈性分析の方法について説明する。

第5章では、実験結果を報告する。5つのプロジェクトにおけるF1スコアなどの評価指標の比較表を示し、提案手法による予測性能の改善を定量的に示す。特徴量の寄与度の分析により、どのような特徴量が予測に貢献しているのかを明らかにする。また、特徴量の分布、Partial Dependence Plotによる各特徴量と陽性確率の関係、決定木による分類条件の可視化、Cost-Benefit Curveによるレビュー労力削減効果を示す。

第6章では、実験結果に基づく考察を行う。予測性能向上のメカニズムを分析し、各特徴量の意味と予測への寄与について解釈する。プロジェクト間での特徴量の影響の違いを考察し、その背景にある開発フェーズやコーディング文化の違いを議論する。また、クラス不均衡の影響やモデルの予測特性、レビュー労力モデルの単純化といった本研究の限界を明示し、今後の研究課題を提示する。

第7章では、本研究の成果を総括し、主要な貢献を再確認する。主要な時系列分析手法との違い、メトリクス設計、実プロジェクトでの性能検証という3つの貢献を振り返り、実用的な意義と今後の展望について述べる。
\clearpage

\chapter{コード品質と関連研究}
本章では、ソフトウェア欠陥予測に関連する研究を概観する前に、本研究で用いる主要な概念と用語について説明する。

ソフトウェアの品質を体系的に評価するため、JIS X 25010:2013\cite{jisx25010}では、製品品質モデルと利用時の品質モデルという2つの品質モデルが定義されている。製品品質モデルは、「ソフトウェアの静的特徴及びコンピュータシステムの動的特徴」に関する品質を8つの特性で分類する。これらの特性は、機能適合性、性能効率性、互換性、使用性、信頼性、セキュリティ、保守性、移植性である。各特性は更に副特性に分割され、より詳細な品質評価を可能にする。

本研究が特に注目するのは、\textbf{保守性}と\textbf{信頼性}である。保守性は「意図した保守者によって、製品又はシステムが修正することができる有効性及び効率性の度合い」と定義され、5つの副特性を持つ。\textbf{モジュール性}(一つの構成要素に対する変更が他の構成要素に与える影響が最小になる度合い)、\textbf{再利用性}(一つ以上のシステムに資産を使用することができる度合い)、\textbf{解析性}(変更の影響を総合評価すること、欠陥若しくは故障の原因を診断すること、又は修正しなければならない部分を識別することが可能であることについての有効性及び効率性の度合い)、\textbf{修正性}(欠陥の取込みも既存の製品品質の低下もなく、有効的に、かつ、効率的に製品又はシステムを修正することができる度合い)、\textbf{試験性}(試験基準を確立し、その基準が満たされているかどうかを決定するために試験を実行することができる有効性及び効率性の度合い)である。

信頼性は「明示された時間帯で、明示された条件下に、システム、製品又は構成要素が明示された機能を実行する度合い」と定義され、4つの副特性を持つ。\textbf{成熟性}(通常の運用操作の下で、システム、製品又は構成要素が信頼性に対するニーズに合致している度合い)、\textbf{可用性}(使用することを要求されたとき、システム、製品又は構成要素が運用操作可能及びアクセス可能な度合い)、\textbf{障害許容性}(ハードウェア又はソフトウェア障害にもかかわらず、システム、製品又は構成要素が意図したように運用操作できる度合い)、\textbf{回復性}(中断時又は故障時に、製品又はシステムが直接的に影響を受けたデータを回復し、システムを希望する状態に復元することができる度合い)である。

本研究では、コードの変更がこれらの品質特性、特に保守性の解析性・修正性と信頼性の成熟性にどのように影響を与えるかを分析する。コードの変化の過程を捉えることが重要である理由は、静的なスナップショットだけでは変更の勢いや不安定性を把握できないためである。例えば、ある時点でのコードメトリクスは問題なくても、短期間に頻繁な変更が繰り返されていれば、それは設計の不安定性や開発者の理解不足を示唆している可能性がある。時系列変化を分析することで、保守性を低下させる要因や、信頼性を脅かす欠陥の混入パターンを明らかにすることを目指す。

ソフトウェアの品質を定量的に評価するため、様々なコードメトリクスが提案されてきた。これらのメトリクスは、大きく3つのカテゴリーに分類できる。

\textbf{複雑度メトリクス}は、コードの構造的な複雑さを測定する。代表的なものとして、McCabeの循環的複雑度(Cyclomatic Complexity)\cite{mccabe1976}がある。これは、プログラムの制御フローグラフにおける独立したパスの数を表し、$V(G) = E - N + 2P$($E$は辺の数、$N$は節の数、$P$は連結成分の数)として計算される。循環的複雑度が高いほど、コードの理解が困難になり、テストすべきパスが増加する。これが欠陥混入リスクを高める理由は、人間の認知的限界により、開発者が複雑なコードの全ての振る舞いを把握することが困難になり、また、テストパスの増加によってテストの網羅性が低下するためである。その結果、保守性の解析性と修正性を低下させる。

\textbf{規模メトリクス}は、コードの量的な大きさを測定する。最も基本的なメトリクスは、LOC(Lines of Code、総コード行数)である。変更規模を示すメトリクスとしては、追加されたコード行数(LA)、削除されたコード行数(LD)、変更前のファイルのコード行数(LT)などがある。規模が大きいほど、より多くのコードの変更や実装が必要となるため、欠陥が発生する可能性が高くなる。

\textbf{結合度・凝集度メトリクス}は、オブジェクト指向プログラムの構造的品質を測定する。Chidamber \& Kemerer\cite{chidamber1994}が提案したCKメトリクスは、オブジェクト指向設計の品質を評価するための標準的な指標である。CBO(Coupling Between Objects、オブジェクト間結合度)は、あるクラスが他のクラスに依存している度合いを示す。RFC(Response For Class、クラスの応答数)は、クラスが呼び出す可能性のあるメソッドの総数を示す。LCOM(Lack of Cohesion of Methods、メソッド凝集度の欠如)は、クラス内のメソッド間の凝集性の欠如を測定する。これらのメトリクスは、保守性のモジュール性や再利用性に直接影響を与える。

本研究では、これらの既存メトリクスに加えて、メソッドレベルでの時系列変化を捉えるための新しいメトリクスを提案する。

ソフトウェア工学における欠陥関連の用語は、ISO/IEC/IEEE 24765\cite{iso24765}において厳密に定義されている。\textbf{欠陥(Defect)}は、成果物が要件または仕様を満たさず、修理や交換が必要となることである。欠陥は、コード内に存在するが、必ずしも実行時に顕在化するとは限らない。\textbf{故障(Failure)}は、システムが要求された機能を実行できない事象のことである。故障は、システムの実行時に観測される異常な振る舞いであり、欠陥が顕在化した結果である。\textbf{誤り(Error)}は、人間が犯す間違いのことであり、欠陥の原因となる。

本研究では、「バグ」という用語を一般的な意味で使用するが、厳密な議論では「欠陥」という用語を用いる。また、「バグ混入コミット」または「欠陥誘発コミット」とは、後に修正が必要となる欠陥を含むコードをリポジトリに追加したコミットを指す。

欠陥混入コミットの特定には、SZZアルゴリズム\cite{sliwerski2005}を用いる。SZZアルゴリズムは、Śliwerski、Zimmermann、Zellerによって提案された手法であり、バグ修正コミットから遡って、そのバグを最初に混入させたコミットを特定する。具体的な手順は以下の通りである。まず、バグレポートシステム(例:JIRA、Bugzilla)とコミットメッセージを関連付けることで、バグ修正コミットを特定する。次に、バグ修正コミットにおいて変更された行を特定し、Gitの\texttt{git blame}機能などを用いて、それらの行が最後に修正されたコミットを遡って追跡する。この追跡によって特定されたコミットが、バグを混入させたと推定される。

本研究では、Gitなどのバージョン管理システム(VCS: Version Control System)に記録された変更履歴を分析対象とする。\textbf{コミット(Commit)}は、ソースコードへの変更をリポジトリに記録する単位である。各コミットには、変更されたファイル、追加・削除された行、コミットメッセージ、作成者、作成日時などの情報が含まれる。\textbf{差分(Diff)}は、コミット間でのファイルの変更内容を表現する方法であり、どの行が追加・削除・変更されたかを示す。

本研究がコミット単位を分析単位として選択した理由は、コミットが開発者の論理的な作業単位を表し、変更の意図やコンテキストが比較的明確であるためである。変更の文脈が明確であることは、欠陥予測において重要な意味を持つ。なぜなら、欠陥の混入は単にコード量の増加だけでなく、変更の目的や背景と密接に関係しているからである。例えば、バグ修正のための変更は新機能追加よりも欠陥を誘発しやすい傾向がある。また、レビュー時には変更の意図を理解することで、より効果的なレビューが可能になる。ファイルやパッケージ単位での予測と比較して、コミット単位での予測は以下の3つの利点がある。第一に、予測単位の粒度が小さいため、開発者は具体的にどのコード断片をレビューすべきかを特定しやすい。第二に、変更を行った開発者が明確であるため、品質保証活動の担当者を容易に割り当てることができる。第三に、開発サイクルの早い段階で予測が行われるため、開発者の記憶が新しいうちにレビューやテストを実施でき、修正コストを低減できる。

欠陥予測モデルの性能を評価するため、本研究では以下の指標を使用する。これらの指標は、\textbf{混同行列(Confusion Matrix)}に基づいて計算される。混同行列は、予測結果と実際の結果を4つのカテゴリーに分類する。\textbf{TP(True Positive、真陽性)}は、バグありと予測し、実際にバグがあったケース、\textbf{TN(True Negative、真陰性)}は、バグなしと予測し、実際にバグがなかったケース、\textbf{FP(False Positive、偽陽性)}は、バグありと予測したが、実際にはバグがなかったケース、\textbf{FN(False Negative、偽陰性)}は、バグなしと予測したが、実際にはバグがあったケースである。

\textbf{適合率(Precision)}は、バグありと予測したもののうち、実際にバグがあった割合を示す。$Precision = \frac{TP}{TP + FP}$として計算される。適合率が高いほど、誤検出(偽陽性)が少ないことを意味する。\textbf{再現率(Recall)}は、実際にバグがあったもののうち、バグありと予測できた割合を示す。$Recall = \frac{TP}{TP + FN}$として計算される。再現率が高いほど、見逃し(偽陰性)が少ないことを意味する。

\textbf{F1スコア(F1-score)}は、適合率と再現率の調和平均であり、$F1 = 2 \times \frac{Precision \times Recall}{Precision + Recall}$として計算される。F1スコアは、適合率と再現率のバランスを評価する指標であり、両者が共に高い値を取るときに高くなる。本研究では、F1スコアを主要な評価指標として使用する。

ソフトウェアの変更履歴では、「バグが含まれるケース」が「正常なケース」に比べて圧倒的に少ないという、データの極端な偏り(クラス不均衡問題)がある。クラス不均衡が問題となる理由は、機械学習モデルが多数クラス(バグなし)に偏った予測を行い、少数クラス(バグあり)の検出精度が著しく低下するためである。この問題に対処するため、\textbf{オーバーサンプリング}(少数クラスのデータを人工的に増やす手法。例: SMOTE)や\textbf{アンダーサンプリング}(多数クラスのデータを削減する手法)といった手法が用いられる。本研究では、ランダムアンダーサンプリングを用いる。この手法を選択した理由は、オーバーサンプリングによる人工データの生成は過学習のリスクを高める可能性があるのに対し、アンダーサンプリングは実データのみを使用するため、モデルの汎化性能が安定しやすいためである。具体的には、バグありクラスとバグなしクラスから同数のデータを取得する。
\section{コミット単位での欠陥予測の必要性}
従来のソフトウェア欠陥予測手法では、ファイルやパッケージ単位での予測が主流であった。このアプローチでは、McCabeの循環的複雑度やCKメトリクス、コード行数といったコードメトリクス、あるいは過去の欠陥数や変更回数といったプロセスメトリクスを用いて、欠陥が発生しやすいモジュールを特定する。しかし、このアプローチには以下の3つの問題点がある。第一に、予測単位の粒度が大きいため、重要なファイルが特定された後も、開発者はレビューやテストの対象となる具体的な関数やコードの断片を見つけるのにかなりの時間を費やす必要がある。第二に、予測がモジュール単位で行われるため、誰がその品質保証活動を担当すべきかが明確でない。第三に、予測が開発サイクルの遅い段階で行われるため、問題が発見された時点では既に多くのコードが書かれており、修正コストが高くなる。

これらの問題に対処するため、Kameiら\cite{kamei2013}は、ファイルやパッケージではなく、コミット単位での欠陥予測を行う「Just-In-Time品質保証」というアプローチを提案した。このアプローチでは、開発者がコードをリポジトリにコミットした直後に、その変更が欠陥を誘発するリスクを予測する。これにより、開発者は記憶が新しいうちにリスクの高い変更をレビューやテストできるため、より効率的かつ効果的な品質保証活動が可能になる。変更単位での予測には、予測単位の粒度が小さいこと、変更を行った開発者への具体的な作業割り当てとして予測を表現できること、開発サイクルの早い段階で予測が行われることという3つの利点がある。

Kameiらは、コード変更から抽出した特徴量を5つのカテゴリーに分類し、合計14個の変更メトリクスを提案した。これらのメトリクスは以下の通りである。

\begin{itemize}
    \item Diffusion
    \begin{itemize}
        \item 変更の拡散範囲を測定するメトリクスであり、NS(変更されたサブシステムの数)、ND(変更されたディレクトリの数)、NF(変更されたファイルの数)、Entropy(各ファイル間の変更されたコードの分散度)が含まれる。広範囲に分散した変更は理解が複雑になり、全ての変更箇所を追跡する必要があるため、欠陥発生リスクが高いと考えられる。
    \end{itemize}
    \item Size
    \begin{itemize}
        \item 変更の規模を測定するメトリクスであり、LA(追加されたコード行数)、LD(削除されたコード行数)、LT(変更前のファイルのコード行数)が含まれる。変更が大きいほど、より多くのコードの変更や実装が必要となるため、欠陥が発生する可能性が高くなる。
    \end{itemize}
    \item Purpose
    \begin{itemize}
        \item 変更の目的を示すメトリクスであり、FIX(変更が欠陥修正であるかどうか)が含まれる。欠陥を修正する変更は、以前の実装で欠陥が混入したことを意味し、その箇所に再び欠陥が混入しやすい可能性がある。
    \end{itemize}
    \item History
    \begin{itemize}
        \item ファイルの変更履歴を測定するメトリクスであり、NDEV(変更されたファイルを変更した開発者数)、AGE(直前の変更と現在の変更の間の平均時間間隔)、NUC(変更されたファイルへのユニークな変更回数)が含まれる。過去の研究では、ファイルに対する過去の変更回数と欠陥修正回数は、ファイルのバグの多さを示す良い指標であることが示されている。
    \end{itemize}
    \item Experience
    \begin{itemize}
        \item 開発者の経験を測定するメトリクスであり、EXP(開発者の全体的な経験)、REXP(開発者の最近の経験)、SEXP(サブシステムにおける開発者の経験)が含まれる。経験が豊富な開発者ほど欠陥を導入しにくいと考えられる。
    \end{itemize}
\end{itemize}

Kameiらは、6つのOSSプロジェクトと5つの商用プロジェクトを対象とした大規模な実証研究を実施した。多重共線性に対処するため、高い相関関係にある因子を除去し、NDとREXPをモデルから除外した。また、LAとLDをLTで割って正規化し(LA/LT、LD/LT)、LTとNUCをNFで割って正規化した(LT/NF、NUC/NF)。その結果、最終的に12個のメトリクスを用いてロジスティック回帰モデルを構築した。10分割交差検証による評価の結果、OSSプロジェクトでは平均適合率37\%、平均再現率67\%、商用プロジェクトでは平均適合率32\%、平均再現率62\%を達成した。

さらに、レビュー労力削減効果を評価するため、変更された行の総数を用いて労力を計算し、総労力の20\%をレビューに使用できると仮定した。予測されたロジスティック確率に基づいて変更を優先順位付けし、労力考慮型モデル(EALR)を構築した。その結果、OSSプロジェクトでは平均28\%、商用プロジェクトでは平均43\%の欠陥を誘発する変更を、総労力の20\%で特定できることが示された。これは、Just-In-Time品質保証が最もリスクの高い変更に集中するための効果的な手法であることを示している。しかし、この手法には重要な単純化が含まれている。レビュー労力を「変更された行の総数」のみで計算しているため、変更の複雑さや影響範囲の広さが考慮されていない。実際のレビュー労力は、変更されたファイル数や変更の分散度(Entropy)といった要因にも依存すると考えられるが、Kameiらの手法ではこれらの要因が労力計算に反映されていない。

特徴量の重要性を分析した結果、OSSプロジェクトではNF、LA/LT、LT/NF、FIXが、商用プロジェクトではNF、LA/LT、LT/NF、NDEV、AGEが、欠陥リスクを増加させる最も重要な要因であることが明らかになった。ただし、これらのメトリクスにはいくつかの問題点が存在する。FIXは特定のコミットが欠陥修正であるかどうかを示すフラグであり、どのメソッドやクラスが欠陥を誘発しやすいかという具体的な情報を提供しない。AGEは商用プロジェクトでは有効だが、OSSプロジェクトではボランティアベースで開発が行われるため、開発頻度が不定期であり効果的ではない。NDEVも商用プロジェクトではチームでの共同作業が多いため有効だが、OSSプロジェクトでは1つの変更は基本的に1人の開発者が担当するため効果的ではない。NSはサブシステムの総数がプロジェクトによって異なるため、プロジェクト間での比較が困難である。Entropyは適切なモジュール分割を行った場合でも、変更が複数のファイルに分散していれば欠陥発生リスクが高いと判断されてしまう。
\section{コミット時点でのコード特性の把握と課題}
コミット単位での欠陥予測は、バージョン管理システム(VCS: Version Control System)に記録された変更履歴を活用し、コードの変化と欠陥の関係を分析する手法である。VCSには、各コミットに対して、変更されたファイル、追加・削除された行、コミットメッセージ、作成者、作成日時などの情報が記録される。これらの情報を活用することで、開発者がコミットした直後に、その変更が欠陥を誘発するリスクを予測することが可能になる。この分野では、どのコミットが欠陥を混入させたかを特定するための手法と、コミット時点でのコードの特性を捉えるためのデータセット構築が重要な課題となっている。

Ferencら\cite{ferenc2020}は、GitHubからバグ情報を自動的に収集し、コミットごと、ソフトウェアの構成要素(メソッド、クラス)ごとのメトリクスを含む「BugHunter Dataset」を構築した。このデータセットの特徴は、従来の研究が特定のリリースバージョンにおける全てのソースコード要素の特性を収集していたのに対し、バグ混入コミットとバグ修正コミットという、バグの存在を特定できる最も狭い期間において、同じソースコード要素のバグあり状態と修正済み状態の両方を捉えることである。

データセット構築において、Ferencらは15のJavaプロジェクトを対象とし、欠陥が混入したコミットを特定するためにSZZアルゴリズムを用いた。この手法により、欠陥混入時と修正時のコードメトリクスを比較することが可能になる。

データセット構築の過程で、同じメトリクス値を持ちながら異なる数のバグが割り当てられているエントリーが存在するという問題に直面した。これは機械学習による欠陥予測の精度に悪影響を与える冗長性を生み出すため、Removal法、Subtract法、Single法、GCF法という4つのフィルタリング手法を比較検証した。その結果、より大きなエントリー数を持つクラスのエントリーを保持するRemoval法が最も高いF1スコアを達成した。また、クラス不均衡問題に対処するため、ランダムアンダーサンプリングを用いて、バグありクラスとバグなしクラスから同数のデータを取得した。

構築されたデータセットを用いた欠陥予測実験では、ナイーブベイズ、ロジスティック回帰、C4.5、ランダムフォレストなど11種類の機械学習アルゴリズムを比較評価した。ファイル、クラス、メソッドという3つのレベルで予測を実施した結果、メソッドレベルではランダムフォレストが最も高い性能を示し、平均F1スコアは約0.63であった。クラスレベルでは単純ロジスティック回帰が最も高く、平均F1スコアは約0.57、ファイルレベルではランダムツリーが最も高く、平均F1スコアは0.55であった。これらの結果は、より細かい粒度(メソッドレベル)での予測が、より高い精度を達成できることを示している。

一方、Hanらは、コードレビューを通じた欠陥検出の実態を調査した。OpenStackプロジェクトのNovaとNeutronを対象に、19,146件のレビューコメントを手動で分析し、将来の不具合を招く恐れのある不適切な構造がどの程度特定されるかを調査した。こうした不適切な構造とは、Martin Fowler が提唱した概念であり、コード内の潜在的な問題を示す特徴的なパターン(例: 肥大化したメソッド、重複コード、過度に複雑なクラス構造など)を指す。これらは必ずしも欠陥ではないが、保守性を低下させ、将来的に欠陥を誘発しやすくする要因となる。その結果、コードレビューでこうした不適切な構造が特定されることは一般的ではないことが明らかになった。約1,200件のレビューのうち、不適切な構造が明示的に指摘されたのは限られた数であった。さらに、レビューの大部分(70\%)では、特定された不適切な構造について説明が提供されておらず、レビュアーは単に問題を指摘するだけで、その理由を詳しく説明していなかった。

これらの研究から、コミットベースの欠陥予測における重要な課題が明らかになる。第一に、Ferencらの研究では各コミットの特性値を用いた予測が行われているが、コミット間の変化は対象外である。すなわち、あるコミット時点でのコードメトリクスは測定されているが、直前のコミットからどのように変化したかという時系列的な情報は活用されていない。第二に、Hanらの研究が示すように、コードレビューでは70\%のケースで欠陥の原因が明示されないため、レビューテキストのみから欠陥を予測することには限界がある。これらの課題は、コミット間の変化を考慮し、特性値同士の関連性を分析する新たなアプローチの必要性を示唆している。
\section{リアルタイム品質検査の実現と課題}
静的解析は、ソフトウェアを実行することなくソースコードを検査し、潜在的な問題を検出する手法である。これは、プログラムを実際に動作させて挙動を観察する動的解析と対比される。静的解析ツールは、コードの構造、複雑度、スタイル、潜在的な欠陥パターンなどを自動的に分析し、JIS X 25010:2013で定義される保守性(特に解析性と修正性)の向上に貢献する。近年では、SonarQube\cite{sonarqube}やCheckstyle\cite{checkstyle}などの静的解析ツールが開発環境に統合され、コーディング中にリアルタイムでコードの問題を検出できるようになっている。

Romanoら\cite{romano2022}は、テスト駆動開発(TDD)において静的解析ツールを使用することがソフトウェア品質に与える影響を実証的に調査した。TDDは、レッドフェーズ(テストを書いて失敗させる)、グリーンフェーズ(テストに合格する最小限のコードを書く)、リファクタリングフェーズ(コードを改善する)という3つのフェーズを繰り返す開発手法である。理論的には、リファクタリングフェーズで\textbf{技術的負債}(将来の保守コストを増加させる設計上の妥協や実装上の問題)を返済し、品質を継続的に向上させることが期待される。

しかし、実際にはこのフェーズがしばしばスキップされることが観察されていた。リファクタリングがスキップされる主な理由は、以下の3点である。第一に、開発スケジュールの時間的プレッシャーにより、機能実装を優先し、コード品質の改善が後回しにされる。第二に、リファクタリングは即座にユーザーに見える価値を提供しないため、優先度が低く見なされる。第三に、技術的負債は可視化されにくく、どの程度の問題があるのかを開発者が認識しづらい。Romanoらは、静的解析ツールを用いることでこれらの問題に対処できるのではないかという仮説を立てた。

Romanoらは、静的解析ツールであるSonarLint(SonarQube for IDE)を使用する群と使用しない群に分けて実験を実施した。ソフトウェア品質を定量化するため、将来の不具合を招く恐れのある不適切な構造の数(Smell)、技術的負債の推定値(SqaleIndex)、循環的複雑度(WMC)、コードの分かりやすさ(CognCompl)、読みやすさ(BW)などの指標を測定した。

実験の結果、SonarLintを使用した群は、使用しない群と比較して、Smell、SqaleIndex、CognComplにおいて統計的に有意な改善が見られた。すなわち、静的解析ツールの使用は、不適切な構造の削減やコードの分かりやすさの向上に寄与することが示された。

静的解析ツールがこのような改善をもたらす理由は、以下の3点にある。第一に、問題の可視化により、開発者が技術的負債の存在に気づくことができる。漠然とした「コードが良くない」という感覚ではなく、具体的な指標として提示されることで、改善の動機が生まれる。第二に、コーディング中の即座のフィードバックにより、問題が小さいうちに修正できる。後から大規模なリファクタリングを行うよりも、その場で修正する方がコストが低い。第三に、ツールが提供する客観的基準により、コードレビューでの議論が建設的になり、チーム内で品質基準が共有される。これらの改善は、JIS X 25010:2013で定義される保守性の副特性、特に解析性(コードの問題点を識別する容易さ)と修正性(欠陥を取り込まずに変更を行う容易さ)の向上に貢献する。

一方で、実験後のアンケートでは、参加者はSonarLintを使用するとTDDがより困難になると認識しており、ツールの使用が追加の認知的負荷をもたらす可能性が示唆された。TDDが困難になる理由として、以下の要因が考えられる。第一に、静的解析ツールの警告に対応するための追加作業が発生し、本来のTDDサイクルが中断される。第二に、グリーンフェーズでは「テストに合格する最小限のコード」を書くことが求められるが、静的解析ツールが即座に品質問題を指摘するため、最小限のコードと品質の高いコードの間で葛藤が生じる。第三に、ツールの誤検出(偽陽性)に対処する必要があり、それが開発者の集中力を削ぐ。

この研究から明らかになる静的解析の課題は、文脈依存のしきい値の未検証である。SonarLintのようなツールは、事前に定義されたルールに基づいて不適切な構造を検出するが、プロジェクトの特性や開発フェーズに応じて適切なしきい値は異なる可能性がある。Romanoらの研究では、このような文脈依存性については検証されておらず、全てのプロジェクトに対して同一の設定が適用されている。

これが問題である理由は、以下の3点にある。第一に、偽陽性(実際には問題でないのに警告される)により、開発者が警告を無視するようになり、ツールへの信頼性が低下する。第二に、プロジェクトの初期段階では厳密な品質基準を適用することが非現実的な場合があり、過度な警告が開発速度を低下させる。第三に、本当に重要な問題が大量の警告に埋もれてしまい、見逃される可能性がある。そのため、静的解析ツールが検出する問題の中には、実際にはプロジェクトの文脈では問題とならない指摘が含まれている可能性がある。
\clearpage

\chapter{提案手法}
\section{コミットの不規則性を考慮した時系列情報の活用}
本研究では、プログラムの変更(コミット)が「いつ起こるか分からない不規則なイベント」である点に着目し、欠陥予測を行う手法を採用する。

従来の時系列分析(株価や気温の予測など)は、一定の間隔で測られたデータを対象としている。例えば、1時間ごとの気温の変化から将来を予測する手法(ARIMAやLSTMなど)が一般的だが、これらはデータが規則正しく並んでいることを前提としている。

しかし、ソフトウェア開発におけるコミットは、決まった間隔で行われるものではない。開発者の作業ペースやプロジェクトの状況によって、頻繁に更新される時期もあれば、長期間動きがない時期もある。このように、いつ起こるか分からない出来事の記録は、点過程データと呼ばれる 。本研究では、この不規則なタイミングそのものに開発のリズムやリスクが隠れていると考え、分析の基盤とする。

具体的なアプローチとして、前回の作業から「何が、どれくらい変わったか」という変化に注目する。例えば、ある関数が書き換えられた際、単に現在の行数を測るのではなく、前回の状態から「何行増えたか」「構造がどれほど複雑になったか」といった時系列の変化を数値化する。これにより、短期間での急激な複雑化といった、静的な解析では見落とされがちな「不具合の予兆」を捉えることが可能になる。

また、本研究では「将来の数値を当てる(回帰)」のではなく、「その変更に欠陥が含まれるか、含まれないか」という「二値分類」の問題として予測を行う。これは、最終的な目的が「限られた確認作業(レビュー)の時間を、リスクの高い箇所に集中させること」にあるためである。

研究の全体像を図\ref{fig:research_approach_overview}に示す。不規則なコミットを点過程データとして整理し、その変化の内容から機械学習(ランダムフォレストなど)を用いて、欠陥が混入している確率を導き出す。

\begin{figure}[htbp]
  \centering
  \includegraphics[width=0.95\textwidth]{figures/research_approach_overview.pdf}
  \caption{研究アプローチ全体の概念図}
  \label{fig:research_approach_overview}
\end{figure}
\section{異なる粒度での変更特性の把握}
欠陥予測精度を向上させるため、コード変更を複数の視点から捉える必要がある。本研究では、メソッド単位とコミット単位という2つの構成要素を用いて特徴量を設計する。

この2つの構成要素を用いる理由は、欠陥は局所的な実装ミスとシステム全体の不整合の両方から生じるためである。メソッド単位のメトリクスは前者を、コミット単位のメトリクスは後者を捉える。また、既存研究において、単一の構成要素のメトリクスを用いるだけでは十分な予測精度を得られないことが示されている。異なる構成要素のメトリクスを組み合わせることで、互いの欠点を補完し合うことができる。

\paragraph{メソッド単位の変更メトリクス}
メソッド単位では、個々のメソッドの局所的な変化を捉えるため、直前のコミットからの変化量を特徴量として用いる。本研究では、コード変更の異なる側面を捉えるため、以下の3つのメトリクスを用いる。

\begin{itemize}
    \item コード行数の変化量: 変更の規模を表し、レビュー労力と相関がある。
    \item トークン数の変化量: 意味のあるコード要素(変数名、演算子、キーワードなど)の変化を捉え、空白行やコメントのみの変更と論理的な変更を区別する。
    \item 循環的複雑度の変化量: 制御フローの複雑さの変化を測定し、条件分岐やループの追加による論理構造の変化を捉える。
\end{itemize}

\paragraph{コミット単位の変更メトリクス}
コミット単位では、特定のコミットに関連付いた変更の影響範囲を捉えるため、以下のメトリクスを用いる。

\begin{itemize}
    \item 変更されたファイル数: 変更範囲の広さを表す。複数ファイルにまたがる変更を加えると、モジュール間の整合性確保が困難になり、欠陥混入リスクが高まる。
    \item コードの追加行数: 変更前のコードに対して追加されたコードの行数。新たなコードの導入が既存のコードに影響する場合、特に欠陥が生まれやすい。
    \item コードの削除行数: 変更前のコードから削除されたコードの行数。大規模な削除は、依存関係の破壊につながる可能性がある。
    \item 変更の広がり: 各ファイルの変更行数から算出されるエントロピーを用いて、変更の広がりを定量化する。変更が複数のファイルに分散しているほど値が大きくなり、それに伴ってレビュー労力が増加する。詳しくは3.4節で述べる。
\end{itemize}

メソッド単位とコミット単位のメトリクスで変化量を採用する理由は、欠陥混入リスクが絶対的な変化に比例するためである。一例として、変化率を用いると小規模なメソッドやコミットにおいて極端な値をとる。例えば、10行のメソッドに対して2行追加した場合の変化率は20\%であり、100行のメソッドに対して2行追加した場合の変化率は2\%であるが、この変化率の違いは欠陥混入リスクとは無関係である。

これらのメトリクスを組み合わせることで、変更の規模と複雑さの両方を捉える。

\paragraph{異なる構成要素に基づく特徴量の活用}
メソッド単位のメトリクスが変更の局所的な影響を捉え、コミット単位のメトリクスが変更の全体的な影響を捉えることで、欠陥予測精度が向上する。例えば、あるメソッドの循環的複雑度が大きく増加した場合(メソッド単位)、それが単一ファイルの変更なのか、複数ファイルにまたがる変更の一部なのか(コミット単位)によって、欠陥混入リスクは異なる。異なる構成要素の特徴を考慮することにより、単一の構成要素では捉えられない欠陥の特徴を識別できる。具体的な活用方法については第4章で述べる。
\section{欠陥パターンの学習とモデル評価}
本研究では、3.1節と3.2節で設計した時系列変化を考慮したメトリクスを用いて欠陥予測モデルを構築する。機械学習アルゴリズムとしてランダムフォレストを採用し、モデルの性能を評価する。

\paragraph{ランダムフォレストの採用}
ランダムフォレストを選定した理由は以下の通りである。

第一に、非線形な関係を捉える能力が高い。ソフトウェア欠陥予測において、特徴量と欠陥の有無の関係は複雑であり、単純な線形モデルでは表現できない相互作用が存在する。例えば、変更行数が少なくても変更ファイル数が多い場合は欠陥リスクが高まるといった、複数の特徴量の組み合わせによる非線形な効果を捉える必要がある。

第二に、アンサンブル学習による予測精度の安定性である。ランダムフォレストは複数の決定木の予測を集約するため、単一の決定木と比較して過学習に強く、安定した予測性能を示す。

第三に、モデルの解釈性である。欠陥予測モデルを実際の開発現場で活用するには、なぜそのコミットが欠陥を含むと予測されたのかを理解できる必要がある。ランダムフォレストは、特徴量重要度により各特徴量の予測への寄与度を定量化でき、Partial Dependence Plot(PDP)により特定の特徴量と予測結果の関係を可視化できる。これらの解釈性ツールは、予測モデルの振る舞いを理解し、開発者が予測結果を信頼できるかを判断する上で重要である。

\paragraph{評価指標}
モデルの性能を評価するため、本研究では主にF1スコアを用いる。F1スコアは、適合率と再現率の調和平均として定義される。適合率は欠陥を含むと予測したデータのうち実際に欠陥を含むデータの割合、再現率は実際に欠陥を含むデータのうち欠陥を含むと予測できた割合を表す。F1スコアは両者のバランスを評価するため、不均衡データにおいても適切な評価が可能である。F1スコアを採用する理由は、欠陥予測のために使用するデータが不均衡であるためである。一般的に、欠陥を含むメソッドやコミットの数は、欠陥を含まないメソッドやコミットの数よりも少ない。このような不均衡データでは、正解率のみでは適切な評価ができない。例えば、全てのデータを欠陥なしと予測しても、高い正確率が得られる可能性がある。

\paragraph{統計的仮説検定}
提案手法がベースライン手法と比較して統計的に有意な改善をもたらしているかを検証するため、マクネマー検定を実施する。マクネマー検定は、同じデータセットに対する2つの分類器の予測結果を比較するための検定手法である。この検定により、提案手法による予測性能の改善が偶然ではなく、統計的に有意であることを確認できる。

\paragraph{交差検証}
モデルの汎化性能を評価するため、10分割交差検証を用いる。データセットを10個のサブセットに分割し、そのうち9個を訓練データ、1個をテストデータとして使用する。この過程を10回繰り返し、各サブセットが1度だけテストデータとして使用されるようにする。これにより、特定のデータ分割に依存しない頑健な性能評価が可能となる。

具体的な実装と評価手順については第4章で述べる。
\section{限られた労力での効率的なバグ検出の実現}
実際の開発現場では、レビューに費やせる労力は少ない。本節では、少ないレビュー労力でより多くの欠陥を検出するための手法を提案する。

\paragraph{従来手法の問題点}
従来のレビュー優先度付け手法では、欠陥予測モデルの予測値(0あるいは1)と変更行数に基づいてレビュー対象のコミットを順位付けし、上位のコミットから順にレビューを行う。\cite{kamei2013}。しかし、この手法には課題が2つある。

まず、レビュー労力の推定方法が不正確である。既存手法では変更行数のみからレビュー労力を計算しているが、実際のレビュー労力は変更の複雑度にも依存する。例えば、10個のファイルに分散した100行の変更は、1個のファイルに集中した100行の変更よりもレビュー労力が大きい。なぜなら、変更が複数のファイルにまたがる場合、ファイル間の整合性を確認したり、モジュール同士の関係性を理解したりするために、より多くの労力が必要になるためである。

次に、レビュー労力の制約の設定が不適切である。従来手法では全てのコミットのレビュー労力の和が総労力として設定されているが、数千行の変更を含む巨大なコミットが存在する場合、この設定はレビューの実態を反映しない。なぜなら、実際の開発現場では、レビューに使える労力には実質的な上限があり、巨大なコミットは分割されることが多いためである。

\paragraph{レビュー対象の優先順位付け}
本研究では、レビュー対象の選択を組み合わせ最適化問題として扱う。これにより、従来の順位付け手法では扱えない制約条件を組み込める。

レビュー対象コミットの選択は、以下の特性を持つ。

\begin{itemize}
    \item レビューに費やせる労力(制約条件)の範囲内で
    \item 各コミットのレビューに必要な労力(コスト)を考慮しながら
    \item 欠陥発見期待値(価値)の合計を最大化する
\end{itemize}

したがって、レビュー対象コミットの選択は、「容量上限のある袋に価値と重さを持つ複数のアイテムを入れるとき、総重量が容量を超えないように総価値を最大化すること」であり、組み合わせ最適化問題に置き換えられる。このことから、レビュー対象コミットの選択を、限られたリソース(総労力)の中で欠陥発見の期待値を最大化する問題として考える。

具体的には、レビューを待つ各コミットに対して、レビューを実施するか否かの二値の意思決定を行う。この際、選択された全てのコミットのレビュー労力の合計が、あらかじめ設定した総労力の上限を超えないという制約条件を設ける。この制約の下で、選択されたコミットに含まれる欠陥混入確率(期待値)の総和が最大となるような組み合わせを特定することを目指す。

これにより、単純な順位付けでは考慮が難しい「少ない労力でいかに多くの欠陥混入リスクをカバーするか」というリソース配分の問題を、組み合わせ最適化問題として扱うことができるようになる。

\paragraph{レビュー労力の計算}
変更の規模と複雑度を考慮した上でレビュー労力$W_i$を計算する。

まず、コミット$i$におけるコードの変更行数$C_i$を、追加行数$LA_i$と削除行数$LD_i$の合計として定義する。次に、変更の広がり$H_i$を、以下の情報理論のエントロピーにより定量化する。

$$H_i = -\sum_{k=1}^{n_i} p_k \log_2 p_k$$

ここで、$n_i$はコミット$i$で変更されたファイル数であり、$p_k$は全ての変更行数に対する各ファイル$k$の変更行数の割合である。このエントロピーを$\log_2 n_i$で割ることで正規化し、正規化された変更の広がり$\bar{H}_i$を得る。

これらを用いて、補正済みレビュー労力$W_i$を次のように算出する。

$$W_i = \log_2(C_i \times N_i^{\bar{H}_i} + 1)$$

この対数変換により、巨大なコミットの影響が緩和され、中小規模のコミットも適切に評価できるようになる。

最後に、各コミットのレビューの優先順位を決定する指標として、単位労力あたりの欠陥混入確率を表す密度$D_i$を求める。

$$D_i = \frac{\hat{y}_i}{W_i}$$

\paragraph{貪欲法による求解}
組み合わせ最適化問題を解くため、本研究では貪欲法を採用する。動的計画法は最適解を求められるが、計算量が$O(NC_{total})$であり、容量$C_{total}$が大きい場合は実用的ではない。一方、貪欲法は計算量が$O(N \log N)$であり、容量に依存せず比較的短い時間で解を得られる。

貪欲法では、各コミット$i$の密度$D_i$(単位労力あたりの欠陥発見期待値)を計算し、密度の高い順にコミットを選択する。アルゴリズムは以下の通りである。

\begin{enumerate}
    \item 全コミットについて密度$D_i$を計算する
    \item 密度の降順にコミットをソートする
    \item 累積労力$W_{\text{total}} = 0$とする
    \item ソートされた順にコミットを選択し、以下を実行する:
    \begin{itemize}
        \item $W_{\text{total}} + W_i \leq C_{total}$であれば、コミット$i$をレビュー対象に追加し、$W_{\text{total}} \leftarrow W_{\text{total}} + W_i$とする
        \item そうでなければ、コミット$i$をスキップする
    \end{itemize}
    \item 累積労力が容量を超えるまで、または全てのコミットを検討するまで繰り返す
\end{enumerate}

\paragraph{レビュー労力の上限設定}
実際にレビューに費やせる労力を考慮した上で、容量$C_{total}$を設定する必要がある。本研究では、労力が小さい順に並べたときの上位80\%のコミットの労力の和を$C_{total}$とする。

\begin{enumerate}
    \item 全コミットをレビュー労力$W_i$の昇順にソートする
    \item 上位80\%のコミットを選択する
    \item 選択されたコミットのレビュー労力の和を$C_{total}$とする
\end{enumerate}

この設定により、レビュー労力が非常に大きいコミットを除外することで、実際の開発現場におけるレビュー労力の制約を反映する。80\%というしきい値は実験的に決定したものであり、プロジェクトの特性に応じて変更できる。

\paragraph{評価方法}
レビュー労力に対する欠陥発見数の累積曲線を用いて、提案手法の効果を評価する。この累積曲線は、横軸に「レビューに費やした労力」を配置し、縦軸に「発見した欠陥の数」を配置したものである。

提案手法では、密度の高い順にコミットを選択してレビューする。各コミットをレビューするごとに、レビューに費やした労力の合計と発見した欠陥の数の合計を記録し曲線を描く。比較対象として、ベースラインモデル(変更メトリクスを追加する前のモデル)の予測確率を用いて同様に貪欲法を適用する。

提案手法がベースライン手法よりも左上に位置する曲線を描く場合、同じレビュー労力でより多くの欠陥を発見できることを意味し、実際のレビューにおいて、提案した特徴量が欠陥発見に役立つ可能性があると示唆される。
\clearpage

\chapter{実験環境}
\section{汎用性検証のための対象選定}
\paragraph{データセットの選定}
本研究では、Ferencらが構築したBugHunterデータセット\cite{ferenc2020}を使用する。このデータセットは、GitHub\cite{github}でホストされている15のJavaプロジェクトから自動的に収集されたバグ情報と、各コミットにおけるソースコード要素(ファイル、クラス、メソッド)のメトリクスを含んでいる。

欠陥予測研究では、これまで複数のベンチマークデータセットが用いられてきた。初期の研究では、NASA MDP(Metrics Data Program)データセットやPROMISEデータセットが広く使用され、多くの予測モデルの評価基盤となった。しかし、NASA MDPデータセットは各プロジェクトの単一スナップショットに基づく静的メトリクスを、PROMISEデータセットは主に各プロジェクトのバージョンごとの静的メトリクスを提供するものであり、バグが「どのコミットで」混入したかという時系列情報が欠落している。近年では、Defects4J\cite{just2014}がソフトウェアテスト研究で広く用いられているが、これは「バグ修正の直前・直後($V_{bug}$と$V_{fix}$)」の状態を高度に隔離して提供するデータセットであり、やはりバグ混入時点の情報は含まれていない。

本研究においてBugHunterデータセットを採用した理由は以下の3点に集約される。第一に、バグのライフサイクル全体を捕捉できる点である。PROMISEやNASAデータセットが提供する静的なスナップショットや、Defects4Jが提供する修正前後の状態に対し、BugHunterはバグ混入コミットとバグ修正コミットをペアで特定しており、バグの発生から収束までの時系列を捉えている。これにより、本研究の核となる「同一メソッド内での時系列的な変化量(変更メトリクス)」の算出が可能となる。第二に、分析粒度の適合性である。PROMISEやNASAデータセットは主にモジュールやファイル単位、Defects4Jはファイルやクラス単位での管理が主であるのに対し、BugHunterはメソッドレベルでの豊富なコードメトリクスを最初から提供しており、本研究が目的とする「メソッド単位の欠陥予測」を前処理のコストを抑えつつ直接実施できる。第三に、現代的な開発環境への適応性である。BugHunterはGitHubでホストされる実際のOSSプロジェクトから構築されており、バージョン管理システムと統合された継続的な開発プロセスを反映している点で、PROMISEやNASAの歴史的データセットよりも現代の開発実態に即している。

\paragraph{対象プロジェクトの選定}
有意性検定で帰無仮説が棄却されない問題を回避するため、BugHunterデータセットに含まれる15のプロジェクトの中から、データセットのレコード数が多いプロジェクトを優先的に選定する。

データセットの構築にはSZZアルゴリズム\cite{sliwerski2005}が用いられているため、解決済みのバグレポートの数とデータセットのレコード数の大きさには相関関係がある。そのため、解決済みのバグレポート数を基準として、データセットの相対的な大きさを推測できる。この基準に基づき、解決済みのバグレポート数が多い上位5つのプロジェクトを対象として選定した。

5つという数を選定した根拠は、統計的検定力とデータ多様性のバランスである。3つ以下では、プロジェクト間の多様性が不足し、提案手法の汎化性能を十分に評価できない。一方、10つ以上では、各プロジェクトの詳細な分析が困難になり、計算コストも増大する。5つのプロジェクトは、異なるドメイン(検索エンジン、分散システム、ネットワークフレームワーク、データベース)をカバーしながら、マクネマー検定などの統計的検定を行うのに十分なデータ量を確保できる。

\begin{table}[ht]
\centering
\caption{各プロジェクトのバグレポート数}
\begin{tabular}{|l|r|}
\hline
プロジェクト & バグレポート数 \\
\hline
Elasticsearch\cite{elasticsearch} & 4,287 \\
Hazelcast\cite{hazelcast} & 3,762 \\
Netty\cite{netty} & 2,207 \\
OrientDB\cite{orientdb} & 1,272 \\
Neo4j\cite{neo4j} & 1,152 \\
\hline
\end{tabular}
\end{table}

Elasticsearchは、分散検索・分析エンジンであり、大規模なログデータやテキストデータの検索に広く使用されている。
Hazelcastは、インメモリデータグリッドを提供する分散コンピューティングプラットフォームである。
Nettyは、高性能な非同期イベント駆動型のネットワークアプリケーションフレームワークである。
OrientDBは、マルチモデルデータベースであり、グラフデータベースとドキュメントデータベースの機能を併せ持つ。
Neo4jは、グラフデータベースの代表的な実装の一つであり、ソーシャルネットワーク分析や推薦システムなど、関係性を重視するアプリケーションで広く使用されている。

これらは、いずれもJavaで記述されている活発なOSSプロジェクトである。ドメインは検索エンジン、分散システム、ネットワークフレームワーク、データベースと多岐にわたり、異なる開発特性を持つ。このような多様性により、提案手法が特定のプロジェクトに依存せず、広範なソフトウェアシステムに適用可能であることを検証できる。

\paragraph{正解ラベルの定義}
本研究では、メソッドレベルでの欠陥予測を行う。メソッドレベルを採用した理由は以下の通りである。第一に、レビュアーの実際の作業単位との対応である。コードレビューでは、レビュアーは個々のメソッドやクラスに注目して変更内容を確認する。ファイルレベルやコミットレベルでは粒度が粗すぎて、具体的なレビュー箇所の特定が困難である。第二に、バグの局所性である。多くのバグは特定のメソッド内のロジックエラーや境界条件の誤りに起因しており、メソッドレベルでの予測がバグの本質を捉えやすい。第三に、予測精度の向上である。メソッドレベルでは、個々のメソッドの複雑度や変更量といった詳細な特徴量を利用できるため、ファイルレベルよりも精緻な予測が可能となる。

メソッドレベルでの欠陥予測を行うため、各メソッドに含まれるバグの数を二値化して正解ラベルとする。具体的には、バグの数が0のメソッドを「バグなし」クラス、バグの数が1以上のメソッドを「バグあり」クラスに分類する。この二値分類により、ランダムフォレストのような分類アルゴリズムを適用できる形式にデータを整形する。

\paragraph{データの前処理}
第一に、値が全て同じであるカラムを削除する。これらのカラムは予測に寄与しないため、モデルの複雑性を低減するために除外する。

第二に、メソッドの識別子をベクトルに変換する。メソッド名やクラス名といった識別子は、そのままでは機械学習モデルに入力できないため、トークン分割などの手法を用いて数値ベクトルに変換する。

\paragraph{使用するレコード数の上限}
交差検証時の評価指標の安定化を図るため、各プロジェクトのデータセットから抽出するデータポイント数の上限を5,000件とする。この設定の根拠は以下の通りである。

第一に、統計的信頼性の確保である。当初は3,000件を上限としていたが、この設定では交差検証のフェーズごとにF1スコアが大きく変動し、最悪と最良のフェーズを比較すると0.1前後の誤差が生じていた。データポイント数を5,000件に増やすことで、各フォルドに含まれるバグありサンプルの数が増加し、この変動を抑制できた。10分割交差検証では各フォルドに約500件が含まれるため、不均衡データにおいても少数クラスのサンプルが十分に確保される。

第二に、計算コストとの兼ね合いである。ランダムフォレストの訓練には$O(n \log n)$の計算量が必要であり、データサイズが大きくなるほど計算時間が増大する。5,000件という設定は、1プロジェクトあたり数分程度での訓練を可能にし、複数回の実験を実行できる実用的な範囲である。また、使用するメモリ量も制限されるため、一般的な計算機環境での実行が可能となる。

ただし、この変動は完全には解消されておらず、データの特性に起因する課題として残っている。各プロジェクトのデータセットから、利用可能なデータが5,000件以下の場合は全データを使用し、5,000件を超える場合は最初の5,000件を使用する。データセット内のレコードは必ずしも時系列順に並んでいるわけではないが、同一メソッドの複数のバージョン(バグ混入時と修正時)が含まれており、これらの情報を用いて変化量を計算することが可能である。
\section{コミット間変化の機械学習可能な形式への変換}
3.2節で定義した変更メトリクスを実際のデータセットに適用するため、以下の実装を行う。

\paragraph{メソッド単位の変更メトリクスの計算}
メソッド単位の変更メトリクスとして、コード行数の変化量、トークン数の変化量、循環的複雑度の変化量を算出する。これらは、同一メソッドのバグ混入時とその1つ前のコミットの値の差分として計算される。

具体的には、BugHunterデータセットに含まれる各メソッドについて、バグ混入コミット(または修正コミット)における値と、その直前のコミットにおける値を取得し、以下の計算を行う。

\begin{itemize}
    \item コード行数の変化量 = 現在のコミットのコード行数 - 直前のコミットのコード行数
    \item トークン数の変化量 = 現在のコミットのトークン数 - 直前のコミットのトークン数
    \item 循環的複雑度の変化量 = 現在のコミットの循環的複雑度 - 直前のコミットの循環的複雑度
\end{itemize}

これらの変化量は、メソッドがどの程度変更されたかを直接的に表現する指標となる。

\paragraph{コミット単位の変更メトリクスの計算}
コミット単位の変更メトリクスとして、変更されたファイル数(NF)、追加行数の割合(LA/LT)、削除行数の割合(LD/LT)、1ファイル当たりの平均行数(LT/NF)を生成する。これらは、BugHunterデータセットに含まれるコミット情報から以下のように計算される。

\begin{itemize}
    \item 変更されたファイル数(NF): コミットで変更されたファイルの総数
    \item 追加行数の割合(LA/LT): $\frac{\text{追加行数}}{\text{変更前の総行数}}$
    \item 削除行数の割合(LD/LT): $\frac{\text{削除行数}}{\text{変更前の総行数}}$
    \item 1ファイル当たりの平均行数(LT/NF): $\frac{\text{変更対象ファイルの総行数}}{\text{変更ファイル数}}$
\end{itemize}

これらの変化率ベースのメトリクスは、コミット全体での変更の影響範囲や性質を捉える。

\paragraph{メソッドの操作タイプのラベル付与}
各メソッドに対して、そのメソッドが追加、変更、削除のいずれの操作を受けたかを示すラベルを付与する。これらの操作タイプは、バグ混入コミットあるいはバグ修正コミットと、その直前のコミットを比較することで判定される。

\begin{itemize}
    \item 追加(Add)
    \begin{itemize}
        \item 直前のコミットには存在せず、当該コミットで新たに追加されたメソッド
    \end{itemize}
    \item 変更(Modify)
    \begin{itemize}
        \item 直前のコミットと当該コミットの両方に存在し、内容が変更されたメソッド
    \end{itemize}
    \item 削除(Delete)
    \begin{itemize}
        \item 直前のコミットには存在したが、当該コミットで削除されたメソッド
    \end{itemize}
\end{itemize}

これらのラベルは、カテゴリカル変数として扱い、One-Hotエンコーディングを用いて数値ベクトルに変換する。例えば、「追加」は [1, 0, 0]、「変更」は [0, 1, 0]、「削除」は [0, 0, 1] のようにエンコードされる。

\paragraph{メソッド識別子の処理}
メソッドの完全修飾名(例: org.elasticsearch.index.fielddata.plain.GeoPointDoubleArrayAtomicFieldData\$Empty.<init>()V)は、そのままでは機械学習モデルに入力できないため、トークン分割を行う。

具体的には、以下の手順で処理する。

第一に、正規表現を用いてメソッドシグネチャからパッケージ名、クラス名、メソッド名を抽出する。

第二に、パッケージ名を.で分割し、各部分をトークンとする。例えば、"org.elasticsearch.index"は["org", "elasticsearch", "index"]に分割される。

第三に、クラス名を\$で分割し、各部分をキャメルケース分割する。例えば、"GeoPointDoubleArrayAtomicFieldData"は["Geo", "Point", "Double", "Array", "Atomic", "Field", "Data"]に分割される。

第四に、メソッド名をスネークケース(\_や-)で分割した後、さらにキャメルケース分割する。ただし、<init>や<clinit>などの特殊なメソッド名は"constructor"というトークンに変換する。

第五に、全てのトークンを小文字に変換する。

さらに、"java"、"util"、"get"、"set"などのJava言語における一般的な単語や、最小トークン長(3文字)未満のトークンをストップワードとして除外する。これにより、メソッドやクラスの本質的な意味を表す有用なトークンのみが抽出される。

抽出されたトークンは、語彙辞書を用いて数値インデックスに変換され、さらにベクトル表現に変換される。この処理により、メソッドが属するパッケージやクラスの名前、メソッド名自体が持つ意味的な情報を特徴量として活用できる。

\paragraph{特徴量の統合}
最終的に、以下の特徴量を統合してモデルの入力とする。

\begin{itemize}
    \item BugHunterデータセットに元々含まれるコードメトリクス
    \item メソッド単位の変更メトリクス(変化量)
    \item コミット単位の変更メトリクス(変化率)
    \item メソッドの操作タイプ(One-Hotエンコーディング)
    \item メソッド識別子のトークンベクトル
\end{itemize}

これらを組み合わせることで、コードの静的な性質、時系列変化、操作の種類、意味的な情報を包括的に捉えた特徴量セットを構築する。
\section{提案手法の効果を検証する比較モデルの準備}
3.3節で述べたランダムフォレストを用いて、欠陥予測モデルを構築する。本節では、ランダムフォレストの実装とモデル構築に必要な準備について述べる。

\paragraph{ランダムフォレストの実装}
ランダムフォレストは、複数の決定木を構築し、それらの予測結果を集約することで高い予測精度と汎化性能を実現するアンサンブル学習手法である。

図\ref{fig:decision_tree}に決定木の構造を示す。決定木は、各ノードにおける特徴量のしきい値に基づいてデータを分割し、葉ノードで最終的な予測クラスを出力する。しかし、単一の決定木は訓練データを過学習しやすく、汎化性能が低下する傾向がある。

\begin{figure}[tb]
  \centering
  \includegraphics[width=0.7\linewidth]{figures/decision_tree.pdf}
  \caption{決定木の構造}
  \label{fig:decision_tree}
\end{figure}

ランダムフォレストでは、この問題を解決するためにブートストラップサンプリングを用いる。図\ref{fig:bootstrap_sampling}に示すように、元の訓練データセットから復元抽出により複数の異なるサンプルを生成し、各サンプルに対して独立に決定木を学習する。

\begin{figure}[tb]
  \centering
  \includegraphics[width=0.85\linewidth]{figures/bootstrap_sampling.pdf}
  \caption{ブートストラップサンプリングによる訓練データの生成}
  \label{fig:bootstrap_sampling}
\end{figure}

図\ref{fig:random_forest}に、ランダムフォレストによる分類の流れを示す。各決定木は独立に予測を行い、分類問題では多数決により最終的な予測クラスが決定される。また、各分岐点ではランダムに選択された特徴量の部分集合のみが使用されるため、決定木間の相関が低減され、モデルの汎化性能が向上する。

\begin{figure}[tb]
  \centering
  \includegraphics[width=0.95\linewidth]{figures/random_forest_classification.pdf}
  \caption{ランダムフォレストによる分類}
  \label{fig:random_forest}
\end{figure}

\paragraph{モデル構築の段階的アプローチ}
提案する特徴量の有効性を検証するため、段階的にモデルを構築する。各ステップで特徴量を追加することにより、その特徴量群が予測性能にどの程度寄与するかを定量的に評価可能である。

以下の3段階でモデルを構築する。

\textbf{ステップ1(既存手法)}: BugHunterデータセットに元々含まれている構造的メトリクスのみを特徴量として使用する。このモデルを既存手法のモデル(後続のモデルとの比較基準)とする。

\textbf{ステップ2}: 既存手法のモデルに対して、メソッド単位の変更メトリクスを追加する。これにより、メソッド単位の時系列変化を考慮することの効果を評価する。

\textbf{ステップ3(提案手法)}: ステップ2のモデルに対して、さらにコミット単位の変更メトリクスを追加する。これにより、局所的な観点と全体的な観点の分析を組み合わせた場合の効果を評価する。

この段階的評価により、1)時系列変化を考慮することで予測性能は向上するか、2)メソッド単位とコミット単位のメトリクスを活用することで、さらなる改善が得られるか、3)各特徴量群の相対的な重要性はどの程度かが明らかになる

\paragraph{モデルのクラス分類の分析手法}
モデルのクラス分類を理解するため、以下の分析手法を準備する。

\textbf{特徴量重要度}: ランダムフォレストが提供する特徴量重要度は、各特徴量が決定木の分岐においてどの程度情報利得をもたらしたかを示す指標である。この値により、どの特徴量が欠陥予測に重要であるかを定量的に評価可能である。

\textbf{Partial Dependence Plot(PDP)}: PDPは、特定の特徴量の値を変化させたときのモデルの予測値の平均的な変化を可視化する手法である。図\ref{fig:pdp}に示すように、特徴量$x_s$に対するPartial Dependence関数は、他の特徴量の値を固定した状態で$x_s$を変化させた際の予測値の平均として計算される。

\begin{figure}[tb]
  \centering
  \includegraphics[width=0.9\linewidth]{figures/partial_dependence_plot.pdf}
  \caption{Partial Dependence Plotの計算方法}
  \label{fig:pdp}
\end{figure}

特徴量$x_s$に対するPartial Dependence関数$f_{\text{PD}}(x_s)$は以下のように定義される。

$$f_{\text{PD}}(x_s) = \frac{1}{n}\sum_{i=1}^{n}\hat{f}(x_s, x_{-s}^{(i)})$$

ここで、$\hat{f}$は学習されたモデル、$x_{-s}^{(i)}$は$i$番目のサンプルにおける特徴量$x_s$以外の特徴量の値を表す。

\textbf{決定木の可視化}: ランダムフォレストを構成する決定木の一つを可視化することで、どのような分類条件で欠陥の有無を判断しているかを確認することが可能である。
\section{労力制約下での評価指標の設計}
3.4節で定義したナップサック問題アプローチに基づき、レビュー労力の計算を実装する。本節では、実装の詳細とデータセットへの適用について述べる。

\paragraph{実装の詳細}
各コミット $i$ のレビュー労力 $W_i$ を計算するため、以下の手順を実行する。

まず、コミットの基本情報を取得する。
\begin{itemize}
    \item 追加されたコード行数 $LA_i$
    \item 削除されたコード行数 $LD_i$
    \item 変更されたファイル数 $N_i$
    \item 各ファイル $k$ の変更行数
\end{itemize}

次に、3.4節で定義した計算式を順次適用する。

コードの変更行数 $C_i$ を計算する。
\[
C_i = LA_i + LD_i
\]

各ファイルが変更全体に占める割合 $p_k$ を計算する。
\[
p_k = \frac{\text{file}_k \text{の変更行数}}{\text{全変更行数}}
\]

変更の広がり $H_i$ を計算する。
\[
H_i = -\sum_{k=1}^{n_i} p_k \log_2 p_k
\]

変更の広がりを正規化する。
\[
H_i^{\text{norm}} = \frac{H_i}{\log_2 n_i}
\]

ベース労力 $E_{\text{raw}, i}$ を計算する。
\[
E_{\text{raw}, i} = C_i \times N_i^{H_i^{\text{norm}}}
\]

対数変換により補正済み労力 $W_i$ を計算する。
\[
W_i = E_{\text{adj}, i} = \ln(E_{\text{raw}, i} + 1)
\]

\paragraph{レビュー総労力の設定}
3.4節で定義した方法に従い、レビューに使える総労力 $C_{total}$ を設定する。全コミットをレビュー労力 $W_i$ の昇順にソートし、上位80\%のコミットの労力の和を $C_{total}$ とする。

\paragraph{密度の計算}
3.4節で定義した密度 $D_i$ を計算する。

\[
D_i = \frac{V_i}{W_i} = \frac{\hat{y}_i}{E_{\text{adj}, i}}
\]

ここで、$\hat{y}_i$ はモデルが予測したコミット $i$ の欠陥混入確率、$E_{\text{adj}, i}$ はコミット $i$ の補正済みレビュー労力である。密度は貪欲法におけるレビュー対象の優先順位付けに用いられる。

\paragraph{データセットへの適用}
BugHunterデータセットに含まれる各コミットについて、上記の手順に従ってレビュー労力 $W_i$ と密度 $D_i$ を算出する。これらの値は、貪欲法によるレビュー対象選択の入力として使用される。

\paragraph{統計的仮説検定}
まず、二値分類の予測性能(F1スコア)について、提案手法とベースライン手法の間に統計的に有意な差があるかを検証するため、マクネマー検定を用いる。マクネマー検定は、対応のある2つの分類器の性能を比較する際に広く用いられる手法であり、各サンプルに対する2つのモデルの予測結果の一致・不一致パターンを分析する。

次に、レビュー労力に対する欠陥発見率について、提案手法がベースライン手法より統計的に有意に優れているかを検証するため、ウィルコクソンの符号順位検定を用いる。本研究では、レビュー労力20\%と40\%の時点における欠陥発見率を評価対象とする。ウィルコクソンの符号順位検定は、対応のある2標本の比較に用いられるノンパラメトリック検定である。本研究では、5つのプロジェクトを独立したサンプルとして扱い、各プロジェクトにおけるベースライン手法と提案手法の欠陥発見率のペアを比較する。この検定により、複数のプロジェクトにわたって提案手法が一貫してベースライン手法を上回るかを統計的に検証できる。

サンプルサイズが5と小さく、データの正規性が保証されないため、正規性を仮定しないノンパラメトリック検定であるウィルコクソンの符号順位検定が適切である。また、本研究の目的は「提案手法がベースライン手法より優れている」ことの検証であるため、片側検定を採用する。

すべての統計的検定において、有意水準は0.05とする。
\clearpage

\chapter{実験結果}
\section{段階的評価による各要素の寄与度検証}
\subsection{評価手順}
本研究では、提案する変更メトリクスの各要素が予測性能に与える影響を明確にするため、段階的な評価を実施する。本節では、評価手順を順に述べる。

\paragraph{データ分割}
まず、各プロジェクトのデータセットを訓練データ80\%、テストデータ20\%に分割する。訓練データはモデルの学習に使用し、テストデータは最終的な性能評価に使用する。テストデータは訓練過程で一切使用しないため、未知データに対する汎化性能を測定可能である。

\paragraph{10分割交差検証}
次に、訓練データに対して10分割交差検証を実施し、モデルの安定性を評価する。交差検証は、データセットを複数のサブセットに分割し、それぞれのサブセットを訓練データあるいは評価データとして用いる手法である。モデルを訓練し、検証データでF1スコア、適合率、再現率を計算する。この過程をサブセットの数だけ繰り返し、各サブセットが1度だけ検証データとして使用されるようにする。交差検証で得られた全ての評価結果の平均値と標準偏差を算出する。標準偏差が小さいほど、データの分割方法に依存しない安定したモデルであることを示す。

\paragraph{評価指標}
評価指標として、3.3節で定義した適合率、再現率、F1スコア、およびAUC(Area Under the Curve)を用いる。適合率は欠陥を含むと予測したコミットのうち実際に欠陥を含む割合、再現率は実際に欠陥を含むコミットのうち正しく予測できた割合を示す。F1スコアは適合率と再現率の調和平均であり、陽性クラス(欠陥を含むコミット)と陰性クラス(欠陥を含まないコミット)の不均衡が大きいデータセットにおいて、モデルの総合的な予測性能を評価するのに適している。AUCは分類器の性能をしきい値によらず評価する指標であり、1に近いほど性能が高い。

\paragraph{最終評価}
交差検証により性能が確認されたモデルを、訓練データ全体で再学習する。その後、このモデルをテストデータに適用し、最終的な性能指標を算出する。

\paragraph{段階的評価の実施}
評価は以下の3段階で実施する。第1段階(既存手法、ステップ1)ではBugHunterデータセットの元のメトリクスのみを使用する。これは構造的メトリクスを中心とした特徴量である。第2段階(ステップ2)ではメソッド単位の変更メトリクスを追加する。これには、メソッドの変更タイプ(追加、削除、修正)やメソッド単位のコード行数の変化量などが含まれる。第3段階(提案手法、ステップ3)ではコミット単位の変更メトリクスをさらに追加する。これには、コミットの変更ファイル数、追加・削除コード行数などが含まれる。各段階の評価結果を比較することで、メソッド単位とコミット単位の変更メトリクスがそれぞれどの程度性能向上に寄与するかを明らかにする。

\paragraph{統計的有意性の検証}
提案手法(ステップ3)と既存手法(ステップ1)の機械学習モデルの性能差が統計的に有意であることを、McNemar検定により検証する。McNemar検定は、同じテストデータに対する2つの分類器の予測結果を比較する検定手法である。この検定では、まず2つのモデルの予測結果から分割表を作成する。分割表では、既存手法で正分類であり提案手法で誤分類であったサンプル数を$n_{12}$、提案手法で正分類であり既存手法で誤分類であったサンプル数を$n_{21}$とする。検定統計量は以下で計算される。

\[
\chi^2 = \frac{(n_{12} - n_{21})^2}{n_{12} + n_{21}}
\]

有意水準0.05で、$p < 0.05$の場合に性能差が統計的に有意であると判断する。

\paragraph{レビュー労力削減効果の評価}
提案手法によるレビュー労力削減効果を評価するため、レビュー労力に対する欠陥発見数の累積曲線を作成する。累積曲線は、横軸に投入したレビュー労力、縦軸に発見した欠陥数をプロットしたグラフである。提案手法の曲線が既存手法より左上に位置する場合、同じ労力でより多くの欠陥を発見可能であることを意味する。

\paragraph{貪欲法によるレビュー対象の選択}
累積曲線を作成するため、各モデル(ステップ1、ステップ2、ステップ3)について、貪欲法によりレビュー対象コミットを選択する。貪欲法は、各段階で局所的に最適な選択を行うアルゴリズムである。このアルゴリズムでは、まず各コミット$i$のレビュー労力を4.4節の方法で計算する。次に、全コミットをレビュー労力の昇順にソートし、上位80\%のコミットの労力の和を総労力$C_{total}$として設定する。各コミット$i$について、モデルが予測した欠陥混入確率$\hat{y}_i$と補正済み労力$W_i$から密度を計算する。

\[
D_i = \frac{\hat{y}_i}{W_i}
\]

密度$D_i$の降順にコミットをソートする。累積労力$W_{\text{total}} = 0$とする。ソートされた順にコミットを検討し、$W_{\text{total}} + W_i \leq C_{total}$であればコミット$i$をレビュー対象に追加し、$W_{\text{total}} \leftarrow W_{\text{total}} + W_i$とする。そうでなければスキップする。累積労力が容量を超えるまで、または全コミットを検討するまで繰り返す。各コミットをレビューするごとに、累積レビュー労力と累積欠陥発見数を記録する。この手順により、少ないレビュー労力で欠陥発見期待値を最大化するレビュー対象を選択することが可能である。

\paragraph{累積曲線の作成}
累積曲線は以下の手順で作成する。各モデルについて、貪欲法によりレビュー対象コミットを選択する。各コミットをレビューする順に、累積レビュー労力と累積欠陥発見数を記録する。横軸を累積レビュー労力、縦軸を累積欠陥発見数として、各モデルの曲線を同一グラフ上に描画する。総労力の20\%, 40\%時点での欠陥発見数を比較する。

\paragraph{評価指標の定義}
評価指標として、欠陥発見率と改善幅を用いる。欠陥発見率は、特定労力時点までに発見した欠陥の数を全欠陥数で割った値であり、以下の式で定義される。

\[
\text{欠陥発見率} = \frac{\text{発見した欠陥の数}}{\text{全欠陥数}} \times 100
\]

改善幅は、提案手法(ステップ3)と既存手法(ステップ1)の欠陥発見率の差であり、以下の式で定義される。

\[
\text{改善幅} = \text{提案手法の欠陥発見率} - \text{既存手法の欠陥発見率}
\]

\paragraph{特徴量重要度の算出}
ランダムフォレストが提供する特徴量重要度を用いて、各特徴量の予測への寄与度を定量化する。特徴量重要度は、各特徴量が決定木の分岐においてどの程度情報利得をもたらしたかを示す指標である。情報利得は、ある特徴量で分岐することにより、データの不純度(通常、ジニ不純度で表される)がどの程度減少するかを測る指標である。ランダムフォレストでは、全ての決定木における各特徴量の情報利得の平均を計算することで特徴量重要度を算出する。本研究では、この指標を用いて、訓練済みの機械学習モデルから各特徴量の重要度を取得する。さらに、各特徴量を特徴量重要度の降順にソートする。上位の特徴量を抽出し、棒グラフで可視化する。これにより、メソッド単位、コミット単位、元のメトリクスという各カテゴリーの特徴量がどの程度重要であるかを分析する。

\paragraph{Partial Dependence Plotの生成}
各特徴量と欠陥混入確率の関係を可視化するため、Partial Dependence Plot(PDP)を生成する。PDPは、特定の特徴量の値を変化させたときに、モデルの予測値がどのように変化するかを示すグラフである。特徴量$x_s$に対するPartial Dependence関数$f_{\text{PD}}(x_s)$を求めるには、まず特徴量$x_s$の値を固定値に設定する。他の全ての特徴量$x_{-s}$は、各サンプルの実際の値を使用する。全サンプルについて予測値を計算し、その平均を取る。

\[
f_{\text{PD}}(x_s) = \frac{1}{n}\sum_{i=1}^{n}\hat{f}(x_s, x_{-s}^{(i)})
\]

特徴量$x_s$の値を変化させながらこの計算を繰り返す。この結果、横軸を特徴量$x_s$の値、縦軸を予測確率とするグラフが得られる。このグラフを用いることで、特定の特徴量が増加した際に欠陥混入確率がどのように変化するかを視覚的に理解することが可能になる。このグラフを生成するために、まず特徴量重要度が上位の特徴量を選択する。各特徴量の値の範囲を等間隔に分割する。各点において上記の計算方法でPartial Dependence値を計算する。横軸を特徴量の値、縦軸を予測確率として、各特徴量のPDPをグラフ化する。

\paragraph{決定木の可視化}
ランダムフォレストを構成する決定木の構造を可視化し、どのような分類条件で欠陥の有無を判断しているかを確認する。決定木の構造を可視化するには、まずランダムフォレストから代表的な決定木を1つ選択する。決定木の各ノードにおける分岐条件(特徴量としきい値)を抽出する。各ノードのサンプル数とクラス分布を取得する。グラフ描画ライブラリを用いて、決定木を描画する。
\section{提案手法による予測性能の向上}
提案手法の予測性能を3段階で評価した。表\ref{tab:evaluation}に、5つのプロジェクトにおけるテストデータでの評価結果を示す。

\begin{table}[ht]
\centering
\caption{テストデータによる評価}
\label{tab:evaluation}
\begin{tabular}{|l|l|r|r|}
\hline
プロジェクト & モデル & F1スコア & AUC \\
\hline
Elasticsearch & ベースライン & 0.575 & 0.775 \\
 & ステップ2 & 0.708 & 0.880 \\
 & ステップ3 & \textbf{0.767} & \textbf{0.925} \\
\hline
Hazelcast & ベースライン & 0.678 & 0.875 \\
 & ステップ2 & 0.733 & 0.900 \\
 & ステップ3 & \textbf{0.790} & \textbf{0.932} \\
\hline
Neo4j & ベースライン & 0.478 & 0.713 \\
 & ステップ2 & 0.616 & 0.8300 \\
 & ステップ3 & \textbf{0.742} & \textbf{0.938} \\
\hline
Netty & ベースライン & 0.455 & 0.705 \\
 & ステップ2 & 0.661 & 0.882 \\
 & ステップ3 & \textbf{0.747} & \textbf{0.934} \\
\hline
OrientDB & ベースライン & 0.483 & 0.738 \\
 & ステップ2 & 0.524 & 0.765 \\
 & ステップ3 & \textbf{0.701} & \textbf{0.914} \\
\hline
\end{tabular}
\end{table}

全プロジェクトでステップを経るごとに性能が一貫して向上した。ベースラインから提案手法(ステップ3)へのF1スコアの改善幅は、Hazelcastの0.11からNettyの0.29まで分布し、平均改善幅は0.21であった。AUCは全プロジェクトで0.91を超え、高い識別性能を達成した。特にNeo4jとNettyでは、ベースラインのF1スコアが0.48程度と低かったが、提案手法で0.74程度まで向上し、欠陥予測の実用性が大きく改善された。

ステップ2からステップ3への改善も全プロジェクトで確認され、コミット単位の変更メトリクスが予測に役立つ情報を提供することが示された。改善幅はプロジェクトによって異なり、Neo4jでは0.13、Nettyでは0.09であった一方、OrientDBでは0.18と最大の改善を示した。これは、プロジェクトの特性によって、メソッド単位とコミット単位の各メトリクスの価値が異なることを示唆している。

McNemar検定により、提案手法とベースラインの性能差の統計的有意性を検証した。全プロジェクトで$p < 0.05$となり、性能向上は統計的に有意であることが確認された。

\paragraph{交差検証による安定性}
10分割交差検証により、データの分割に依存しない安定性を評価した。表\ref{tab:crossvalidation}に結果を示す。

\begin{table}[ht]
\centering
\caption{10分割交差検証の結果(平均値 ± 標準偏差)}
\label{tab:crossvalidation}
\begin{tabular}{|l|l|r|r|}
\hline
プロジェクト & モデル & F1スコア & AUC \\
\hline
Elasticsearch & ステップ3 & 0.791 ± 0.024 & 0.931 ± 0.016 \\
Hazelcast & ステップ3 & 0.783 ± 0.019 & 0.931 ± 0.011 \\
Neo4j & ステップ3 & 0.739 ± 0.033 & 0.936 ± 0.011 \\
Netty & ステップ3 & 0.708 ± 0.028 & 0.925 ± 0.016 \\
OrientDB & ステップ3 & 0.664 ± 0.025 & 0.899 ± 0.019 \\
\hline
\end{tabular}
\end{table}

各プロジェクトでF1スコアの標準偏差は0.02から0.03程度、AUCの標準偏差は0.01から0.02程度に収まっており、提案手法はデータの分割方法に大きく依存せず安定した性能を発揮することが確認された。
\section{提案したメトリクスの高い予測寄与度の確認}
モデルが予測において重視した特徴量を明らかにするため、特徴量重要度を分析した。

分析の結果、以下の傾向が全てのプロジェクトで共通して確認された。

\begin{enumerate}
    \item \textbf{コミット規模の影響:} \texttt{lines\_added}(追加行数)や\texttt{num\_files}(変更ファイル数)が上位を占めた。これは、変更規模の大きさが欠陥混入リスクの主要な指標であることを示している。
    \item \textbf{変更の分散:} \texttt{entropy}も上位に位置し、変更が特定の箇所に集中しているか、分散しているかが重要であることが示唆された。
    \item \textbf{メソッド単位の詳細:} Netty等では\texttt{tokens\_change}(トークン変化量)が最重要となるなど、粒度の細かい変更情報も予測に大きく寄与していた。
\end{enumerate}

図\ref{fig:feature_importance_neo4j}に、代表例としてNeo4jプロジェクトにおける重要度上位の特徴量を示す。

\begin{figure}[ht]
\centering
\includegraphics[width=0.7\textwidth]{figures/neo4j/feature_importance_chart.png}
\caption{特徴量重要度(代表例: Neo4j)}
\label{fig:feature_importance_neo4j}
\end{figure}
\section{提案メトリクスの分布特性と予測への示唆}
Feature Importance上位の特徴量について統計情報を分析した結果、コミット単位のメトリクス(\texttt{lines\_added}等)は、大多数の値は小さいがごく一部に極端に大きな値がある偏りの大きい分布を示した。これは、少数の大規模な変更がデータセットに含まれていることを反映している。一方、メソッド単位のメトリクスは中央値が0であり、多くのメソッドは変更されないか、微細な変更に留まることが確認された。
\section{特徴量と予測確率の関係性の解明}
Partial Dependence Plot(PDP)を用いて、重要な特徴量と欠陥混入確率の関係を分析した。PDPは特定の特徴量の値を変化させたときに予測確率がどのように変化するかを示す。

コミット単位の変更メトリクスにおいて、値が増加するにつれて欠陥混入確率が低下する傾向が全プロジェクトで一貫して観察された。メソッド単位のメトリクスでは、\texttt{tokens\_change}(トークン数の変化量)が0付近に近づくにつれて、欠陥混入確率が上昇する傾向が確認された。

これらの分析から、コミット全体の規模が小さく、欠陥にかかわるメソッドへの変更も小さいという組み合わせが、最も高い欠陥混入リスクを示すことが明らかになった。
\section{直感的に理解可能な欠陥予測基準の抽出}
ランダムフォレストを構成する決定木を可視化し、モデルがどのように欠陥の有無を予測しているかを分析した。決定木の可視化により、特徴量重要度だけでは把握できない具体的な分類条件を抽出可能である。

提案手法(ステップ3)の決定木では、ルートノード付近で\texttt{lines\_added}(追加行数)や\texttt{tokens\_change}(トークン数の変化量)といった変更メトリクスが頻繁に使用されていた。これらは、コミットの規模やメソッドの変化量という時系列的な情報を表す特徴量である。決定木の浅い階層でこれらの特徴量が選択されることは、コードの構造的な変化が欠陥予測において重要な特徴となっていることを示している。
\section{同一労力でのバグ発見数増加の実証}
提案手法の有効性を評価するため、レビュー労力に対する欠陥発見数を分析した。表\ref{tab:cost_benefit_summary}に、レビュー労力を20\%および40\%に制限した場合の欠陥発見率を示す。

\begin{table}[ht]
\centering
\caption{レビュー労力20\%および40\%時点での欠陥発見率}
\label{tab:cost_benefit_summary}
\small
\begin{tabular}{|l|r|r|r|r|}
\hline
 & \multicolumn{2}{c|}{レビュー労力 20\%時} & \multicolumn{2}{c|}{レビュー労力 40\%時} \\
\cline{2-5}
プロジェクト & ベースライン & \textbf{提案手法} & ベースライン & \textbf{提案手法} \\
\hline
Elasticsearch & 64.7\% & \textbf{68.1\%} & 78.8\% & \textbf{86.6\%} \\
Hazelcast & 56.1\% & \textbf{56.1\%} & 77.6\% & \textbf{80.0\%} \\
Neo4j & 53.0\% & \textbf{68.8\%} & 72.5\% & \textbf{89.5\%} \\
Netty & 55.8\% & \textbf{75.6\%} & 77.4\% & \textbf{93.5\%} \\
OrientDB & 54.0\% & \textbf{65.8\%} & 72.6\% & \textbf{85.7\%} \\
\hline
\end{tabular}
\end{table}

全プロジェクトで提案手法がベースラインを上回り、同一のレビュー労力における欠陥発見率の増加が確認された。レビュー労力20\%の時点では、ベースラインが平均56.7\%の欠陥を発見するのに対し、提案手法では66.9\%を発見し、10.2ポイントの改善を達成した。レビュー労力40\%では、ベースラインの75.8\%に対し提案手法は87.0\%となり、11.3ポイントの改善が見られた。

特にNeo4jとNettyでは顕著な効果が確認された。Neo4jではレビュー労力20\%時点で15.8\%、40\%時点で17.0\%の改善を示した。Nettyでは20\%時点で19.8\%、40\%時点で16.1\%の改善となり、わずか20\%の労力で全体の75\%以上の欠陥を検出できるようになった。一方、Hazelcastでは20\%時点で改善が見られなかったものの、40\%時点では2.4\%の改善が確認された。

図\ref{fig:comparison_cost_benefit_neo4j}に、Neo4jにおけるレビュー労力に対する欠陥発見数の累積曲線を示す。提案手法の曲線がベースラインより左上に位置しており、同一労力でより多くの欠陥を発見できることが視覚的に確認できる。

\begin{figure}[ht]
\centering
\includegraphics[width=0.7\textwidth]{figures/neo4j/comparison_cost_benefit_curve.png}
\caption{レビュー労力に対する欠陥発見数の比較(代表例: Neo4j)}
\label{fig:comparison_cost_benefit_neo4j}
\end{figure}

\paragraph{統計的有意性の検証}
ウィルコクソンの符号順位検定により、改善の統計的有意性を検証した。表\ref{tab:wilcoxon_results}に結果を示す。

\begin{table}[ht]
\centering
\caption{欠陥発見率に対するウィルコクソンの符号順位検定の結果}
\label{tab:wilcoxon_results}
\begin{tabular}{|l|r|r|r|l|}
\hline
労力 & ベースライン & 提案手法 & 平均改善幅 & p値 \\
\hline
20\% & 56.7\% & 66.9\% & +10.2\% & 0.0625 \\
40\% & 75.8\% & 87.0\% & +11.3\% & 0.0313 \\
\hline
\end{tabular}
\end{table}

レビュー労力40\%の時点で統計的に有意な改善が確認された($p=0.0313$)。一方、20\%の時点では有意差は認められなかった($p=0.0625$)。これは、20\%時点ではプロジェクト間のばらつきが大きく、Hazelcastで改善が見られなかった一方でNettyやNeo4jでは大幅な改善が見られたためである。
\clearpage

\chapter{考察}
\section{観察された現象と疑問の整理}
第5章で得られた主要な結果を簡潔に要約し、本章で考察すべき疑問を明確化する。提案手法(ステップ3)は、5つのプロジェクトすべてでベースラインと比較してF1スコアが向上した。Cost-Benefit分析では、20\%のレビュー労力で全バグの70〜75\%を検出可能であることが示された。しかし、これらの結果からいくつかの疑問が生じる。

\paragraph{なぜソフトウェアの変更サイズが小さく変更ファイル数が少ないほど欠陥混入確率が高くなるのか?}
Partial Dependence Plot分析により、追加行数や変更ファイル数などの変更規模を表すメトリクスの値が増加するにつれて、バグ混入確率が低下する傾向が全プロジェクトで一貫して観察された。これは直感に反する結果である。

\paragraph{なぜプロジェクトごとに欠陥予測に影響を与える特徴量が異なるのか?}
特徴量重要度を分析した結果、Nettyではトークン数の変化量が最重要特徴量であったのに対し、他のプロジェクトでは追加行数や変更ファイル数が上位を占めた。プロジェクト間で重要な特徴量が異なる理由は何か。

\paragraph{なぜプロジェクト間で欠陥予測の精度が異なるのか?}
F1スコアの改善幅はNettyで0.29、Hazelcastで0.11と最大2.6倍の差が見られた。この性能差は何に起因するのか。

\paragraph{なぜコミット単位とメソッド単位の変更メトリクスを統合すると性能が向上するのか?}
ステップ2(メソッド単位のみ)からステップ3(コミット単位を追加)への性能向上のメカニズムは何か。

\paragraph{本研究の評価方法とデータセットは、これらの疑問に答えるために妥当か?}
SZZアルゴリズムやBugHunter Datasetの制約が結果にどのように影響しているか。

本章では、これらの疑問に対して既存研究の知見と本研究のデータを用いて考察する。
\section{変更規模と欠陥混入確率の関係}
第5章のPartial Dependence Plot分析で観察された「変更規模が小さいほど欠陥混入確率が高い」という傾向について考察する。この現象は全5プロジェクトで一貫して確認されており、偶然ではなく何らかの原因が存在すると考えられる。

本現象を説明する可能性として、以下の仮説が考えられる。

\paragraph{欠陥密度とサンプル分布の影響}
この結果は、一見すると直感に反するように思われる。一般的に、ソフトウェア工学の研究では、変更規模が大きいほど欠陥密度(単位コード量あたりの欠陥数)が高くなることが知られている。しかし、本研究で構築した機械学習モデルが予測しているのは欠陥密度ではなく、各コミットが欠陥を含む確率である。

この2つの概念の違いを理解する鍵は、実際のソフトウェア開発におけるコミットサイズの分布にある。ソフトウェア開発では、変更の大部分が小規模であり、大規模な変更は相対的に少数である。機械学習モデルは訓練データにおける欠陥の絶対的な出現パターンから学習するため、サンプル数が多い小規模変更の領域において、より多くの欠陥例とその特徴パターンを学習することになる。したがって、個々のコミットの欠陥密度が低くても、サンプル数が圧倒的に多ければ、その領域で観察される欠陥の絶対数は多くなり、結果として予測モデルはこの領域で高い欠陥混入確率を出力する可能性がある。

言い換えれば、大規模変更は欠陥密度が高い可能性があるものの、その絶対数が少ないため、モデルの予測において支配的な影響を持たない。一方、小規模変更は密度こそ低いが、その数の多さゆえに多くの欠陥を含んでおり、モデルはこのパターンを学習していると考えられる。

\paragraph{変更の性質と目的の違い}
小規模な変更と大規模な変更では、その目的や性質が根本的に異なる可能性がある。
Hindleらは、大規模なコミットをカテゴリーごとに分類することでコミットの性質を分析し、変更の規模によって変更の種類に顕著な違いがあることを示した\cite{hindle2008}。

彼らの調査によると、適応的(Adaptive)な変更はいずれの規模においても最も頻繁に見られたが、小規模な変更と大規模な変更を比較すると興味深い逆転現象が確認された。小規模な変更では、完全な(Perfective)変更よりも修正的な(Corrective)変更が行われる傾向が強かったのに対し、大規模な変更では、完全な変更の方が多く含まれていた。

この結果について、Hindleらは直感的な解釈を与えている。すなわち、エラーの修正はしばしば「外科手術的」で局所的な小規模変更にとどまることが多い一方で、システムの改善や再構築を伴う完全な変更は、その影響範囲が広く、多くのファイルに波及するため大規模な変更になりやすいと考えられる。一見すると、修正的な変更である小規模な変更の方が欠陥混入確率がより低いように思われる。しかし、コードの結合度が高い場合、このような局所的な修正がシステム全体の不整合を引き起こす原因となる可能性がある。

\paragraph{開発者の注意力とレビュープロセスの違い}
変更の規模によって、開発者とレビュアーが払う注意の程度が異なる可能性がある。Bosuらは、レビューの有効性は変更のサイズとともに低下することを発見しており\cite{bosu2015}、大規模な変更では詳細な検査が困難になることを示唆している。同様に、Kononenkoらは、変更サイズがレビュー時間に大きく影響すると報告しており\cite{kononenko2016}、レビュアーがより多くの時間を費やす必要があることを示している。このレビューの困難さの違いは、欠陥の検出率に影響を与える可能性がある。小規模な変更はレビューが容易であるため、潜在的な欠陥が発見されやすく、後続のコミットで修正される可能性が高い。一方、大規模な変更では詳細な検査が困難なため、欠陥が見逃されやすく、修正されないまま残存する可能性がある。本研究で使用するBugHunterデータセットはSZZアルゴリズムに基づいて欠陥混入コミットを特定しているため、実際に修正された欠陥のみが「欠陥あり」として検出される。したがって、小規模な変更で観察される高い欠陥混入確率は、真の欠陥混入確率の違いではなく、レビューによる検出率の違いを反映している可能性がある。

\paragraph{開発者の経験と役割分担}
小規模な変更は経験の浅い開発者が担当することが多い可能性がある。Juら\cite{ju2021}は、オンボーディング戦略に関する調査において、新人開発者に対してはSimple-Complex戦略が広く採用されていることを報告している。この戦略では、マネージャーは最初の1週間にシンプルなタスク(バグ修正や設定変更)を割り当て、その後2〜3週間で小さなバグやシンプルな機能に取り組ませることで、段階的にコードベースへの理解を深めさせる。一方、経験豊富な開発者に対しては、Exploration-Based戦略が用いられ、定義が曖昧で不確実性の高いタスクが割り当てられる傾向にある。このように、開発者の経験レベルに応じてタスクの複雑さや性質が調整されることが示されている。

彼らの調査から、小規模な変更の欠陥混入確率の高さは、経験の浅い開発者がコードベース全体への理解が不十分なまま変更を行うことに起因すると考えられる。Simple-Complex戦略では、新人開発者は最初の数週間で小規模なタスクに取り組むが、この段階ではまだシステムの全体像や他のコンポーネントとの相互作用を十分に把握できていない。その結果、局所的には正しく見える変更であっても、他の部分への影響を見落とし、予期しない欠陥を混入させてしまう可能性が高まる。
\section{特徴量重要度のプロジェクト間差異}
特徴量重要度を分析した結果、プロジェクトによって重要な特徴量が異なることが確認された。例えば、Nettyではトークン数の変化量が最も重要な特徴量であったのに対し、他の多くのプロジェクトでは追加行数や変更ファイル数が上位を占めた。本節では、この差異が生じる理由について考察する。

\paragraph{プロジェクトドメインとアーキテクチャの影響}
プロジェクトのドメインやアーキテクチャ特性が、重要な特徴量に影響を与えている可能性がある。Nettyは非同期ネットワークフレームワークであり、責務が明確で焦点が絞られている。ネットワークプロトコルの実装では、わずかなトークンレベルの変更(例: バイトオーダーの処理、フラグビットの操作)が重大な欠陥を引き起こす可能性がある。そのため、トークン数の変化量のような詳細な変更メトリクスが重要になると考えられる。
一方、ElasticsearchやHazelcastのような分散システム基盤は、広範な機能セットを持ち、変更の影響範囲が広い。このようなプロジェクトでは、コミット全体の規模が、変更の複雑性や影響範囲をより適切に捉える指標となる。
Neo4jやOrientDBのようなデータベースシステムでは、トランザクション処理や状態管理の複雑性が高い。これらのプロジェクトでは、変更の広がりが重要な指標となる傾向が見られた。

\paragraph{開発プロセスとツールの影響}
プロジェクトの開発プロセスや使用しているツールも、重要な特徴量に影響を与える可能性がある。高いテストカバレッジを持つプロジェクトでは、自動テストによって構造的メトリクスだけでもある程度の欠陥検出が可能である。このような場合、変更メトリクスによる追加的な情報の価値が相対的に低くなる。また、厳格なコードレビュープロセスを持つプロジェクトでは、小規模な変更でも慎重にレビューされるため、変更サイズと欠陥リスクの関係が弱まる可能性がある。

\paragraph{欠陥の種類による影響}
プロジェクトで発生する欠陥の種類によって、重要な特徴量が異なる可能性がある。ロジックバグは、条件分岐の複雑さや変更の頻度と関連が深く、コードの理解しやすさや保守性を測る特徴量が重要になる。セキュリティバグは、特定のセキュリティ問題(例: 入力検証の欠如、認証処理の不備)を持つコード箇所で発生しやすく、コードの複雑性や外部ライブラリとの連携に関する特徴量が重要になると考えられる。本研究では欠陥の種類を区別せず一律に扱っているが、欠陥の種類ごとに異なる予測モデルを構築することで、より高精度な欠陥予測が可能になると考えられる。これは今後の重要な研究課題である。
\section{予測性能のプロジェクト間差異}
本研究では、5つのプロジェクトに提案手法を適用した結果、F1スコアの改善幅に最大2.6倍の差(Hazelcast: 0.11 vs Netty: 0.29)が観察された。本節では、この性能差が何に起因するかを体系的に分析する。

\paragraph{バグとコミットの特性}
ステップ1(静的コードメトリクスのみ)での予測精度と、ステップ3(変更メトリクス追加後)での改善幅の間に明確な関係が観察された。静的コードメトリクスのみを使ったモデル(ステップ1)の予測精度が比較的高いプロジェクトでは、変更メトリクス追加後のモデル(ステップ3)でもあまり予測精度が改善しない一方で、ステップ1の予測精度が比較的低いプロジェクトでは、ステップ3で予測精度が改善しやすいことが分かる。具体的には、Hazelcastはステップ1で既にF1スコア0.68と比較的高い性能を達成しており、改善の余地が限定的である。一方、Nettyはステップ1でF1スコア0.45と低い性能であったが、ステップ3で0.75まで向上し、0.29という最大の改善幅を示した。

この関係は、静的コードメトリクスと変更メトリクスの相補性を示している。静的メトリクスのみで予測が困難なプロジェクトでは、変更メトリクスが提供する時系列情報が特に有効に機能する。逆に、静的メトリクスである程度予測可能なプロジェクトでは、変更メトリクスの価値が限定的となる。

重要なのは、改善幅の違いにかかわらず、すべてのプロジェクトでステップ3の最終的なF1スコアが0.70以上に達していることである。これは、静的コードメトリクスと変更メトリクスの組み合わせにより、プロジェクト特性の違いを超えて、より安定した予測ができるようになることを示している。ステップ1のF1スコアは0.45から0.68と1.5倍の範囲でばらついていたが、ステップ3では0.70から0.79と1.1倍程度に収束している。このばらつきの減少は、提案手法が多様なプロジェクトに対して一貫した性能を提供できることを意味する。

さらに、ベースライン性能と改善幅の関係を詳しく分析するため、各プロジェクトのデータセットにおける陽性クラス(バグあり)の割合を算出し、F1スコアの改善幅との相関を調査した。表\ref{tab:imbalance_vs_improvement}に、各プロジェクトのバグあり割合とF1改善幅を示す。

\begin{table}[ht]
\centering
\caption{データセットの陽性クラス割合とF1改善幅の関係}
\label{tab:imbalance_vs_improvement}
\begin{tabular}{|l|r|r|}
\hline
プロジェクト & バグあり割合 & F1改善幅 \\
\hline
Netty & 0.22 & 0.29 \\
Neo4j & 0.26 & 0.26 \\
OrientDB & 0.26 & 0.22 \\
Elasticsearch & 0.35 & 0.19 \\
Hazelcast & 0.37 & 0.11 \\
\hline
\end{tabular}
\end{table}

この結果から、バグあり割合とF1改善幅の間に強い負の相関(Pearson相関係数 $r = -0.93$, $p = 0.020$)が観察された。すなわち、データセット中のバグを含むメソッドの割合が低いほど(データの不均衡が大きいほど)、変更メトリクスの追加による性能改善効果が大きい傾向がある。

この関係は以下のように解釈できる。データが比較的バランスの取れているプロジェクト(Hazelcast: バグあり37.5\%、Elasticsearch: 35.0\%)では、静的コードメトリクスのみでも陽性クラスと陰性クラスの判別が比較的容易であり、ベースライン性能が高い。一方、データの不均衡が大きいプロジェクト(Netty: バグあり21.8\%、Neo4j: 25.8\%)では、陽性クラスが少数派であるため静的メトリクスのみでは識別が困難であり、ベースライン性能が低くなる。このような不均衡データに対して、変更メトリクスが提供する時系列的な変化パターンは、少数派クラスを識別するための追加的な判別情報として特に有効に機能すると考えられる。

データの不均衡は機械学習における一般的な課題であるが、本研究の結果は、変更メトリクスの統合が不均衡データに対する有効な対処法の一つとなりうることを示唆している。
\section{提案手法の有効性のメカニズム}
\input{6_5_mechanism_of_effectiveness.tex}
\section{研究の限界と妥当性}
\input{6_6_limitations_and_validity.tex}
\clearpage

\chapter{おわりに}
本研究では、ソフトウェア開発におけるコードレビューの効率化を目的として、コミット単位とメソッド単位の変更メトリクスを活用した欠陥予測手法を提案した。提案手法では、コミットを不規則なタイミングで発生するイベントとして扱い、直前の状態からの差分を特徴量として学習する。さらに、新たなレビュー労力の計算式とレビュー対象コミットの選択基準を設けることで、少ないレビュー労力で高い欠陥発見率を実現する。

5つのオープンソースプロジェクトを対象とした実験により、提案手法の有効性が示された。全てのプロジェクトにおいて、構造的メトリクスのみのベースラインと比較してF1スコアが向上し、平均改善幅は0.21、最終的なF1スコアは0.70以上を達成した。ROC-AUCはすべてのプロジェクトで0.91を超え、高い識別性能を示した。レビュー労力の分析では、20\%のレビュー労力で、全ての欠陥のうち平均66.9\%を検出できることが示され、提案手法がリソース配分の効率化に有効であることが確認された。

今後の課題として、より多様なプログラミング言語やドメインへの適用が挙げられる。本研究はJavaプロジェクトを対象としたが、型システムやメモリ管理が異なる言語では、メトリクスの調整が必要となる可能性がある。また、本研究で対象としていない金融システムや医療システムでは、より厳格な品質要求や規制要件が存在する。これらのドメインでは、欠陥の特徴が異なる可能性があり、ドメイン固有の調整が必要となる。さらに、プロジェクトの進行に伴うコードベースの変化に対応するため、欠陥予測モデルを継続的に改善する仕組みが求められる。


\renewcommand{\bibname}{参考文献}
\begin{thebibliography}{99}
\bibitem{kim2014}
Miryung Kim, Thomas Zimmermann, and Nachiappan Nagappan,
``An Empirical Study of Refactoring Challenges and Benefits at Microsoft,''
\textit{IEEE Transactions on Software Engineering},
vol. 40, no. 7, pp. 633--649, July 2014.

\bibitem{kamei2013}
Yasutaka Kamei, Emad Shihab, Bram Adams, Ahmed E. Hassan, Audris Mockus, and Anand Sinha,
``A Large-Scale Empirical Study of Just-in-Time Quality Assurance,''
\textit{IEEE Transactions on Software Engineering},
vol. 39, no. 6, pp. 757--773, June 2013.

\bibitem{ferenc2020}
Rudolf Ferenc, P\'{e}ter Gyimesi, G\'{a}bor Gyimesi, Zolt\'{a}n T\'{o}th, and Tibor Gyim\'{o}thy,
``An Automatically Created Novel Bug Dataset and Its Validation in Bug Prediction,''
\textit{Journal of Systems and Software},
vol. 169, p. 110691, November 2020.

\bibitem{han2021}
Xiaofeng Han, Amjed Tahir, Peng Liang, Steve Counsell, and Yajing Luo,
``Understanding Code Smell Detection via Code Review: A Study of the OpenStack Community,''
\textit{Proc. 2021 IEEE/ACM 29th International Conference on Program Comprehension (ICPC)},
pp. 323--334, May 2021.

\bibitem{romano2022}
Simone Romano, Fiorella Zampetti, Maria Teresa Baldassarre, Massimiliano Di Penta, and Giuseppe Scanniello,
``Do Static Analysis Tools Affect Software Quality when Using Test-driven Development?''
\textit{Proc. 16th ACM/IEEE International Symposium on Empirical Software Engineering and Measurement (ESEM)},
pp. 80--91, September 2022.

\bibitem{jisx25010}
日本産業標準調査会,
``JIS X 25010: 2013 システム及びソフトウェア製品の品質要求及び評価(SQuaRE)-システム及びソフトウェア品質モデル,''
日本規格協会,2013年.

\bibitem{mccabe1976}
Thomas J. McCabe,
``A Complexity Measure,''
\textit{IEEE Transactions on Software Engineering},
Vol. SE-2, No. 4, pp. 308--320, December 1976.

\bibitem{chidamber1994}
Shyam R. Chidamber and Chris F. Kemerer,
``A Metrics Suite for Object Oriented Design,''
\textit{IEEE Transactions on Software Engineering},
Vol. 20, No. 6, pp. 476--493, June 1994.

\bibitem{iso24765}
ISO/IEC/IEEE 24765:2017,
``Systems and Software Engineering -- Vocabulary,''
International Organization for Standardization/International Electrotechnical Commission/Institute of Electrical and Electronics Engineers, 2017.

\bibitem{sliwerski2005}
Jacek \'Sliwerski, Thomas Zimmermann, and Andreas Zeller,
``When Do Changes Induce Fixes?''
\textit{Proc. 2005 International Workshop on Mining Software Repositories (MSR)},
pp. 1--5, May 2005.

\bibitem{fowler1999}
Martin Fowler,
\textit{Refactoring: Improving the Design of Existing Code},
Addison-Wesley Professional, 1999.

\bibitem{sonarqube}
SonarSource,
``SonarQube,''
\url{https://www.sonarsource.com/products/sonarqube/},
(2025-12-18 参照).

\bibitem{checkstyle}
Checkstyle Project Team,
``Checkstyle,''
\url{https://checkstyle.sourceforge.io/},
(2025-12-18 参照).

\bibitem{jisx0161}
日本産業標準調査会,
``JIS X 0161: 2008 ソフトウェア技術−ソフトウェアライフサイクルプロセス−保守,''
日本規格協会,2008年.

\bibitem{hindle2008}
Abram Hindle, Daniel M. German, and Ric Holt,
``What do large commits tell us? a taxonomical study of large commits,''
\textit{Proc. 2008 International Working Conference on Mining Software Repositories (MSR)},
pp. 99--108, May 2008.

\bibitem{bosu2015}
Amiangshu Bosu, Michaela Greiler, and Christian Bird,
``Characteristics of useful code reviews: an empirical study at microsoft,''
\textit{Proc. 2015 IEEE/ACM 12th Working Conference on Mining Software Repositories (MSR)},
pp. 146--156, 2015.

\bibitem{kononenko2016}
Oleksii Kononenko, Olga Baysal, and Michael W. Godfrey,
``Code review quality: how developers see it,''
\textit{Proc. 38th International Conference on Software Engineering (ICSE)},
pp. 1028--1038, 2016.

\bibitem{ju2021}
An Ju, Hitesh Sajnani, Scot Kelly, and Kim Herzig,
``A Case Study of Onboarding in Software Teams: Tasks and Strategies,''
\textit{Proc. 43rd International Conference on Software Engineering (ICSE)},
pp. 613--623, 2021.

\end{thebibliography}
\end{document}
