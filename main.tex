\documentclass[12pt,a4paper]{ltjsreport}
\usepackage{graphicx}
\usepackage{fontspec}
\usepackage{url}
\usepackage{colortbl,array,xcolor}
\usepackage{here}
\begin{document}
\thispagestyle{empty}
\begin{center}                                                        
eラーニングシステムにおける\\
利用形態の多様性に対応可能な\\
アーキテクチャに関する調査研究\\
\vfill
サブタイトル

\vfill
学籍番号 氏名\\
\vfill
主指導教員  教員名\\
\vfill
北陸先端科学技術大学院大学\\
先端科学技術研究科\\
情報科学\\ 
\vfill
令和4年3月\\ % 学位授与年月
\vfill
\end{center}

% 目次の出力
\tableofcontents

\clearpage
\centerline{概要}
%%概要

本研究は、現代社会で様々な利用背景に対応できる、
利用形態が多様化するeラーニングシステムのアーキテクチャの構築を目的にする。
具体的には、eラーニングシステム例を対象にする、
必要とされる機能を実現するための仕組み
(アーキテクチャ)を調査し、
適切なアーキテクチャを選択した上で、
アーキテクチャの改良も行う。
最終的にステークホルダーを確定し、
アーキテクチャを提案する。
提案したアーキテクチャを評価するために、
既存システムと同様の機能及び特徴を持つシステムを作成し、
不足している機能の実現を図る。
その不足している機能を実現するための
システムの改良に必要な作業量をプロトタイプ上で計測する。
プロトタイプの作成にはJavaを使用する。

\clearpage

\chapter{はじめに}
\section{背景}
%% 背景


企業などでIT人材の不足に対し、
eラーニングによる教育システムが注目されている。
例えば脆弱性対策教育を目的として、
教材の提供と攻撃・防御の演習を可能にした実例\cite{el5}がある。
しかし受講者の達成目標が様々であり、
適切な教材の提供を実現することは困難であることが指摘されている。

その理由として受講者ごとに学習の動機づけが異なるが、
それをシステムに対して与える仕組みは不十分である。
他にも\cite{el3}では学習の達成度に応じて教材を選択する方法、
\cite{el1}では学習者の要求に応じて教材を柔軟に構成する手法が述べられているが、
いずれもシステムとしての実現は限定的である。

\clearpage
\section{目的}
%% 目的

本課題研究の目的は、
実行形態の多様性を実現可能な
eラーニングシステムの
アーキテクチャについて調査し、
機能や実行環境が異なる利用形態においても
対応可能なアーキテクチャを提案・評価することである。
ただし教育内容については対象とせず、
システムにより提供されている内容の中から
利用者の目的に沿ったものを選択する仕組みに
限定するものとする。

教材作成者の立場では、
学習者の要求に応じた教材を構成するためには
それぞれの構成要素について様々な特性を定義する必要がある。

例えば、テキストAには初級者用、
テキストBには中級者用という特性が定義されている。
初級者用の教材を構成するためには、
初級者用という特性を持つ全ての構成要素を収集、組み合わせる。
中級用の教材では、
中級者特性を持つ構成要素のみを組み合わせるが、
学習者が中級者用の教材を十分に理解できない場合には、
初級者用教材も提供する必要がある。

従来は学習者の理解度を確認するための仕組みと
教材を切り替える仕組みを個別に作成する必要があったが、
本研究が示すガイドラインに従えば、
収集する構成要素の誤りや見落としを回避することができる。
作成者がこのガイドラインに従うことにより、
学習者が学習の進捗に応じて柔軟に教材を選択、
構成する機能を容易に提供可能になる。
構成要素の収集と組み合わせの機能の一部を
学習者に公開することにより、
学習者がそれぞれの目的に応じて、
複数の教材の中から必要な部分のみを抽出して
独自の教材を構成することができる。
その結果学習者は学習効率を、
教材作成者はより多くの利用者を得ることにより
教材の品質をそれぞれ向上させることが可能になる。

\clearpage

\section{構成}
%% 構成 書き直し土曜の朝

本論文の構成は以下の通りである。
第2章ではeランニングシステムの定義、実例と特性について、
複数のシステムの調査結果を述べる。
第3章では調査結果に基づいてアーキテクチャを提案し、
詳しくアーキテクチャに関する学習の対象、ステークホルダー、
機能要求、システム構成、ユースケース、システム動作を説明する。
第4章では提案したアーキテクチャによりシステムを実装し、
システムの構成と挙動を説明する。
第5章で実装したシステムにより、
調査したシステム例から獲得した機能要求の完成と比較し、
提案したアーキテクチャを評価する。


\clearpage

\chapter{調査事例}
\section{ELSの定義}
%% \subsection{ELSの達成目標}

ELS(e-Learning System)とは情報技術を用いて
利用者が自分の目的や都合にあわせて
学習する内容を登録したり閲覧したりすることを可能にする
学習支援システムである。
ELSの利用者は大きく2つに分けられる。
それぞれを学習者、教育担当者と呼ぶことにする。

\begin{enumerate}

\item{学習者}

学習者はELSを使って学習し、スキルを取得する人である。
例えばオンライン教育に参加する学生または会社新入社員などが該当する。

\item{教育担当者}

教育担当者は教材を作成し学習者に提供する。
学習結果を監視し、必要ならば教材の改善をする。
例えば教師及び新人教育部門などが該当する。

\end{enumerate}


%% 理解度を確認するために最後に試験をうける。
%% 試験で決められた数の正解をすると、合格証明を取得する。
%% 予め定められた教材の集まり(課程)の合格証明をすべて取得したら、修了試験をうけることができる。
%% 合格したら修了証明を取得する。これにより学習目標が達成されたことを示す。

図\ref{fig:ELSによる効果}は利用者の役割を示したものである。
教材は教育担当者が作成しELSに登録する。
学習者は目標の達成に必要な
学習内容を含む教材を選択する。
教材の最後には学習者が目標を達成したかを確認する試験があり、
学習者の成績は学習記録としてELSに保存され、
教育担当者は学習記録を分析することで
学習者が目標を達成するための援助を効率的に実現できる。

\begin{figure}[H]
  \vspace{0.5cm}
  \hspace*{0.5cm}
%%  \includegraphics[width=13cm,bb=0 350 800 600]{../usage.pdf}
  % \includegraphics[width=13cm]{../usage.pdf}
  \caption{ELSによる効果}
  \label{fig:ELSによる効果}
\end{figure}

%% ELSを利用して達成すべき目標

%% 利用者にとっては...

%% 教育担当者にとっては...

\clearpage
\section{ELSの実例}

\begin{enumerate}
\item 情報セキュリティ教育用教材

%% ELSEC(東京電機大学2011)
川上, 佐々木らは拡張性や柔軟性が高い情報セキュリティ教育システムとして
ELSEC(E-Learning system %
for SECurity)を提案している\cite{el1}.
フィッシング対策教育を行うためのシナリオを作成し、その評価を行っている。
ELSECは達成目標に対応した教材の選択と学習者の達成度の保証手段として、
教材リストが定義されており、全部学習しないと、修了試験を受けられない。
これを参照し、本研究におけるシステムも課程を設定し、修了試験を含む。

\item 拡張可能な学習支援システムアーキテクチャ

%% ELECOA(千葉工業大学2017)
仲林,森本らは相互運用性と機能拡
張性の両立を図ることを目的とし
ELECOA(Extensible Learning Environment %
with Courseware Object Architecture)を提案している\cite{el2}.
ELECOAはグループ学習における学習制御機能を実現した。
例えば、他学習者の状態を条件とする分岐、
または他学習者の状態を条件とする強制移動などの機能がある。
この中、他学習者の間の連携、援助する機能が本研究に役に立つものと考える。


\item 情報セキュリティ教育の強制実行

%% SEC-EL(近畿大学2017)
坂上,森山,上田らは教育実施者・対象者への負担が少ない、
効果的な強制実行型情報セキュリティ教育システムとして、
SEC-ELを提案している\cite{el3}.
SEC-ELは教育対象者の理解度に適した難易度の問題を出題することができる。
手段としては、問題の難易度と教育対象者の理解度を計算し、
教材選択の際に条件を満たすもののみを可能にする。
例えば:学生と教材の双方にランクを付加、自身のランクを超える教材が選択できない。
また教材ごとに前提となる他の教材のリスト付加、一つでも履修しないなら選択できない。
本研究は、教材リストを参考し、課程を設計する機能を付加する。


\item 学習習熟度を用いた自主学習支援システム

%% AR-PBL(公立はこだて未来大学2016)
花田,大場らは学生が自身の学習に適した資料を適切に選択することを目的とし
AR-PBLを提案している\cite{el4}.
このシステムは習熟度に応じた教材の提供を実現する。
具体な手段は専門用語の出現を自動計測し、教材の難易度を推定する。
本研究にでもこのような難易度推定機能の利用を考えている。


\item 脆弱性対策教育システム

%% VulCES(東京電機大学,IPA2011)
竹下,小林,佐々木らは適切な防止策を理解するために
攻撃方法理解用のツールと組み合わせた
VulCES(Vulnerability Countermeasures %
E-learning System)を提案している\cite{el5}.
VulCESは実験システムとの連携し、学習の行動を拡張する。
学習者が防衛側となり、予め決められている攻撃を防衛した割合で評価する。
本研究も、実験システムの導入を考えている。


\item 生徒同士の協調学習

%% MultiVNC(アルファシステムズ2004)
北川,上原らは教師・生徒間や生徒同士による協調作業を可能とするシステムとして
MultiVNC(Virtual Network Computing)を提案している\cite{el6}.
VNCとはネットワーク上の離れたコンピュータを遠隔操作するためのRFBプロトコルを利用する、
リモートデスクトップソフトである。
VNCは通常は1対1であるが、これを教師側の画面に複数の画面を同時表示するための拡張したものが
MultiVNCである。そのゆえ、複数の学習者の中から理解度不足のものの発見と改善することができる。
本研究でも、この手段に通じて、教師側は複数の学習者の理解を把握することを望ている。

\end{enumerate}
\clearpage
\section{ELSの特性}
%% 



\begin{itemize}

\item 達成度の保証

\begin{itemize}
\item 学習者のレベルに合った教材の選択(ELSEC,SEC-EL,AR-PBL)

ELSECは達成目標に対応した教材リストあり、
学習者が自分レベルにより順番に学習することができる。

SEC-ELは問題の難易度と教育対象者の理解度をランクを付加、
教育対象者の理解度に適した難易度の問題を出題する。
教材選択の際も条件を満たすもののみを可能にする。

AR-PBLは専門用語の出現を自動計測し、教材の難易度を推定する。
そのゆえ、習熟度に応じた教材の提供を実現する。

\item ナビゲーションの制約(ELSEC,SEC-EL)

ELSECは学習者の達成度の保証手段として、
教材リストが定義されており、全部学習しないと、修了試験を受けられない。

SEC-ELは似たような、教材ごとに前提となる他の教材のリスト付加、
一つでも履修しないなら選択できない。

\end{itemize}

\item グループ学習(ELECOA)

ELECOAはグループ学習における学習制御機能として
複数の学習者間での連携、援助が可能である。

\item ユーザビリティ(MultiVNC)

MultiVNCは複数の学習者の中から理解度不足ものの発見と改善することができるので、
教育担当者は理解度を把握し、学習者の学習に適切なサポートの提供が可能である。


\item 他システムとの連携(VulCES)

VulCESは実験システムとの連携により、学習者の活動の選択肢を追加する。

\end{itemize}






\chapter{アーキテクチャ}
本章では調査事例で上げた様々な利用形態に対応する
システムを実現可能なアーキテクチャについて検討する。

\section{学習の対象}

最初に学習者の達成目標に従って学習の対象を分類する。
学習者は学習の際に教材を選択するが、
目標を達成するためには特定の知識の集まりを習得する必要がある。

このため学習内容は
複数の科目及び必要な科目をすべて習得した後に、
目標の達成を確認する試験を含む必要がある。
これを課程(コース)と定義する。

以下に各要素について詳しく述べる。

\begin{itemize}\itemsep=-5pt

\item 課程

課程とはある目標を達成するために学習者が習得する必要のある学習内容の集まりである。
例えばセキュリティ課程には
セキュリティ初級(セキュリティ障害の発見)、
セキュリティ中級(解決手段)、
セキュリティ上級(セキュリティ防衛環境の構築)を
含む内容が含まれる。
課程の最後には試験が含まれており、
これに合格することが学習者の最終目標である。
この試験を修了試験と呼ぶことにする。

\item 試験

試験は目標とした学習内容を理解したかを確認するためのものである。

%% 他の試験が必要な理由
しかし、課程に含まれている学習内容の量が多くなめ、
課程修了を示す修了試験の他に、
段階的な試験を設置する必要性がある。
と同時に、修了試験の合格率が低い時に
理解不足のところを発見しやすい利点もある。

そのゆえ、課程の最後にある修了試験の他に、
教材に含まれている理解度確認をするための完了試験と、
学習単位に含まれてている学習項目の理解度ための確認試験も設置する。

% 完了試験の結果が基準点に達したら学習者がその教材を理解した証明として合格証明書を取得する。
% 合格済の教材は学習する際に選択することができない。

\item 学習項目

学習項目とはある学習内容を解説する最小単位である。
例えばセキュリティ初級の教材には、セキュリティ障害の検出と
それに対する対策などが学習項目として含まれている。
学習者は学習項目を最初から最後まで順に学習することもできるし、
前に戻って復習したり、先に進むこともできる。

\item 学習単位

学習単位は複数の学習項目と理解度確認のための確認試験から構成される。


\item 教材

教材は特定の内容の習得を目的とし、
学習すべき内容を説明した文章や図などを含む
学習学習単位と完了試験から構成される。




\end{itemize}




\section{ステークホルダー}

前節での学習対象の定義に従い
利用者を役割に従って分類し、それぞれが必要とする機能要求を定義する。
システム構成とユースケースを決定し、
動作とそれに必要なデータを決定する。

ELSの利用者をその役割に応じて分類したもの(ステークホルダー)を以下に示す。

\begin{itemize}

\item{学習者}

学習者は学習習熟度を向上させることで目標を達成する。
学習者とは。教材を勉強し、試験を受ける人である。
課程で決められた教材を全て習得した後で修了試験に合格することが目標となる。

\item{教育担当者}

教育担当者とは教材を教えるまたは修正する責任者である。
教育担当者は学習者の学習進捗状況を把握して教材を改善する。
学習記録を参照し理解度が低い箇所を修正することが目標である。

\item{課程設計者}

課程設計者とはコースを設計する責任者である。
教材改善進捗状況を把握し、必要ならば課程に新しい教材を追加する。
%% これを必要の理由を書く

\end{itemize}

\begin{figure}[H]
  \vspace{0.5cm}
  \hspace*{0.5cm}
  % \includegraphics[width=12cm,bb=0 50 800 600]{../operation.pdf}
  \caption{ステークホルダーの役割}
  \label{fig:ステークホルダーの役割}
\end{figure}

図\ref{fig:ステークホルダーの役割}はステークホルダーの役割を示したものである。
課程設計者は課程を設計し、教育担当者が設計した課程に対応する教材を作成する。
学習者が作成した教材を学習し、試験をうける。
システムは受験者全員に対する試験の問題ごとの正答率及び合格率を計算する。

この中で、教育担当者が合格率を向上させるために教材を改善する役割を担当する。
もし学習者の合格率は基準値に到達していない場合は、教育担当者は教材を改善する。
学習者が改善された教材を再学習し受験することで、理解度が向上したかどうかを確認する。

改善された教材に対する学習及び受験を一定数に繰り返しても合格率が基準値に達しない場合は、
課程設計者は課程に新しい教材を追加することを教育担当者に依頼する。



\begin{figure}[H]
  \vspace{0.5cm}
  \hspace*{0.5cm}
  % \includegraphics[width=13cm,bb=0 100 800 400]{../overview.pdf}
  \caption{ステークホルダーの目標}
  \label{fig:ステークホルダーの目標}
\end{figure}

各ステークホルダーの目標を図\ref{fig:ステークホルダーの目標}に示す。
学習者は試験に合格したことによりスキル習得し、目標を達成する。
教育担当者は学生の正答率の向上を目標として教材を改善する。
課程設計者は学習が停滞している理由を分析し、
必要ならば課程を改善することにより目標を達成する。


\clearpage
\section{機能要求}
機能要求は各ステークホルダーが必要とする情報の提供及び分析の支援である。
図\ref{fig:システムが保持する情報}に保持すべき情報を示す。
これらの作成と利用の際にシステムが提供する機能を情報ごとに列挙する。

\begin{itemize}

\item 学習記録

学習記録は学習者が教材を学習者する度に作成され、教材ごとに履歴として保持される。


\item 学習進捗状況

教育担当者が利用できるように正答率及び完了試験の合格率を計算する。
その時に全体集合を履歴中特定の条件を満たすものとして与えることを可能にする。

\item 教材改善進捗状況

教材が改善される度に、その教材を新たに学習した学習者に対する学習進捗状況として作成され、
教材ごとに履歴として保持される。
課程設計者が利用できるように正答率及び合格率の改善の効果を計算する。
全体集合は同様に特定の条件により選択可能にする。

\end{itemize}


\begin{figure}[H]
  \vspace{0.5cm}
  \hspace*{0.5cm}
  % \includegraphics[width=12cm,bb=0 150 800 500]{../structure.pdf}
  \caption{システムが保持する情報}
  \label{fig:システムが保持する情報}
\end{figure}


\clearpage

\section{システム構成}

システムは学習対象を保持する部分、学習を管理する部分、情報提供する部分から構成される。
図\ref{fig:システム構成}にシステム構成を示す。

\begin{itemize}

\item{情報保持部(Library)}

情報保持部は課程、教材、学習の過程で作成される全ての情報を保持する。
学習対象で最大のものは課程(Course)である。
課程は複数の教材(Book)から構成される。
教材には初級、中級などの段階(rank)が付随する。
教材の最後には完了試験(CompletionExam)が、
課程内の全ての教材を習得した後には修了試験(QualificationExam)が含まれる。

学習が進行すると最初に教材に対応した
学習記録(ScoreRec)が作成される。
複数の学習者が同じ教材を学習すると、結果が学習進捗記録(ScoreBook)に追加される。
教育担当者が教材を改善すると改善前後の正答率の変換が教材改善進捗記録(ScoreHistory)に追加される。


\item{学習管理部(Runner)}

学習管理部は学習者に教材選択、学習、完了試験受験の各機能を提供する。
システムが起動し、教材が登録された後でまだ学習者がいない場合は学習記録は空である。
学習が行われると学習者ごとに学習記録が追加される。
教材を複数の学習者が学習した時に、教材ごとに学習進捗状況が追加される。
教材が更新される時に課程ごとに学習進捗状況が教材改善進捗状況に追加される。

\item{情報分析部(Viewer)}

情報分析部は各利用者が必要とする情報を分析した結果を提供する。
学習者には学習記録から教材の学習状況(合否)を、
教育担当者には教材ごと、学習者集合ごとの正答率及び合格率を、
課程設計者には課程に対する教材の改善履歴及び改善による正答率及び合格率の上昇を与える。

\end{itemize}

\clearpage

\begin{figure}[H]
  \vspace{0.5cm}
  \hspace*{0.5cm}
%%  \includegraphics[width=13cm,bb=0 350 800 600]{../system.pdf}
  % \includegraphics[width=13cm]{../system.pdf}
  \caption{システム構成}
  \label{fig:システム構成}
\end{figure}


課程(course1)とそれに含まれる教材群はすでに登録されているとする。
学習者1(st1)が初めてこの課程を学習する場合を考える。
課程の最初の教材(book1)を選択する。
完了試験後に結果が学習記録(sr1)に記録される。
別の学習者2(st2)が同様に最初の教材book1を学習し、
試験結果がsr2に記録される。

教育担当者(pf1)が自分が作成した教材(book1)の学習進捗記録を要求する。
今までbook1を学習した学習記録(sr1,sr2)を含む記録s-book1を返す。
ある程度の人数が学習した後で
合格率が必要な場合は学習記録に含まれる情報から計算した結果を返す。
この時に正答数が著しく低い問題が発見されたら、
教材の改善を流すメッセージを返す。
教育担当者は説明を改善した教材(book1n)を
book1の更新として課程に登録する。
以降の学習者はbook1の代わりにbook1nを学習し、
学習進捗はs-book1nに記録される。

課程設計者の要求に応じて
その課程に含まれるすべての教材に対する改善進捗記録s-history1を返す。
教材の改善前後における合格率の上昇度を必要とする場合は
s-history1から計算した結果を共に返す。

課程の更新の一例として最初に学習する教材(book1/book1n/...)の前に
より基礎的な内容の教材(book0)が追加された場合を考える。
以降の学習者はbook1の代わりにbook0から学習を開始し、
その完了試験に合格した後でbook1を学習する。
これによりbook1の学習効果が改善されることが期待される。

\clearpage
\section{ユースケース}
各ステークホルダーが必要とする機能を実現するために決定したユースケースを以下に示す。

\begin{enumerate}
\item 教材の登録

\medskip
\begin{tabular}{|l|} \hline
起動アクター:教育担当者\\%
教育担当者は作成または修正した教材を登録する         \\%
これは失敗しない\\ \hline
\end{tabular}
\medskip

\item 学習

\medskip
\begin{tabular}{|l|} \hline
起動アクター:学習者\\%
成功時:\\%
\; 学習者は学習する課程を選択する\\%
\; 学習者が学習目標による選択した課程の中で、\\%
\; 自分のランクに合った教材を閲覧して学習する\\%
\; 教材のランクが学習者のランクより大きい場合に失敗する    \\%
失敗時:\\%
\; 学習者にメッセージを返す\\ \hline
\end{tabular}
\medskip

\item 完了試験の受験

\medskip
\begin{tabular}{|l|} \hline
起動アクター:学習者\\%
事前条件:学習者が教材の学習を完了している          \\%
システムは学習者の受験結果を成績として得る\\%
成績が合格基準以上:\\%
\; システムは合格証明書を発行する\\%
成績が合格基準未満:\\%
\; 学習者は理解が不足している箇所を再学習する\\ \hline
\end{tabular}
\medskip

\clearpage

\item 修了試験の受験

\medskip
\begin{tabular}{|l|} \hline
起動アクター:学習者\\%
事前条件:学習者は課程の全て教材の合格証明書を得ている    \\%
システムは学習者の受験結果を成績として得る\\%
成績が合格基準以上:\\%
\; システムは修了証明書を発行する\\%
成績が合格基準未満:\\%
学習者は理解が不足している箇所を再学習する\\ \hline
\end{tabular}
\medskip


\item 教材改善のための分析

\medskip
\begin{tabular}{|l|} \hline
起動アクター:教育担当者\\%
システムは教材に対応する学習進捗記録を取得する        \\%
システムは正答率を計算する\\%
これは失敗しない\\ \hline
\end{tabular}
\medskip


\item 課程改善のための分析

\medskip
\begin{tabular}{|l|} \hline
起動アクター:課程設計者\\%
システムは課程に対応する教材改善進捗記録を取得する      \\%
システムは教材改善前後の正答率の増加を計算する\\%
増加が基準値以上:\\%
\; システムは改善不要を通知する\\%
増加が基準値未満:\\%
\; システムは改善の必要があると提示する\\ \hline
\end{tabular}
\medskip


\end{enumerate}
\clearpage
\section{システムの動作}

図\ref{fig:システム動作例}に教育担当者が教材を登録してから、
学習者の学習、教育担当者の教材改善、課程設計者の課程設計改善までのシステムの動作を示す。
以下のように準備、学習者、教育担当者、課程設計者に分けて詳しく説明する。

\begin{figure}[H]
  \vspace{0.5cm}
  \hspace*{0.5cm}
  % \includegraphics[width=12cm,bb=0 200 800 500]{../role.pdf}
  \caption{システム動作例}
  \label{fig:システム動作例}
\end{figure}

\clearpage

\begin{enumerate}
\item{準備}

学習者が利用する前に、教育担当者は課程と教材をシステムに登録しておく必要がある。

\begin{enumerate}
\item 課程に新しい教材を追加する

\medskip
\begin{tabular}{|l|l|l|}\hline
入力 & (課程,教材リスト) & \verb|(CO1,[BK1, BK2, …])| \\ \hline
出力 & なし & \verb|ok|\\ \hline
\end{tabular}
\medskip

\item 学習進捗記録を初期化する

\medskip
\begin{tabular}{|l|l|l|}\hline
入力 & 教材 & \verb|BK1| \\ \hline
出力 & 学習進捗記録 & \verb|SB1(BK1, [])|\\ \hline
\end{tabular}
\medskip

\end{enumerate}

\item{学習者}

\begin{enumerate}
\item 課程一覧の中からまだ修了していないものを検索する

\medskip
\begin{tabular}{|l|l|l|}\hline
入力 & 学生 & \verb|ST1| \\ \hline
出力 & 課程リスト & \verb|[CO1,CO2, …]| \\ \hline
\end{tabular}
\medskip

\item 選択した課程の中でまだ完了していない教材を学習する

\medskip
\begin{tabular}{|l|l|l|}\hline
入力 & 課程 & \verb|[CO1]|\\ \hline
出力 & 教材リスト & \verb|(CO1,[BK1, BK2, …])| \\ \hline
\end{tabular}
\medskip

学習記録が存在しない場合は作成される。
例えば上の例で\verb|BK1|を初めて学習するならば、
\verb|SB1(BK1,[SR1(ST1,[])])|が作成される。
学習者の完了試験の結果はその教材の学習記録に追加される。

\end{enumerate}

\item{教育担当者}

\begin{enumerate}

\item 教材を学習したすべての学習者の学習記録を取得する

学習記録は学習が遅い、または正答率が低い学生を見つけ、指導するために使われる。

\medskip
\begin{tabular}{|l|l|l|}\hline
入力 & 教材 & \verb|BK1|\\ \hline
出力 & 学習記録 & 
\verb|SB1(BK1,[SR1(ST1,[...70...]), |\\
 & &
\verb|         SR2(ST2,[...30...])])|\\ \hline
\end{tabular}
\medskip

上の例では\verb|ST2|の正答率が低いことから
学習が遅れていることを発見し、指導する。


\item 教材の改善箇所を発見する

完了試験の問題ごとの正答率を計算し、境界値を下回るものがあれば
その問題に対応する学習項目を説明している箇所を改善対象とする。

\medskip
\begin{tabular}{|l|l|l|}\hline
入力& (教材, 試験問題) & \verb|(BK1,Q1)|\\ \hline
履歴& 学習進捗記録 & 
\verb|[SR1(ST1,[Q1(OOXX)]), |\\
 & &
\verb| SR2(ST2,[Q1(OXOX)]),...])|\\ \hline
出力& 問題ごとの正答率 &
\verb|Q1(0.7, 0.5, 0.5, 0.1)|\\ \hline
\end{tabular}
\medskip

上の例では教材\verb|BK1|の完了試験\verb|Q1|に対し、
受験者全員の問題ごとの正解/不正解の履歴を取得し、正答率を計算する。
この時境界値を\verb|0.2|とすると、問題4は期待される正答率を下回っているので、
対応する学習項目の改善が必要になると判定する。

\end{enumerate}

\item{課程設計者}

\begin{enumerate}
\item 課程改善(教材の追加)の必要性を発見する

教材改善進捗記録を取得し、
教材の改善により変化した正答率を境界値と比べる。
改善を繰り返しても正答率が境界値を下回る場合には
課程自身の改善が必要と判定する。
具体的には学習者の理解度を向上させる教材を新たに作成し課程に
追加することを教育担当者に依頼する。

\verb|SH1|を課程\verb|CO1|に対応する教材改善進捗記録、
\verb|BK2...BK5|を改善された教材、
\verb|SB2...SB5|を対応する学習進捗記録とする。
教材\verb|BK1|の完了試験を\verb|Q1|とする。
\verb|Q2|以降も同様である。

\medskip
\begin{tabular}{|l|l|l|}\hline
入力&課程& \verb|CO1|\\ \hline
対象&教材改善進捗記録& \verb|SH1([SB1, ..., SB5])|\\ \hline
履歴& 学習進捗記録 & 
\verb|SB1(BK1,|\\
 & &
\verb|[SR1(ST1,[Q1(OOXX)]),|\\ 
& & 
\verb| SR2(ST2,[Q1(OXOX)]),]),|\\ 
& & 
\verb|SB5(BK5,|\\
 & &
\verb|[SR1(ST1,[Q5(OOOX)]), |\\
 & &
\verb| SR2(ST2,[Q5(OXOX)]),]),|\\ \hline
出力& 問題ごとの正答率 &
\verb|Q1(0.7, 0.5, 0.1, 0.1)|\\ 
& &
\verb|Q2(0.7, 0.5, 0.2, 0.1)|\\ 
& &
\verb|Q5(0.7, 0.5, 0.5, 0.2)|\\ \hline
\end{tabular}
\medskip

上の例では、課程設計者が\verb|CO1|の改善を検討する場合である。
最初に登録した教材\verb|BK1|を複数の学習者が学習し、学習記録が作成された後で、
問題ごとの正答率を確認する。
その結果問題3と問題4の正答率が低いことが発見され、
教育担当者が該当する学習項目の内容を改善した教材\verb|BK2|を作成し登録する。


問題3と問題4に対してこの課程を繰り返した結果
問題3に対しては\verb|BK5|では期待される正答率を得られているが、問題4に対しては改善の効果が
ほとんど見られないことが判明した。
課程設計者が問題4を分析した結果から\verb|CO1|で最初に学習する教材をより理解が容易なものにすることを決定し
教育担当者に作成を依頼する。この結果、課程が以下のように変更される。

\medskip
\begin{tabular}{|l|l|l|}\hline
改善前& (課程,教材リスト) & \verb|(CO1,[BK1])| \\ \hline
改善後&& \verb|(CO1,[BK0, BK1])|\\ \hline
\end{tabular} 
\medskip

改善された課程を別の学習者の集合に学習してもらった後で、
同様に正答率の上昇率を確認する。
\verb|BK0|を問題4の理解に必要な基礎知識を含むように作成することで、
改善後は\verb|BK1|を学習するのは\verb|BK0|の試験の合格者のみであるから、
\verb|Q1|の問題4は理解している学習者の割合が増えるため正答率の上昇効果が期待できる。

\end{enumerate}

\end{enumerate}

\clearpage



\chapter{実装}
%% \section{実装}
%%概要

システムを実装する目的は各ステークホルダーがシステムに対して操作を行う際に
必要な情報が全て与えられているかを確認すること、
操作に対して必要な情報が正しく返されるかを確認することである。
このため入出力は全て文字情報(CUI)とする。
また認証機能なども省略する。

システムは以下の部分から構成される。

\begin{enumerate}

\item 学習管理部

ユーザが機能を選択した時に、必要な情報の入力と出力を対話的に実現する。
選択可能な項目は文字として表示され、
ユーザは番号を入力することで対応する項目を選択する。
ナビゲーションは「進む」「戻る」などを番号で選択する。
試験の際の選択肢の出力と解答の入力も同じ仕組みを用いる。

\item 情報保持部

最初に教材関連情報(課程、教材、試験など)を入力して登録する。
選択肢が発生する場合は必要に応じて
学習管理部のメニュー選択機能を利用する。

\item 情報分析部

学習が可能(まだ完了試験に合格していない)な教材を示し、
その中から学習する教材を選択する仕組みを持つ。


\end{enumerate}


\clearpage



\section{構成}

システムの構成要素である学習管理部のクラス図を図\ref{fig:学習管理部}に、
情報保持部のクラス図を図\ref{fig:情報保持部}に、
情報分析部のクラス図を図ref{fig:情報分析部}に示す。

学習管理部ではランナー(\verb|Runner|)が起動クラスとなりシステムを制御する。
他にはユーザ(\verb|User|)、ライブラリ(\verb|Library|)とスクリーン(\verb|Screen|)のクラスがある。
ユーザは課程設計者(\verb|Designer|)、教育担当者(\verb|Professor|)、学習者(\verb|Student|)をサブクラスにもつ。


\begin{figure}[H]
  \vspace{0.5cm}
  \hspace*{0.5cm}
  % \includegraphics[width=13cm]{../UI.pdf}
  \caption{学習管理部}
  \label{fig:学習管理部}
\end{figure}

情報保持部で最大のものが課程(\verb|Course|)である。これは達成目標を表す文書(\verb|text|)を含む。
以下すべてのクラスは文書を持ち、必要に応じて画面に表示する。

課程は複数の教材(\verb|Book|)と修了試験(\verb|QualificationExam|)から構成される。
教材にはそれを学習できる制限を表すランク(\verb|rank|)が付随する。
学習者も同様にランクを持っており、自分のランクより高い教材を選択することができない。

\begin{figure}[H]
  \vspace{0.5cm}
  \hspace*{0.5cm}
  % \includegraphics[width=13cm]{../教材関連.pdf}
  \caption{情報保持部}
  \label{fig:情報保持部}
\end{figure}

教材は複数の学習単位(\verb|Chapter|)と完了試験(\verb|CompletionExam|)から構成される。
学習単位は複数の学習項目(\verb|Section|)と確認試験(\verb|ConfirmationExam|)から構成される。
学習単位は理解するまで何度でも繰り返し学習ができる。
これは前の学習(\verb|prev|)と次の学習単位(\verb|next|)をリンクで保持することで実現される。

試験はすべて\verb|Exam|のサブクラスとして実現する。合格基準(\verb|border|)を保持し、
受験者がこれ以上の得点を得た場合に合格とする。
試験は複数の設問(\verb|Question|)から構成される。
設問はその内容を文書及び図として含む\verb|Box|を保持する。
解答(\verb|Answer|)は複数存在し、それぞれに対して得点(\verb|point|)が決められている。
正解/不正解の他に部分的正解として部分点を与えることが可能である。

このように試験を通じて、学習者が学習内容の理解度を確認し、
次の学習内容へ進む条件を満たすかどうかを判断する。
理解度が不足する場合は、学習者まだ把握していない内容を確定し、再学習ができる。
教育担当者側も、説明が不適切な内容を確定し修正できる。


\begin{figure}[H]
  \vspace{0.5cm}
  \hspace*{0.5cm}
  % \includegraphics[width=13cm]{../履歴関連.pdf}
  \caption{情報分析部}
  \label{fig:情報分析部}
\end{figure}


情報分析部で最大のものが教材改善進捗記録(\verb|ScoreHistory|)である。
これは学習した人数を表す数字(\verb|count|)と正答率及び合格率の上昇率(\verb|Improvment|)を含む。
教材改善進捗記録は課程設計者(\verb|Designer|)に利用され、課程(\verb|Course|)の改善に役に立つ。

教材改善進捗記録は複数の学習進捗記録(\verb|ScoreBook|)から構成される。
学習進捗記録にはそれを学習した人数を表す数字(\verb|count|)と正答率及び正答率(\verb|ratios|)が付随する。
学習進捗記録は教育担当者(\verb|Professor|)に利用され、
改善された教材はバージョン(\verb|version|)を更新された上で課程に登録される。
学習者は常に最新のバージョンのみ閲覧できる。

学習進捗記録は複数の学習記録(\verb|ScoreRec|)から構成される。
学習記録は試験内容(\verb|mark|)学習時間(\verb|date|)、獲得した点数(\verb|Score|)を含む、
学習者(\verb|Student|)から利用する。

このような履歴内容による、学習者の学習状況を把握できる。
教育担当者と課程設計者も教材と課程の修正に役に立つデータを獲得できる。


\clearpage


\section{挙動}


複数の学習者が
ユースケース「学習」を起動した後で、
教育担当者が「教材改善のための分析」を起動し、
すべての学習者の正答率を取得するまでの
呼び出しと戻り値を図\ref{fig:シーケンス図}に示す。

ユースケース「学習」の最初に学習記録が作成される。
完了時に完了試験の成績が学習記録に追加される。

教育担当者は教材に対応する学習進捗記録に新しい成績が追加される度に、
正答率が低い学習項目があるかを調べる。

\begin{figure}[H]
  \vspace{0.5cm}
  \hspace*{0.5cm}
  % \includegraphics[width=13cm]{../シーケンス図0.pdf}
  \caption{シーケンス図}
  \label{fig:シーケンス図}
\end{figure}

項目が発見されたら、その項目を対象に
記述が簡潔で分かりやすく改善した教材を用意し、
それをユースケース「教材の登録」により情報保持部に追加する。









\chapter{評価}
%% 評価

%% 4章のアーキテクチャが事例に上げた5個のシステムを実現しよう時に
%% そのまま適用できるものは何ですか、適用できないものは?、


% %% 例3強制:毎日最初に試験をうけることを強制するシステム。
% 試験はアーキテクチャ内に含まれている
% 但しアーキテクチャでは試験は学習完了時しかできない
% 変更点は利用者が試験をうけられる機会を増やす
% 学習管理部を修正し、最初に試験をうけるように修正する。

% 上記の変更はやさしいのか難しいのか。
前章で提案したアーキテクチャが2章で上げたシステムの再構成に適用可能かどうかを評価する。
ただし最後の協調学習監視システムは
提案アーキテクチャの性質
(各ステークホルダーが自由な時間にアクセスを行い結果を得られる)
では考慮していない上、
対象システムで使用される学習場面のビデオ画像を表示する機能を想定していないので、
評価の対象外にする。


\begin{enumerate}\itemsep=-1ex

\item 情報セキュリティ教育用教材(ELSEC)

\item 拡張可能な学習支援システムアーキテクチャ(ELECOA)

\item 情報セキュリティ教育の強制実行(SEC-EL)

\item 学習習熟度を用いた自主学習支援システム(AR-PBL)

\item 脆弱性対策教育システム(VulCES)

\end{enumerate}

\begin{description}

\item{ELSEC}

提案アーキテクチャは課程に含む教材を確定し、
全部学習しないと修了試験を受けられない点においては
ELSECと同様である。
これは達成目標に対応した教材の選択と学習者の達成度を保証する手段となる。
しかし提案アーキテクチャの中に達成目標という概念が含まれているが、
課程と達成目標の関連付けが正しいかどうかを確認する基準がまだ明確になっていないので、
学習者が求めるものを提供する機能の実現が困難である。

\item{ELECOA}

独習用教材をグループ学習でも再利用できるように学習制御機能を標準化
することにより、相互運用性を実現しているのが特徴である。
提案アーキテクチャではナビゲーション機能に相当する。
教材を選択した後では任意の学習項目の閲覧などが可能であるが、
複数の教材を予め決められた順序で閲覧したり、
グループに所属するすべての学習者が同時に学習を開始し、
それぞれの進捗状況を共有する機能は持っていない。
提案アーキテクチャでこれを実現するためには
ナビゲーション機能を拡張する方法が考えられる。
例えばランナーに複数の課程に所属する
教材の学習順序を記述した情報を与える、
ある教材の学習を完了する時に、
次に学習することをすすめる教材リストをあげる。
と同時に、複数の教材の学習順序が強制的に確定すれば、
ある教材を学習した後、
次の教材を強制的にする必要があるけど、
現行のシステムで強制ができないので、
現在のシステムの上でどのよう拡張すれば良いのか、
もっと慎重に検討しなければならない。
%% 進捗状況の共有する(制限あり)






\item{SEC-EL}

毎日最初に試験をうけることを強制するシステム。
試験はアーキテクチャ内に含まれている
但しアーキテクチャでは試験は学習完了時しかできない
変更点は利用者が試験をうけられる機会を増やす
学習管理部を修正し、最初に試験をうけるように修正する。	
と同時に、提案アーキテクチャは教材の最後まで学習するのを前提として実行しているので、時間がかかる。
それゆえ、毎日強制実行なら、普段の生活あるいは仕事に影響があるので、実現が不可能である。
しかし、学習項目ごとに実行する理解度確認テストに変えれば、実現可能と考える。
または強制実行の件については、業務システムと連携が必要であるが、
ユーザの行動を制限するためには業務システムのメニューを制限するなど内部を変更する必要性が高い、
しかし業務システムはほとんどの場合でクローズド、すなわち外部からの修正を想定していないので、
技術的に実現が困難である。

\item{AR-PBL}

学生が自身の学習に適した資料を適切に選択することができる。
具体な手段は専門用語の出現を自動計測し、教材の難易度を推定する。
提案アーキテクチャでは
教材の難易度を計測する方法は考慮していないが、
特定の方法で計測可能とするシステムを用意し、
その出力を利用できるように教材の難易度を表す
情報の構造を変更することで実現が可能である。

%% 運用方法の変更(新しい制約)
具体的には教育担当者が教材を登録する時に初級と設定したが、
計測システムの判定により初級と合わない場合では、
難易度が合っていないことを教育担当者に通知する。
ほかにも学習者が新しい教材を選択する時に同様に難易度の不一致を通知する。

\item{VulCES}

様々なセキュリティ攻撃に対して適切な防衛スキルを習得するために
実験用ツールと組み合わせたシステムである。
実システムと完全に分離された実験用システムの上で
様々な攻撃とそれに対して防衛を実体験することで
攻撃の検出方法や防御の有効性を理解する。

提案アーキテクチャでこれと同様な教育を行うには
理解度確認テストする前に実験環境に入るための仕組みを追加する必要がある。
しかし繋がる前に実験環境内に実験装置を構築するために必要な情報を決めて渡す必要がある。
ほかに、実験に成功した場合は、
実験装置は提案アーキテクチャ側に結果に伝える必要があるが
その方法に理解度確認テストの成績として与えることが考えられる。
ただし、現在の試験は選択肢が決められており、また配点も決められている。
例えば防御実習の成績をこの試験の枠組みに適合させることが可能か、
もし可能なら選択肢と配点をどのように決定するかなどの問題を解決する必要がある。
検討の結果現状では適合が困難だと判断される場合は、新しい試験の形態を構築し
それをアーキテクチャで実現する(例:試験のサブクラス)方法を考える必要がある。

\end{description}

\clearpage


以上5つのシステムを提案したアーキテクチャを用いて再構成する際に
そのままでよい部分と追加変更が必要な部分をまとめたものを
表\ref{tbl:アーキテクチャを利用した従来システムの再構成}に示す。

\medskip
\medskip

\begin{table}[H]
  \caption{アーキテクチャを利用した従来システムの再構成}
  \label{tbl:アーキテクチャを利用した従来システムの再構成}

\medskip

\begin{center}
\begin{tabular}{|l|c|c|c|c|c|} \hline
&ELSEC&ELECOA&SEC-EL&AR-PBL&VulCES\\ \hline
学習管理部&○&×&×&△&△\\ \hline
情報保持部&△&○&△&△&△\\ \hline
情報分析部&△&○&○&○&△\\ \hline
\end{tabular}
\medskip
\medskip

\begin{tabular}{ll} 
○&提案したアーキテクチャでそのまま、あるいはデータを変更\\
 &するのみで実現可能\\
△&ユースケースや制御フローはそのままでよいが、データ構造\\
 &の変更、あるいは新しいデータとそれらは扱う操作の追加が必要\\
×&制御構造の大幅な変更、ユースケースの変更、情報の変更\\
 &などの大規模の変更が必要\\
\end{tabular}
\end{center}
\medskip

\end{table}

%% ここにそれぞれに必要な変更点を書く。
\begin{itemize}
\item ELSEC情報保持部

情報保持部の
教材(\verb|Book|)には
難易度に対応した値(\verb|rank|)が設定されているが、
学習者の達成目標と教材中の各学習項目の詳細度の対応を表す情報を保持していない。
これを実現するには\verb|rank|に相当する情報を
\verb|Book|のほかに学習項目(\verb|Section|)にも
追加すればよい。
必要ならば、対応関係を保持するクラスを追加し、
目的に応じた教材を選択するメソッドを持たせるが可能である。

\item ELSEC情報分析部

達成目標と学習項目の内容との関連は学習進捗記録などの構造及び
改善に必要な情報提供を行うユースケースには影響しない。
ただし、適切な教材を選択するためのユースケースが追加される。

% 学習進捗記録(\verb|ScoreBook|)と
% 教材改善進捗記録(\verb|ScoreHistory|)では、
% それぞれ、正答率(\verb|ratios|)と
% 正答率及び合格率の上昇率(\verb|Improvment|)を含む。
% そのゆえ、情報分析部正答率をめぐって分析するが、
% 達成目標との遠回しかどうかの判断、
% または正答率への影響も考慮すべきと考える。

\item ELECOA学習管理部

学習管理部では、ランナー(\verb|Runner|)が
スクリーン(\verb|Screen|)の表示順を制御している。
%% \verb|Screen|の切り替えはまたメニュー(\verb|Menu|)で選択肢があり。
ELECOAでは複数の教材を決められた順序で提示する機能が必要であり、
これはナビゲーション機能の拡張で実現する。
例えば教材の順序を記述したデータ(スクリプト)の読み込みと、
任意のページから閲覧を開始するための仕組みを\verb|Runner|に追加する。

% 課程で教材の学習順番を確定することだけでなく、
% ある教材の学習完了後、
% 他学習者の学習経験から得る、
% 次におすすめ教材へのナビゲーションなどを提供することが有利と考える。
% そのゆえ、\verb|Runner|と\verb|Menu|の変更が必要とする。



\item SEC-EL学習管理部

SEC-ELは起動時に解説と確認テストを強制実行することが求められるため、
ELECOAと同様の\verb|Runner|の拡張が必要である。

ただし現在の\verb|Runner|は特定の順序で\verb|Screen|を表示する機能と
ナビゲーションを強制する(他の操作を選択不可能にする)仕組みを持っていない。
また\verb|Runner|を使用した1つの学習項目が完了するまで
業務システムの操作を制限する仕組みも必要であるが、
これは業務システムがオープンであることが前提条件である。

\item SEC-EL情報保持部

情報保持部には毎日の業務開始前に行われる
理解度確認テストの結果を保持する領域を拡張する。
分析方法も異なる場合は情報分析部にも影響する可能性がある。

\item AR-PBL学習管理部

現在の学習管理部は教材を選択する際に
学習者の理解度に一致するかのみを判断していて、
詳細の難易度を表す情報とそれを計測する仕掛けを持っていない。
AR-PBLで用いる外部計測システムから詳細情報を取得し、
それを教材難易度ランクに変更する基準を必要とする。
また計測結果を\verb|Book|,\verb|Chapter|,\verb|Section|
それぞれ対応する箇所に格納するとともに、
それらの値と教育担当者が登録時に設定した\verb|rank|の値と不整合が発生した場合に、
教育担当者に再登録を促す仕組みも必要とする。

\item AR-PBL情報保持部

外部システムの計測結果に基づき、
学習者の理解度をより詳細に表現し、
教材選択に活用する仕組みが必要である。
例えばこの学習者は\verb|Course1|は用語をほとんど知らないので
初級から始めるべきであるが、
\verb|Course2|は基本的な概念は理解しているので
中級からはじめてもよい、という内容の情報を扱えるようにする。


\item VulCES学習管理部

学習した時点でVulCESのような実験装置を利用するを前提としたら、
学習管理部は、学習者を実験環境に移すを要求するので、
学習過程を制御する\verb|Runner|を変わる必要がある。
そのゆえ、\verb|Runner|に学習項目完了する時点で、
実験環境へ移す条件と要求を渡して、判断する、
判断結果の上で、自動的に実験あるいは試験を実行ことを要求する。

\item VulCES情報保持部

確認テストの代わりに実験を行う場合に、
学習項目によるどのような実験環境が必要か、
これらの情報は、教材とどもに情報保持部に保持する必要がある。
例えば理解度確認試験(\verb|ConfirmationExam|)の
内容と構造を変わって、実験環境のデータも加入する。


\item VulCES情報分析部

情報分析部では試験後正答率を分析する仕組みがあり、
そのゆえ、テストの代わりに実験を行う場合に、
どのよう外部実験環境から戻る情報を分析するか、
特定の分析方法が必要とする。
例えば実験環境からの戻り値は\verb|ratios|に変える。
状況判断にかかる時間による点数付けなどの方式で実現する。

\end{itemize}


%% \chapter{議論}


\clearpage


\chapter{おわりに}
%% 終わりに

本研究の目的にはeラーニングに対応する
様々な利用者の望みを答えるために、
機能や実行環境が異なる利用形態においても
対応可能なアーキテクチャを提案・評価することである。
機能要求は調査したシステム例により様々であった。
研究方法としては調査したシステムによりELSの特性をまとめ、
まとめた結果に基づいて、アーキテクチャを提案し、評価する。
結果としては適切な学習内容の選択、
最後までの学習保証、学習効果の確認と
教育内容不足の発見と改善、ナビゲーション機能などを実現した。
しかしながら、課程と学習目標の関連付け、複数の教材の学習順序、
教材の難易度と学習者能力の測りがまだ不足しているので、
実現するためには、まだ慎重な検討が必要となる。
またシステムの強制実行と実験環境と連携するの実現は
現在のアーキテクチャ構成と現実の面で実現が難しいことと考えられる。

本研究の社会的意義はIT業界など人材不足の現状を緩和するために、
自由度が高いeラーニングの利用形態を多様化したら、
現場により役に立つ学習道具になると考える。

最後に、本研究の残された課題と今後の発展については、
まず課程と学習目標の関連付け、複数の教材の学習順序、
教材の難易度と学習者能力の測りに関することの実現をより詳しく検討すること。
これ以外に、システムの強制実行と
実験環境との連携を実現することが困難であるため、
他の要求が満たされる、
実現可能な手段があるかどうかを検討することである。



\renewcommand{\bibname}{参考文献}
\begin{thebibliography}{99}
\bibitem{el1} 川上, 佐々木:情報セキュリティ教育のための
e ラーニング教材作成システム
ELSEC のフィッシング対策教育への適用,2011,
情報処理学会論文誌 Vol.52 No.3 1266-1278(Mar.2011)
\bibitem{el2} 仲林,森本:拡張性を有する
学習支援システムにおける再利用性向上のための
教材オブジェクトデザインパターンの設計と実装, 2018,
教育システム情報学会誌 Vol. 35, No. 3 2018 pp. 248‒259
\bibitem{el3} 坂上,森山,上田:強制実行型
情報セキュリティ教育システムの開発と評価, 2017,
情報システム学会 第 13 回全国大会・研究発表大会
\bibitem{el4} 花田,大場:学習習熟度を用いた
PBL 向け自主学習支援システムの構築, 2016,
「情報教育シンポジウム」2016年8月
\bibitem{el5} 竹下,小林,佐々木:脆弱性対策教育のためのeラーニングシステムの開発と評価, 2011,
日本セキュリティ・マネジメント学会誌 24(1), 17-26, 2010-04
\bibitem{el6} 北川,上原...:IT教育サポートツール「MultiVNC」の開発, 
2004,社団法人 情報処理学会 研究報告 2004-CE-77 (13) 2004/11/20
\end{thebibliography}
\end{document}
