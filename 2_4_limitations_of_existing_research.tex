前述した関連研究は、ソフトウェア欠陥予測において重要な知見を提供してきたが、いくつかの問題点が残されている。

第一に、Kameiらの研究で提案された変更メトリクスには、プロジェクト特性や開発体制への依存性という問題がある。例えば、AGE(変更間隔)は商用プロジェクトでは有効だが、OSSプロジェクトではボランティアベースで開発が行われるため、開発頻度が不定期であり効果的ではない。NDEV(開発者数)も、商用プロジェクトではチームでの共同作業が多いため有効だが、OSSプロジェクトでは1つの変更は基本的に1人の開発者が担当するため効果的ではない。NS(サブシステム数)はプロジェクトによってサブシステムの総数が異なるため、プロジェクト間での比較が困難である。これらのメトリクスは、特定の開発体制やプロジェクト特性に依存しており、汎用性に欠ける。

第二に、最も重要な問題点として、既存研究ではコミット間の変化量が十分に考慮されていないことが挙げられる。Kameiらの研究では、コミット単位での特徴量(変更されたファイル数、追加・削除行数など)を用いているが、これらは単一のコミットの特性を表すものであり、前回のコミットからどのように変化したかという時系列的な情報は含まれていない。同様に、Ferencらの研究でも、各コミット時点でのコードメトリクスは測定されているが、コミット間の変化量は対象外である。コードの変化の過程には、バグ混入リスクを示す重要な情報が含まれている可能性があるが、この観点からの体系的な分析は不足している。例えば、循環的複雑度が短期間に大きく増加したメソッドや、頻繁にコード行数が変動するメソッドは、バグ混入リスクが高い可能性があるが、こうした変化のパターンは従来の研究では捉えられていない。

第三に、特性値同士の関連性分析が欠如している。既存研究では、個々のメトリクスが独立して評価されることが多く、複数のメトリクスがどのように相互作用してバグ混入リスクに影響を与えるかについての分析は限られている。例えば、変更されたファイル数が多く、かつ各ファイルの変更規模も大きい場合、これらの特徴量の組み合わせがバグ混入リスクにどのような影響を与えるかは明らかになっていない。Kameiらの研究では多重共線性への対処として高相関の因子を除去しているが、これは統計的な問題を回避するための処理であり、メトリクス間の本質的な関連性を分析するものではない。

第四に、Hanらの研究が示すように、コードレビューでは70\%のケースで欠陥の原因が明示されないため、レビューテキストのみから欠陥を予測することには限界がある。レビュアーは単に問題を指摘するだけで、その理由を詳しく説明していないことが多い。このため、自然言語処理によるレビューテキスト分析だけでは、バグ混入の原因を特定することは困難である。

これらの問題点は、コミット間の変化量を考慮し、異なる粒度でのメトリクスを統合し、メトリクス間の関連性を分析する新たなアプローチの必要性を示唆している。本研究では、これらの課題に対処するため、メソッド単位とコミット単位という異なる粒度での時系列変化を捉える手法を提案する。