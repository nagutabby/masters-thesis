ソフトウェア開発では継続的な変更が不可避であり、各変更はバグ混入のリスクを伴う。変更により既存コードとの整合性が崩れ、副作用が発生しやすくなるためである。Microsoft Researchの調査では76\%の開発者がリファクタリングによるバグ混入を懸念している\cite{kim2014}。

また、ソフトウェアライフサイクルにおいて、保守プロセスはライフサイクル費用のかなりの部分を占める重要なフェーズである。保守には、是正保守、予防保守、適応保守、完全化保守の4つのタイプがあり\cite{jisx0161}、それぞれがソフトウェアの品質維持に不可欠な役割を果たす。保守プロセスでは、どの修正依頼や問題報告を優先的に処理すべきかの判断が重要な課題となる。特に是正保守では、限られた保守リソースをいかに効率的に配分し、重大な欠陥を早期に発見・修正するかが、ライフサイクルコスト削減の鍵となる。しかし、保守対象のコードは継続的に変化し、各変更が新たな欠陥を混入させるリスクを伴うため、静的な分析だけでは十分な予測精度が得られない。

機械学習による欠陥予測の研究が進められているが、既存手法はコードの静的特徴量に依存している。静的特徴量では変更頻度や変更パターンといった動的な品質リスク要因を捉えられないため、予測精度に限界がある。コードがどのように変化してきたかという時系列情報は、開発プロセスの動的な側面を反映し、欠陥との強い相関が期待されるが、十分に活用されていない。