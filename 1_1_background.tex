現代のソフトウェア開発プロセスは、従来のウォーターフォール型から、アジャイル開発やCI/CD(継続的インテグレーション/継続的デリバリー)を軸とした迅速かつ反復的なプロセスへと変化している。これは、市場の不確実性が高まり、ユーザーニーズの多様化に対応し続けることがビジネスの成否を分ける要素となったためである。このような環境下では、リリース後もソースコードの継続的な変更が不可欠となる。

しかし、頻繁な変更は新しいコードと既存のコードとの整合性を損なう可能性があり、副作用として新たな欠陥が混入する要因となる。ソフトウェアの欠陥の修正コストは、発見が遅れるほど指数的に増大することが知られている。開発の初期段階での修正に比べ、テスト工程や本番環境での発見は、修正作業のみならず、影響範囲の特定、再テスト、デプロイの再実行を伴うため、多くの工数を消費し、ユーザー体験の低下を招く。したがって、欠陥を早期に特定して対処することは品質保証における重要な課題となっている。

これまで、データ分析に基づく欠陥予測の研究では、ソースコードの構造的な特徴に着目していた。しかし、コードが絶えず変化する現代の開発現場においては、ソフトウェアの構造のみを分析するだけでは構造の変化に関する特徴を十分に捉えきれず、予測精度の向上に寄与しにくい。また、プロジェクトごとに異なる変更単位や変更頻度を考慮した分析手法も確立されていない。

さらに、レビュー工数のの配分も課題の1つである。全ての変更に対して詳細なレビューを行うことは労力の制約上困難であるため、開発現場では「どの変更を優先的に確認すべきか」という意思決定を支援するための指標が求められている。