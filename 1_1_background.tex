ソフトウェア開発では継続的な変更が不可避であり、各変更はバグ混入のリスクを伴う。変更により既存コードとの整合性が崩れ、副作用が発生しやすくなるためである。Microsoft Researchの調査では76\%の開発者がリファクタリングによるバグ混入を懸念している\cite{kim2014}。

従来の品質保証手法には限界がある。静的解析はコードの時系列変化を捉えられないため、頻繁に変更される不安定な箇所や、変更が集中する高リスク領域を識別できない。コードレビューでは全ての変更を詳細に検査する時間的な余裕がなく、どの変更を優先的にレビューすべきかの判断基準が不明確である。特に大規模プロジェクトでは、限られたレビューリソースの効率的配分が重要な課題となっている。

機械学習による欠陥予測の研究が進められているが、既存手法はコードの静的特徴量に依存している。静的特徴量では変更頻度や変更パターンといった動的な品質リスク要因を捉えられないため、予測精度に限界がある。コードがどのように変化してきたかという時系列情報は、開発プロセスの動的な側面を反映し、欠陥との強い相関が期待されるが、十分に活用されていない。