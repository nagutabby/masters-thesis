ソフトウェア開発において、システムは一度完成したら終わりではなく、継続的な機能拡張や改善が繰り返される。ユーザーの要求は時間とともに変化し、ビジネス環境や技術環境も日々進化するため、既存のコードベースに対する変更作業は避けられない。こうした継続的な開発サイクルにおいて、既存コードへの変更は常にバグ混入のリスクを伴う。

特に、コードの品質を改善するために実施されるリファクタリングにおいても、このリスクは避けられない。Microsoft Researchの調査によれば、76\%の開発者がリファクタリングによるバグ混入のリスクを認識している\cite{kim2014}。これは、既存のコードベースに変更を加えることが、予期しない副作用を引き起こし、システムの品質を低下させる可能性があることを示している。システムの規模が大きくなり、機能拡張が重ねられるほど、コードの複雑性は増大し、変更の影響範囲を正確に把握することが困難になる。

従来、ソフトウェアの品質保証には静的解析やコードレビューといった手法が用いられてきた。静的解析は、コードを実行することなく構文や構造を検査し、潜在的な問題を検出する手法である。しかし、モジュール内部や外部の構造が変化した場合、静的解析だけでは品質改善が困難になる。これは、静的解析が特定時点でのコードスナップショットに基づいており、コードがどのように変化してきたかという時系列的な情報を考慮しないためである。

一方、コードレビューは、開発者が互いのコードを確認し合うことで品質を担保する手法である。しかし、限られた時間とリソースの中で全ての変更を詳細にレビューすることは現実的ではない。特に大規模なプロジェクトでは、日々多数のコミットが行われるため、どの変更を優先的にレビューすべきかを判断することが重要な課題となっている。効率的なレビューを実現するためには、バグ混入リスクの高い変更を事前に特定し、限られたレビューリソースを適切に配分する仕組みが必要である。

これらの課題に対して、近年では機械学習を用いたソフトウェア欠陥予測の研究が注目されている。しかし、既存の多くの研究では、コードの静的な特徴量のみを用いた予測や、コミット単位での分析が主流であり、コードがどのように変化してきたかという時系列的な情報を十分に活用できていない。継続的な機能拡張によるコードの変化の過程には、バグ混入リスクを示す重要な情報が含まれている可能性があるが、この観点からの体系的な分析は不足している。