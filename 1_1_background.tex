ソフトウェア開発プロセスは、従来のウォーターフォール型から、アジャイル開発やCI/CD(継続的インテグレーション / 継続的デリバリー)を中心としたプロセスへと変化してきた。これは、ユーザーニーズの多様化に伴い、リリース後も市場の要求に対応し続けることがビジネス上の課題となったためである。

このような環境においては、ソフトウェアへの継続的な変更は避けられない。しかし、頻繁に行われるソフトウェア変更は、既存のコードとの整合性を損なうリスクを常に孕んでおり、副作用として新たな欠陥を混入させる要因となる。

従来、機械学習を用いた欠陥予測の研究が多く行われてきたが、その多くはコードの静的な構造に依存している。しかし、ソフトウェアの構造が絶えず変化する現代の開発現場においては、静的な分析だけではその構造を反映するコード変更の特徴を十分に捉えきれず、さらなる予測精度の向上が難しい。