ソフトウェア開発では継続的な変更が不可避であり、各変更はバグ混入のリスクを伴う。Microsoft Researchの調査では76\%の開発者がリファクタリングによるバグ混入を懸念している\cite{kim2014}。

従来の品質保証手法には限界がある。静的解析はコードの時系列変化を捉えられず、コードレビューは全変更を詳細に検査する時間的余裕がない。特に大規模プロジェクトでは、どの変更を優先的にレビューすべきかの判断が重要な課題となっている。

機械学習による欠陥予測の研究が進められているが、既存手法はコードの静的特徴量に依存し、コードがどのように変化してきたかという時系列情報を十分に活用できていない。