ソフトウェア開発において、継続的な変更は避けられない。各変更は既存コードとの整合性を崩し、副作用を引き起こすリスクを伴う。Microsoft Researchの調査では、76\%の開発者がリファクタリングによる欠陥混入を懸念している\cite{kim2014}。

ソフトウェアライフサイクルにおいて、保守フェーズはコスト全体の大部分を占める\cite{jisx0161}。特に是正保守では、限られたリソースで重大な欠陥を早期に発見・修正することがコスト削減の鍵となる。しかし、コードが継続的に変化する環境では、静的な分析だけでは予測精度に限界がある。

機械学習による欠陥予測の研究が進められているが、既存手法の多くはコードの静的特徴量に依存している。静的特徴量では、変更頻度や変更パターンといった動的なリスク要因を捉えられない。コードの時系列的な変化は開発プロセスの動的な側面を反映し、欠陥との強い相関が期待されるが、十分に活用されていない。