本論文の構成は以下の通りである。

第2章では、ソフトウェア品質と欠陥予測に関する関連研究を示す。構造的メトリクス、変更メトリクス、および機械学習を用いた欠陥予測手法の動向を整理し、既存手法における時系列情報の扱いとレビュー労力の推定における課題を明確にする。

第3章では、本研究が提案する欠陥予測およびレビュー優先度付け手法について述べる。変更の不規則性を考慮した特徴量設計、メソッド単位とコミット単位の特徴量の活用方法、およびレビュー労力の推定モデルの構築について述べる。

第4章では、提案手法の有効性を検証するための実験環境について述べる。データセットの選定理由、前処理の手順、および評価指標の設定について説明する。

第5章では、実験結果を報告する。段階的な評価を通じて、各メトリクス群が予測性能に与える寄与度を明らかにするとともに、レビュー労力に対する欠陥発見率の累積曲線を用いて、提案手法がレビューに費やせる労力の配分の効率化にどの程度貢献するかを検証する。

第6章では、実験結果に基づき、変更の規模と欠陥混入確率の関係性や、プロジェクト間での特性の相違について考察する。また、本研究で採用した手法やデータセットに関連する制約についても議論する。

第7章では、本研究の結論をまとめ、今後の展望を述べる。