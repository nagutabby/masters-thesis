ナップサック問題を解くための代表的なアルゴリズムとして、動的計画法と貪欲法がある。

動的計画法は最適解を求めることができるが、計算量が $O(NC_{total})$ であり、容量 $C_{total}$ が大きいほど計算時間が増加する。レビュー労力の場合、$C_{total}$ は全コミットの補正済み労力の和(あるいはその一部)となるため、非常に大きな値になる可能性がある。実際に動的計画法を実装して実験を試みたところ、メモリ容量が不足し、プログラムが終了してしまった。

貪欲法は近似解を求める手法であり、計算量は $O(N \log N)$(ソートが必要な場合)である。貪欲法では、アイテムを「価値と重さの比(密度)」の降順にソートし、密度が高いものから順にナップサックに入れていく。最適解は保証されないが、計算時間が容量に依存せず、実用的な時間で解を得られる。実験では、貪欲法により数秒で処理が完了した。

本研究では、計算時間とメモリ効率を考慮し、貪欲法を採用する。

各コミット $i$ の密度 $D_i$ を以下のように定義する。

\[
D_i = \frac{V_i}{W_i} = \frac{\hat{y}_i}{E_{\text{adj}, i}}
\]

ここで、$\hat{y}_i$ はモデルが予測したコミット $i$ のバグ混入確率、$E_{\text{adj}, i}$ はコミット $i$ の補正済みレビュー労力である。

貪欲法のアルゴリズムは以下の通りである。

\begin{enumerate}
    \item 全てのコミットについて密度 $D_i$ を計算する
    \item 密度の降順にコミットをソートする
    \item 累積労力 $W_{\text{累積}} = 0$ とする
    \item ソートされた順にコミットを選択し、以下を実行する:
    \begin{itemize}
        \item $W_{\text{累積}} + W_i \leq C_{total}$ であれば、コミット $i$ をレビュー対象に追加し、$W_{\text{累積}} \leftarrow W_{\text{累積}} + W_i$ とする
        \item そうでなければ、コミット $i$ をスキップする
    \end{itemize}
    \item 累積労力が容量を超えるまで、または全てのコミットを検討するまで繰り返す
\end{enumerate}

この貪欲法により、限られたレビュー労力の中で、バグ発見期待値を効率的に最大化できる。