%% 



\begin{itemize}

\item 達成度の保証

\begin{itemize}
\item 学習者のレベルに合った教材の選択(ELSEC,SEC-EL,AR-PBL)

ELSECは達成目標に対応した教材リストあり、
学習者が自分レベルにより順番に学習することができる。

SEC-ELは問題の難易度と教育対象者の理解度をランクを付加、
教育対象者の理解度に適した難易度の問題を出題する。
教材選択の際も条件を満たすもののみを可能にする。

AR-PBLは専門用語の出現を自動計測し、教材の難易度を推定する。
そのゆえ、習熟度に応じた教材の提供を実現する。

\item ナビゲーションの制約(ELSEC,SEC-EL)

ELSECは学習者の達成度の保証手段として、
教材リストが定義されており、全部学習しないと、修了試験を受けられない。

SEC-ELは似たような、教材ごとに前提となる他の教材のリスト付加、
一つでも履修しないなら選択できない。

\end{itemize}

\item グループ学習(ELECOA)

ELECOAはグループ学習における学習制御機能として
複数の学習者間での連携、援助が可能である。

\item ユーザビリティ(MultiVNC)

MultiVNCは複数の学習者の中から理解度不足ものの発見と改善することができるので、
教育担当者は理解度を把握し、学習者の学習に適切なサポートの提供が可能である。


\item 他システムとの連携(VulCES)

VulCESは実験システムとの連携により、学習者の活動の選択肢を追加する。

\end{itemize}




