%% \subsection{ELSの達成目標}

ELS(e-Learning System)とは情報技術を用いて
利用者が自分の目的や都合にあわせて
学習する内容を登録したり閲覧したりすることを可能にする
学習支援システムである。
ELSの利用者は大きく2つに分けられる。
それぞれを学習者、教育担当者と呼ぶことにする。

\begin{enumerate}

\item{学習者}

学習者はELSを使って学習し、スキルを取得する人である。
例えばオンライン教育に参加する学生または会社新入社員などが該当する。

\item{教育担当者}

教育担当者は教材を作成し学習者に提供する。
学習結果を監視し、必要ならば教材の改善をする。
例えば教師及び新人教育部門などが該当する。

\end{enumerate}


%% 理解度を確認するために最後に試験をうける。
%% 試験で決められた数の正解をすると、合格証明を取得する。
%% 予め定められた教材の集まり(課程)の合格証明をすべて取得したら、修了試験をうけることができる。
%% 合格したら修了証明を取得する。これにより学習目標が達成されたことを示す。

図\ref{fig:ELSによる効果}は利用者の役割を示したものである。
教材は教育担当者が作成しELSに登録する。
学習者は目標の達成に必要な
学習内容を含む教材を選択する。
教材の最後には学習者が目標を達成したかを確認する試験があり、
学習者の成績は学習記録としてELSに保存され、
教育担当者は学習記録を分析することで
学習者が目標を達成するための援助を効率的に実現できる。

\begin{figure}[H]
  \vspace{0.5cm}
  \hspace*{0.5cm}
%%  \includegraphics[width=13cm,bb=0 350 800 600]{../usage.pdf}
  % \includegraphics[width=13cm]{../usage.pdf}
  \caption{ELSによる効果}
  \label{fig:ELSによる効果}
\end{figure}

%% ELSを利用して達成すべき目標

%% 利用者にとっては...

%% 教育担当者にとっては...

\clearpage