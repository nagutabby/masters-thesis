%% 目的

本課題研究の目的は、
実行形態の多様性を実現可能な
eラーニングシステムの
アーキテクチャについて調査し、
機能や実行環境が異なる利用形態においても
対応可能なアーキテクチャを提案・評価することである。
ただし教育内容については対象とせず、
システムにより提供されている内容の中から
利用者の目的に沿ったものを選択する仕組みに
限定するものとする。

教材作成者の立場では、
学習者の要求に応じた教材を構成するためには
それぞれの構成要素について様々な特性を定義する必要がある。

例えば、テキストAには初級者用、
テキストBには中級者用という特性が定義されている。
初級者用の教材を構成するためには、
初級者用という特性を持つ全ての構成要素を収集、組み合わせる。
中級用の教材では、
中級者特性を持つ構成要素のみを組み合わせるが、
学習者が中級者用の教材を十分に理解できない場合には、
初級者用教材も提供する必要がある。

従来は学習者の理解度を確認するための仕組みと
教材を切り替える仕組みを個別に作成する必要があったが、
本研究が示すガイドラインに従えば、
収集する構成要素の誤りや見落としを回避することができる。
作成者がこのガイドラインに従うことにより、
学習者が学習の進捗に応じて柔軟に教材を選択、
構成する機能を容易に提供可能になる。
構成要素の収集と組み合わせの機能の一部を
学習者に公開することにより、
学習者がそれぞれの目的に応じて、
複数の教材の中から必要な部分のみを抽出して
独自の教材を構成することができる。
その結果学習者は学習効率を、
教材作成者はより多くの利用者を得ることにより
教材の品質をそれぞれ向上させることが可能になる。

\clearpage
