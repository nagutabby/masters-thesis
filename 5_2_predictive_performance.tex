本節では、提案手法の予測性能を評価した結果を報告する。評価は、ベースライン、ステップ2(メソッドメトリクス追加)、ステップ3(コミットメトリクス追加)の3段階で実施した。

\paragraph{テストデータによる評価結果}
表\ref{tab:evaluation}に、5つのプロジェクトにおける各段階のF1スコアとROC-AUCを示す。

\begin{table}[ht]
\centering
\caption{テストデータによる評価(F1スコアとROC-AUCの向上)}
\label{tab:evaluation}
\begin{tabular}{|l|l|r|r|}
\hline
プロジェクト & モデル & F1スコア & ROC-AUC \\
\hline
Elasticsearch & ベースライン & 0.5754 & 0.7746 \\
 & ステップ2 & 0.7075 & 0.8802 \\
 & ステップ3 & \textbf{0.7669} & \textbf{0.9254} \\
\hline
Hazelcast & ベースライン & 0.6783 & 0.8747 \\
 & ステップ2 & 0.7332 & 0.8998 \\
 & ステップ3 & \textbf{0.7901} & \textbf{0.9316} \\
\hline
Neo4j & ベースライン & 0.4782 & 0.7130 \\
 & ステップ2 & 0.6158 & 0.8296 \\
 & ステップ3 & \textbf{0.7420} & \textbf{0.9382} \\
\hline
Netty & ベースライン & 0.4548 & 0.7054 \\
 & ステップ2 & 0.6607 & 0.8815 \\
 & ステップ3 & \textbf{0.7466} & \textbf{0.9337} \\
\hline
OrientDB & ベースライン & 0.4833 & 0.7376 \\
 & ステップ2 & 0.5241 & 0.7647 \\
 & ステップ3 & \textbf{0.7012} & \textbf{0.9141} \\
\hline
\end{tabular}
\end{table}

実験の結果、全てのプロジェクトにおいて、ステップを経るごとに予測性能が一貫して向上した。特にステップ3(提案手法)はベースラインと比較して、F1スコアで最大0.29ポイント(Netty)、ROC-AUCにおいても全プロジェクトで0.91を超える高い値を達成した。
また、マクネマー検定の結果、全てのプロジェクトにおいて$p < 0.05$となり、この性能向上は統計的に有意であることが確認された。

\paragraph{交差検証による安定性}
10分割交差検証の結果を表\ref{tab:crossvalidation}に示す。各フォールドにおける標準偏差は小さく、提案手法がデータの分割に依存せず安定した性能を発揮することを確認した。

\begin{table}[ht]
\centering
\caption{10分割交差検証の結果(平均値 ± 標準偏差)}
\label{tab:crossvalidation}
\begin{tabular}{|l|l|r|r|}
\hline
プロジェクト & モデル & F1スコア & ROC-AUC \\
\hline
Elasticsearch & ステップ3 & 0.7910 ± 0.0242 & 0.9313 ± 0.0161 \\
Hazelcast & ステップ3 & 0.7833 ± 0.0185 & 0.9314 ± 0.0111 \\
Neo4j & ステップ3 & 0.7393 ± 0.0329 & 0.9360 ± 0.0111 \\
Netty & ステップ3 & 0.7076 ± 0.0275 & 0.9252 ± 0.0160 \\
OrientDB & ステップ3 & 0.6636 ± 0.0251 & 0.8985 ± 0.0192 \\
\hline
\end{tabular}
\end{table}