提案手法の予測性能を3段階で評価した。表\ref{tab:evaluation}に、5つのプロジェクトにおけるテストデータでの評価結果を示す。

\begin{table}[ht]
\centering
\caption{テストデータによる評価}
\label{tab:evaluation}
\begin{tabular}{|l|l|r|r|}
\hline
プロジェクト & モデル & F1スコア & AUC \\
\hline
Elasticsearch & ベースライン & 0.575 & 0.775 \\
 & ステップ2 & 0.708 & 0.880 \\
 & ステップ3 & \textbf{0.767} & \textbf{0.925} \\
\hline
Hazelcast & ベースライン & 0.678 & 0.875 \\
 & ステップ2 & 0.733 & 0.900 \\
 & ステップ3 & \textbf{0.790} & \textbf{0.932} \\
\hline
Neo4j & ベースライン & 0.478 & 0.713 \\
 & ステップ2 & 0.616 & 0.8300 \\
 & ステップ3 & \textbf{0.742} & \textbf{0.938} \\
\hline
Netty & ベースライン & 0.455 & 0.705 \\
 & ステップ2 & 0.661 & 0.882 \\
 & ステップ3 & \textbf{0.747} & \textbf{0.934} \\
\hline
OrientDB & ベースライン & 0.483 & 0.738 \\
 & ステップ2 & 0.524 & 0.765 \\
 & ステップ3 & \textbf{0.701} & \textbf{0.914} \\
\hline
\end{tabular}
\end{table}

全プロジェクトでステップを経るごとに性能が一貫して向上した。ベースラインから提案手法(ステップ3)へのF1スコアの改善幅は、Hazelcastの0.11からNettyの0.29まで分布し、平均改善幅は0.21であった。AUCは全プロジェクトで0.91を超え、高い識別性能を達成した。特にNeo4jとNettyでは、ベースラインのF1スコアが0.48程度と低かったが、提案手法で0.74程度まで向上し、欠陥予測の実用性が大きく改善された。

ステップ2からステップ3への改善も全プロジェクトで確認され、コミット単位の変更メトリクスが予測に役立つ情報を提供することが示された。改善幅はプロジェクトによって異なり、Neo4jでは0.13、Nettyでは0.09であった一方、OrientDBでは0.18と最大の改善を示した。これは、プロジェクトの特性によって、メソッド単位とコミット単位の各メトリクスの価値が異なることを示唆している。

McNemar検定により、提案手法とベースラインの性能差の統計的有意性を検証した。全プロジェクトで$p < 0.05$となり、性能向上は統計的に有意であることが確認された。

\paragraph{交差検証による安定性}
10分割交差検証により、データの分割に依存しない安定性を評価した。表\ref{tab:crossvalidation}に結果を示す。

\begin{table}[ht]
\centering
\caption{10分割交差検証の結果(平均値 ± 標準偏差)}
\label{tab:crossvalidation}
\begin{tabular}{|l|l|r|r|}
\hline
プロジェクト & モデル & F1スコア & AUC \\
\hline
Elasticsearch & ステップ3 & 0.791 ± 0.024 & 0.931 ± 0.016 \\
Hazelcast & ステップ3 & 0.783 ± 0.019 & 0.931 ± 0.011 \\
Neo4j & ステップ3 & 0.739 ± 0.033 & 0.936 ± 0.011 \\
Netty & ステップ3 & 0.708 ± 0.028 & 0.925 ± 0.016 \\
OrientDB & ステップ3 & 0.664 ± 0.025 & 0.899 ± 0.019 \\
\hline
\end{tabular}
\end{table}

各プロジェクトでF1スコアの標準偏差は0.02から0.03程度、AUCの標準偏差は0.01から0.02程度に収まっており、提案手法はデータの分割方法に大きく依存せず安定した性能を発揮することが確認された。