第5章で得られた主要な結果を簡潔に要約し、本章で考察すべき疑問を明確化する。提案手法(ステップ3)は、5つのプロジェクトすべてでベースラインと比較してF1スコアが向上した。Cost-Benefit分析では、20\%のレビュー労力で全バグの70〜75\%を検出可能であることが示された。しかし、これらの結果からいくつかの疑問が生じる。

\paragraph{なぜソフトウェアの変更サイズが小さく変更ファイル数が少ないほど欠陥混入確率が高くなるのか?}
Partial Dependence Plot分析により、追加行数や変更ファイル数などの変更規模を表すメトリクスの値が増加するにつれて、バグ混入確率が低下する傾向が全プロジェクトで一貫して観察された。これは直感に反する結果である。

\paragraph{なぜプロジェクトごとに欠陥予測に影響を与える特徴量が異なるのか?}
特徴量重要度を分析した結果、Nettyではトークン数の変化量が最重要特徴量であったのに対し、他のプロジェクトでは追加行数や変更ファイル数が上位を占めた。プロジェクト間で重要な特徴量が異なる理由は何か。

\paragraph{なぜプロジェクト間で欠陥予測の精度が異なるのか?}
F1スコアの改善幅はNettyで0.29、Hazelcastで0.11と最大2.6倍の差が見られた。この性能差は何に起因するのか。

\paragraph{なぜコミット単位とメソッド単位の変更メトリクスを統合すると性能が向上するのか?}
ステップ2(メソッド単位のみ)からステップ3(コミット単位を追加)への性能向上のメカニズムは何か。

\paragraph{本研究の評価方法とデータセットは、これらの疑問に答えるために妥当か?}
SZZアルゴリズムやBugHunter Datasetの制約が結果にどのように影響しているか。

本章では、これらの疑問に対して既存研究の知見と本研究のデータを用いて考察する。