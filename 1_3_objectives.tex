本研究の目的は、コードの時系列変化を考慮した欠陥予測により、限られたレビューリソースの効率的な配分を支援することである。具体的には以下を達成する。

\textbf{目的1: 時系列メトリクスの設計}

メソッド・コミット単位の変化を組み合わせ、ミクロとマクロの両視点から変更特性を捉える。メソッドレベルの分析だけでは全体の変更パターンや開発の勢いを見逃し、コミットレベルの分析だけでは局所的な変更の詳細が不明確になる。両者を統合することで、リスク要因を包括的に捉える。

\textbf{目的2: 高精度予測モデルの構築}

時系列特徴量を用いた機械学習モデルにより、従来手法を上回る予測性能を実現する。

\textbf{目的3: レビュー労力削減効果の定量評価}

ナップサック問題としての定式化により、同一労力においてバグ発見数が増加することを検証する。この定式化は、限られたレビュー時間で最大のバグ発見数を得るという問題構造が、ナップサック問題の定義と一致するため適切である。