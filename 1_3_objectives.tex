本研究の目的は、コードの時系列変化を考慮した欠陥予測により、限られたレビューリソースの効率的配分を支援することである。具体的には以下を達成する。

\textbf{目的1: 時系列メトリクスの設計}

メソッド単位の変化量とコミット単位の変化率を組み合わせ、ミクロとマクロの両視点から変更特性を捉える。

\textbf{目的2: 高精度予測モデルの構築}

時系列特徴量を用いた機械学習モデルにより、従来手法を上回る予測性能を実証する。

\textbf{目的3: レビュー労力削減効果の定量評価}

ナップサック問題としての定式化により、同一労力でのバグ発見数向上をCost-Benefit Curveで検証する。