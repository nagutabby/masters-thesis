本研究の目的は、コードの時系列変化を考慮したソフトウェア欠陥予測手法を確立し、バグ混入リスクの高い変更を事前に特定することで、限られたレビューリソースの効率的な配分を支援することである。

具体的には、以下の3つの目的を設定する。

第一に、コードの時系列変化を捉えるための適切なメトリクス設計を行う。従来の欠陥予測研究では、特定時点でのコードの静的な特徴量が主に用いられてきたが、本研究では、コミット間の変化に着目した時系列的な特徴量を導入する。特に、メソッド単位での変化量とコミット単位での変更特性という、異なる粒度での特徴量を組み合わせることで、ミクロ的視点とマクロ的視点の両面から変更の特性を捉える。メソッド単位の変化量は、個々のメソッドがどの程度変更されたかを示し、コミット単位の変更特性は、1つのコミット全体での変更の影響範囲や性質を表現する。これらのメトリクスを設計することで、コード変更に伴うバグ混入リスクをより正確に捉えることを目指す。

第二に、設計したメトリクスを用いて、機械学習による高精度な欠陥予測モデルを構築する。時系列変化を考慮した特徴量を機械学習モデルに組み込むことで、従来の静的な特徴量のみを用いた手法と比較して、予測性能がどの程度向上するかを実証的に評価する。特に、アンサンブル学習手法を用いることで、複雑な特徴量間の非線形な関係を捉え、実用的な予測精度を達成する。また、構築したモデルの解釈性を高めるため、どの特徴量が予測に寄与しているかを分析し、実務での活用可能性を示す。

第三に、提案手法によるレビュー労力削減効果を定量的に評価する。従来のレビュー優先度付け手法では、全てのコミットのレビュー労力が等しいと仮定していたが、実際にはコミットごとにレビューに必要な労力は大きく異なる。本研究では、レビュー対象のコミット選択をナップサック問題として定式化し、限られたレビュー労力の中でバグ発見期待値を最大化する手法を提案する。具体的には、各コミットのレビュー労力をコードチャーン、変更ファイル数、Entropyから算出し、モデルが予測したバグ混入確率をその労力で除した密度に基づいて、貪欲法によりレビュー対象コミットを選択する。また、同じレビュー労力でより多くの欠陥を発見できることを、Cost-Benefit Curveを用いて確認する。

これらの目的を達成することで、本研究は、継続的な機能拡張が行われるソフトウェア開発において、バグ混入リスクの事前予測と効率的な品質保証活動を支援する実用的な手法を提供する。