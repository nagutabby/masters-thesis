本研究の目的は、コードの時系列変化を考慮した欠陥予測により、限られたレビューリソースの効率的な配分を支援することである。具体的には以下を達成する。

\textbf{目的1: 変更メトリクスの統合}

メソッド単位とコミット単位の変更メトリクスを統合し、局所的な変化とプロジェクト全体の動向の両面から変更特性を捉える。

\textbf{目的2: 高精度予測モデルの構築}

時系列特徴量を用いた機械学習モデルにより、従来手法を上回る予測性能を実現する。

\textbf{目的3: レビュー労力削減効果の定量評価}

ナップサック問題として定式化し、同一労力で欠陥発見数が増加することを検証する。