特徴量重要度を分析した結果、プロジェクトによって重要な特徴量が異なることが確認された。例えば、Nettyではトークン数の変化量が最も重要な特徴量であったのに対し、他の多くのプロジェクトでは追加行数や変更ファイル数が上位を占めた。本節では、この差異が生じる理由について考察する。

\paragraph{プロジェクトドメインとアーキテクチャの影響}
プロジェクトのドメインやアーキテクチャ特性が、重要な特徴量に影響を与えている可能性がある。Nettyは非同期ネットワークフレームワークであり、責務が明確で焦点が絞られている。ネットワークプロトコルの実装では、わずかなトークンレベルの変更(例: バイトオーダーの処理、フラグビットの操作)が重大な欠陥を引き起こす可能性がある。そのため、トークン数の変化量のような詳細な変更メトリクスが重要になると考えられる。
一方、ElasticsearchやHazelcastのような分散システム基盤は、広範な機能セットを持ち、変更の影響範囲が広い。このようなプロジェクトでは、コミット全体の規模が、変更の複雑性や影響範囲をより適切に捉える指標となる。
Neo4jやOrientDBのようなデータベースシステムでは、トランザクション処理や状態管理の複雑性が高い。これらのプロジェクトでは、変更の広がりが重要な指標となる傾向が見られた。

\paragraph{開発プロセスとツールの影響}
プロジェクトの開発プロセスや使用しているツールも、重要な特徴量に影響を与える可能性がある。高いテストカバレッジを持つプロジェクトでは、自動テストによって構造的メトリクスだけでもある程度の欠陥検出が可能である。このような場合、変更メトリクスによる追加的な情報の価値が相対的に低くなる。また、厳格なコードレビュープロセスを持つプロジェクトでは、小規模な変更でも慎重にレビューされるため、変更サイズと欠陥リスクの関係が弱まる可能性がある。

\paragraph{欠陥の種類による影響}
プロジェクトで発生する欠陥の種類によって、重要な特徴量が異なる可能性がある。ロジックバグは、条件分岐の複雑さや変更の頻度と関連が深く、コードの理解しやすさや保守性を測る特徴量が重要になる。セキュリティバグは、特定のセキュリティ問題(例: 入力検証の欠如、認証処理の不備)を持つコード箇所で発生しやすく、コードの複雑性や外部ライブラリとの連携に関する特徴量が重要になると考えられる。本研究では欠陥の種類を区別せず一律に扱っているが、欠陥の種類ごとに異なる予測モデルを構築することで、より高精度な欠陥予測が可能になると考えられる。これは今後の重要な研究課題である。