ナップサックの容量 $C_{total}$(レビューに使える総労力)の設定には注意が必要である。

素朴なアプローチとして、全コミットの労力の和を $C_{total}$ とする方法が考えられる。しかし、極端に大きな労力を要するコミット(例えば、数千行の変更を含むコミット)が存在する場合、この設定は不適切である。実際の開発現場では、レビューに使える労力には上限があり、巨大なコミットは分割されるか、別の品質保証プロセスが適用される。

本研究では、以下の手順で $C_{total}$ を設定する。

\begin{enumerate}
    \item 全コミットをレビュー労力 $W_i$ の昇順にソートする
    \item 上位80\%のコミット(レビュー労力が小さい方から80\%)を選択する
    \item 選択されたコミットのレビュー労力の和を $C_{total}$ とする
\end{enumerate}

\[
C_{total} = \sum_{i \in S_{80\%}} W_i
\]

ここで、$S_{80\%}$ は労力の小さい順に並べた上位80\%のコミットの集合である。

この設定により、極端に大きな労力を要するコミット(上位20\%)の影響を除外し、より現実的なレビュー労力の制約をモデル化できる。80\%というしきい値は、実験的に決定されたものであり、プロジェクトの特性に応じて調整可能である。