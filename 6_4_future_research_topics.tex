本研究の成果を発展させるため、以下の研究課題が考えられる。

まず、より多様なプロジェクトでの評価が必要である。特に、異なるプログラミング言語、開発規模、ドメインにおける提案手法の有効性を検証することで、汎用性を高められる。また、企業との共同研究により、商用プロジェクトでの実証実験を行うことも重要である。

次に、動的な環境への適応が課題となる。プロジェクトの進行に伴うコードベースの変化や開発チームの学習効果を考慮し、予測モデルを継続的に更新する仕組みが求められる。オンライン学習やアクティブラーニングの手法を取り入れることで、この課題に対処できる可能性がある。

さらに、レビュープロセス全体の最適化に向けた拡張が考えられる。本研究ではレビュー対象の選定に焦点を当てたが、レビュアーの割り当てやレビューの深さの調整なども含めた総合的な最適化が、より大きな効果をもたらすだろう。

最後に、他の品質保証活動との統合も重要な研究方向である。コードレビューは品質保証活動の一部であり、自動テストや静的解析など他の活動と組み合わせることで、さらなる効率化が期待できる。複数の品質保証活動を統合的に最適化する枠組みの構築は、実務上の価値が高い研究テーマである。