本研究の成果は、ソフトウェア開発プロセスにおいて以下の適用可能性を持つ。

まず、限られたレビューリソースを最大限に活用するための具体的な戦略を提供できる。多くの開発組織では、すべてのコード変更を詳細にレビューする時間的余裕がない。提案手法を用いることで、欠陥を含む可能性が高いコード変更を優先的にレビューし、全体としての品質向上とリソース効率化を両立できる。

次に、コストベネフィット分析の枠組みは、レビュープロセスの継続的改善に活用できる。プロジェクトの進行に伴い、レビューにかけられる工数やプロジェクトの品質要求は変化する。本研究で提案した分析手法により、現状のレビュー戦略が適切かどうかを定量的に評価し、必要に応じて調整することが可能となる。

さらに、メトリクスに基づく欠陥予測は、開発者への教育的なフィードバックとしても機能しうる。どのような特徴を持つコード変更が欠陥を引き起こしやすいかを可視化することで、開発者はより品質の高いコードを書くための指針を得られる。