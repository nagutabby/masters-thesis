%% 終わりに

本研究の目的にはeラーニングに対応する
様々な利用者の望みを答えるために、
機能や実行環境が異なる利用形態においても
対応可能なアーキテクチャを提案・評価することである。
機能要求は調査したシステム例により様々であった。
研究方法としては調査したシステムによりELSの特性をまとめ、
まとめた結果に基づいて、アーキテクチャを提案し、評価する。
結果としては適切な学習内容の選択、
最後までの学習保証、学習効果の確認と
教育内容不足の発見と改善、ナビゲーション機能などを実現した。
しかしながら、課程と学習目標の関連付け、複数の教材の学習順序、
教材の難易度と学習者能力の測りがまだ不足しているので、
実現するためには、まだ慎重な検討が必要となる。
またシステムの強制実行と実験環境と連携するの実現は
現在のアーキテクチャ構成と現実の面で実現が難しいことと考えられる。

本研究の社会的意義はIT業界など人材不足の現状を緩和するために、
自由度が高いeラーニングの利用形態を多様化したら、
現場により役に立つ学習道具になると考える。

最後に、本研究の残された課題と今後の発展については、
まず課程と学習目標の関連付け、複数の教材の学習順序、
教材の難易度と学習者能力の測りに関することの実現をより詳しく検討すること。
これ以外に、システムの強制実行と
実験環境との連携を実現することが困難であるため、
他の要求が満たされる、
実現可能な手段があるかどうかを検討することである。
