\paragraph{データセットに起因する限界}
BugHunter DatasetではSZZアルゴリズムを用いてバグ混入コミットを特定しているが、このアルゴリズムには既知の問題がある。Herboldらの調査によれば、SZZはバグ修正コミットの約5分の1を見逃し、SZZがバグ修正として識別したコミットのうち、実際にバグ修正であるのは約半分である\cite{herbold2022}。これはSZZが、イシュートラッカーに記録されたバグのみをラベルなどのメタデータに基づいて追跡することが原因である。そのため、本研究で得られたモデルの予測精度が実際のJIT品質保証における予測精度よりも高い可能性がある。

さらに、本研究では各プロジェクトのデータセットから抽出するレコード数の上限を5,000行としているが、この制限が結果に影響を与えるかもしれない。より多くのデータを使用すれば性能がさらに向上する可能性がある。

\paragraph{手法に起因する限界}
本研究では機械学習アルゴリズムとしてランダムフォレストを採用したが、他の機械学習アルゴリズム(XGBoost、LightGBM、ニューラルネットワークなど)との比較は行っていない。ランダムフォレストを選択した理由は、特徴量重要度やPartial Dependence Plotといった解釈可能性の高い分析をするためだが、予測精度の観点では他のアルゴリズムが優れている可能性がある。また、ハイパーパラメーター(決定木の数、深さ、最小サンプル数など)の最適化は限定的である。より細かなハイパーパラメーターチューニングにより、性能がさらに向上する可能性がある。

次に、特徴量エンジニアリングにおいてメソッドの変更タイプを導入したが、変更の意図が見逃されている。例えば、機能追加やリファクタリングという目的を特徴量として加えることで、性能が向上する可能性がある。

最後に、Cost-Benefit分析におけるレビュー労力の計算方法は実際のレビュー労力と完全には一致しない。なぜなら、レビュー労力は変更行数、変更範囲、変更の複雑さだけでなく、レビュアーの経験やコードベースの知識にも依存するためである。

\paragraph{一般化可能性の限界}
本研究は、Javaで記述されたプロジェクトを対象としている。オブジェクト指向言語特有の性質(継承、ポリモーフィズムなど)が結果に影響している可能性があり、他の言語では異なる傾向が見られる可能性がある。

また、プロジェクト選択に関するバイアスも存在する。本研究では、BugHunter Datasetに含まれる15プロジェクトの中から、バグレポート数が多い上位5プロジェクトを選定した。これらは全て活発なOSSプロジェクトであり、商用プロジェクトや小規模プロジェクトには本研究で提示した手法が効果的でない可能性がある。特にOSSプロジェクトと商用プロジェクトでは、開発プロセス、コードレビューの厳格さ、テストカバレッジ、ドキュメントの質などが異なる。