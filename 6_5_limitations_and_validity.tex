\paragraph{データセットに起因する制約}
BugHunterデータセットでは、欠陥混入コミットの特定にSZZアルゴリズムを採用している。SZZは、バグ修正コミットから変更履歴を遡って欠陥を混入したコミットを特定する手法である。

SZZアルゴリズムの長所は、大規模なデータセットを自動的に構築できる点と、明確なアルゴリズムに基づくため再現性が高い点である。数千から数万のコミットを含むプロジェクトにおいて、人手によるラベル付けは現実的でないが、SZZはVCSと課題管理システムの情報を活用することで効率的にラベル付きデータを生成できる。

一方で、SZZには精度に関する課題も存在する。Herboldらの調査によれば、SZZは欠陥修正コミットの約5分の1を見逃し、SZZが欠陥修正として識別したコミットのうち実際に欠陥修正であるのは約半分である\cite{herbold2022}。これは、SZZが課題管理システムに記録されたバグのみを追跡するため、軽微な修正や開発中に発見された問題を検出できないことに起因する。

本研究では、これらのトレードオフを考慮した上で、以下の理由からBugHunterデータセットを採用した。SZZの短所はベースラインと提案手法に等しく影響するため、相対的な性能比較には大きな影響を与えない。ただし、本研究で得られたモデルの絶対的な予測精度は、実際のJIT品質保証における真の予測精度と乖離している可能性があることに留意する必要がある。

さらに、本研究では各プロジェクトのデータセットから抽出するレコード数の上限を5,000行としているが、この制限が欠陥予測精度を低下させ、レビュー対象のコミットの選択を誤らせる恐れがある。より多くのデータを使用すれば性能がさらに向上する可能性がある。

\paragraph{手法に起因する制約}
本研究では機械学習アルゴリズムとしてランダムフォレストを採用したが、近年の関連研究では、ディープニューラルネットワーク(DNN)がより高いF1スコアを達成することが報告されている。DNNが優れた性能を示す理由として、複数の隠れ層を通じて非線形な特徴表現を自動的に学習できる点と、特徴間の高次の相互作用を効果的にモデル化できる点が挙げられる。例えば、変更の規模、開発者の経験、コードの結合度が複雑に組み合わさって欠陥混入リスクが上昇するといった非線形な関係を、DNNはより柔軟に捉えられる可能性がある。

一方、本研究がランダムフォレストを選択した理由は、解釈可能性を重視したためである。本研究の目的は予測精度の最大化だけでなく、どの特徴が欠陥予測に寄与するかを理解することにある。ランダムフォレストは特徴量重要度やPartial Dependence Plotといった分析手法を提供し、これらは6.2節の変更規模と予測確率の関係や、6.3節のプロジェクト間の特徴量重要度の違いといった考察に不可欠である。DNNは予測精度が高い一方で、内部表現がブラックボックス化しやすく、このような詳細な分析が困難である。

また、本研究では、ランダムフォレストのハイパーパラメーター(決定木の数、深さ、最小サンプル数など)の最適化を実施していない。ハイパーパラメーターチューニングにより、性能がさらに向上する可能性がある。

次に、特徴量エンジニアリングにおいてメソッドの変更タイプを導入したが、変更の意図が見逃されている。例えば、機能追加やリファクタリングという目的を特徴量として加えることで、性能が向上する可能性がある。

最後に、レビュー労力に対する欠陥発見数の分析におけるレビュー労力の計算方法は実際のレビュー労力と完全には一致しない。なぜなら、レビュー労力は変更行数、変更範囲、変更の複雑さだけでなく、レビュアーの経験やコードベースの知識にも依存するためである。

\paragraph{ケーススタディの制約}
本研究は、Javaで記述されたプロジェクトを対象としている。プログラミング言語の特性は、欠陥の予測精度に影響を与える可能性がある。

第一に、型システムの違いが影響する。Javaは静的型付け言語であり、コンパイル時に型エラーを検出できるため、型関連の欠陥は少ない。PythonやJavaScriptのような動的型付け言語では、型関連の欠陥が実行時まで検出されず、欠陥の種類と分布が異なる可能性がある。

第二に、メモリ管理の違いが影響する。Javaはガベージコレクションによる自動メモリ管理を採用しているため、メモリリークやポインタエラーは発生しにくい。C++のような手動メモリ管理を行う言語では、これらの欠陥が頻繁に発生し、変更規模と欠陥リスクの関係が異なる可能性がある。

第三に、プログラミングパラダイムの違いが影響する。Javaはオブジェクト指向を基本とするため、継承やポリモーフィズムに関連する欠陥が発生しやすい。本研究で用いた一部の構造的メトリクスは、オブジェクト指向言語に特化しており、関数型言語では異なるメトリクスが重要となる可能性がある。

したがって、本研究で得られた知見をJava以外の言語に適用する際には、言語特性に応じてメトリクスや予測モデルを調整する必要がある。

また、プロジェクト選択に関するバイアスも存在する。本研究では、BugHunterデータセットに含まれる15プロジェクトの中から、バグレポート数が多い上位5プロジェクトを選定した。これらは全て活発なOSSプロジェクトであり、商用プロジェクトや小規模プロジェクトには本研究で提示した手法が効果的でない可能性がある。特にOSSプロジェクトと商用プロジェクトでは、コードレビューの厳格さやドキュメントの質などが異なる。