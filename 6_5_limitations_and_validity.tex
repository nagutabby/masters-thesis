\paragraph{データセットに起因する制約}
BugHunterデータセットでは、欠陥混入コミットの特定にSZZアルゴリズムを採用している。SZZは、バグ修正コミットから変更履歴を遡って欠陥を混入したコミットを特定する手法である。

SZZアルゴリズムの長所は、大規模なデータセットを自動的に構築可能である点と、明確なアルゴリズムに基づくため再現性が高い点である。数千から数万のコミットを含むプロジェクトにおいて、人手によるラベル付けは現実的でないが、SZZはVCSと課題管理システムの情報を活用することで効率的にラベル付きデータを生成可能である。

一方で、SZZには精度に関する課題も存在する。Herboldらの調査によれば、SZZは欠陥修正コミットの約5分の1を見逃し、SZZが欠陥修正として識別したコミットのうち実際に欠陥修正であるのは約半分である\cite{herbold2022}。これは、SZZが課題管理システムに記録されたバグのみを追跡するため、軽微な修正や開発中に発見された問題を検出できないことに起因する。

本研究では、これらのトレードオフを考慮した上で、以下の理由からBugHunterデータセットを採用した。SZZの短所はベースラインと提案手法に等しく影響するため、相対的な性能比較には大きな影響を与えない。ただし、本研究で得られたモデルの絶対的な予測精度は、実際のJIT品質保証における真の予測精度と乖離している可能性があることに留意する必要がある。

さらに、本研究では各プロジェクトのデータセットから抽出するレコード数の上限を5,000行としているが、この制限が欠陥予測精度を低下させ、レビュー対象のコミットの選択を誤らせる恐れがある。より多くのデータを使用すれば性能がさらに向上する可能性がある。

\paragraph{手法に起因する制約}
本研究では機械学習アルゴリズムとしてランダムフォレストを採用したが、近年の関連研究では、ディープニューラルネットワーク(DNN)がより高いF1スコアを達成することが報告されている。DNNが優れた性能を示す理由として、複数の隠れ層を通じて非線形な特徴表現を自動的に学習可能である点と、特徴間の高次の相互作用を効果的にモデル化可能である点が挙げられる。例えば、変更の規模、開発者の経験、コードの結合度が複雑に組み合わさって欠陥混入リスクが上昇するといった非線形な関係を、DNNはより柔軟に捉えられる可能性がある。

一方、本研究がランダムフォレストを選択した理由は、解釈可能性を重視したためである。本研究の目的は予測精度の最大化だけでなく、どの特徴が欠陥予測に寄与するかを理解することにある。ランダムフォレストは特徴量重要度やPartial Dependence Plotといった分析手法を提供し、これらは6.2節の変更規模と予測確率の関係や、6.3節のプロジェクト間の特徴量重要度の相違といった考察に不可欠である。DNNは予測精度が高い一方で、内部表現がブラックボックス化しやすく、このような詳細な分析が困難である。

また、本研究では、ランダムフォレストのハイパーパラメーター(決定木の数、深さ、最小サンプル数など)の最適化を実施していない。ハイパーパラメーターチューニングにより、性能がさらに向上する可能性がある。

次に、特徴量エンジニアリングにおいてメソッドの変更タイプを導入したが、変更の意図が見逃されている。例えば、機能追加やリファクタリングという目的を特徴量として加えることで、性能が向上する可能性がある。

最後に、レビュー労力に対する欠陥発見数の分析におけるレビュー労力の計算方法は実際のレビュー労力と完全には一致しない。なぜなら、レビュー労力は変更の規模や複雑度だけでなく、レビュアーの経験やコードベースの知識にも依存するためである。

\paragraph{ケーススタディの制約}
ケーススタディで使用されたプログラミング言語であるJavaの言語特性が、欠陥の傾向に影響を与えた可能性がある。Javaは静的型付けを採用し、ガベージコレクションによる自動メモリ管理を備えているため、本研究で用いたデータセット内の欠陥は、型不整合やメモリ管理不備といった低レイヤーの問題ではなく、主に論理構造の誤りに限定されている。また、本研究で採用した構造的メトリクスはオブジェクト指向パラダイムを前提として設計されており、Java特有の継承構造やクラス間の結合度を捉えるには適しているが、異なるパラダイムを持つ言語の分析にはそのまま適用できない可能性がある。

以上の理由から、本研究で提示した機械学習モデルは「Javaプロジェクトにおける典型的な欠陥混入リスク」を見つけることに特化しているといえる。したがって、本研究で得られた知見をJava以外の異なる言語特性を持つプロジェクトに適用する際には、対象言語の欠陥の傾向やパラダイムに応じたメトリクスの再選定が必要となる。

また、プロジェクト選択に関するバイアスも存在する。本研究では、BugHunterデータセットに含まれる15プロジェクトの中から、バグレポート数が多い上位5プロジェクトを選定した。これらは全て活発なOSSプロジェクトであり、商用プロジェクトや小規模プロジェクトには本研究で提示した手法が効果的でない可能性がある。特にOSSプロジェクトと商用プロジェクトでは、コードレビューの厳格さやドキュメントの質などが異なる。