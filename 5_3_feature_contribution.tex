特徴量重要度を分析した結果、提案した変更メトリクスが予測に大きく寄与することが確認された。全プロジェクトで共通して、コミット単位の変更メトリクスである\texttt{lines\_added}(追加行数)と\texttt{num\_files}(変更ファイル数)が上位を占めた。これらは変更規模を表す指標であり、欠陥混入リスクの主要な予測因子となっている。また、\texttt{entropy}(変更の広がり)も多くのプロジェクトで重要な特徴量として機能し、変更が特定箇所に集中しているか、複数箇所に分散しているかが予測に影響することが示された。

メソッド単位の変更メトリクスも重要な予測情報を提供する。特にNettyでは\texttt{tokens\_change}(トークン数の変化量)が最も重要な特徴量となった。一方、ElasticsearchやHazelcastでは\texttt{lines\_added}や\texttt{num\_files}といったコミット単位のメトリクスが上位を占める傾向が見られた。この違いは、プロジェクトのドメインや開発特性を反映していると考えられる。

図\ref{fig:feature_importance_neo4j}に、代表例としてNeo4jプロジェクトにおける特徴量重要度を示す。Neo4jでは\texttt{lines\_added}(追加行数)が最も重要な特徴量であり、次いで\texttt{num\_files}(変更ファイル数)、\texttt{entropy}(変更の広がり)が続いている。メソッド単位のメトリクスでは\texttt{tokens\_change}(トークン数の変化量)が上位に位置し、局所的な変化の情報も予測に寄与している。この結果から、コミット単位とメソッド単位の変更メトリクスを組み合わせることで、変更の全体像と局所的な変化の両面を捉えることが可能となり、予測精度の向上につながることが確認された。

\begin{figure}[ht]
\centering
\includegraphics[width=0.7\textwidth]{figures/neo4j/feature_importance_chart.png}
\caption{特徴量重要度(代表例: Neo4j)}
\label{fig:feature_importance_neo4j}
\end{figure}