
\begin{enumerate}
\item 情報セキュリティ教育用教材

%% ELSEC(東京電機大学2011)
川上, 佐々木らは拡張性や柔軟性が高い情報セキュリティ教育システムとして
ELSEC(E-Learning system %
for SECurity)を提案している\cite{el1}.
フィッシング対策教育を行うためのシナリオを作成し、その評価を行っている。
ELSECは達成目標に対応した教材の選択と学習者の達成度の保証手段として、
教材リストが定義されており、全部学習しないと、修了試験を受けられない。
これを参照し、本研究におけるシステムも課程を設定し、修了試験を含む。

\item 拡張可能な学習支援システムアーキテクチャ

%% ELECOA(千葉工業大学2017)
仲林,森本らは相互運用性と機能拡
張性の両立を図ることを目的とし
ELECOA(Extensible Learning Environment %
with Courseware Object Architecture)を提案している\cite{el2}.
ELECOAはグループ学習における学習制御機能を実現した。
例えば、他学習者の状態を条件とする分岐、
または他学習者の状態を条件とする強制移動などの機能がある。
この中、他学習者の間の連携、援助する機能が本研究に役に立つものと考える。


\item 情報セキュリティ教育の強制実行

%% SEC-EL(近畿大学2017)
坂上,森山,上田らは教育実施者・対象者への負担が少ない、
効果的な強制実行型情報セキュリティ教育システムとして、
SEC-ELを提案している\cite{el3}.
SEC-ELは教育対象者の理解度に適した難易度の問題を出題することができる。
手段としては、問題の難易度と教育対象者の理解度を計算し、
教材選択の際に条件を満たすもののみを可能にする。
例えば:学生と教材の双方にランクを付加、自身のランクを超える教材が選択できない。
また教材ごとに前提となる他の教材のリスト付加、一つでも履修しないなら選択できない。
本研究は、教材リストを参考し、課程を設計する機能を付加する。


\item 学習習熟度を用いた自主学習支援システム

%% AR-PBL(公立はこだて未来大学2016)
花田,大場らは学生が自身の学習に適した資料を適切に選択することを目的とし
AR-PBLを提案している\cite{el4}.
このシステムは習熟度に応じた教材の提供を実現する。
具体な手段は専門用語の出現を自動計測し、教材の難易度を推定する。
本研究にでもこのような難易度推定機能の利用を考えている。


\item 脆弱性対策教育システム

%% VulCES(東京電機大学,IPA2011)
竹下,小林,佐々木らは適切な防止策を理解するために
攻撃方法理解用のツールと組み合わせた
VulCES(Vulnerability Countermeasures %
E-learning System)を提案している\cite{el5}.
VulCESは実験システムとの連携し、学習の行動を拡張する。
学習者が防衛側となり、予め決められている攻撃を防衛した割合で評価する。
本研究も、実験システムの導入を考えている。


\item 生徒同士の協調学習

%% MultiVNC(アルファシステムズ2004)
北川,上原らは教師・生徒間や生徒同士による協調作業を可能とするシステムとして
MultiVNC(Virtual Network Computing)を提案している\cite{el6}.
VNCとはネットワーク上の離れたコンピュータを遠隔操作するためのRFBプロトコルを利用する、
リモートデスクトップソフトである。
VNCは通常は1対1であるが、これを教師側の画面に複数の画面を同時表示するための拡張したものが
MultiVNCである。そのゆえ、複数の学習者の中から理解度不足のものの発見と改善することができる。
本研究でも、この手段に通じて、教師側は複数の学習者の理解を把握することを望ている。

\end{enumerate}
\clearpage