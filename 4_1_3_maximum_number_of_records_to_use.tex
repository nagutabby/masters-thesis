交差検証時の評価指標の安定化を図るため、各プロジェクトのデータセットから抽出するデータポイント数の上限を5,000件とする。当初は3,000件を上限としていたが、この設定では交差検証のフェーズごとにF1スコアが大きく変動し、最悪と最良のフェーズを比較すると0.1前後の誤差が生じていた。データポイント数を5,000件に増やすことで、この変動を抑制し、より安定した性能評価が可能になった。ただし、この変動は完全には解消されておらず、データの特性に起因する本質的な課題として残っている。

各プロジェクトのデータセットから、利用可能なデータが5,000件以下の場合は全データを使用し、5,000件を超える場合は最初の5,000件を使用する。データセット内のレコードは必ずしも時系列順に並んでいるわけではないが、同一メソッドの複数のバージョン(バグ混入時と修正時)が含まれており、これらの情報を用いて変化量を計算することが可能である。