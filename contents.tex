本章では調査事例で上げた様々な利用形態に対応する
システムを実現可能なアーキテクチャについて検討する。

\section{学習の対象}

最初に学習者の達成目標に従って学習の対象を分類する。
学習者は学習の際に教材を選択するが、
目標を達成するためには特定の知識の集まりを習得する必要がある。

このため学習内容は
複数の科目及び必要な科目をすべて習得した後に、
目標の達成を確認する試験を含む必要がある。
これを課程(コース)と定義する。

以下に各要素について詳しく述べる。

\begin{itemize}\itemsep=-5pt

\item 課程

課程とはある目標を達成するために学習者が習得する必要のある学習内容の集まりである。
例えばセキュリティ課程には
セキュリティ初級(セキュリティ障害の発見)、
セキュリティ中級(解決手段)、
セキュリティ上級(セキュリティ防衛環境の構築)を
含む内容が含まれる。
課程の最後には試験が含まれており、
これに合格することが学習者の最終目標である。
この試験を修了試験と呼ぶことにする。

\item 試験

試験は目標とした学習内容を理解したかを確認するためのものである。

%% 他の試験が必要な理由
しかし、課程に含まれている学習内容の量が多くなめ、
課程修了を示す修了試験の他に、
段階的な試験を設置する必要性がある。
と同時に、修了試験の合格率が低い時に
理解不足のところを発見しやすい利点もある。

そのゆえ、課程の最後にある修了試験の他に、
教材に含まれている理解度確認をするための完了試験と、
学習単位に含まれてている学習項目の理解度ための確認試験も設置する。

% 完了試験の結果が基準点に達したら学習者がその教材を理解した証明として合格証明書を取得する。
% 合格済の教材は学習する際に選択することができない。

\item 学習項目

学習項目とはある学習内容を解説する最小単位である。
例えばセキュリティ初級の教材には、セキュリティ障害の検出と
それに対する対策などが学習項目として含まれている。
学習者は学習項目を最初から最後まで順に学習することもできるし、
前に戻って復習したり、先に進むこともできる。

\item 学習単位

学習単位は複数の学習項目と理解度確認のための確認試験から構成される。


\item 教材

教材は特定の内容の習得を目的とし、
学習すべき内容を説明した文章や図などを含む
学習学習単位と完了試験から構成される。




\end{itemize}


