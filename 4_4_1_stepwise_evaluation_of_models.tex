提案手法の各要素がどの程度予測性能に寄与するかを明らかにするため、3段階の比較評価を行う。

ステップ1: はじめに、変更メトリクスを追加する前のモデルを構築する。このモデルは、BugHunter Datasetに元々含まれているコードメトリクスのみを特徴量として使用する。このモデルをベースラインとし、後続のモデルとの比較基準とする。

ステップ2: ベースラインモデルに対して、メソッド単位の変更メトリクスを追加したモデルを構築する。これにより、メソッド単位の時系列変化を考慮することの効果を評価する。

ステップ3: ステップ2のモデルに対して、さらにコミット単位の変更メトリクスを追加したモデルを構築する。これにより、ミクロ的視点とマクロ的視点を統合することの効果を評価する。