有意性検定で帰無仮説が棄却されない問題を回避するため、BugHunterデータセットに含まれる15のプロジェクトの中から、データセットのレコード数が多いプロジェクトを優先的に選定する。


データセットの構築にはSZZアルゴリズムが用いられているため、解決済みのバグレポートの数とデータセットのレコード数の大きさには相関関係がある。そのため、解決済みのバグレポート数を基準として、データセットの相対的な大きさを推測できる。この基準に基づき、解決済みのバグレポート数が多い上位5つのプロジェクトを対象として選定した。

\begin{table}[h]
\centering
\caption{各プロジェクトのバグレポート数}
\begin{tabular}{|l|r|}
\hline
プロジェクト & バグレポート数 \\
\hline
Elasticsearch & 4,287 \\
Hazelcast & 3,762 \\
Netty & 2,207 \\
OrientDB & 1,272 \\
Neo4j & 1,152 \\
\hline
\end{tabular}
\end{table}

Elasticsearchは、分散検索・分析エンジンであり、大規模なログデータやテキストデータの検索に広く使用されている。
Hazelcastは、インメモリデータグリッドを提供する分散コンピューティングプラットフォームである。
Nettyは、高性能な非同期イベント駆動型のネットワークアプリケーションフレームワークである。
OrientDBは、マルチモデルデータベースであり、グラフデータベースとドキュメントデータベースの機能を併せ持つ。
Neo4jは、グラフデータベースの代表的な実装の一つであり、ソーシャルネットワーク分析や推薦システムなど、関係性を重視するアプリケーションで広く使用されている。

これらのプロジェクトは、いずれもJavaで記述されており、GitHubでホストされている活発なOSSプロジェクトである。ドメインも検索エンジン、分散システム、ネットワークフレームワーク、データベースと多岐にわたり、異なる開発特性を持つ。このような多様性により、提案手法が特定のプロジェクトタイプに依存せず、広範なソフトウェアシステムに適用可能であることを検証できる。