%% 評価

%% 4章のアーキテクチャが事例に上げた5個のシステムを実現しよう時に
%% そのまま適用できるものは何ですか、適用できないものは?、


% %% 例3強制:毎日最初に試験をうけることを強制するシステム。
% 試験はアーキテクチャ内に含まれている
% 但しアーキテクチャでは試験は学習完了時しかできない
% 変更点は利用者が試験をうけられる機会を増やす
% 学習管理部を修正し、最初に試験をうけるように修正する。

% 上記の変更はやさしいのか難しいのか。
前章で提案したアーキテクチャが2章で上げたシステムの再構成に適用可能かどうかを評価する。
ただし最後の協調学習監視システムは
提案アーキテクチャの性質
(各ステークホルダーが自由な時間にアクセスを行い結果を得られる)
では考慮していない上、
対象システムで使用される学習場面のビデオ画像を表示する機能を想定していないので、
評価の対象外にする。


\begin{enumerate}\itemsep=-1ex

\item 情報セキュリティ教育用教材(ELSEC)

\item 拡張可能な学習支援システムアーキテクチャ(ELECOA)

\item 情報セキュリティ教育の強制実行(SEC-EL)

\item 学習習熟度を用いた自主学習支援システム(AR-PBL)

\item 脆弱性対策教育システム(VulCES)

\end{enumerate}

\begin{description}

\item{ELSEC}

提案アーキテクチャは課程に含む教材を確定し、
全部学習しないと修了試験を受けられない点においては
ELSECと同様である。
これは達成目標に対応した教材の選択と学習者の達成度を保証する手段となる。
しかし提案アーキテクチャの中に達成目標という概念が含まれているが、
課程と達成目標の関連付けが正しいかどうかを確認する基準がまだ明確になっていないので、
学習者が求めるものを提供する機能の実現が困難である。

\item{ELECOA}

独習用教材をグループ学習でも再利用できるように学習制御機能を標準化
することにより、相互運用性を実現しているのが特徴である。
提案アーキテクチャではナビゲーション機能に相当する。
教材を選択した後では任意の学習項目の閲覧などが可能であるが、
複数の教材を予め決められた順序で閲覧したり、
グループに所属するすべての学習者が同時に学習を開始し、
それぞれの進捗状況を共有する機能は持っていない。
提案アーキテクチャでこれを実現するためには
ナビゲーション機能を拡張する方法が考えられる。
例えばランナーに複数の課程に所属する
教材の学習順序を記述した情報を与える、
ある教材の学習を完了する時に、
次に学習することをすすめる教材リストをあげる。
と同時に、複数の教材の学習順序が強制的に確定すれば、
ある教材を学習した後、
次の教材を強制的にする必要があるけど、
現行のシステムで強制ができないので、
現在のシステムの上でどのよう拡張すれば良いのか、
もっと慎重に検討しなければならない。
%% 進捗状況の共有する(制限あり)






\item{SEC-EL}

毎日最初に試験をうけることを強制するシステム。
試験はアーキテクチャ内に含まれている
但しアーキテクチャでは試験は学習完了時しかできない
変更点は利用者が試験をうけられる機会を増やす
学習管理部を修正し、最初に試験をうけるように修正する。	
と同時に、提案アーキテクチャは教材の最後まで学習するのを前提として実行しているので、時間がかかる。
それゆえ、毎日強制実行なら、普段の生活あるいは仕事に影響があるので、実現が不可能である。
しかし、学習項目ごとに実行する理解度確認テストに変えれば、実現可能と考える。
または強制実行の件については、業務システムと連携が必要であるが、
ユーザの行動を制限するためには業務システムのメニューを制限するなど内部を変更する必要性が高い、
しかし業務システムはほとんどの場合でクローズド、すなわち外部からの修正を想定していないので、
技術的に実現が困難である。

\item{AR-PBL}

学生が自身の学習に適した資料を適切に選択することができる。
具体な手段は専門用語の出現を自動計測し、教材の難易度を推定する。
提案アーキテクチャでは
教材の難易度を計測する方法は考慮していないが、
特定の方法で計測可能とするシステムを用意し、
その出力を利用できるように教材の難易度を表す
情報の構造を変更することで実現が可能である。

%% 運用方法の変更(新しい制約)
具体的には教育担当者が教材を登録する時に初級と設定したが、
計測システムの判定により初級と合わない場合では、
難易度が合っていないことを教育担当者に通知する。
ほかにも学習者が新しい教材を選択する時に同様に難易度の不一致を通知する。

\item{VulCES}

様々なセキュリティ攻撃に対して適切な防衛スキルを習得するために
実験用ツールと組み合わせたシステムである。
実システムと完全に分離された実験用システムの上で
様々な攻撃とそれに対して防衛を実体験することで
攻撃の検出方法や防御の有効性を理解する。

提案アーキテクチャでこれと同様な教育を行うには
理解度確認テストする前に実験環境に入るための仕組みを追加する必要がある。
しかし繋がる前に実験環境内に実験装置を構築するために必要な情報を決めて渡す必要がある。
ほかに、実験に成功した場合は、
実験装置は提案アーキテクチャ側に結果に伝える必要があるが
その方法に理解度確認テストの成績として与えることが考えられる。
ただし、現在の試験は選択肢が決められており、また配点も決められている。
例えば防御実習の成績をこの試験の枠組みに適合させることが可能か、
もし可能なら選択肢と配点をどのように決定するかなどの問題を解決する必要がある。
検討の結果現状では適合が困難だと判断される場合は、新しい試験の形態を構築し
それをアーキテクチャで実現する(例:試験のサブクラス)方法を考える必要がある。

\end{description}

\clearpage


以上5つのシステムを提案したアーキテクチャを用いて再構成する際に
そのままでよい部分と追加変更が必要な部分をまとめたものを
表\ref{tbl:アーキテクチャを利用した従来システムの再構成}に示す。

\medskip
\medskip

\begin{table}[H]
  \caption{アーキテクチャを利用した従来システムの再構成}
  \label{tbl:アーキテクチャを利用した従来システムの再構成}

\medskip

\begin{center}
\begin{tabular}{|l|c|c|c|c|c|} \hline
&ELSEC&ELECOA&SEC-EL&AR-PBL&VulCES\\ \hline
学習管理部&○&×&×&△&△\\ \hline
情報保持部&△&○&△&△&△\\ \hline
情報分析部&△&○&○&○&△\\ \hline
\end{tabular}
\medskip
\medskip

\begin{tabular}{ll} 
○&提案したアーキテクチャでそのまま、あるいはデータを変更\\
 &するのみで実現可能\\
△&ユースケースや制御フローはそのままでよいが、データ構造\\
 &の変更、あるいは新しいデータとそれらは扱う操作の追加が必要\\
×&制御構造の大幅な変更、ユースケースの変更、情報の変更\\
 &などの大規模の変更が必要\\
\end{tabular}
\end{center}
\medskip

\end{table}

%% ここにそれぞれに必要な変更点を書く。
\begin{itemize}
\item ELSEC情報保持部

情報保持部の
教材(\verb|Book|)には
難易度に対応した値(\verb|rank|)が設定されているが、
学習者の達成目標と教材中の各学習項目の詳細度の対応を表す情報を保持していない。
これを実現するには\verb|rank|に相当する情報を
\verb|Book|のほかに学習項目(\verb|Section|)にも
追加すればよい。
必要ならば、対応関係を保持するクラスを追加し、
目的に応じた教材を選択するメソッドを持たせるが可能である。

\item ELSEC情報分析部

達成目標と学習項目の内容との関連は学習進捗記録などの構造及び
改善に必要な情報提供を行うユースケースには影響しない。
ただし、適切な教材を選択するためのユースケースが追加される。

% 学習進捗記録(\verb|ScoreBook|)と
% 教材改善進捗記録(\verb|ScoreHistory|)では、
% それぞれ、正答率(\verb|ratios|)と
% 正答率及び合格率の上昇率(\verb|Improvment|)を含む。
% そのゆえ、情報分析部正答率をめぐって分析するが、
% 達成目標との遠回しかどうかの判断、
% または正答率への影響も考慮すべきと考える。

\item ELECOA学習管理部

学習管理部では、ランナー(\verb|Runner|)が
スクリーン(\verb|Screen|)の表示順を制御している。
%% \verb|Screen|の切り替えはまたメニュー(\verb|Menu|)で選択肢があり。
ELECOAでは複数の教材を決められた順序で提示する機能が必要であり、
これはナビゲーション機能の拡張で実現する。
例えば教材の順序を記述したデータ(スクリプト)の読み込みと、
任意のページから閲覧を開始するための仕組みを\verb|Runner|に追加する。

% 課程で教材の学習順番を確定することだけでなく、
% ある教材の学習完了後、
% 他学習者の学習経験から得る、
% 次におすすめ教材へのナビゲーションなどを提供することが有利と考える。
% そのゆえ、\verb|Runner|と\verb|Menu|の変更が必要とする。



\item SEC-EL学習管理部

SEC-ELは起動時に解説と確認テストを強制実行することが求められるため、
ELECOAと同様の\verb|Runner|の拡張が必要である。

ただし現在の\verb|Runner|は特定の順序で\verb|Screen|を表示する機能と
ナビゲーションを強制する(他の操作を選択不可能にする)仕組みを持っていない。
また\verb|Runner|を使用した1つの学習項目が完了するまで
業務システムの操作を制限する仕組みも必要であるが、
これは業務システムがオープンであることが前提条件である。

\item SEC-EL情報保持部

情報保持部には毎日の業務開始前に行われる
理解度確認テストの結果を保持する領域を拡張する。
分析方法も異なる場合は情報分析部にも影響する可能性がある。

\item AR-PBL学習管理部

現在の学習管理部は教材を選択する際に
学習者の理解度に一致するかのみを判断していて、
詳細の難易度を表す情報とそれを計測する仕掛けを持っていない。
AR-PBLで用いる外部計測システムから詳細情報を取得し、
それを教材難易度ランクに変更する基準を必要とする。
また計測結果を\verb|Book|,\verb|Chapter|,\verb|Section|
それぞれ対応する箇所に格納するとともに、
それらの値と教育担当者が登録時に設定した\verb|rank|の値と不整合が発生した場合に、
教育担当者に再登録を促す仕組みも必要とする。

\item AR-PBL情報保持部

外部システムの計測結果に基づき、
学習者の理解度をより詳細に表現し、
教材選択に活用する仕組みが必要である。
例えばこの学習者は\verb|Course1|は用語をほとんど知らないので
初級から始めるべきであるが、
\verb|Course2|は基本的な概念は理解しているので
中級からはじめてもよい、という内容の情報を扱えるようにする。


\item VulCES学習管理部

学習した時点でVulCESのような実験装置を利用するを前提としたら、
学習管理部は、学習者を実験環境に移すを要求するので、
学習過程を制御する\verb|Runner|を変わる必要がある。
そのゆえ、\verb|Runner|に学習項目完了する時点で、
実験環境へ移す条件と要求を渡して、判断する、
判断結果の上で、自動的に実験あるいは試験を実行ことを要求する。

\item VulCES情報保持部

確認テストの代わりに実験を行う場合に、
学習項目によるどのような実験環境が必要か、
これらの情報は、教材とどもに情報保持部に保持する必要がある。
例えば理解度確認試験(\verb|ConfirmationExam|)の
内容と構造を変わって、実験環境のデータも加入する。


\item VulCES情報分析部

情報分析部では試験後正答率を分析する仕組みがあり、
そのゆえ、テストの代わりに実験を行う場合に、
どのよう外部実験環境から戻る情報を分析するか、
特定の分析方法が必要とする。
例えば実験環境からの戻り値は\verb|ratios|に変える。
状況判断にかかる時間による点数付けなどの方式で実現する。

\end{itemize}