10分割交差検証の結果を表\ref{tab:crossvalidation}に示す。各プロジェクトにおいて、ベースライン、ステップ2、ステップ3の順にF1スコアとROC-AUCの平均値と標準偏差を報告する。

\begin{table}[ht]
\centering
\caption{10分割交差検証の結果(平均値 ± 標準偏差)}
\label{tab:crossvalidation}
\begin{tabular}{|l|l|r|r|}
\hline
プロジェクト & モデル & F1スコア & ROC-AUC \\
\hline
Elasticsearch & ベースライン & 0.6016 ± 0.0152 & 0.7988 ± 0.0179 \\
 & ステップ2 & 0.7165 ± 0.0242 & 0.8918 ± 0.0136 \\
 & ステップ3 & 0.7910 ± 0.0242 & 0.9313 ± 0.0161 \\
\hline
Hazelcast & ベースライン & 0.6934 ± 0.0234 & 0.8919 ± 0.0160 \\
 & ステップ2 & 0.7489 ± 0.0250 & 0.9215 ± 0.0145 \\
 & ステップ3 & 0.7833 ± 0.0185 & 0.9314 ± 0.0111 \\
\hline
Neo4j & ベースライン & 0.4472 ± 0.0244 & 0.6775 ± 0.0186 \\
 & ステップ2 & 0.5765 ± 0.0232 & 0.8101 ± 0.0136 \\
 & ステップ3 & 0.7393 ± 0.0329 & 0.9360 ± 0.0111 \\
\hline
Netty & ベースライン & 0.4257 ± 0.0280 & 0.6864 ± 0.0312 \\
 & ステップ2 & 0.6212 ± 0.0331 & 0.8707 ± 0.0290 \\
 & ステップ3 & 0.7076 ± 0.0275 & 0.9252 ± 0.0160 \\
\hline
OrientDB & ベースライン & 0.4734 ± 0.0392 & 0.7246 ± 0.0364 \\
 & ステップ2 & 0.5086 ± 0.0402 & 0.7535 ± 0.0326 \\
 & ステップ3 & 0.6636 ± 0.0251 & 0.8985 ± 0.0192 \\
\hline
\end{tabular}
\end{table}

Elasticsearchでは、ベースラインのF1スコアが0.6016±0.0152、ステップ2で0.7165±0.0242、ステップ3で0.7910±0.0242となった。ROC-AUCはベースラインの0.7988±0.0179からステップ2で0.8918±0.0136、ステップ3で0.9313±0.0161へと向上した。標準偏差は全ての段階で0.02前後と比較的小さく、性能が安定していることが確認された。

Hazelcastでは、F1スコアがベースラインの0.6934±0.0234からステップ2で0.7489±0.0250、ステップ3で0.7833±0.0185へと向上した。ROC-AUCはベースラインの0.8919±0.0160からステップ2で0.9215±0.0145、ステップ3で0.9314±0.0111へと改善された。特にステップ3では標準偏差が0.0111と最も小さく、性能の安定性が向上したことが示された。

Neo4jでは、F1スコアがベースラインの0.4472±0.0244からステップ2で0.5765±0.0232、ステップ3で0.7393±0.0329へと大幅に向上した。ROC-AUCはベースラインの0.6775±0.0186からステップ2で0.8101±0.0136、ステップ3で0.9360±0.0111へと改善された。ステップ3では標準偏差が0.0329とやや大きいが、これはプロジェクトの特性やデータの偏りに起因する可能性がある。

Nettyでは、F1スコアがベースラインの0.4257±0.0280からステップ2で0.6212±0.0331、ステップ3で0.7076±0.0275へと向上した。ROC-AUCはベースラインの0.6864±0.0312からステップ2で0.8707±0.0290、ステップ3で0.9252±0.0160へと改善された。ステップ3では標準偏差が0.0160と小さくなり、性能の安定性が向上した。

OrientDBでは、F1スコアがベースラインの0.4734±0.0392からステップ2で0.5086±0.0402、ステップ3で0.6636±0.0251へと向上した。ROC-AUCはベースラインの0.7246±0.0364からステップ2で0.7535±0.0326、ステップ3で0.8985±0.0192へと改善された。ステップ3では標準偏差が0.0251と小さくなり、性能の安定性が向上した。

全てのプロジェクトにおいて、交差検証の標準偏差は0.04以下と比較的小さく、提案手法が異なるデータ分割に対しても安定した性能を発揮することが確認された。特に、ステップ3ではROC-AUCの標準偏差が全プロジェクトで0.02以下となり、モデルの識別能力が安定していることが示された。