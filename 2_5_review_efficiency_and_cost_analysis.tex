ソフトウェア開発において、レビューは品質保証活動の中心となる重要な活動であるが、レビューに費やせる労力が少ないため、全てのコードを詳細にレビューすることは困難である。一方で、欠陥の事後修正は開発者にとっては大きな負担である。開発後期の修正コストは初期段階の数倍に達することもあり、本番環境で欠陥が見つかるとユーザー体験を損なう可能性がある。これは、欠陥が他のモジュールに波及し、テストの再実行やデプロイの繰り返しが必要となるためである。

\begin{figure}[tb]
\centering
\begin{subfigure}[b]{0.9\textwidth}
\centering
\includegraphics[width=\textwidth]{figures/defect_detection.pdf}
\caption{欠陥予測を行わない場合の欠陥の広がり}
\label{fig:defect_detection}
\end{subfigure}

\vspace{1em}

\begin{subfigure}[b]{0.9\textwidth}
\centering 
\includegraphics[width=\textwidth]{figures/defect_prediction.pdf}
\caption{欠陥予測を用いた場合の早期対応}
\label{fig:defect_prediction}
\end{subfigure}
\caption{欠陥予測の有無による欠陥の広がりの比較}
\label{fig:defect_comparison}
\end{figure}

図\ref{fig:defect_comparison}に示すように、欠陥予測手法を用いて早い段階で欠陥を見つけることで、欠陥の波及を防ぎ、修正コストを削減可能である。
MendeとKoschkeは、欠陥予測を労力を考慮した順位付け問題として捉え直し、Effort-Aware Defect Prediction(レビュー労力に基づいた欠陥予測)の概念を提案した\cite{mende2010}。従来手法は各構成要素のレビュー労力が同一であると仮定していたが、実際にはそれぞれの規模や複雑度によってレビュー労力が異なる。彼らは循環的複雑度を用いてレビュー労力を計算し、少ない労力でレビューでき、かつ欠陥混入率が高い構成要素を優先的にレビューする戦略を提案した。Kameiらは、このMendeとKoschkeらの手法をJust-In-Time欠陥予測に適用した\cite{kamei2013}。彼らは6つのオープンソースプロジェクトと5つの商用プロジェクトを用いた実証研究により、平均正解率68\%、平均再現率64\%で欠陥混入コミットを予測可能であることを示し、20\%のレビュー労力で欠陥混入コミットの35\%を特定可能であることを示した。