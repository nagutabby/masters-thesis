ソフトウェア開発において、コードレビューは品質保証の中核を担うが、時間的・人的リソースの制約により、すべてのコードを十分にレビューすることは困難である。Ostrandら\cite{ostrand2005}は、欠陥密度が最も高いファイルの20\%に平均83\%の欠陥が集中することを示した。この知見は、限られたリソースを効果的に配分するには、欠陥を含む可能性の高いコード領域を優先的にレビューする戦略が有効であることを示唆している。

この課題に対し、MendeとKoschkeは、欠陥予測を単純な分類問題ではなく労力を考慮した順位付け問題として捉え直し、Effort-Aware Defect Predictionの概念を提案した\cite{mende2010}。従来の欠陥予測モデルは各モジュールが欠陥を含むか否かの予測に焦点を当て、レビューコストが同一であると暗黙的に仮定していたが、実際には小さなモジュールと大きなモジュールではレビューに必要な労力が大きく異なる。彼らは、初めに循環的複雑度を用いてレビューに必要な労力を計算し、次にそれを用いて、少ない労力でレビューでき、かつ欠陥混入率も高いモジュールを優先的にレビューする戦略を採用した。この戦略の登場により、レビューで行われる単体テストなどの労力を反映した欠陥予測モデルを構築できるようになった。

Kameiらは、この労力考慮型アプローチをJust-In-Time欠陥予測に適用し、6つのオープンソースプロジェクトと5つの商用プロジェクトを対象とした大規模実証研究を実施した\cite{kamei2013}。彼らが提案したEffort-Aware Linear Regression(EALR)モデルは、各変更における欠陥混入の有無を予測した上でコード行数に対する欠陥混入の有無として変更のレビュー優先度を順位付けし、20\%の労力で全欠陥の約35\%を発見できることを実証した。

しかし、既存研究には重要な課題が残されている。これまでの手法は、レビュー労力がコード行数や循環的複雑度に比例するという仮定に基づいていた。実際のソフトウェア開発では、この仮定が必ずしも成立しない。なぜなら、変更が複数のファイルに分散している場合はより多くのコードベースの理解が必要になるためである。したがって、このような変更の広がりも考慮してレビュー労力を計算する必要がある。