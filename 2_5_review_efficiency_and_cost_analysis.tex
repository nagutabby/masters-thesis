ソフトウェア開発において、コードレビューは品質保証の中核を担うが、リソースの制約により全てのコードを十分にレビューすることは困難である。Ostrandらは、欠陥密度が最も高いファイルの20\%に平均83\%の欠陥が集中することを示し、限られたリソースを効果的に配分する戦略の重要性を示唆した\cite{ostrand2005}。

MendeとKoschkeは、欠陥予測を労力を考慮した順位付け問題として捉え直し、Effort-Aware Defect Predictionの概念を提案した\cite{mende2010}。従来手法は各モジュールのレビューコストが同一であると暗黙的に仮定していたが、実際には規模や複雑度によって必要な労力が異なる。彼らは循環的複雑度を用いてレビュー労力を計算し、少ない労力でレビューでき、かつ欠陥混入率も高いモジュールを優先的にレビューする戦略を提案した。

Kameiらは、この労力考慮型アプローチをJust-In-Time欠陥予測に適用した\cite{kamei2013}。Effort-Aware Linear Regression(EALR)モデルは、欠陥混入の有無を予測した上でコード行数に基づいて変更の優先度を順位付けし、20\%の労力で全欠陥の約35\%を発見できることを実証した。

しかし、既存手法はレビュー労力がコード行数や循環的複雑度に比例するという仮定に基づいており、変更が複数のファイルに分散する場合の影響を十分に考慮していない。このような変更の広がりも含めてレビュー労力を計算する必要がある。