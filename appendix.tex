\chapter{ソースコード}

本章では、第3章および第4章で説明した処理のソースコードを示す。

\section{メソッド単位の変更メトリクスの抽出}
\label{appendix:method_metrics}

メソッド単位の変更メトリクスを抽出する処理を以下に示す。この処理では、連続するコミット間でのメソッドの変更を検出し、変更タイプ(追加、削除、修正)および循環的複雑度・コード行数・トークン数の変化量を計算する。

\lstinputlisting[language=Python]{code/method_metrics.py}

\clearpage
\section{コミット単位の変更メトリクスの抽出}
\label{appendix:commit_metrics}

コミット単位の変更メトリクスを抽出する処理を以下に示す。この処理では、各コミットにおける全体的な変更の特徴(変更ファイル数、コードの追加・削除行数、変更の広がり)を計算する。

\subsection{変更ファイル数とコードの追加・削除行数の計算}

\lstinputlisting[language=Python]{code/commit_stats.py}

\subsection{変更の広がりの計算}

\lstinputlisting[language=Python]{code/entropy.py}

\subsection{コミット単位の変更メトリクスの計算}

\lstinputlisting[language=Python]{code/vcs_metrics.py}

\clearpage
\section{レビュー労力の計算}
\label{appendix:review_effort}

レビュー労力を計算する処理を以下に示す。この処理では、コードの変更行数、変更ファイル数、および変更の広がりに基づいてレビューに必要な労力を推定する。

\subsection{レビュー労力の推定}

\lstinputlisting[language=Python]{code/effort_calculation.py}

\subsection{貪欲法によるレビュー対象コミットの選択}

少ないレビュー労力で高い欠陥発見率を達成するために、貪欲法を用いてレビュー対象コミットを選択する。

\lstinputlisting[language=Python]{code/knapsack_greedy.py}

\subsection{欠陥発見率の計算}

レビュー労力に対する欠陥発見率を計算する関数を以下に示す。

\lstinputlisting[language=Python]{code/cost_benefit.py}