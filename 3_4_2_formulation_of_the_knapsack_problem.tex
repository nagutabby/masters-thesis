本研究では、レビュー対象コミットの選択をナップサック問題として定式化する。ナップサック問題は、容量制約のある袋(ナップサック)に、価値と重さを持つ複数のアイテムを入れるとき、総重量が容量を超えないように、総価値を最大化するアイテムの組み合わせを求める最適化問題である。

レビュー対象の選択問題をナップサック問題に対応付けると、以下のようになる。

\begin{itemize}
    \item アイテム: レビュー待ちの各コミット $i$($i = 1, 2, ..., N$)
    \item アイテム数 $N$: レビュー待ちの全てのコミット数
    \item アイテムの重さ $W_i$: コミット $i$ のレビューに必要な労力
    \item アイテムの価値 $V_i$: コミット $i$ のレビューによるバグ発見期待値(モデルが予測したバグ混入確率 $\hat{y}_i$)
    \item ナップサックの容量 $C_{total}$: レビューに使える総労力
    \item 目的: レビュー労力の合計が $C_{total}$ を超えないように、レビューするコミットの組み合わせを選び、バグ発見期待値の合計 $V_{total}$ を最大化
\end{itemize}

数式で表現すると、以下の最適化問題となる。

\begin{align}
\text{maximize} \quad & \sum_{i=1}^{N} V_i x_i \\
\text{subject to} \quad & \sum_{i=1}^{N} W_i x_i \leq C_{total} \\
& x_i \in \{0, 1\}
\end{align}

ここで、$x_i$ は二値変数であり、コミット $i$ をレビューする場合は $x_i = 1$、レビューしない場合は $x_i = 0$ となる。