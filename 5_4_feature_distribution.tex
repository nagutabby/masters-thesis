重要度の高い特徴量の分布特性を分析した結果、コミット単位とメソッド単位のメトリクスで対照的な分布が観察された。

コミット単位のメトリクスは、右裾が長い分布を示した。大半のコミットは小規模な変更であるが、一部に大規模な変更が含まれる。この分布の偏りは、日常的な変更と大規模な機能追加やリファクタリングという異なる性質の変更が混在していることを反映している。

メソッド単位のメトリクスは異なる特性を示した。\texttt{tokens\_change}(トークン数の変化量)や\texttt{lines\_change}(行数の変化量)の中央値は0であり、欠陥混入や欠陥修正にかかわるメソッドが変更される場合は、小さな修正に留まることが確認された。

この分布特性は、提案手法の予測性能に影響を与える。コミット単位やメソッド単位のメトリクスの偏った分布により、ランダムフォレストは、主に小規模な変更を含む訓練データから欠陥の傾向を学習する。その結果、実際の開発で頻繁に発生する小規模な変更による欠陥混入を効果的に捉えることが可能となる。