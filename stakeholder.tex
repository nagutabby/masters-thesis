
前節での学習対象の定義に従い
利用者を役割に従って分類し、それぞれが必要とする機能要求を定義する。
システム構成とユースケースを決定し、
動作とそれに必要なデータを決定する。

ELSの利用者をその役割に応じて分類したもの(ステークホルダー)を以下に示す。

\begin{itemize}

\item{学習者}

学習者は学習習熟度を向上させることで目標を達成する。
学習者とは。教材を勉強し、試験を受ける人である。
課程で決められた教材を全て習得した後で修了試験に合格することが目標となる。

\item{教育担当者}

教育担当者とは教材を教えるまたは修正する責任者である。
教育担当者は学習者の学習進捗状況を把握して教材を改善する。
学習記録を参照し理解度が低い箇所を修正することが目標である。

\item{課程設計者}

課程設計者とはコースを設計する責任者である。
教材改善進捗状況を把握し、必要ならば課程に新しい教材を追加する。
%% これを必要の理由を書く

\end{itemize}

\begin{figure}[H]
  \vspace{0.5cm}
  \hspace*{0.5cm}
  % \includegraphics[width=12cm,bb=0 50 800 600]{../operation.pdf}
  \caption{ステークホルダーの役割}
  \label{fig:ステークホルダーの役割}
\end{figure}

図\ref{fig:ステークホルダーの役割}はステークホルダーの役割を示したものである。
課程設計者は課程を設計し、教育担当者が設計した課程に対応する教材を作成する。
学習者が作成した教材を学習し、試験をうける。
システムは受験者全員に対する試験の問題ごとの正答率及び合格率を計算する。

この中で、教育担当者が合格率を向上させるために教材を改善する役割を担当する。
もし学習者の合格率は基準値に到達していない場合は、教育担当者は教材を改善する。
学習者が改善された教材を再学習し受験することで、理解度が向上したかどうかを確認する。

改善された教材に対する学習及び受験を一定数に繰り返しても合格率が基準値に達しない場合は、
課程設計者は課程に新しい教材を追加することを教育担当者に依頼する。



\begin{figure}[H]
  \vspace{0.5cm}
  \hspace*{0.5cm}
  % \includegraphics[width=13cm,bb=0 100 800 400]{../overview.pdf}
  \caption{ステークホルダーの目標}
  \label{fig:ステークホルダーの目標}
\end{figure}

各ステークホルダーの目標を図\ref{fig:ステークホルダーの目標}に示す。
学習者は試験に合格したことによりスキル習得し、目標を達成する。
教育担当者は学生の正答率の向上を目標として教材を改善する。
課程設計者は学習が停滞している理由を分析し、
必要ならば課程を改善することにより目標を達成する。


\clearpage