従来のソフトウェア品質保証のアプローチは、主に事後対応型である。すなわち、バグが発生してから検出し、修正するという流れが一般的であった。しかし、バグが本番環境で発見された場合、その影響はユーザー体験の低下やシステムの停止など、組織にとって大きな損失につながる可能性がある。さらに、開発が進んだ後の段階でバグを修正するコストは、開発初期段階での修正に比べて桁違いに高くなることが知られている。そのため、バグが混入する前に、あるいは混入直後に検出し対処する事前予防型のアプローチへの転換が求められている。

事前予防型のアプローチを実現するためには、コード変更時点でバグ混入リスクを評価し、リスクの高い偏光を早期に特定する必要がある。これにより、開発者は変更を本番環境に反映する前に、より慎重なレビューやテストを実施できる。また、問題が小規模なうちに対処することで、後の段階での大規模な修正を回避し、開発コストを抑制できる可能性がある。

しかし、現実の開発現場では、全ての変更に対して十分な時間をかけてレビューやテストを実施することは困難である。特に、アジャイル開発やDevOpsのような短いサイクルでの開発が主流となる中、レビューに割ける時間やリソースは限られている。そのため、限られたレビューリソースをどのように配分するかという問題が重要になる。バグ混入リスクの高い変更を優先的にレビューすることで、同じリソースでより多くの欠陥を発見できれば、開発効率と品質の両立が可能になる。

このような効率的なレビュー優先度付けを実現するためには、コード変更の特性に基づいてバグ混入リスクを定量的に評価する手法が必要である。ここで重要なのは、単に現時点でのコードの静的な特性を見るだけでなく、コードがどのように変化してきたかという時系列的な情報に着目することである。例えば、短期間に頻繁に変更されているコード、あるいは大規模な変更が加えられたコードは、バグ混入リスクが高い可能性がある。また、過去にバグが多く発見された箇所への変更も、同様にリスクが高いと考えられる。

コードの変化過程に着目した時系列情報の活用は、静的解析では捉えきれない動的なリスク要因を明らかにする可能性がある。変更の頻度、変更の規模、変更されたファイル間の関連性など、時系列的な観点から抽出できる特徴量は多数存在する。これらの情報を機械学習モデルに組み込むことで、より精度の高い欠陥予測が可能になると期待される。

本研究は、こうした背景から、コードの時系列変化に着目したソフトウェア欠陥予測手法の確立を目指す。時系列情報を活用することで、バグ混入リスクの事前予測精度を向上させ、限られたレビューリソースの効率的な配分を支援する実用的な手法を提供することが本研究の動機である。