欠陥の事後修正は高コストである。開発後期の修正コストは初期段階の数倍に達し、本番環境での発見はユーザー体験を損なう。これは、欠陥が他のモジュールに波及し、テストケースの再実行やデプロイプロセスの繰り返しが必要となるためである。

図\ref{fig:defect_comparison}に示すように、欠陥予測による早期対応は、欠陥の波及を防ぎ、修正コストを削減できる。

\begin{figure}[tb]
  \centering
  \begin{subfigure}[b]{0.9\textwidth}
    \centering
    \includegraphics[width=\textwidth]{figures/defect_detection.pdf}
    \caption{欠陥予測を行わない場合の欠陥の広がり}
    \label{fig:defect_detection}
  \end{subfigure}

  \vspace{1em}

  \begin{subfigure}[b]{0.9\textwidth}
    \centering
    \includegraphics[width=\textwidth]{figures/defect_prediction.pdf}
    \caption{欠陥予測を用いた場合の早期対応}
    \label{fig:defect_prediction}
  \end{subfigure}
  \caption{欠陥予測の有無による欠陥の広がりの比較}
  \label{fig:defect_comparison}
\end{figure}

しかし、すべての変更を詳細にレビューすることは現実的ではない。限られたレビューリソースを効率的に配分するには、リスクの高い変更を事前に特定する必要がある。既存研究では、変更の頻度や規模が欠陥混入と相関することが示されている\cite{graves2000}。頻繁に変更される箇所では開発者の理解が不完全になりやすく、変更が集中する時期には注意力が分散するためである。本研究は、コードの時系列変化を活用した欠陥予測により、リスクの高い変更を特定し、レビューリソースの適切な配分を実現する。