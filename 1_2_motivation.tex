バグの事後修正は高コストである。本番環境での発見はユーザー体験を損ない、開発後期の修正コストは初期段階の数倍に達する。事前予防型アプローチへの転換が必要である。

図\ref{fig:defect_comparison}に示すように、欠陥予測により早期対応が可能となり、欠陥の波及を防げる。

\begin{figure}[tb]
  \centering
  \begin{subfigure}[b]{0.9\textwidth}
    \centering
    \includegraphics[width=\textwidth]{figures/defect_detection.pdf}
    \caption{欠陥予測を行わない場合の欠陥の広がり}
    \label{fig:defect_detection}
  \end{subfigure}
  
  \vspace{1em}
  
  \begin{subfigure}[b]{0.9\textwidth}
    \centering
    \includegraphics[width=\textwidth]{figures/defect_prediction.pdf}
    \caption{欠陥予測を用いた場合の早期対応}
    \label{fig:defect_prediction}
  \end{subfigure}
  \caption{欠陥予測の有無による欠陥の広がりの比較}
  \label{fig:defect_comparison}
\end{figure}

しかし、限られたレビューリソースの効率的配分が課題である。コードの時系列変化(変更頻度、変更規模、変更箇所の関連性など)は、静的解析では捉えられない動的なリスク要因を含む。本研究は、時系列情報を活用した欠陥予測により、リスクの高い変更を事前特定し、レビューリソースの最適配分を実現する。