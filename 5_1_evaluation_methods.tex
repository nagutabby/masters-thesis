本研究では、提案する変更メトリクスの各要素が予測性能に与える影響を明確にするため、3段階の評価戦略を採用する。第1段階(既存手法)では構造的メトリクスのみを使用し、第2段階ではメソッド単位の変更メトリクスを追加、第3段階(提案手法)ではコミット単位の変更メトリクスをさらに追加する。この段階的なアプローチにより、各メトリクス群の予測性能への寄与度を定量的に測定可能である。

本節では、予測精度、レビュー労力の削減効果、モデルの判断基準の透明性という3つの観点から評価手順を述べる。

\paragraph{データ分割}
各プロジェクトのデータセットを訓練データ80\%、テストデータ20\%に分割する。訓練データはモデルの学習に使用し、テストデータは最終的な性能評価に使用する。テストデータは訓練過程で一切使用しないため、未知データに対する汎化性能を測定可能である。

\paragraph{10分割交差検証}
訓練データに対して10分割交差検証を実施し、モデルの安定性を評価する。具体的な手順は以下の通りである。

\begin{enumerate}
    \item データをサブセットに分割し、9割を訓練用、1割を検証用として使用する
    \item モデルを訓練し、検証データでF1スコア、適合率、再現率を計算する
    \item この過程をサブセットの数だけ繰り返し、各サブセットが1度だけ検証データとして使用されるようにする
    \item 全ての評価結果の平均値と標準偏差を算出する
\end{enumerate}

標準偏差が小さいほど、データの分割方法に依存しない安定したモデルであることを示す。

\paragraph{評価指標}
評価指標として、3.3節で定義した適合率、再現率、F1スコア、およびAUCを用いる。

\paragraph{最終評価}
交差検証により性能が確認されたモデルを、訓練データ全体で再学習する。その後、このモデルをテストデータに適用し、最終的な性能指標を算出する。

\paragraph{段階的評価の実施}
以下の3段階で評価を実施する。

\begin{enumerate}
    \item \textbf{既存手法(ステップ1)}: BugHunterデータセットの元のメトリクスのみを使用
    \item \textbf{ステップ2}: メソッド単位の変更メトリクス
    \item \textbf{提案手法(ステップ3)}: コミット単位の変更メトリクスをさらに追加
\end{enumerate}

各段階の評価結果を比較することで、メソッド単位とコミット単位の変更メトリクスがそれぞれどの程度性能向上に寄与するかを明らかにする。

\paragraph{統計的有意性の検証}
提案手法(ステップ3)と既存手法(ステップ1)の機械学習モデルの性能差が統計的に有意であることを、McNemar検定により検証する。McNemar検定は、同じテストデータに対する2つの分類器の予測結果を比較する統計的検定手法である。

検定手順は以下の通りである。まず、2つのモデルの予測結果から以下の分割表を作成する。

\begin{table}[ht]
\centering
\caption{McNemar検定のための分割表}
\begin{tabular}{cc|c|c|}
\cline{3-4}
& & \multicolumn{2}{c|}{提案手法} \\
\cline{3-4}
& & 正分類 & 誤分類 \\
\hline
\multicolumn{1}{|c|}{\multirow{2}{*}{既存手法}} & 正分類 & $n_{11}$ & $n_{12}$ \\
\cline{2-4}
\multicolumn{1}{|c|}{} & 誤分類 & $n_{21}$ & $n_{22}$ \\
\hline
\end{tabular}
\end{table}

ここで、$n_{12}$は既存手法で正分類であり、提案手法で誤分類であったサンプル数、$n_{21}$は提案手法で正分類であり、既存手法で誤分類であったサンプル数を表す。検定統計量は以下で計算される。

\[
\chi^2 = \frac{(n_{12} - n_{21})^2}{n_{12} + n_{21}}
\]

有意水準0.05で、$p < 0.05$の場合に性能差が統計的に有意であると判断する。

\paragraph{評価の概要}
提案手法によるレビュー労力削減効果を評価するため、レビュー労力に対する欠陥発見数の累積曲線を作成する。横軸に投入したレビュー労力、縦軸に発見した欠陥数をプロットする。提案手法の曲線が既存手法より左上に位置する場合、同じ労力でより多くの欠陥を発見可能であることを意味する。

\paragraph{貪欲法によるレビュー対象の選択手順}
各モデル(ステップ1、ステップ2、ステップ3)について、以下の手順でレビュー対象コミットを選択する。

\begin{enumerate}
    \item 各コミット$i$のレビュー労力を4.4節の方法で計算する
    \item 全コミットをレビュー労力の昇順にソートし、上位80\%のコミットの労力の和を総労力$C_{total}$として設定する
    \item 各コミット$i$について、モデルが予測した欠陥混入確率$\hat{y}_i$と補正済み労力$W_i$から密度を計算する
    \[
    D_i = \frac{\hat{y}_i}{E_{\text{adj}, i}}
    \]
    \item 密度$D_i$の降順にコミットをソートする
    \item 累積労力$W_{\text{累積}} = 0$とする
    \item ソートされた順にコミットを検討し、$W_{\text{total}} + W_i \leq C_{total}$であればコミット$i$をレビュー対象に追加し、$W_{\text{total}} \leftarrow W_{\text{total}} + W_i$とする。そうでなければスキップする
    \item 累積労力が容量を超えるまで、または全コミットを検討するまで繰り返す
    \item 各コミットをレビューするごとに、累積レビュー労力と累積発見欠陥数を記録する
\end{enumerate}

この手順により、少ないレビュー労力で欠陥発見期待値を最大化するレビュー対象を選択することが可能である。

\paragraph{累積曲線の作成手順}
\begin{enumerate}
    \item 各モデルについて、貪欲法によりレビュー対象コミットを選択する
    \item 各コミットをレビューする順に、累積レビュー労力と累積発見欠陥数を記録する
    \item 横軸を累積レビュー労力、縦軸を累積発見欠陥数として、各モデルの曲線を同一グラフ上に描画する
    \item 総労力の20\%, 40\%時点での欠陥発見数を比較する
\end{enumerate}

\paragraph{評価指標}
\textbf{欠陥発見率}: 特定労力時点で発見した欠陥数を全欠陥数で割った値。
\[
\text{欠陥発見率} = \frac{\text{発見した欠陥数}}{\text{全欠陥数}} \times 100\%
\]

\textbf{改善幅}: 提案手法(ステップ3)と既存手法(ステップ1)の欠陥発見率の差。
\[
\text{改善幅} = \text{提案手法の欠陥発見率} - \text{既存手法の欠陥発見率}
\]

\paragraph{特徴量重要度の算出手順}
ランダムフォレストが提供する特徴量重要度を用いて、各特徴量の予測への寄与度を定量化する。

\textbf{計算方法}: 特徴量重要度は、各特徴量が決定木の分岐においてどの程度情報利得をもたらしたかを示す指標である。ランダムフォレストでは、全ての決定木における各特徴量の情報利得の平均を計算することで特徴量重要度を算出する。

\textbf{分析手順}:
\begin{enumerate}
    \item 訓練済みのランダムフォレストモデルから、各特徴量の重要度を取得する
    \item 重要度の高い順に特徴量をソートする
    \item 上位の特徴量を抽出し、棒グラフで可視化する
    \item メソッド単位、コミット単位、元のメトリクスの各カテゴリーの特徴量がどの程度重要であるかを分析する
\end{enumerate}

\paragraph{Partial Dependence Plot(PDP)の生成手順}
各特徴量と欠陥混入確率の関係を可視化するため、PDPを生成する。PDPは、特定の特徴量の値を変化させたときに、モデルの予測値がどのように変化するかを示すグラフである。

\textbf{計算方法}: 特徴量$x_s$に対するPartial Dependence関数$f_{\text{PD}}(x_s)$は、以下の手順で計算される。

\begin{enumerate}
    \item 特徴量$x_s$の値を固定値に設定する
    \item 他の全ての特徴量$x_{-s}$は、各サンプルの実際の値を使用する
    \item 全サンプルについて予測値を計算し、その平均を取る
    \[
      f_{\text{PD}}(x_s) = \frac{1}{n}\sum_{i=1}^{n}\hat{f}(x_s, x_{-s}^{(i)})
    \]
    \item 特徴量$x_s$の値を変化させながらこの計算を繰り返す
\end{enumerate}

この結果、横軸を特徴量$x_s$の値、縦軸を予測確率とするグラフが得られる。このグラフにより、特定の特徴量が増加すると欠陥混入確率がどのように変化するかを視覚的に理解することが可能である。

\textbf{生成手順}:
\begin{enumerate}
    \item 特徴量重要度において上位の特徴量を選択する
    \item 各特徴量の値の範囲を等間隔に分割する
    \item 各点において上記の計算方法でPartial Dependence値を計算する
    \item 横軸を特徴量の値、縦軸を予測確率として、各特徴量のPDPをグラフ化する
\end{enumerate}

\paragraph{決定木の可視化手順}
ランダムフォレストを構成する決定木の一つを可視化し、どのような分類条件で欠陥の有無を判断しているかを確認する。

\begin{enumerate}
    \item ランダムフォレストから代表的な決定木を1つ選択する
    \item 決定木の各ノードにおける分岐条件(特徴量としきい値)を抽出する
    \item 各ノードのサンプル数、クラス分布を取得する
    \item グラフ描画ライブラリを用いて、決定木を可視化する
\end{enumerate}