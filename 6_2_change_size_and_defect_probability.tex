第5章のPartial Dependence Plot分析で観察された「変更規模が小さいほど欠陥混入確率が高い」という傾向について考察する。この現象は全5プロジェクトで一貫して確認されており、偶然ではなく何らかの原因が存在すると考えられる。

本現象を説明する可能性として、以下の仮説が考えられる。

\paragraph{欠陥密度とサンプル分布の影響}
この結果は、一見すると直感に反するように思われる。一般的に、ソフトウェア工学の研究では、変更規模が大きいほど欠陥密度(単位コード量あたりの欠陥数)が高くなることが知られている。しかし、本研究で構築した機械学習モデルが予測しているのは欠陥密度ではなく、各コミットが欠陥を含む確率である。

この2つの概念の違いを理解する鍵は、実際のソフトウェア開発におけるコミットサイズの分布にある。ソフトウェア開発では、変更の大部分が小規模であり、大規模な変更は相対的に少数である。機械学習モデルは訓練データにおける欠陥の絶対的な出現パターンから学習するため、サンプル数が多い小規模変更の領域において、より多くの欠陥例とその特徴パターンを学習することになる。したがって、個々のコミットの欠陥密度が低くても、サンプル数が圧倒的に多ければ、その領域で観察される欠陥の絶対数は多くなり、結果として予測モデルはこの領域で高い欠陥混入確率を出力する可能性がある。

言い換えれば、大規模変更は欠陥密度が高い可能性があるものの、その絶対数が少ないため、モデルの予測において支配的な影響を持たない。一方、小規模変更は密度こそ低いが、その数の多さゆえに多くの欠陥を含んでおり、モデルはこのパターンを学習していると考えられる。

\paragraph{変更の性質と目的の違い}
小規模な変更と大規模な変更では、その目的や性質が根本的に異なる可能性がある。
Hindleらは、大規模なコミットをカテゴリーごとに分類することでコミットの性質を分析し、変更の規模によって変更の種類に顕著な違いがあることを示した\cite{hindle2008}。

彼らの調査によると、適応的(Adaptive)な変更はいずれの規模においても最も頻繁に見られたが、小規模な変更と大規模な変更を比較すると興味深い逆転現象が確認された。小規模な変更では、完全な(Perfective)変更よりも修正的な(Corrective)変更が行われる傾向が強かったのに対し、大規模な変更では、完全な変更の方が多く含まれていた。

この結果について、Hindleらは直感的な解釈を与えている。すなわち、エラーの修正はしばしば「外科手術的」で局所的な小規模変更にとどまることが多い一方で、システムの改善や再構築を伴う完全な変更は、その影響範囲が広く、多くのファイルに波及するため大規模な変更になりやすいと考えられる。一見すると、修正的な変更である小規模な変更の方が欠陥混入確率がより低いように思われる。しかし、コードの結合度が高い場合、このような局所的な修正がシステム全体の不整合を引き起こす原因となる可能性がある。

\paragraph{開発者の注意力とレビュープロセスの違い}
変更の規模によって、開発者とレビュアーが払う注意の程度が異なる可能性がある。Bosuらは、レビューの有効性は変更のサイズとともに低下することを発見しており\cite{bosu2015}、大規模な変更では詳細な検査が困難になることを示唆している。同様に、Kononenkoらは、変更サイズがレビュー時間に大きく影響すると報告しており\cite{kononenko2016}、レビュアーがより多くの時間を費やす必要があることを示している。このレビューの困難さの違いは、欠陥の検出率に影響を与える可能性がある。小規模な変更はレビューが容易であるため、潜在的な欠陥が発見されやすく、後続のコミットで修正される可能性が高い。一方、大規模な変更では詳細な検査が困難なため、欠陥が見逃されやすく、修正されないまま残存する可能性がある。本研究で使用するBugHunterデータセットはSZZアルゴリズムに基づいて欠陥混入コミットを特定しているため、実際に修正された欠陥のみが「欠陥あり」として検出される。したがって、小規模な変更で観察される高い欠陥混入確率は、真の欠陥混入確率の違いではなく、レビューによる検出率の違いを反映している可能性がある。

\paragraph{開発者の経験と役割分担}
小規模な変更は経験の浅い開発者が担当することが多い可能性がある。Juら\cite{ju2021}は、オンボーディング戦略に関する調査において、新人開発者に対してはSimple-Complex戦略が広く採用されていることを報告している。この戦略では、マネージャーは最初の1週間にシンプルなタスク(バグ修正や設定変更)を割り当て、その後2〜3週間で小さなバグやシンプルな機能に取り組ませることで、段階的にコードベースへの理解を深めさせる。一方、経験豊富な開発者に対しては、Exploration-Based戦略が用いられ、定義が曖昧で不確実性の高いタスクが割り当てられる傾向にある。このように、開発者の経験レベルに応じてタスクの複雑さや性質が調整されることが示されている。

彼らの調査から、小規模な変更の欠陥混入確率の高さは、経験の浅い開発者がコードベース全体への理解が不十分なまま変更を行うことに起因すると考えられる。Simple-Complex戦略では、新人開発者は最初の数週間で小規模なタスクに取り組むが、この段階ではまだシステムの全体像や他のコンポーネントとの相互作用を十分に把握できていない。その結果、局所的には正しく見える変更であっても、他の部分への影響を見落とし、予期しない欠陥を混入させてしまう可能性が高まる。