本論文の構成は以下の通りである。

第2章では、ソフトウェア欠陥予測に関する関連研究を概観する。Just-In-Time品質保証の分野におけるKameiらの研究を取り上げ、14個の変更メトリクスとその問題点におついて詳述する。また、Ferencらのコミットベースの欠陥予測手法、Hanらのコードレビューテキスト分析、Romanoらの静的解析による品質改善研究を紹介する。これらの既存研究を整理することで、コミット間変化量の考慮不足や特性値間の関連性分析の欠如といった問題点を明らかにする。

第3章では、本研究で提案する時系列変化を考慮した欠陥予測手法について説明する。まず、コミットを点過程データとして位置付け、従来の等間隔時系列分析との違いを明確化する。次に、メソッド単位での変化量メトリクスとコミット単位での変更率メトリクスの設計方針を述べ、それぞれのメトリクスで変化量と変化率を使い分ける理論的根拠を示す。最後に、機械学習モデルの選定理由、モデルの評価指標、有意性検定について説明する。

第4章では、実験設定について詳述する。データセットにおける正解ラベルの定義方法、機械学習モデルの学習における追加の制約、プロジェクト選定基準を説明する。対象とする5つのプロジェクト(Elasticsearch、Hazelcast、Netty、OrientDB、Neo4j)の特性を示し、直前コミットとの差分に基づく特徴量生成方法と前処理手順を述べる。さらに、ベースライン、変化量メトリクスの追加後、変化率メトリクスの追加後の3段階での評価を行う実験手順と、モデル解釈性分析の方法について説明する。

第5章では、実験結果を報告する。5つのプロジェクトにおけるF1スコアなどの評価指標の比較表を示し、提案手法による予測性能の改善を定量的に示す。特徴量の寄与度の分析により、どのような特徴量が予測に貢献しているのかを明らかにする。また、特徴量の分布、Partial Dependence Plotによる各特徴量と陽性確率の関係、決定木による分類条件の可視化、Cost-Benefit Curveによるレビュー労力削減効果を示す。

第6章では、実験結果に基づく考察を行う。予測性能向上のメカニズムを分析し、各特徴量の意味と予測への寄与について解釈する。プロジェクト間での特徴量の影響の違いを考察し、その背景にある開発フェーズやコーディング文化の違いを議論する。また、クラス不均衡の影響やモデルの予測特性、レビュー労力モデルの単純化といった本研究の限界を明示し、今後の研究課題を提示する。

第7章では、本研究の成果を総括し、主要な貢献を再確認する。主要な時系列分析手法との違い、メトリクス設計、実プロジェクトでの性能検証という3つの貢献を振り返り、実用的な意義と今後の展望について述べる。