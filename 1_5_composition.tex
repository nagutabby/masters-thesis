本論文の構成は以下の通りである。

第2章では関連研究を概観し、既存手法の限界を明確化する。

第3章では提案手法を説明する。不規則な時系列データとしての分析、メトリクス設計、機械学習モデルの選定について述べる。

第4章では実験設定を示す。5つのOSSプロジェクトを対象とし、3段階の評価とモデル解釈性分析を行う。

第5章では実験結果を報告する。予測性能、特徴量寄与度、Cost-Benefit Curveによる効果検証を示す。

第6章では結果を考察し、限界と今後の課題を議論する。

第7章で本研究を総括する。