Partial Dependence Plot (PDP)を用いて、各特徴量が予測陽性確率に与える影響を分析した。PDPは、特定の特徴量の値を変化させたときに予測される陽性確率がどのように変化するかを可視化する手法であり、モデルの解釈性を高めるために有効である。ここでは、各特徴量の四分位数における陽性確率を計算し、特徴量と予測の関係性を定量的に評価した。表\ref{tab:partial_dependence}に、各プロジェクトにおけるFeature Dependence上位10個の特徴量の四分位数での陽性確率を示す(ハルステッド系メトリクスおよびMaintainability Index関連の特徴量は抜粋)。

\begin{table}[ht]
\centering
\caption{Feature Importance上位10個の特徴量のPartial Dependence(抜粋)}
\label{tab:partial_dependence}
\small
\begin{tabular}{|l|l|r|r|r|}
\hline
プロジェクト & 特徴量 & Q1陽性確率 & Q2陽性確率 & Q3陽性確率 \\
\hline
\textbf{Elasticsearch} & lines\_added & 0.406 & 0.399 & 0.382 \\
 & entropy & 0.402 & 0.396 & 0.397 \\
 & num\_files & 0.414 & 0.400 & 0.397 \\
 & lines\_deleted & 0.413 & 0.397 & 0.397 \\
 & tokens\_change & 0.440 & 0.440 & 0.440 \\
 & length\_change & 0.427 & 0.427 & 0.427 \\
 & operation\_type\_added & 0.413 & 0.413 & 0.413 \\
\hline
\textbf{Hazelcast} & lines\_added & 0.420 & 0.404 & 0.393 \\
 & num\_files & 0.416 & 0.406 & 0.393 \\
 & lines\_deleted & 0.416 & 0.406 & 0.396 \\
 & entropy & 0.400 & 0.400 & 0.398 \\
 & tokens\_change & 0.434 & 0.434 & 0.434 \\
\hline
\textbf{Neo4j} & lines\_added & 0.443 & 0.404 & 0.388 \\
 & lines\_deleted & 0.412 & 0.392 & 0.359 \\
 & entropy & 0.423 & 0.397 & 0.393 \\
 & num\_files & 0.416 & 0.408 & 0.402 \\
 & tokens\_change & 0.469 & 0.469 & 0.469 \\
 & length\_change & 0.447 & 0.447 & 0.447 \\
 & operation\_type\_added & 0.430 & 0.430 & 0.394 \\
\hline
\textbf{Netty} & tokens\_change & 0.374 & 0.374 & 0.374 \\
 & lines\_deleted & 0.305 & 0.291 & 0.288 \\
 & lines\_added & 0.311 & 0.295 & 0.291 \\
 & length\_change & 0.337 & 0.337 & 0.337 \\
 & num\_files & 0.311 & 0.311 & 0.300 \\
 & operation\_type\_added & 0.311 & 0.311 & 0.311 \\
 & ccn\_change & 0.309 & 0.309 & 0.309 \\
 & entropy & 0.300 & 0.294 & 0.299 \\
\hline
\textbf{OrientDB} & lines\_added & 0.423 & 0.437 & 0.410 \\
 & lines\_deleted & 0.418 & 0.407 & 0.409 \\
 & num\_files & 0.426 & 0.419 & 0.412 \\
 & entropy & 0.431 & 0.411 & 0.412 \\
\hline
\end{tabular}
\end{table}

コミット単位の変更メトリクスであるlines\_added、lines\_deleted、num\_filesは、全てのプロジェクトで一貫した傾向を示した。これらの特徴量は、値が増加するにつれて陽性確率が低下する単調減少の傾向が観察された。特に、lines\_addedとlines\_deletedでは、大規模な変更を伴うコミットほどバグ混入リスクが低いという結果が得られた。これは、大規模な追加や削除を伴う変更が、機能追加や大規模リファクタリングといった計画的な作業である可能性を示唆している。num\_filesにおいても同様の傾向が確認され、変更が多数のファイルにまたがるほど陽性確率が低下している。これは、広範囲に影響する変更が慎重に計画され実施されているためと考えられる。逆に、少数のファイルへの集中的な変更の方が、局所的な修正やバグフィックスである可能性が高く、バグ混入リスクが高いことを示している。entropyにおいても、変更の分散度が高いほど陽性確率が低下する傾向が見られ、広範囲な変更が慎重に実施される傾向を裏付けている。ただし、OrientDBのlines\_addedでは、中央値付近で陽性確率が最も高くなるという異なるパターンも観察された。

メソッド単位の変更メトリクスであるtokens\_change、length\_change、operation\_type\_addedは、値の変化に対する陽性確率の変動が小さく、ほぼ一定の傾向を示した。これは、多くのメソッドが変更されていないことを反映している。Neo4jのoperation\_type\_addedでは、新規メソッドの追加がバグ混入リスクを低下させる傾向が見られた。これは、新規メソッドが既存コードとの相互作用が少ないため、既存メソッドの変更よりもリスクが低い可能性を示唆している。

PDPの分析から、プロジェクト間で特徴量の影響が異なることが明らかになった。コミット単位の変更メトリクスは、全てのプロジェクトで一貫して陽性確率を低下させる傾向を示したが、その低下の程度はプロジェクトによって異なる。これは、プロジェクトの開発文化やコーディング規約の違いが、特徴量の重要性に影響を与えている可能性を示唆している。メソッド単位の変更メトリクスは、変化に対する陽性確率の応答が小さいという共通の特性を持つが、変更が発生した場合の影響の大きさはプロジェクトによって異なると考えられる。

図\ref{fig:pdp_elasticsearch}から図\ref{fig:pdp_orientdb}に、各プロジェクトにおける主要な特徴量のPartial Dependence Plotを示す。

\begin{figure}[ht]
\centering
\includegraphics[width=0.9\textwidth]{figures/elasticsearch/partial_dependence_plots.png}
\caption{主要特徴量のPartial Dependence Plot(Elasticsearch)}
\label{fig:pdp_elasticsearch}
\end{figure}

\begin{figure}[ht]
\centering
\includegraphics[width=0.9\textwidth]{figures/hazelcast/partial_dependence_plots.png}
\caption{主要特徴量のPartial Dependence Plot(Hazelcast)}
\label{fig:pdp_hazelcast}
\end{figure}

\begin{figure}[ht]
\centering
\includegraphics[width=0.9\textwidth]{figures/neo4j/partial_dependence_plots.png}
\caption{主要特徴量のPartial Dependence Plot(Neo4j)}
\label{fig:pdp_neo4j}
\end{figure}

\begin{figure}[ht]
\centering
\includegraphics[width=0.9\textwidth]{figures/netty/partial_dependence_plots.png}
\caption{主要特徴量のPartial Dependence Plot(Netty)}
\label{fig:pdp_netty}
\end{figure}

\begin{figure}[ht]
\centering
\includegraphics[width=0.9\textwidth]{figures/orientdb/partial_dependence_plots.png}
\caption{主要特徴量のPartial Dependence Plot(OrientDB)}
\label{fig:pdp_orientdb}
\end{figure}