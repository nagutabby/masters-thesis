PDPを用いて各特徴量とバグ混入確率の関係を分析した。その結果、\texttt{lines\_added}などの変更規模に関するメトリクスは、値が増加するにつれて陽性確率(バグ混入リスク)が低下する傾向が見られた。これは、大規模な変更は機能追加やリファクタリングとして計画的に実施されるため、相対的にリスクが低くなる(あるいは慎重にレビューされる)ためと解釈できる。逆に、小規模で分散的な変更においてリスクが高まる傾向が確認された。