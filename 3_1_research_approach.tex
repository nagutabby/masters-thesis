本研究では、プログラムの変更(コミット)が「いつ起こるか分からない不規則なイベント」である点に着目し、欠陥予測を行う手法を採用する。

従来の時系列分析(株価や気温の予測など)は、一定の間隔で測られたデータを対象としている。例えば、1時間ごとの気温の変化から将来を予測する手法(ARIMAやLSTMなど)が一般的だが、これらはデータが規則正しく並んでいることを前提としている。

しかし、ソフトウェア開発におけるコミットは、決まった間隔で行われるものではない。開発者の作業ペースやプロジェクトの状況によって、頻繁に更新される時期もあれば、長期間動きがない時期もある。このように、いつ起こるか分からない出来事の記録は、点過程データと呼ばれる 。本研究では、この不規則なタイミングそのものに開発のリズムやリスクが隠れていると考え、分析の基盤とする。

具体的なアプローチとして、前回の作業から「何が、どれくらい変わったか」という変化に注目する。例えば、ある関数が書き換えられた際、単に現在の行数を測るのではなく、前回の状態から「何行増えたか」「構造がどれほど複雑になったか」といった時系列の変化を数値化する。これにより、短期間での急激な複雑化といった、静的な解析では見落とされがちな「不具合の予兆」を捉えることが可能になる。

また、本研究では「将来の数値を当てる(回帰)」のではなく、「その変更に欠陥が含まれるか、含まれないか」という「二値分類」の問題として予測を行う。これは、最終的な目的が「限られた確認作業(レビュー)の時間を、リスクの高い箇所に集中させること」にあるためである。

研究の全体像を図\ref{fig:research_approach_overview}に示す。不規則なコミットを点過程データとして整理し、その変化の内容から機械学習(ランダムフォレストなど)を用いて、欠陥が混入している確率を導き出す。

\begin{figure}[htbp]
  \centering
  \includegraphics[width=0.95\textwidth]{figures/research_approach_overview.pdf}
  \caption{研究アプローチ全体の概念図}
  \label{fig:research_approach_overview}
\end{figure}