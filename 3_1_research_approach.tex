ソフトウェア開発におけるコミットは、不規則なタイミングで発生する。開発者の作業ペースやプロジェクトの状況により、頻繁に更新される時期もあれば、長期間動きがない時期もある。この不規則性は、従来の時系列分析手法が前提とする等間隔データとは性質が異なる。

株価や気温の予測に用いられる従来の時系列分析手法(ARIMAやLSTMなど)は、データが一定間隔で測定されることを前提としている。しかし、コミットのような不規則に発生するイベントの記録は点過程データと呼ばれ、イベントの発生タイミング自体が重要な情報を持つ。本研究では、コミットを点過程データとして扱い、発生タイミングの不規則性が開発のリズムやリスクを反映すると考える。

点過程データの特性を活用するため、本研究では各コミット時点での絶対値ではなく、直前のコミットからの変化に着目する。具体的には、メソッドが書き換えられた際、現在の行数ではなく前回の状態から「何行増えたか」「構造がどれほど複雑になったか」を数値化する。この差分ベースのアプローチにより、短期間での急激な複雑化といった構造的メトリクスでは捉えられない欠陥の予兆を検出できる。変化量や変化率が重要な理由は、変更が既存コードとの整合性を崩すリスクを伴うためである。また、急激な変化は開発者の理解が不完全な状態での作業を示唆し、欠陥混入の可能性を高める可能性もある。

本研究では欠陥予測を「将来の数値を当てる回帰問題」ではなく「その変更に欠陥が含まれるか否かの二値分類問題」として扱う。これは、レビューに費やせる労力をリスクの高い箇所に集中させることが最終目的であるためである。回帰問題として定式化した場合、欠陥数の予測値は得られるが、レビュー対象の優先順位付けには欠陥混入確率の方がより有用である。

研究の全体像を図\ref{fig:research_approach_overview}に示す。不規則なコミットを点過程データとして整理し、直前のコミットからの変化を特徴量として抽出する。これらの特徴量から機械学習により欠陥混入確率を予測し、レビュー対象コミットの優先順位付けを行う。

\begin{figure}[htbp]
  \centering
  \includegraphics[width=0.95\textwidth]{figures/research_approach_overview.pdf}
  \caption{研究アプローチ全体の概念図}
  \label{fig:research_approach_overview}
\end{figure}