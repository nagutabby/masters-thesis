本研究では、コードの時系列変化を考慮したソフトウェア欠陥予測を実現するため、コミットを点過程データとして扱うアプローチを採用する。

従来の時系列分析は、株価や気温のように等間隔で観測されるデータを対象とし、ARIMA、LSTM、Prophetといった手法を用いて数値の将来的な変化を予測するものである。これらの手法は、データが規則的な時間間隔で取得されることを前提としている。

一方、ソフトウェア開発におけるコミットは、等間隔で発生するデータではなく、不規則な時間間隔で発生するイベントデータである。特にOSSプロジェクトでは、開発者の生活パターンに依存し、プロジェクトの活動期と休止期が不規則に存在する。さらに、複数の開発者が非同期的に作業するため、周期的な活動パターンを仮定することが困難である。このような特性を持つデータは、点過程データ(point process data)として位置づけられる。点過程データとは、時間軸上の不規則な点(イベント)の発生を記録したデータであり、各イベントの発生時刻とその属性が記録される。

本研究では、このようなコミットの不規則性を考慮し、直前のコミットとの差分に基づく変化量を特徴量として用いることで、時系列情報を欠陥予測に活用する。具体的には、あるメソッドがコミットnとコミットn+1で変更された場合、コミットn+1における特徴量として、コミットnからの変化量(コード行数の増減、トークン数の増減、循環的複雑度の増減など)を計算する。これにより、コードがどのように変化したかという動的な情報を捉えることができる。

予測問題の定義としては、本研究では分類問題として欠陥予測を扱う。すなわち、あるメソッドがバグを含むか否かを二値分類する。これは、従来の時系列分析が主に扱う回帰問題(数値の予測)とは異なる。バグの有無という離散的な結果を予測するため、分類アルゴリズムが適している。

図\ref{fig:research_approach_overview}に研究アプローチの全体像を示す。コミットを点過程データとして扱い、コミット間の差分から特徴量を抽出し、ランダムフォレストによりバグ混入確率を予測する。

\begin{figure}[htbp]
  \centering
  \includegraphics[width=0.95\textwidth]{figures/research_approach_overview.pdf}
  \caption{研究アプローチ全体の概念図}
  \label{fig:research_approach_overview}
\end{figure}