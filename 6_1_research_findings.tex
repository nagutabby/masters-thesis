本研究では、Cost-Benefit Curveを用いた効率的なコードレビュー戦略の提案と評価を行った。実験結果から、提案手法は従来手法と比較して、限られたレビューリソースの下でより多くの欠陥を検出できることが示された。

提案手法が高い性能を示した要因として、以下の2点が考えられる。第一に、メトリクスの選定において、コミットレベルとメソッドレベルの両方の情報を活用したことで、欠陥の予測精度が向上した。第二に、労力考慮型アプローチの採用により、実際の開発現場におけるレビュー工数の制約を適切にモデル化した。

一方で、プロジェクトによって性能にばらつきが見られた点は重要な知見である。これは、プロジェクトごとに開発プロセスやコーディング規約が異なることに起因すると考えられる。

これらの結果は、提案手法が理論的な有効性だけでなく、実際のソフトウェア開発環境においても適用可能であることを示している。特に、レビューリソースが制約される状況下での品質保証活動において、提案手法は有用な選択肢となりうる。