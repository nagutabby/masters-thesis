提案手法で導入した特徴量がどのような値の範囲と分布を持つかを把握するため、Feature Importance上位10個の特徴量について統計情報を分析した。各特徴量の最小値、最大値、中央値を算出し、特徴量の分布特性を明らかにする。表\ref{tab:feature_statistics}に、5つのプロジェクトにおける主要な特徴量の統計情報を示す。

\begin{table}[ht]
\centering
\caption{Feature Importance上位10個の特徴量の統計情報(抜粋)}
\label{tab:feature_statistics}
\small
\begin{tabular}{|l|l|r|r|r|}
\hline
プロジェクト & 特徴量 & 最小値 & 最大値 & 中央値 \\
\hline
\textbf{Elasticsearch} & lines\_added & 0.00 & 203,237.00 & 104.00 \\
 & entropy & 0.00 & 1.00 & 0.73 \\
 & num\_files & 0.00 & 3,651.00 & 5.00 \\
 & lines\_deleted & 0.00 & 325,455.00 & 33.00 \\
 & tokens\_change & -395.00 & 357.00 & 0.00 \\
 & length\_change & -99.00 & 35.00 & 0.00 \\
 & operation\_type\_added & 0.00 & 1.00 & 0.00 \\
\hline
\textbf{Hazelcast} & lines\_added & 0.00 & 13,739.00 & 139.00 \\
 & num\_files & 0.00 & 549.00 & 5.00 \\
 & lines\_deleted & 0.00 & 8,368.00 & 32.00 \\
 & entropy & 0.00 & 1.00 & 0.73 \\
 & tokens\_change & -209.00 & 158.00 & 0.00 \\
\hline
\textbf{Neo4j} & lines\_added & 0.00 & 3,893.00 & 203.00 \\
 & lines\_deleted & 0.00 & 149,409.00 & 53.00 \\
 & entropy & 0.00 & 1.00 & 0.79 \\
 & num\_files & 0.00 & 1,009.00 & 8.00 \\
 & tokens\_change & -130.00 & 180.00 & 0.00 \\
 & length\_change & -37.00 & 60.00 & 0.00 \\
 & operation\_type\_added & 0.00 & 1.00 & 0.00 \\
\hline
\textbf{Netty} & tokens\_change & -529.00 & 190.00 & 0.00 \\
 & lines\_deleted & 0.00 & 4,202.00 & 34.00 \\
 & lines\_added & 0.00 & 3,327.00 & 119.00 \\
 & length\_change & -83.00 & 46.00 & 0.00 \\
 & num\_files & 1.00 & 780.00 & 4.00 \\
 & operation\_type\_added & 0.00 & 1.00 & 0.00 \\
 & ccn\_change & -22.00 & 12.00 & 0.00 \\
 & entropy & 0.00 & 1.00 & 0.76 \\
\hline
\textbf{OrientDB} & lines\_added & 0.00 & 11,029.00 & 102.00 \\
 & lines\_deleted & 0.00 & 8,234.00 & 29.00 \\
 & num\_files & 0.00 & 124.00 & 3.00 \\
 & entropy & 0.00 & 1.00 & 0.59 \\
\hline
\end{tabular}
\end{table}

コミット単位の変更メトリクスであるlines\_addedとlines\_deletedは、全てのプロジェクトで広い値の範囲を持つことが確認された。lines\_addedの最大値は、Elasticsearchで203,237行、Hazelcastで13,739行、Neo4jで3,893行、Nettyで3,327行、OrientDBで11,029行である。一方、中央値はそれぞれ104行、139行、203行、119行、102行と比較的小さい。これは、大部分のコミットは小規模な変更であるが、一部に非常に大規模な変更を含むコミットが存在することを示している。lines\_deletedも同様の傾向を示しており、最大値はElasticsearchで325,455行と極めて大きいが、中央値は33行と小さい。この分布の非対称性は、ソフトウェア開発において大規模なリファクタリングやモジュール削除が稀に発生することを反映している。

num\_files(変更ファイル数)も同様に広い値の範囲を持つ。最大値はElasticsearchで3,651ファイル、Hazelcastで549ファイル、Neo4jで1,009ファイル、Nettyで780ファイル、OrientDBで124ファイルである。中央値は3から8ファイルの範囲にあり、多くのコミットは少数のファイルを変更するが、一部のコミットでは非常に多くのファイルにまたがる変更が行われることを示している。特にElasticsearchでは、最大値と中央値の差が極めて大きく、プロジェクト全体に影響を与える大規模な変更が存在することが分かる。

entropy(変更の分散度)は、0から1の範囲で正規化された値を取る。全てのプロジェクトで最小値は0、最大値は1である。中央値は0.59から0.79の範囲にあり、多くのコミットで変更が複数のファイルに分散していることを示している。特にNeo4jでは中央値が0.79と高く、変更が広範囲に分散する傾向が強いことが分かる。

メソッド単位の変更メトリクスであるtokens\_change(トークン数の変化量)は、全てのプロジェクトで中央値が0である。これは、多くのメソッドが変更されないか、変更されても小規模であることを意味する。最小値は負の値を取り、Elasticsearchで-395、Hazelcastで-209、Neo4jで-130、Nettyで-529である。負の値は、メソッドからトークンが削除されたことを示している。最大値はElasticsearchで357、Hazelcastで158、Neo4jで180、Nettyで190であり、一部のメソッドでは大幅にトークン数が増加していることが分かる。

length\_change(コード行数の変化量)も、tokens\_changeと同様に中央値が0である。最小値は負の値を取り、Elasticsearchで-99、Neo4jで-37、Nettyで-83である。最大値はElasticsearchで35、Neo4jで60、Nettyで46であり、メソッドのコード行数の変化は比較的小規模であることが示されている。

operation\_type\_added(追加操作のフラグ)は、0または1の二値を取るバイナリ特徴量である。中央値は全てのプロジェクトで0であり、メソッドが新規追加されないコミットの方が多いことが分かる。

ccn\_change(McCabeの循環的複雑度の変化量)は、Nettyでのみ上位10位以内に含まれている。最小値は-22、最大値は12、中央値は0である。この分布は、大部分のメソッドで循環的複雑度が変化しないか、変化しても小規模であることを示している。

全体として、コミット単位の変更メトリクスは広い値の範囲を持ち、最大値と中央値の差が大きいことから、分布が右に歪んでいることが確認された。これは、少数の大規模なコミットが全体の統計に大きく影響を与えていることを示している。一方、メソッド単位の変更メトリクスは、中央値が0であり、大部分のメソッドが変更されないか小規模な変更にとどまることが示された。この分布特性は、機械学習モデルが少数の大規模変更や特定のメソッドの変化パターンを捉えることで、バグ混入リスクを予測していることを示唆している。