メソッド単位では、個々のメソッドがどの程度変更されたかを捉えるため、以下の変化量を特徴量として用いる。

\begin{itemize}
    \item コード行数の変化量
    \begin{itemize}
        \item 直前のコミットと比較したときのコード行数の増減
    \end{itemize}
    \item トークン数の変化量
    \begin{itemize}
        \item 直前のコミットと比較したときのトークン数の増減
    \end{itemize}
    \item 循環的複雑度の変化量
    \begin{itemize}
        \item 直前のコミットと比較したときの循環的複雑度の増減
    \end{itemize}
\end{itemize}

これらのメトリクスで変化量を採用する理論的根拠は以下の通りである。第一に、メソッドは比較的スケールが均一である。個々のメソッドは通常、特定の機能を実装するために設計されるため、極端に大きなものや小さなものが混在することは少ない。第二に、小規模メソッドでの変化率の不安定性を回避できる。例えば、10行のメソッドに2行追加した場合、変化率は20\%となるが、100行のメソッドに2行追加した場合、変化率は2\%となる。このように、変化率を用いると小規模なメソッドでは値が極端に大きくなり、予測モデルにノイズをもたらす可能性がある。第三に、実際の作業負荷との対応である。「2行の追加」と「50行の追加」では、明らかに後者の方がレビュイーとレビュアーの双方にとって作業負荷が高く、欠陥発生率が高くなる傾向にある。変化量はこうした実際の作業負荷を直接的に反映する。