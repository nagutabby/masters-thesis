本研究の目的は、コード変更の特徴を考慮した欠陥予測手法を構築し、レビューに費やせる労力をリスクの高い変更箇所に効率的に配分する意思決定支援モデルを提示することである。具体的には、以下の3つの目的を達成することを目指す。

1つ目は、異なる構成要素からなる変更メトリクスの活用である。本研究では、メソッド単位での局所的な変化と、コミット単位での全体的な変化を考慮した特徴量を検討する。これにより、単一の構成要素では見落とされがちな「局所的な修正が全体に及ぼすリスク」を捉える。

2つ目は、不規則な時系列変化を考慮した欠陥予測モデルの構築である。コード変更を等間隔なデータではなく、不規則なタイミングで発生するイベントとして扱い、直前の状態からの差分を学習することで、変更の特徴に基づいて欠陥混入リスクを検出する。

3つ目は、レビュー労力の測定と欠陥発見率の向上である。単なる予測精度の向上に留まらず、変更の規模と複雑さに基づくレビュー労力を定義し、少ないレビュー労力で欠陥発見率を改善する。これにより、同一のレビュー労力において、従来手法よりも高い欠陥発見率を達成できることを示す。