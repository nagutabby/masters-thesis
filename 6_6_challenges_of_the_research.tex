本節では、1.2節で設定した3つの研究目的に対する達成状況を振り返り、特に達成できなかった点とその理由について考察する。

1つ目の目的「異なる構成要素からなる変更メトリクスの活用」については、メソッド単位の局所的な変化とコミット単位の全体的な変化を組み合わせた特徴量を設計し、全プロジェクトで予測精度の向上を確認できた。5.2節で示したように、両者を組み合わせることで単独使用時よりも高いF1スコアを達成しており、複数の構成要素の変化を同時に捉えることの有効性を示すことができた。

2つ目の目的「不規則な時系列変化を考慮した欠陥予測モデルの構築」については、コミットを点過程データとして扱い、直前の状態からの差分を用いて特徴量を抽出する手法を提案した。5.3節の結果から、この差分ベースのアプローチが構造的メトリクスのみの手法と比較して一貫した性能向上をもたらすことが示された。

3つ目の目的「レビュー労力の測定と欠陥発見率の向上」については、変更の規模と複雑度に基づくレビュー労力指標を定義し、5.4節で示したように、同一のレビュー労力でベースライン手法よりも多くの欠陥を発見できることを確認した。

一方で、変更の目的を変更の特徴と関連付け、欠陥の原因を特定することはできていない。6.2節で議論したように、小規模な変更と大規模な変更ではその目的が異なる。変更の目的の相違は、欠陥混入リスクに大きく影響を与える可能性があり、欠陥の原因とも関連性が高いが、本研究ではこの変更の意図を抽出し、活用することができなかった。

変更の目的を特定することの利点はいくつかある。例えば、6.2節で観察された変更の規模と欠陥混入確率の逆転現象について、変更の目的に応じた説明ができるようになる。具体的には、「欠陥修正を目的とした小規模な変更」と「機能追加を目的とした小規模な変更」を区別して欠陥混入確率を算出することで、変更規模と欠陥混入確率の関係が変更の目的によってどのように異なるかを明らかにできる。Hindleらの研究\cite{hindle2008}が示唆するように、小規模な変更は修正を意図したものであることが多いという傾向が確認されれば、小規模な変更に見られる高い欠陥混入確率は、変更の目的そのもの(過去の欠陥を修正する過程で新たな欠陥を混入させる)に起因すると結論付けることができる。逆に、機能追加を目的とした小規模な変更の欠陥混入確率が低ければ、変更規模が小さいこと自体が安全性を保証するわけではなく、変更の目的が重要であることが明確になる。

このように、変更の目的を明らかにするための手段を考えることは、欠陥が混入した原因を突き止めるのに役立つ。しかし、変更の目的を短時間で特定するのは容易ではない。例えば、ソフトウェア欠陥の分類手法として広く用いられているOrthogonal Defect Classification(ODC)を適用した研究\cite{agnelo2020}では、NoSQLデータベースの4096件の欠陥を手動で分類するために、まず300件の欠陥を用いて研究者をトレーニングし、その後、各欠陥に対して複数の属性(Activity、Trigger、Impact、Target、Defect Type、Qualifier)を手作業で追加している。

しかし、このような詳細な欠陥分類を実施するには、専門的な知識と訓練が必要であり、大量のコミットに対して手動で適用する場合はさらに多くの時間が必要になる。欠陥の詳細な分類には多大な労力が必要であり、本研究で扱う数千から数万のコミットに対して同様のアプローチを適用することは困難である。さらに、ODCのような既存の欠陥分類手法は、欠陥が発見された後に分析が行われることを前提としている。一方、本研究で必要なのは、コミット時点での変更の目的の特定であり、欠陥がまだ顕在化していない可能性のある変更の意図を推定する必要がある。

これらの課題を解決するために、欠陥に対するラベル付けの負担を最小限に抑えつつ、変更が加えられたときにその目的を推定する手法が求められる。これまでは分析対象となるソフトウェアについての深い知識を持つ開発者が欠陥を手動で分類していた。これは、彼らが経験の蓄積によって、特定の変更の背景を推測することに長けていたからであると考えられる。このことから、これまでに行われた変更の目的を理解した上で、新しい変更の目的を推測することが、欠陥分類の精度を担保する上で重要であるといえる。

この知見から、変更目的の推定を効率的かつ迅速に行うためには、熟練した開発者が暗黙的に参照している情報を明らかにし、いくつかの要素に整理する必要がある。変更目的の推定に必要な要素には、以下のようなものがある。

1つ目は、コミットに関する情報である。これには、コミットメッセージのテキスト、コミットに紐付けられた課題管理システムの課題番号とそのラベル("bug", "enhancement", "refactoring"など)、レビューでのコメントや議論の内容が含まれる。これらの情報は、開発者が変更の目的を明示的に記述したものであり、最も直接的な情報源である。ただし、プロジェクトや開発者によって記述の詳細度が大きく異なるため、これらの情報が実際にどの程度役立つのかを事前に評価する必要がある。

2つ目は、コード変更の構造に関する情報である。これには、変更された構成要素の機能(テストコード、ドキュメント、設定ファイルなど)、変更を構成する処理(エラーハンドリングなど)が含まれる。例えば、テストファイルのみが変更されている場合は「テスト追加」である可能性が高く、エラーハンドリングのコードが追加されている場合は「欠陥修正」である可能性が高い。これらの情報は、静的コード解析ツールや抽象構文木(AST)の比較により自動的に抽出できるため、コミットメッセージが不十分な場合でも有効な判断材料となる。

3つ目は、変更の文脈に関する情報である。これには、変更された構成要素の過去の変更履歴(過去にどのような目的の変更が多かったか)、その構成要素における過去の欠陥報告の有無と頻度、変更箇所の周辺コードの特性(複雑度、結合度など)、プロジェクト全体における最近の変更傾向(リリース前の欠陥修正フェーズか、新機能開発フェーズかなど)が含まれる。熟練した開発者は、これらの文脈情報を参照することで、同じコード変更でも異なる解釈を行う。例えば、過去に頻繁に欠陥が報告されている構成要素への変更は「欠陥修正」である可能性が高いが、新規に作成されたファイルへの変更は「機能追加」である可能性が高い。

4つ目は、開発者に関する情報である。これには、変更を行った開発者のプロジェクトへの貢献履歴(貢献期間、過去のコミット数)、その開発者が過去に行った変更の目的の傾向(主に欠陥修正を担当しているか、新機能開発を担当しているか)、その開発者の専門領域(特定のモジュールやコンポーネントに精通しているか)が含まれる。6.2節で議論したように、経験の浅い開発者は小規模な欠陥修正を担当することが多いという仮説があり、開発者の特性は変更目的の推定において重要な手がかりとなる。

これらの情報を組み合わせることで、効率的かつ迅速に変更目的を推定できるようになると考えられる。変更目的を推定する際は、利用できる情報の質と量がプロジェクトによって大きく異なることを認識し、各プロジェクトにおいて最も信頼性の高い情報源を優先的に活用することが重要である。例えば、コミットメッセージが詳細に記述されているプロジェクトでは、コミットに関する情報を重視すべきであり、コミットメッセージが簡潔なプロジェクトでは、コード変更の構造に関する情報や変更の文脈に関する情報により重点を置くべきである。