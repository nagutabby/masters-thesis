機械学習技術の進展により、欠陥予測の精度は大きく向上してきた。Menziesらは、ナイーブベイズ、決定木、ロジスティック回帰など複数のアルゴリズムを比較し、適切なアルゴリズム選択の重要性を示した\cite{menzies2007}。

ランダムフォレストは、欠陥予測において広く用いられるアルゴリズムである。Ghotraらは、決定木ベースのアルゴリズムが他のアルゴリズムと比較して安定した高い性能を示すことを確認した\cite{ghotra2015}。その利点は、過学習に強く、非線形な関係を捉えられ、モデルの解釈が可能なことである。特に、特徴量重要度により各特徴量の予測への寄与度を定量化でき、Partial Dependence Plot(PDP)\cite{friedman2001}により特定の特徴量と予測結果の関係を可視化できる。これらの解釈性ツールは、予測モデルの振る舞いを理解し、実務への適用を判断する上で重要である。

欠陥予測における重要な課題の一つは、データの不均衡である。実際のソフトウェアでは、欠陥を含むコードは全体の一部であり、陽性クラスと陰性クラスの比率が大きく偏っている。

不均衡データへの対処として、サンプリング技術が用いられる。アンダーサンプリングは多数クラスのサンプル数を削減し、オーバーサンプリングは少数クラスのサンプル数を増加させる。アンダーサンプリングは計算コストが低いが情報が失われる可能性があり、オーバーサンプリングは情報を保持できるが計算コストが増加し、過学習のリスクがある。