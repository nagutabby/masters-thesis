ソフトウェアの品質に影響を与える主な要因としては、コードの構造と開発プロセスがある。これらと欠陥の間に存在する複雑な関係をモデル化するために、機械学習技術が活用されている。先行研究では、多様なアルゴリズムの比較を通じた適切なモデル選定によって予測精度を向上させられることが示されており、人間が定義するルールでは捉えきれない潜在的な欠陥混入リスクを自動的に抽出する重要性が論じられている\cite{menzies2007}。また、決定木を用いた機械学習モデルが、過学習を抑制しながら安定した高い分類性能を発揮することも確認されており、変数間の非線形関係を正確に捉える能力が、ソフトウェアの欠陥の判別に役立つことが示されている\cite{ghotra2015}。さらに、予測結果を実際の品質保証活動に繋げるためには、モデルの判断基準を可視化し、開発者が納得できる客観的な根拠を提示することが必要である。そのため、Partial Dependence Plotなどの手法を用いて、予測基準を可視化することが求められる\cite{friedman2001}。